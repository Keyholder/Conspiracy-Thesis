\pdfbookmark{Общая характеристика работы}{characteristic}             % Закладка pdf
\section*{Общая характеристика работы}

\newcommand{\actuality}{\pdfbookmark[1]{Актуальность}{actuality}\underline{\textbf{\actualityTXT}}}
\newcommand{\progress}{\pdfbookmark[1]{Разработанность темы}{progress}\underline{\textbf{\progressTXT}}}
\newcommand{\aim}{\pdfbookmark[1]{Цели}{aim}\underline{{\textbf\aimTXT}}}
\newcommand{\tasks}{\pdfbookmark[1]{Задачи}{tasks}\underline{\textbf{\tasksTXT}}}
\newcommand{\aimtasks}{\pdfbookmark[1]{Цели и задачи}{aimtasks}\aimtasksTXT}
\newcommand{\novelty}{\pdfbookmark[1]{Научная новизна}{novelty}\underline{\textbf{\noveltyTXT}}}
\newcommand{\influence}{\pdfbookmark[1]{Практическая значимость}{influence}\underline{\textbf{\influenceTXT}}}
\newcommand{\methods}{\pdfbookmark[1]{Методология и методы исследования}{methods}\underline{\textbf{\methodsTXT}}}
\newcommand{\defpositions}{\pdfbookmark[1]{Положения, выносимые на защиту}{defpositions}\underline{\textbf{\defpositionsTXT}}}
\newcommand{\reliability}{\pdfbookmark[1]{Достоверность}{reliability}\underline{\textbf{\reliabilityTXT}}}
\newcommand{\probation}{\pdfbookmark[1]{Апробация}{probation}\underline{\textbf{\probationTXT}}}
\newcommand{\contribution}{\pdfbookmark[1]{Личный вклад}{contribution}\underline{\textbf{\contributionTXT}}}
\newcommand{\publications}{\pdfbookmark[1]{Публикации}{publications}\underline{\textbf{\publicationsTXT}}}


{\actuality} В теории игр конфликты с участием трёх и более преследующих собственные цели сторон по многим причинам считаются существенно более трудными для моделирования в сравнении с классическими парными противостояниями. Среди этих причин следует особо подчеркнуть влияние, оказываемое на их ход информационной асимметрией. В играх с двумя участниками её эффект, в частности, сводится к последствиям априорной неполноты их знаний о параметрах конфликта, и обычно описывается при помощи байесовских моделей. Платёжная функция в таких случаях не является общим знанием игроков "--- каждый из них действуют исходя из собственных, возможно различающихся предположений, выраженных в форме распределения вероятностей на пространстве всевозможных платёжных функций с заданным пространством стратегий.

Ещё одним источником информационной асимметрии в парном конфликте может выступать присутствие в действиях оппонентов тайной составляющей в тех случаях, когда он проходит в несколько стадий. При этом действия, совершённые игроком на ранних стадиях, позднее становятся полностью или частично известны его противнику, чья стратегия может варьироваться в зависимости от них. Это естественным образом формализуется при помощи информационных разбиений дерева ходов в развёрнутой форме игры. По сути, двумя упомянутыми аспектами исчерпывается влияние информационной асимметрии на конфликты с двумя сторонами. Однако, увеличение количества участников ещё хотя бы на одного порождает новый феномен, замеченный ещё Робертом Ауманом в статье \fixme{[1]}, где впервые было сформулировано коррелированное расширение игр в нормальной форме.

Выражается этот феномен в том, как для одних и тех же игр соотносятся множества равновесий по Нэшу в смешанных стратегиях и с использованием внешних механизмов корреляции. Для случая двух игроков все вектора математических ожиданий выплат в точках коррелированных равновесий принадлежат выпуклой оболочке множества смешанных, т.\:е. механизмы корреляции можно рассматривать просто как способ получения линейных комбинаций классических решений. С появлением третьего игрока картина меняется "--- в некоторых играх присутствие непубличного корреляционного механизма позволяет достичь точек равновесия по Нэшу с выплатами за пределами выпуклой оболочки решений в смешанных стратегиях. Фактически, это означает, что асимметрия знаний может оказывать существенное влияние на исход многостороннего конфликта даже в тех случаях, когда собственно предмет знаний вообще не имеет к нему отношения. Назовём игры, подверженные этому эффекту, \emph{чувствительными к дополнительной информационной асимметрии}.

В качестве простейшего примера приведём игру <<трёхсторонний чёт"~нечет>>, где каждый из трёх участников тайно выбирает <<орла>> или <<решку>> на своей монете и прижимает её к столу соответствующей стороной вверх, после чего все одновременно поднимают ладони и в зависимости от сложившейся комбинации делят фиксированный банк. Когда все три монеты лежат одной и той же стороной, раунд считается сыгранным вничью и игроки делят банк поровну. Если же совпали только две из них, то оказавшийся в меньшинстве игрок считается проигравшим и не получает доли при дележе банка. В матричной форме это можно описать так:
\begin{table} [htbp]
	\centering
	\begin{threeparttable}
		\caption{Трёхсторонний чёт"~нечет}
		\label{tab:coin3}
		\begin{tabular}{ |c|c|c|c|c| }
			\cline{1-2} \cline{4-5}
			\rule[-7pt]{0pt}{2em}$4, 4, 4$ &
			\rule[-7pt]{0pt}{2em}$6, 0, 6$ & \qquad\qquad\qquad &
			\rule[-7pt]{0pt}{2em}$6, 6, 0$ &
			\rule[-7pt]{0pt}{2em}$0, 6, 6$ \\
			\cline{1-2} \cline{4-5}
			\rule[-7pt]{0pt}{2em}$0, 6, 6$ &
			\rule[-7pt]{0pt}{2em}$6, 6, 0$ & \qquad\qquad\qquad &
			\rule[-7pt]{0pt}{2em}$6, 0, 6$ &
			\rule[-7pt]{0pt}{2em}$4, 4, 4$ \\
			\cline{1-2} \cline{4-5}
		\end{tabular}
	\end{threeparttable}
\end{table}

В таблице \ref{tab:coin3} первый игрок выбирает строку, второй "--- столбец, а третий "--- матрицу. Решением этой игры в чистых стратегиях являются два равновесия Нэша, соответствующие синхронным выборам одинаковых сторон всеми игроками. В смешанных стратегиях добавляется ещё одно вырожденное решение, когда каждый игрок делает случайный выбор между орлом и решкой с равными вероятностями. Все эти решения, очевидно, дают математическое ожидание платежей равное $(4,4,4)$. В рамках классической теории игр этим анализ игры и исчерпывается, однако, добавление фактора информационной асимметрии делает ситуацию интереснее. Представим, что принятие решения игроками предваряется случайным событием, исход которого становится известен только двум игрокам из трёх "--- к примеру, некий доброжелатель подбрасывает симметричную монету и по секрету сообщает результат первому и второму игроку. Хотя это событие никак не влияет на платёжную матрицу, оно позволяет обоим узнавшим о нём игрокам использовать условные стратегии вида <<если выпала сторона $s$, выбрать сторону $f(s)$>>. Таким образом, если первый и второй игроки договорятся положить свои монеты на стол той же стороной, какой выпала монета доброжелателя, то их решения будут совпадать всегда, а вот любая стратегия третьего игрока даст совпадение с остальными только в половине случаев. Математическое ожидание платежей при этом составляет $(5,5,2)$, причём ни для одного из игроков нет выгодного индивидуального отклонения.

%Для большей формальности можно предложить следующее:
%\begin{definition}
%	Пусть $\Gamma = \langle A, S^a, u^a(s), a \in A \rangle$ "--- игра в нормальной форме c $m$ участниками, а $U \subseteq \mathbb{R}^m$ "--- множество всех векторов выплат, достижимых в её смешанных равновесиях по Нэшу. Игра $\Gamma$ называется \emph{чувствительной к дополнительной информационной асимметрии}, когда существует пространство корреляции $\Phi = \langle A, \Omega, \mathfrak{I}^a, \mathbb{P}, a \in A \rangle$ такое, что в игре $\Gamma | \Phi$ найдётся коррелированное равновесие по Нэшу с вектором выплат, не принадлежащим выпуклой оболочке множества $U$.
%\end{definition}

%Здесь коррелированное расширение игры в нормальной форме $\Gamma = \langle A, S^a, u^a(s), a \in A \rangle$ записывается в нотации, адаптированной к русскоязычным традициям. Далее, если не сказано иного, конечное множество игроков обозначается как $A = \{1, \ldots, m\}$, а конечное множество наборов чистых стратегий "--- $S = S^1 \times \ldots \times S^m$. Помимо множества стратегий $S^a$ каждый игрок определяется платёжной функцией $u^a : S \rightarrow \mathbb{R}$. В вероятностном пространстве\cite{kolmog74} $\langle \Omega, \mathfrak{B}, \mathbb{P} \rangle$, в реализуется наблюдаемое игроками состояние природы. Здесь $\Omega$ "--- множество всевозможных таких состояний, $\mathfrak{B}$ "--- $\sigma$"~алгебра подмножеств $\Omega$, а $\mathbb{P} : \mathfrak{B} \rightarrow \mathbb{R}_{\ge 0}$ "--- вероятностная мера. Каждому игроку $a \in A$ поставим в соответствие \emph{собственное подпространство} $\langle \Omega, \mathfrak{I}^a, \mathbb{P} \rangle$ такое, что $\mathfrak{I}^a \subseteq \mathfrak{B}$. При этом набор $\sigma$"~алгебр $\mathfrak{I} = (\mathfrak{I}^a, a \in A)$ отражает информированность игроков о состоянии природы. В описываемой ситуации это состояние не влияет на функции выигрышей непосредственно, выступая исключительно как способ синхронизации действий игроков. Это значит, что $\sigma$"~алгебра $\mathfrak{B}$ сама по себе не является существенным параметром модели, и измеримость по ней для $\mathbb{P}$ можно заменить измеримостью по $\mathfrak{I}^a, \forall a \in A$.

%Отдельно следует заметить, что в оригинале для собственных подпространств игроков формализм Аумана предполагал индивидуальность не только $\sigma$"~алгебр, но и соответствующих им мер, учитывая тем самым возможную субъективность оценок вероятности наступления тех или иных событий, что немаловажно в случаях, когда в качестве механизма корреляции выступают процессы, слишком сложные для объективного анализа (например спортивные соревнования). Однако, для целей данного исследования этот аспект не имеет большого смысла, поскольку предлагающаяся модель подразумевает, что участники конфликта могут произвольным образом выбирать механизм корреляции, а в такой ситуации разумно ожидать, что они будут использовать простые источники случайности с известным распределением (рулетки, кости, жребий и т.\:д.). По этой причине здесь и далее формализм коррелированных стратегий используется в его упрощённой форме, с общей для всех игроков объективной вероятностной мерой в пространстве состояний природы.

Хотя описанный феномен известен уже давно, при построении моделей большинство исследователей обходили его стороной, рассматривая скорее как курьёзное свойство некоторых игр многих игроков. Важным исключением при этом выступает, пожалуй, наиболее значимая из активно использующих формализм коррелированного расширения игр в нормальной форме область теории игр "--- <<дизайн механизмов>> \fixme{[2]} Леонида Гурвича, Эрика Маскина и Роджера Майерсона. Их подход ставит своей целью создание экономических инструментов, стимулирующих эгоистичных рациональных агентов к поведению, оптимальному с точки зрения общих целевых функций, формализующих различные социальные блага. Будучи чрезвычайно плодотворной областью исследований, дизайн механизмов породил множество направлений и ответвлений, объединённых тем не менее рядом неотъемлемых общих черт, проистекающих из информационной структуры игр, для которых доказываются его основные положения. Типичная схема взаимодействий выглядит так "--- игроки"~агенты, знающие свои предпочтения и возможности, но находящиеся в неведении относительно этих параметров у других участников, информируют о них центр, формирующий на основе этой информации набор коррелированных стратегий. Далее центр реализует его для конкретного случая в виде набора чистых стратегий и инструктирует каждого из агентов, которые в свою очередь и принимают окончательное решение о том или ином действии. При этом подразумевается, что агенты могут лгать на первом этапе и не подчиняться на последнем. Главной задачей дизайна механизмов в этой парадигме становится создание таких алгоритмов поведения центра, что стратегии правдивости и послушания образуют для агентов равновесие Нэша. Несложно заметить, что вышеописанная игра с доброжелателем вполне может быть сформулирована в терминах дизайна механизмов, если предположить, что по какой-то причине центр благоволит первому и второму игроку (в его целевой функции их выигрыш имеет больший вес).

%Опишем предложенную ситуацию в терминах коррелированного расширения, дополнив игру вероятностным пространством, в котором реализуется состояние природы, использующееся игроками в качестве сигнала для синхронизации их действий. Таким образом, стратегиями игроков становятся функции, отображающие множество всевозможных сигналов в соответствующие множества чистых стратегий. При этом информационная асимметрия достигается за счёт требования измеримости стратегии каждого игрока относительно его индивидуальной $\sigma$"~алгебры, вложенной в $\sigma$"~алгебру общего вероятностного пространства. То есть, некоторые состояния природы могут быть различимы с точки зрения одних игроков и идентичны для других.

Не вдаваясь в детали, можно сказать, что дизайн механизмов базируется на частном случае информационной асимметрии "--- своего рода звёздчатой структуре связей, где выделенный центральный агент может в своих интересах распоряжаться общим механизмом корреляции, а подчинённые ему агенты находятся в полной изоляции как друг от друга, так и от остального мира. Название здесь действительно неплохо отражает свойственный для модели взгляд на конфликты "--- через её призму стороны рассматриваются как взаимозаменимые детали единого рукотворного механизма, не связанные ничем кроме участия в нём. Хотя моделирование в рамках подобного упрощения вполне может быть полезно при конструировании формализованных способов решения конфликтов, оно никак не может помочь в тех случаях, когда на них оказывает существенное влияние информационная асимметрия, складывающаяся не в результате сознательного дизайна, а естественным образом, по мере спонтанного взаимодействия агентов в неконтролируемой, внешней с точки зрения модели среде. К примеру, сложно переоценить значимость влияния коррупции на политические и экономические институты, а ведь она складывается именно из таких незапланированных информационных связей, внешних по отношению к самим институтам.

По описанной причине исследование влияния дополнительной информационной асимметрии на решения игр многих игроков никак нельзя сводить только к конструктивным моделям. Увы, но за пределами дизайна механизмов сложилась традиция игнорировать этот феномен. К примеру, в статье \fixme{[3]} Дрю Фуденберга и Эрика Маскина можно найти следующую сноску: <<Actually, if $n \ge 3$, the other players may be able to keep player $j$'s payoff even lower by using a correlated strategy against $j$, where the outcome of the correlating device is not observed by $j$ (...). In keeping with the rest of the literature on repeated games, however, we shall rule out such correlated strategies.>> А ведь казалось бы, в контексте повторяющихся игр с дисконтированием проблематика использования секретности механизма корреляции для усиления стратегий наказания довольно любопытна "--- наверное в каждой области исследований, использующей народную теорему, от антропологии до международной политики несложно отыскать примеры того, как группы агентов усиливали свою коллективную долгосрочную позицию при помощи необходимо тайного согласования действий. Увы, но приходится констатировать, что теории игр до сих пор почти нечего предложить другим наукам в качестве инструмента анализа описанного феномена.

%Обзор, введение в тему, обозначение места данной работы в
%мировых исследованиях и~т.\:п., можно использовать ссылки на~другие
%работы~\autocite{Gosele1999161,Lermontov}
%(если их~нет, то~в~автореферате
%автоматически пропадёт раздел <<Список литературы>>). Внимание! Ссылки
%на~другие работы в~разделе общей характеристики работы можно
%использовать только при использовании \verb!biblatex! (из-за технических
%ограничений \verb!bibtex8!. Это связано с тем, что одна
%и~та~же~характеристика используются и~в~тексте диссертации, и в
%автореферате. В~последнем, согласно ГОСТ, должен присутствовать список
%работ автора по~теме диссертации, а~\verb!bibtex8! не~умеет выводить в~одном
%файле два списка литературы).
%При использовании \verb!biblatex! возможно использование исключительно
%в~автореферате подстрочных ссылок
%для других работ командой \verb!\autocite!, а~также цитирование
%собственных работ командой \verb!\cite!. Для этого в~файле
%\verb!common/setup.tex! необходимо присвоить положительное значение
%счётчику \verb!\setcounter{usefootcite}{1}!.
%
%Для генерации содержимого титульного листа автореферата, диссертации
%и~презентации используются данные из файла \verb!common/data.tex!. Если,
%например, вы меняете название диссертации, то оно автоматически
%появится в~итоговых файлах после очередного запуска \LaTeX. Согласно
%ГОСТ 7.0.11-2011 <<5.1.1 Титульный лист является первой страницей
%диссертации, служит источником информации, необходимой для обработки и
%поиска документа>>. Наличие логотипа организации на~титульном листе
%упрощает обработку и~поиск, для этого разметите логотип вашей
%организации в папке images в~формате PDF (лучше найти его в векторном
%варианте, чтобы он хорошо смотрелся при печати) под именем
%\verb!logo.pdf!. Настроить размер изображения с логотипом можно
%в~соответствующих местах файлов \verb!title.tex!  отдельно для
%диссертации и автореферата. Если вам логотип не~нужен, то просто
%удалите файл с~логотипом.
%
%\ifsynopsis
%Этот абзац появляется только в~автореферате.
%Для формирования блоков, которые будут обрабатываться только в~автореферате,
%заведена проверка условия \verb!\!\verb!ifsynopsis!.
%Значение условия задаётся в~основном файле документа (\verb!synopsis.tex! для
%автореферата).
%\else
%Этот абзац появляется только в~диссертации.
%Через проверку условия \verb!\!\verb!ifsynopsis!, задаваемого в~основном файле
%документа (\verb!dissertation.tex! для диссертации), можно сделать новую
%команду, обеспечивающую появление цитаты в~диссертации, но~не~в~автореферате.
%\fi

% {\progress}
% Этот раздел должен быть отдельным структурным элементом по
% ГОСТ, но он, как правило, включается в описание актуальности
% темы. Нужен он отдельным структурынм элемементом или нет ---
% смотрите другие диссертации вашего совета, скорее всего не нужен.

{\aim} данной работы является создание новой модели многосторонних конфликтов, учитывающей влияние на их ход дополнительной информационной асимметрии.

Для~достижения поставленной цели необходимо было решить следующие {\tasks}:
\begin{enumerate}[beginpenalty=10000] % https://tex.stackexchange.com/a/476052/104425
	\item Исследовать формализм коррелированного обобщения игр в нормальной форме с точки зрения проблематики работы.
	\item Разработать способ описания информационных структур, достаточно разнообразным образом связывающих участников произвольного конфликта.
	\item Исследовать влияние дополнительной информационной асимметрии на соответствие равновесий критериям коллективной рациональности.
	\item Разработать приемлемую концепцию решения с учётом связей между агентами для игр с дополнительной информационной асимметрией.
%  \item Исследовать, разработать, вычислить и~т.\:д. и~т.\:п.
%  \item Исследовать, разработать, вычислить и~т.\:д. и~т.\:п.
%  \item Исследовать, разработать, вычислить и~т.\:д. и~т.\:п.
%  \item Исследовать, разработать, вычислить и~т.\:д. и~т.\:п.
\end{enumerate}


{\novelty}
\begin{enumerate}[beginpenalty=10000] % https://tex.stackexchange.com/a/476052/104425
	\item Были впервые выделены в качестве самостоятельного объекта исследования игры многих игроков, проявляющие чувствительность к дополнительной информационной асимметрии.
	\item Был впервые предложен формализм пространства заговоров, специальным образом сужающий в целях моделирования дополнительной информационной асимметрии формализм пространства корреляции.
	\item Была впервые сформулирована концепция структурно согласованного равновесия, позволяющая во многих случаях выделять среди решений игр в пространствах заговоров отвечающие достаточно тонкому принципу коллективной рациональности.  
%  \item Впервые \ldots
%  \item Впервые \ldots
%  \item Было выполнено оригинальное исследование \ldots
\end{enumerate}

{\influence} работы проистекает из явной необходимости учитывать при моделировании многосторонних конфликтов тот факт, что состав их участников не является в большинстве случаев случайной выборкой никак не связанных друг с другом агентов. Классический формализм игр в нормальной форме опирается на неявное допущение, состоящее в том, что единственной значимой характеристикой каждого игрока является порядок его предпочтений относительно исхода розыгрыша, выражающийся в форме платёжной функции. Совершенно очевидно при этом, что реальных людей, вступающих в противостояние, зачастую связывают значимые для его исхода отношения, структура которых не может быть выражена простым сочетанием платёжных функций. В качестве наглядной иллюстрации такой связи можно сравнить две воображаемые партии в бридж или преферанс с участием одинаково сильных игроков, различающиеся тем, что в одном случае за столом сидят незнакомцы, а в другом часть из них играют вместе уже много лет. Любой достаточно опытный картёжник скажет, что при равных навыках фактор <<сыгранности>> с партнёром надёжно обеспечивает решающее преимущество. Естественным образом этот феномен можно обобщить и на более значимые конфликты: политика, бизнес, дипломатия "--- везде, где исход противостояния существенно зависит от согласованности и непредсказуемости действий, взаимопонимание не требующее коммуникации зачастую может превратить поражение в победу. Таким образом, для более точного предсказания исходов многосторонних конфликтов насущно необходимы модели, позволяющие учитывать этот фактор.

{\methods} В работе используются методы теории игр, теории вероятности и топологии.

{\defpositions}
\begin{enumerate}[beginpenalty=10000] % https://tex.stackexchange.com/a/476052/104425
  \item Фактор дополнительной информационной асимметрии важен при исследовании многосторонних конфликтов, поскольку получаемые с его помощью новые решения отражают реальные преимущества, которые даёт агентам возможность тайной координации действий.
  \item Формализм игр с заговорами позволяет, оставаясь в рамках пространств с конечным описанием, моделировать широкий класс испытывающих влияние дополнительной информационной асимметрии конфликтов.
  \item Концепция структурно согласованного равновесия позволяет во многих случаях находить в играх с заговорами решения, удовлетворяющие более тонкому в сравнении с эффективностью по Парето критерию коллективной рациональности.
  \item В повторяющихся играх эффект влияния дополнительной информационной асимметрии может проявляться даже при ограничении пространства стратегий игроков классическим смешанным случаем.
\end{enumerate}
%В папке Documents можно ознакомиться с решением совета из Томского~ГУ
%(в~файле \verb+Def_positions.pdf+), где обоснованно даются рекомендации
%по~формулировкам защищаемых положений.

{\reliability} полученных результатов обеспечивается при помощи формальных доказательств с применением топологических инструментов. Результаты находятся в соответствии с результатами, полученными другими авторами.


{\probation}
Основные результаты работы докладывались~на:
перечисление основных конференций, симпозиумов и~т.\:п.

{\contribution} Автор принимал активное участие \ldots

\ifnumequal{\value{bibliosel}}{0}
{%%% Встроенная реализация с загрузкой файла через движок bibtex8. (При желании, внутри можно использовать обычные ссылки, наподобие `\cite{vakbib1,vakbib2}`).
    {\publications} Основные результаты по теме диссертации изложены
    в~XX~печатных изданиях,
    X из которых изданы в журналах, рекомендованных ВАК,
    X "--- в тезисах докладов.
}%
{%%% Реализация пакетом biblatex через движок biber
    \begin{refsection}[bl-author, bl-registered]
        % Это refsection=1.
        % Процитированные здесь работы:
        %  * подсчитываются, для автоматического составления фразы "Основные результаты ..."
        %  * попадают в авторскую библиографию, при usefootcite==0 и стиле `\insertbiblioauthor` или `\insertbiblioauthorgrouped`
        %  * нумеруются там в зависимости от порядка команд `\printbibliography` в этом разделе.
        %  * при использовании `\insertbiblioauthorgrouped`, порядок команд `\printbibliography` в нём должен быть тем же (см. biblio/biblatex.tex)
        %
        % Невидимый библиографический список для подсчёта количества публикаций:
        \printbibliography[heading=nobibheading, section=1, env=countauthorvak,          keyword=biblioauthorvak]%
        \printbibliography[heading=nobibheading, section=1, env=countauthorwos,          keyword=biblioauthorwos]%
        \printbibliography[heading=nobibheading, section=1, env=countauthorscopus,       keyword=biblioauthorscopus]%
        \printbibliography[heading=nobibheading, section=1, env=countauthorconf,         keyword=biblioauthorconf]%
        \printbibliography[heading=nobibheading, section=1, env=countauthorother,        keyword=biblioauthorother]%
        \printbibliography[heading=nobibheading, section=1, env=countregistered,         keyword=biblioregistered]%
        \printbibliography[heading=nobibheading, section=1, env=countauthorpatent,       keyword=biblioauthorpatent]%
        \printbibliography[heading=nobibheading, section=1, env=countauthorprogram,      keyword=biblioauthorprogram]%
        \printbibliography[heading=nobibheading, section=1, env=countauthor,             keyword=biblioauthor]%
        \printbibliography[heading=nobibheading, section=1, env=countauthorvakscopuswos, filter=vakscopuswos]%
        \printbibliography[heading=nobibheading, section=1, env=countauthorscopuswos,    filter=scopuswos]%
        %
        \nocite{*}%
        %
        {\publications} Основные результаты по теме диссертации изложены в~\arabic{citeauthor}~печатных изданиях,
        \arabic{citeauthorvak} из которых изданы в журналах, рекомендованных ВАК\sloppy%
        \ifnum \value{citeauthorscopuswos}>0%
            , \arabic{citeauthorscopuswos} "--- в~периодических научных журналах, индексируемых Web of~Science и Scopus\sloppy%
        \fi%
        \ifnum \value{citeauthorconf}>0%
            , \arabic{citeauthorconf} "--- в~тезисах докладов.
        \else%
            .
        \fi%
        \ifnum \value{citeregistered}=1%
            \ifnum \value{citeauthorpatent}=1%
                Зарегистрирован \arabic{citeauthorpatent} патент.
            \fi%
            \ifnum \value{citeauthorprogram}=1%
                Зарегистрирована \arabic{citeauthorprogram} программа для ЭВМ.
            \fi%
        \fi%
        \ifnum \value{citeregistered}>1%
            Зарегистрированы\ %
            \ifnum \value{citeauthorpatent}>0%
            \formbytotal{citeauthorpatent}{патент}{}{а}{}\sloppy%
            \ifnum \value{citeauthorprogram}=0 . \else \ и~\fi%
            \fi%
            \ifnum \value{citeauthorprogram}>0%
            \formbytotal{citeauthorprogram}{программ}{а}{ы}{} для ЭВМ.
            \fi%
        \fi%
        % К публикациям, в которых излагаются основные научные результаты диссертации на соискание учёной
        % степени, в рецензируемых изданиях приравниваются патенты на изобретения, патенты (свидетельства) на
        % полезную модель, патенты на промышленный образец, патенты на селекционные достижения, свидетельства
        % на программу для электронных вычислительных машин, базу данных, топологию интегральных микросхем,
        % зарегистрированные в установленном порядке.(в ред. Постановления Правительства РФ от 21.04.2016 N 335)
    \end{refsection}%
    \begin{refsection}[bl-author, bl-registered]
        % Это refsection=2.
        % Процитированные здесь работы:
        %  * попадают в авторскую библиографию, при usefootcite==0 и стиле `\insertbiblioauthorimportant`.
        %  * ни на что не влияют в противном случае
        \nocite{vakbib2}%vak
        \nocite{patbib1}%patent
        \nocite{progbib1}%program
        \nocite{bib1}%other
        \nocite{confbib1}%conf
    \end{refsection}%
        %
        % Всё, что вне этих двух refsection, это refsection=0,
        %  * для диссертации - это нормальные ссылки, попадающие в обычную библиографию
        %  * для автореферата:
        %     * при usefootcite==0, ссылка корректно сработает только для источника из `external.bib`. Для своих работ --- напечатает "[0]" (и даже Warning не вылезет).
        %     * при usefootcite==1, ссылка сработает нормально. В авторской библиографии будут только процитированные в refsection=0 работы.
}

При использовании пакета \verb!biblatex! будут подсчитаны все работы, добавленные
в файл \verb!biblio/author.bib!. Для правильного подсчёта работ в~различных
системах цитирования требуется использовать поля:
\begin{itemize}
        \item \texttt{authorvak} если публикация индексирована ВАК,
        \item \texttt{authorscopus} если публикация индексирована Scopus,
        \item \texttt{authorwos} если публикация индексирована Web of Science,
        \item \texttt{authorconf} для докладов конференций,
        \item \texttt{authorpatent} для патентов,
        \item \texttt{authorprogram} для зарегистрированных программ для ЭВМ,
        \item \texttt{authorother} для других публикаций.
\end{itemize}
Для подсчёта используются счётчики:
\begin{itemize}
        \item \texttt{citeauthorvak} для работ, индексируемых ВАК,
        \item \texttt{citeauthorscopus} для работ, индексируемых Scopus,
        \item \texttt{citeauthorwos} для работ, индексируемых Web of Science,
        \item \texttt{citeauthorvakscopuswos} для работ, индексируемых одной из трёх баз,
        \item \texttt{citeauthorscopuswos} для работ, индексируемых Scopus или Web of~Science,
        \item \texttt{citeauthorconf} для докладов на конференциях,
        \item \texttt{citeauthorother} для остальных работ,
        \item \texttt{citeauthorpatent} для патентов,
        \item \texttt{citeauthorprogram} для зарегистрированных программ для ЭВМ,
        \item \texttt{citeauthor} для суммарного количества работ.
\end{itemize}
% Счётчик \texttt{citeexternal} используется для подсчёта процитированных публикаций;
% \texttt{citeregistered} "--- для подсчёта суммарного количества патентов и программ для ЭВМ.

Для добавления в список публикаций автора работ, которые не были процитированы в
автореферате, требуется их~перечислить с использованием команды \verb!\nocite! в
\verb!Synopsis/content.tex!.
 % Характеристика работы по структуре во введении и в автореферате не отличается (ГОСТ Р 7.0.11, пункты 5.3.1 и 9.2.1), потому её загружаем из одного и того же внешнего файла, предварительно задав форму выделения некоторым параметрам

%Диссертационная работа была выполнена при поддержке грантов \dots

%\underline{\textbf{Объем и структура работы.}} Диссертация состоит из~введения,
%четырех глав, заключения и~приложения. Полный объем диссертации
%\textbf{ХХХ}~страниц текста с~\textbf{ХХ}~рисунками и~5~таблицами. Список
%литературы содержит \textbf{ХХX}~наименование.

\pdfbookmark{Содержание работы}{description}                          % Закладка pdf
\section*{Содержание работы}
Во \underline{\textbf{введении}} обосновывается актуальность
исследований, проводимых в~рамках данной диссертационной работы,
приводится обзор научной литературы по~изучаемой проблеме,
формулируется цель, ставятся задачи работы, излагается научная новизна
и практическая значимость представляемой работы. В~последующих главах сперва в терминах коррелированного расширения матричных игр формулируется модель заговоров, позволяющая анализировать влияние на многосторонние конфликты дополнительной информационной асимметрии, а~потом идёт
дополнение этой модели специализированным принципом коллективной рациональности и анализ её взаимодействия с повторяющимися играми.

\underline{\textbf{Первая глава}} посвящена обоснованию центрального формализма работы. Она начинается с обзора концепции коррелированного расширения игр в нормальной форме и введения понятия пространства корреляции:
\begin{equation*}
	\Phi = \langle A, \Omega, \mathfrak{I}^a, \mathbb{P}, a \in A \rangle
\end{equation*}

Здесь $A$ "--- конечное множество игроков, $\Omega$ "--- произвольное множество возможных состояний природы, $(\mathfrak{I}^a \subseteq 2^\Omega, a \in A)$ "--- набор индивидуальных $\sigma$"~алгебр событий, характеризующих информированность игроков о реализованном состоянии природы, а $\mathbb{P} : 2^\Omega \rightarrow [0, 1]$ "--- вероятностная мера на множестве состояний природы. Любую игру в нормальной форме $\Gamma = \langle A, S^a, u^a(s), a \in A \rangle$ можно дополнить совместимым по множеству игроков пространством корреляции $\Phi$ для получения её коррелированного расширения $\Gamma | \Phi$. В полученном расширении множеством стратегией каждого игрока $a \in A$ становится множество  всевозможных отображений $\Omega \rightarrow S^a$, измеримых по $\sigma$"~алгебре информированности $\mathfrak{I}^a$, а выигрыши вычисляются по формуле математического ожидания для заданной вероятностной меры $\mathbb{P}$. Отсюда вытекает феномен чувствительности игр к дополнительной информационной асимметрии:
\begin{definition}{1.1.1}
	Пусть $\Gamma$ --- игра в нормальной форме c $m$ участниками, а $U \subseteq \mathbb{R}^m$ --- множество всех векторов выплат, достижимых в её смешанных равновесиях по Нэшу. Игра $\Gamma$ называется \emph{чувствительной к дополнительной информационной асимметрии}, когда существует пространство корреляции $\Phi$ такое, что в игре $\Gamma | \Phi$ найдётся коррелированное равновесие по Нэшу с вектором выплат, не принадлежащим выпуклой оболочке множества $U$.
\end{definition}

Необходимые нам утверждения о пространствах корреляции удобно формулировать и доказывать вне теоретико-игрового контекста, рассматривая их как тривиальное обобщение колмогоровских вероятностных пространств. При этом вместо того, чтобы говорить о профилях коррелированных стратегий, отображающих состояние природы в набор чистых стратегий, построим их прямой аналог, лишённый дополнительных ненужных смыслов:
\begin{definition}{1.2.1}
	Разбиением пространства корреляции $\Phi = \langle A, \Omega, \mathfrak{I}^a, \mathbb{P}, a \in A \rangle$ в произвольное конечное множество исходов (кодомен) $X = X^1 \times ... \times X^m$ называется отображение $f : \Omega \rightarrow X$, состоящее из набора функций $(f^1, ..., f^m)$, где каждая $f^a : \Omega \rightarrow X^a$ измерима в $\mathfrak{I}^a$. Далее разбиение $f$ пространства корреляции $\Phi$ будем сокращённо обозначать $f \models \Phi$.
\end{definition}

Это позволяет ввести на множестве всевозможных пространств корреляции отношение изоморфизма, группирующее неразличимые с теоретико-игровой точки зрения пространства в соответствующие классы эквивалентности:

\begin{definition}{1.2.2}
	Пространство $\Phi_1$ с мерой $\mathbb{P}_1$ называется отобразимым на $\Phi_2$ с мерой $\mathbb{P}_2$ (далее $\Phi_1 \precsim \Phi_2$), если их множества игроков совпадают и для любого разбиения $f_1 \models \Phi_1$ существует разбиение $f_2 \models \Phi_2$ с тем же кодоменом такое, что $\mathbb{P}_1 \circ f_1^{-1} = \mathbb{P}_2 \circ f_2^{-1}$. Взаимно отобразимые друг на друга пространства корреляции называются изоморфными (далее $\Phi_1 \sim \Phi_2$).
\end{definition}

Символ $\circ$ используется здесь для обозначения композиции функций, т.е. $\mathbb{P}_1 \circ f_1^{-1} = \mathbb{P}_2 \circ f_2^{-1}$ означает, что у разбиений $f_1$ и $f_2$ совпадают не только множества исходов, но и вероятности реализации каждого отдельного исхода. 

\begin{definition}{1.2.3}
	Для игры $\Gamma = \langle A, S^a, u^a(s), a \in A \rangle$ множеством достижимых выплат по отклонениям группы игроков $A_*$ от профиля стратегий $s$ будем называть
	\begin{equation*}
		U_\Gamma^{A_*}(s) = \{\bar{u} \mid \exists s_* \in S = \times \prod_{a \in A} S^a : u(s_*) = \bar{u}, \forall a \in A \setminus A_*, s^a = s_*^a\}.
	\end{equation*}
\end{definition}

\begin{theorem}{1.2.1}[Об изоморфных пространствах]
	Пусть $\Phi_1 \sim \Phi_2$. Тогда для любой игры в нормальной форме $\Gamma$ с конечными множествами стратегий игроков её коррелированные расширения $\Gamma | \Phi_1$ и $\Gamma | \Phi_2$ обладают следующим свойством. Пусть $\mathbf{s}_1$ "--- некоторый профиль стратегий игры $\Gamma | \Phi_1$. Тогда существует $\mathbf{s}_2$ "--- профиль стратегий игры $\Gamma | \Phi_2$ такой, что $U_{\Gamma | \Phi_1}^{A_*}(\mathbf{s}_1) = U_{\Gamma | \Phi_2}^{A_*}(\mathbf{s}_2)$ для любой группы игроков $A_*$.
\end{theorem}

Назовём тайной группы игроков $A_* \subseteq A$ такую $\sigma$"~алгебру событий $\mathfrak{S}_\Phi^{A_*} \subseteq 2^\Omega$, что
\begin{equation*}
	\mathfrak{S}_\Phi^{A_*} = \{U \in \bigcap\limits_{a \in A_*} \mathfrak{I}^a \mid \mathbb{P}(U \cap V) = \mathbb{P}(U) \mathbb{P}(V), \forall V \in \sigma(\bigcup\limits_{a \in A \setminus A_*} \mathfrak{I}^a)\},
\end{equation*}
т.е. содержащую события, известные одновременно всем членам группы и никому кроме них (даже при объединении знаний всех аутсайдеров). Полными тайнами будем называть безатомические алгебры, а пустыми "--- вырожденные, содержащие атом вероятности $1$. Это позволяет дать для важных (в смысле дополнительной информационной асимметрии) классов способ конечного описания их структуры в виде семейства подмножеств игроков, способных на тайное согласование действий:
\begin{definition}{1.3.1}
	Пространство корреляции $\langle A, \Omega, \mathfrak{I}^a, \mathbb{P}, a \in A \rangle$ назовём пространством заговоров структуры $\mathfrak{A} = \{A_1, ..., A_n\} \subseteq 2^A$, когда
	\begin{itemize}
		\item $\bigcup\limits_{i=0}^n A_i = A$;
		\item $\forall A_* \in \mathfrak{A}$ тайна группы игроков $A_*$ полна;
		\item $\forall A_* \notin \mathfrak{A}$ тайна группы игроков $A_*$ пуста;
		\item $\mathfrak{I}^a = \sigma(\bigcup\limits_{a \in A_* \in \mathfrak{A}}\mathfrak{S}_\Phi^{A_*})$, т.е. $\mathfrak{I}^a$ "--- наименьшая $\sigma$"~алгебра, включающая все $\sigma$"~алгебры тайн групп, в которые входит игрок $a$.
	\end{itemize}
\end{definition}

\begin{theorem}{1.3.1}
	Все пространства заговоров одной структуры изоморфны.
\end{theorem}

\begin{definition}{1.3.2}
	Стандартным пространством структуры $\mathfrak{A} = \{A_1, A_2, ..., A_n\}$ называется пространство корреляции $\Phi_{\mathfrak{A}} = \langle A, \Omega, \mathfrak{I}^a, \mathbb{P}, a \in A \rangle$ со следующими параметрами:
	\begin{itemize}
		\item $A = \bigcup\limits_{i=1}^n A_i$,
		\item $\Omega = \left[0, 1\right)^n$,
		\item $\mathfrak{I}^a = \sigma(\{\prod\limits_{i=1}^n\left[ 0, p_i \right) \mid \operatorname{\mathbf{if}} \: a \in A_i \: \operatorname{\mathbf{then}} \: 0 < p_i \leq 1 \: \operatorname{\mathbf{else}} \: p_i = 1 \})$,
		\item $\mathbb{P}$ "--- мера Лебега.
	\end{itemize}
\end{definition}

Функционирование построенной модели иллюстрируется несложной, на первый взгляд, матричной игрой в <<трёхсторонний чёт"~нечет>>, получающей существенно новые точки коррелированного равновесия в невырожденных пространствах заговоров:
\begin{table} [htbp]
	\centering
	\captionsetup{labelformat=empty}
	\begin{threeparttable}
%		\caption{Трёхсторонний чёт"~нечет}
		\begin{tabular}{ |c|c|c|c|c| }
			\cline{1-2} \cline{4-5}
			\rule[-7pt]{0pt}{2em}$4, 4, 4$ &
			\rule[-7pt]{0pt}{2em}$6, 0, 6$ & \qquad\qquad\qquad &
			\rule[-7pt]{0pt}{2em}$6, 6, 0$ &
			\rule[-7pt]{0pt}{2em}$0, 6, 6$ \\
			\cline{1-2} \cline{4-5}
			\rule[-7pt]{0pt}{2em}$0, 6, 6$ &
			\rule[-7pt]{0pt}{2em}$6, 6, 0$ & \qquad\qquad\qquad &
			\rule[-7pt]{0pt}{2em}$6, 0, 6$ &
			\rule[-7pt]{0pt}{2em}$4, 4, 4$ \\
			\cline{1-2} \cline{4-5}
		\end{tabular}
	\end{threeparttable}
\end{table}

Отдельно освещается вопрос о необходимости использования при применении модели нетривиальных формализмов колмогоровской теории вероятностей, проистекающей из невозможности отождествить пространство стратегических профилей и множество исходов игры.

\underline{\textbf{Вторая глава}} посвящена исследованию распространения принципов коллективной рациональности на игры с заговорами. Она начинается с формулировки классической проблемы планирования заданий в терминах игры нормальной формы
\begin{equation*}
	\Gamma = \langle A, S^a, u^a(s), a \in A \rangle
\end{equation*}
с параметрами:
\begin{itemize}
	\item $A = \{ 1, \ldots, m \}$;
	\item $S^1 = \ldots = S^m = \{ 1, \ldots, n \}$;
	\item $u^a(s) = v^a(t_{s^a}(s))$, где $t_i(s) = \sum\limits_{a \in A, s^a = i} t_i^a$, а $v^a(t)$ "--- монотонно невозрастающая функция оплаты за срочность по заданию сотрудника $a$.
\end{itemize}

Здесь в вычислительном центре работают $m$ сотрудников, каждому из которых поручено произвести некое вычисление. В их распоряжении находятся $n$ компьютеров, на каждом из которых может быть запущена одна или несколько программ, производящих вычисления сотрудников. Машины отличаются архитектурными особенностями, что задаётся матрицей констант $t_i^a \ge 0$, обозначающих время выполнения программы сотрудника $a=\overline{1,\ldots,m}$ на компьютере $i=\overline{1,\ldots,n}$. Каждое вычисление может производиться только одним устройством. Несколько программ на одном компьютере выполняются последовательно, но результаты их работы выводятся одновременно после остановки последней из них. В зависимости от суммарной продолжительности вычислений $t$ на выбранной игроком $a$ машине, игрок получает оплату $v^a(t)$. В классической формулировке, при монотонно невозрастающих функциях оплаты, выгода каждого сотрудника заключается в выборе такого компьютера для своего вычисления, что для него оказывается минимальным суммарное время выполнения всех запущенных программ.

Для проявления искомого свойства чувствительности к дополнительной информационной асимметрии рассматривается обобщение проблемы, допускающее немонотонные $v^a(t)$. В качестве конкретного примера строится игра $\Gamma^3_n$, с $3$ однотипными заданиями и $n \ge 2$ одинаковыми компьютерами:
\begin{equation*}
	v^1(t) = v^2(t) = v^3(t) = \begin{cases}
		\begin{aligned}
			0, \qquad\qquad & \ t < 3 ;\\
			3, \qquad\ 3 \le & \ t < 5 ;\\
			2, \qquad\ 5 \le & \ t ;
		\end{aligned}
	\end{cases}
\end{equation*}
\begin{equation*}
	t_i^1 = t_i^2 = t_i^3 = 2, i = \overline{1,n}.
\end{equation*}

В этой ситуации когда все трое выбирают один компьютер на всех, выигрыш каждого игрока составляет $2$. При использовании машины вдвоём, игроки получают по $3$, а игрок, использующий компьютер в одиночку, остаётся с $0$. Для $\Gamma^3_n$ находятся все равновесия в смешанных стратегиях:
\begin{lemma}{2.3.1}
	Пусть $T \subseteq \{1, \ldots, n\}$ "--- произвольное непустое подмножество компьютеров. Тогда в игре $\Gamma^3_n$ набор одинаковых смешанных стратегий $s^1 = s^2 = s^3 = \left(\frac{[1 \in T]}{\left| T \right|}, \ldots, \frac{[n \in T]}{\left| T \right|}\right)$, где каждый игрок независимо и равновероятно выбирает одну из машин множества $T$, является равновесием по Нэшу.
\end{lemma}

\begin{lemma}{2.3.2}
	В игре $\Gamma^3_n$ набор смешанных стратегий $s = (s^1, s^2, s^3)$ может быть равновесием по Нэшу только в том случае, когда $s^1 = s^2 = s^3$.
\end{lemma}

\begin{lemma}{2.3.3}
	В игре $\Gamma^3_n$ набор одинаковых смешанных стратегий может быть равновесием по Нэшу только в том случае, когда все компьютеры, выбираемые с ненулевой вероятностью, выбираются с равными вероятностями.
\end{lemma}

Далее показывается, как переход от классического случая к розыгрышу в невырожденных пространствах заговоров существенно пополняет множество решений игры $\Gamma^3_n | \{\{1,2\}\}$ целыми наборами точек с выплатами, лежащими за пределами выпуклой оболочки смешанных равновесий. На этом примере и демонстрируется применение нового принципа коллективной рациональности, названного структурной согласованностью равновесий:
\begin{definition}{2.5.1}
	В игре с заговорами $\Gamma | \mathfrak{A}$ равновесие по Нэшу $\mathbf{s}$ называется структурно согласованным, если для всех заговоров $A_* \in \mathfrak{A}$ отсутствуют приемлемые отклонения от ситуации $\mathbf{s}$.
\end{definition}

\begin{definition}{2.5.2}
	В игре с заговорами $\Gamma | \mathfrak{A}$ ситуацию $\mathbf{s}_* \neq \mathbf{s}$ назовём отклонением от $\mathbf{s}$, приемлемым для заговора $A_* \in \mathfrak{A}$, если
	\begin{itemize}
		\item $\forall a \notin A_* \quad \mathbf{s}^a = \mathbf{s}_*^a$;
		\item $\forall a \in A_* \quad u^a(\mathbf{s}_*) \geq u^a(\mathbf{s})$;
		\item $\forall a : \mathbf{s}^a \neq \mathbf{s}_*^a \quad u^a(\mathbf{s}_*) > u^a(\mathbf{s})$.
	\end{itemize}
\end{definition}

Этот критерий позволяет фильтровать множества решений в играх с заговорами, когда антагонизм интересов не позволяет использовать привычные принципы Парето и Слейтера. Поднимается также вопрос о возможных способах примирения необходимости игрокам согласовывать коллективные отклонения для поддержания структурной согласованности равновесий с необходимостью сохранения структуры информированности игроков о значениях тайных сигналов. Завершается же глава построением примеров более сложных конфликтов планирования (политических и экономических) с немонотонной функцией отдачи, наглядно иллюстрирующих интуитивный принцип, стоящий за решениями игр в пространствах заговоров.

\underline{\textbf{Третья глава}} посвящена исследованию дополнительных решений повторяющихся игр, чувствительных к дополнительной информационной асимметрии, возникающих, когда выбор агентами стратегий на очередном шаге розыгрыша ограничен по вычислительной сложности. Первым делом в ней формулируется обобщение одной из классических версий <<народной>> теоремы на игры в пространствах заговоров:
\begin{theorem}{3.1.1}
	Пусть $\Gamma = \langle A, S^a, u^a(s), a \in A \rangle$ "--- игра в нормальной форме с конечным множеством исходов, $V$ "--- выпуклая оболочка множества платёжных векторов её матрицы, а $\mathfrak{A} \subseteq 2^A$ "--- произвольное пространство заговоров. Если вектор выплат $v \in V$ строго доминирует точку минимакса игры $\Gamma | \mathfrak{A}$, то найдётся такой коэффициент дисконтирования $0 < \delta < 1$, что в бесконечно повторяющейся игре $\Gamma | \mathfrak{A}$ будет существовать равновесие Нэша с выплатами, сходящимися к $v$. Если же вектор $v$ доминирует ещё и точку равновесного минимакса той же игры, то в бесконечно повторяющейся игре с достаточно большим коэффициентом дисконтирования будет существовать совершенное подыгровое равновесие с выплатами, сходящимися к $v$.
\end{theorem}

Далее на примере повторяющегося трёхстороннего чёт-нечета показывается, как её применение позволяет существенно расширить пространство решений в розыгрышах с невырожденным семейством заговоров. Модель повторяющейся игры с учётом стоимости вычислений формулируется в общем виде, для произвольных тьюринг-полных устройств с фиксированной ценой одного шага, работу которых игроки оплачивают из своего выигрыша с тем же темпом дисконтирования будущих раундов:
\begin{figure}[ht]
	\centerfloat{
		\ifdefmacro{\tikzsetnextfilename}{\tikzsetnextfilename{tikz_repeat_compiled}}{}
		\begin{tikzpicture}[scale=1.5]
			\node[circle,draw] (game1) at (1, 0) {$\Gamma$};
			\node[circle,draw] (game2) at (3, 0) {$\Gamma$};
			\node[circle,draw] (game3) at (5, 0) {$\Gamma$};
			\node[rectangle,draw] (calc11) at (0,  1) {$M^1$};
			\node[rectangle,draw] (calc12) at (2,  1) {$M^1$};
			\node[rectangle,draw] (calc13) at (4,  1) {$M^1$};
			\node[] (calc1i) at (6, 1) {$\cdots$};
			\node[rectangle,draw] (calc21) at (0, -1) {$M^2$};
			\node[rectangle,draw] (calc22) at (2, -1) {$M^2$};
			\node[rectangle,draw] (calc23) at (4, -1) {$M^2$};
			\node[] (calc2i) at (6, -1) {$\cdots$};
			\draw [->,thick,decorate,decoration={snake,post length=1mm}] (calc11) -- (game1) node[midway,sloped,below] {$s^1_1$};
			\draw [->,thick,decorate,decoration={snake,post length=1mm}] (calc21) -- (game1) node[midway,sloped,above] {$s^2_1$};
			\draw [->,thick,decorate,decoration={snake,post length=1mm}] (calc12) -- (game2) node[midway,sloped,below] {$s^1_2$};
			\draw [->,thick,decorate,decoration={snake,post length=1mm}] (calc22) -- (game2) node[midway,sloped,above] {$s^2_2$};
			\draw [->,thick,decorate,decoration={snake,post length=1mm}] (calc13) -- (game3) node[midway,sloped,below] {$s^1_3$};
			\draw [->,thick,decorate,decoration={snake,post length=1mm}] (calc23) -- (game3) node[midway,sloped,above] {$s^2_3$};
			\draw [->,thick,decorate,decoration={snake,post length=1mm}] (game1) -- (calc12) node[midway,sloped,below] {$s_1$};
			\draw [->,thick,decorate,decoration={snake,post length=1mm}] (game1) -- (calc22) node[midway,sloped,above] {$s_1$};
			\draw [->,thick,decorate,decoration={snake,post length=1mm}] (game2) -- (calc13) node[midway,sloped,below] {$s_2$};
			\draw [->,thick,decorate,decoration={snake,post length=1mm}] (game2) -- (calc23) node[midway,sloped,above] {$s_2$};
			\draw [->,thick,decorate,decoration={snake,post length=1mm}] (game3) -- (calc1i) node[midway,sloped,below] {$s_3$};
			\draw [->,thick,decorate,decoration={snake,post length=1mm}] (game3) -- (calc2i) node[midway,sloped,above] {$s_3$};
			\node[diamond,draw] (sum1) at (-1, 2) {$\Sigma^1$};
			\node[] (sum10) at (0,  2) [label=$-\delta \mathfrak{w}^1_1$] {};
			\node[] (sum11) at (1,  2) [label=$+\delta u^1(s_1)$] {};
			\node[] (sum12) at (2,  2) [label=$-\delta^2 \mathfrak{w}^1_2$] {};
			\node[] (sum13) at (3,  2) [label=$+\delta^2 u^1(s_2)$] {};
			\node[] (sum14) at (4,  2) [label=$-\delta^3 \mathfrak{w}^1_3$] {};
			\node[] (sum15) at (5,  2) [label=$+\delta^3 u^1(s_3)$] {};
			\node[] (sum1i) at (6,  2) {$\cdots$};
			\node[diamond,draw] (sum2) at (-1, -2) {$\Sigma^2$};
			\node[] (sum20) at (0, -2) [label=below:$-\delta \mathfrak{w}^2_1$] {};
			\node[] (sum21) at (1, -2) [label=below:$+\delta u^2(s_1)$] {};
			\node[] (sum22) at (2, -2) [label=below:$-\delta^2 \mathfrak{w}^2_2$] {};
			\node[] (sum23) at (3, -2) [label=below:$+\delta^2 u^2(s_2)$] {};
			\node[] (sum24) at (4, -2) [label=below:$-\delta^3 \mathfrak{w}^2_3$] {};
			\node[] (sum25) at (5, -2) [label=below:$+\delta^3 u^2(s_3)$] {};
			\node[] (sum2i) at (6, -2) {$\cdots$};
			\draw [->,thick] (sum1i) -- (sum1);
			\draw [->,thick] (calc11) -- (sum10);
			\draw [->,thick] (game1) -- (sum11);
			\draw [->,thick] (calc12) -- (sum12);
			\draw [->,thick] (game2) -- (sum13);
			\draw [->,thick] (calc13) -- (sum14);
			\draw [->,thick] (game3) -- (sum15);
			\draw [->,thick] (sum2i) -- (sum2);
			\draw [->,thick] (calc21) -- (sum20);
			\draw [->,thick] (game1) -- (sum21);
			\draw [->,thick] (calc22) -- (sum22);
			\draw [->,thick] (game2) -- (sum23);
			\draw [->,thick] (calc23) -- (sum24);
			\draw [->,thick] (game3) -- (sum25);
			\fill [white] (1,  1) circle (2pt);
			\fill [white] (3,  1) circle (2pt);
			\fill [white] (5,  1) circle (2pt);
			\fill [white] (1, -1) circle (2pt);
			\fill [white] (3, -1) circle (2pt);
			\fill [white] (5, -1) circle (2pt);
			\draw [->,double,thick] (-1,  1) -- (calc11) node[midway,above] {$\psi^1_0$};
			\draw [->,double,thick] (calc11) -- (calc12) node[near end,above] {$\psi^1_1$};
			\draw [->,double,thick] (calc12) -- (calc13) node[near end,above] {$\psi^1_2$};
			\draw [->,double,thick] (calc13) -- (calc1i) node[near end,above] {$\psi^1_3$};
			\draw [->,double,thick] (-1, -1) -- (calc21) node[midway,below] {$\psi^2_0$};
			\draw [->,double,thick] (calc21) -- (calc22) node[near end,below] {$\psi^2_1$};
			\draw [->,double,thick] (calc22) -- (calc23) node[near end,below] {$\psi^2_2$};
			\draw [->,double,thick] (calc23) -- (calc2i) node[near end,below] {$\psi^2_3$};
		\end{tikzpicture}
	}
	\legend{}
%	\caption[Повторяющаяся игра с учётом стоимости вычислений]{Повторяющаяся игра с учётом стоимости вычислений}
\end{figure}

Диаграмма на рисунке схематично изображает моделируемый процесс для двух игроков (естественным образом обобщающийся на любое конечное их число). В узлах, помеченных буквой $\Gamma$, происходят последовательные розыгрыши произвольной игры $\Gamma = \langle A, S^a, u^a(s), a \in A \rangle$. В $i$-м розыгрыше игрок $a$ выбирает свою стратегию $s^a_i$, применяя вероятностный алгоритм $M^a$ к результату предыдущей итерации $(\psi^a_{i-1}, s_{i-1})$, включающему сохранённое состояние памяти самого алгоритма и набор сыгранных на итерации $i-1$ стратегий. Узлы $\Sigma^a$ изображают последовательное суммирование разностей выигрыша $u^a(s_i)$ и затрат на произведённое вычисление $\mathfrak{w}^a_i$, с учётом экспоненциально уменьшающегося коэффициента дисконтирования $\delta^i$. При такой схеме взаимодействий имеет смысл рассматривать только универсальные алгоритмы $M^a$, что позволяет закодировать любой набор вычислимых стратегий повторяющейся игры в начальных состояниях памяти $\psi^a_0$.

В рамках модели показывается, что даже в отсутствие априорной информационной асимметрии (т.е. без заговоров), пары игроков могут использовать для построения стратегий наказания примитивы современной криптографии, достигая такого согласования действий, что третий не может к ним присоединиться без выполнения неприемлемо дорогого вычисления. Для рассматриваемой игры показывается, как таким образом существенно пополняется множество совершенных подыгровых равновесий:
\begin{theorem}{3.4.1}
	В повторяющемся трёхстороннем чёт-нечете с дисконтированием и оплатой каждым игроком вычислительного ресурса, использованного им для выбора очередного шага стратегии, множество векторов выплат, достижимых в совершенных подыгровых равновесиях, представляет собой участок поверхности конечной положительной площади.
\end{theorem}

Завершается глава рассуждениями о значении описанного феномена, обобщая который можно сделать пару смелых предположений как о его родстве с наблюдаемыми, но до сих пор толком не описанными особенностями поведения мастеров некоторых карточных игр, так и о его возможной роли в моделях популяционных конфликтов:
\begin{conjecture}{3.5.1}
	Если в популяции периодически (много раз на протяжении жизни одной особи) возникают конфликтные ситуации, модель которых чувствительна к дополнительной информационной асимметрии, и вероятность продолжения рода отдельными особями существенно зависит от их успеха в этих конфликтах, то давление отбора закрепляет в популяции признаки, способствующие увеличению когнитивного потенциала следующих поколений (понимаемого в общем смысле, как способность производить Тьюринг-полные вычисления над произвольными данными).
\end{conjecture}

%Можно сослаться на свои работы в автореферате. Для этого в файле
%\verb!Synopsis/setup.tex! необходимо присвоить положительное значение
%счётчику \verb!\setcounter{usefootcite}{1}!. В таком случае ссылки на
%работы других авторов будут подстрочными.
%Изложенные в третьей главе результаты опубликованы в~\cite{vakbib1, vakbib2}.
%Использование подстрочных ссылок внутри таблиц может вызывать проблемы.

%В \underline{\textbf{четвертой главе}} приведено описание

%\FloatBarrier
%\pdfbookmark{Заключение}{conclusion}                                  % Закладка pdf
%В \underline{\textbf{заключении}} приведены основные результаты работы, которые заключаются в следующем:
%%% Согласно ГОСТ Р 7.0.11-2011:
%% 5.3.3 В заключении диссертации излагают итоги выполненного исследования, рекомендации, перспективы дальнейшей разработки темы.
%% 9.2.3 В заключении автореферата диссертации излагают итоги данного исследования, рекомендации и перспективы дальнейшей разработки темы.
\begin{enumerate}
  \item На основе анализа понятия коррелированного равновесия в контексте многосторонних конфликтов было сформулировано свойство чувствительности игр к дополнительной информационной асимметрии.
  \item Исследование изоморфизма пространств корреляции позволило ввести сужающий их формализм пространства заговоров, с применением которого удобно рассуждать о влиянии дополнительной информационной асимметрии на решения игр.
  \item Моделирование проблемы планирования заданий при допущении немонотонности функций отдачи показало, что в пространствах заговоров концепция структурной согласованности равновесий может использоваться как функциональный аналог классических критериев коллективной рациональности по отношению к равновесиям Нэша в смешанных стратегиях. 
  \item Для демонстрации значимости феномена чувствительности игр к дополнительной информационной асимметрии была построена модель повторяющихся конфликтов с учётом стоимости вычисления очередного шага стратегии.
  \item В рамках построенной модели было показано, как в повторяющихся играх можно даже без дополнительной информационной асимметрии как таковой использовать современные криптографические примитивы для конструирования эффективных стратегий наказания, использующих чувствительность к ней.
\end{enumerate}


После подводящего итоги диссертации \underline{\textbf{заключения}} и технических разделов (\underline{\textbf{словаря терминов}}, \underline{\textbf{списка литературы}} и \underline{\textbf{списка рисунков}}) следуют 4 приложения с материалом, дополняющим основные идеи работы, но необязательным для их понимания. \underline{\textbf{Приложение А}} содержит краткий обзор литературы, посвящённой коррелированному расширению игр в нормальной форме, по которому можно проследить развитие мысли других исследователей, развивавших ауманновский формализм. В \underline{\textbf{приложения Б}} и \underline{\textbf{В}} вынесены доказательства теорем из первой главы, представляющие собой упражнения в топологии без тесной связи с теорией игр. \underline{\textbf{Приложение Г}} посвящено новой карточной игре <<Тессеракт>> с дизайном, целиком подчинённым задаче, поставленной при обобщении итогов третьей главы.
%, т.е. с одной стороны достаточно интересной, чтобы выступать в качестве салонного развлечения, а с другой "--- не имеющей сложных тактических элементов помимо опоры на предполагаемый феномен <<сыгранности>> участников, что и должно в перспективе способствовать его изучению.

\pdfbookmark{Литература}{bibliography}                                % Закладка pdf
\ifdefmacro{\microtypesetup}{\microtypesetup{protrusion=false}}{} % не рекомендуется применять пакет микротипографики к автоматически генерируемому списку литературы
\urlstyle{rm}                               % ссылки URL обычным шрифтом
\ifnumequal{\value{bibliosel}}{0}{% Встроенная реализация с загрузкой файла через движок bibtex8
    \renewcommand{\bibname}{\large \bibtitleauthor}
    \nocite{*}
    \insertbiblioauthor           % Подключаем Bib-базы
    %\insertbiblioexternal   % !!! bibtex не умеет работать с несколькими библиографиями !!!
}{% Реализация пакетом biblatex через движок biber
    % Цитирования.
    %  * Порядок перечисления определяет порядок в библиографии (только внутри подраздела, если `\insertbiblioauthorgrouped`).
    %  * Если не соблюдать порядок "как для \printbibliography", нумерация в `\insertbiblioauthor` будет кривой.
    %  * Если цитировать каждый источник отдельной командой --- найти некоторые ошибки будет проще.
    %
    %% authorvak
    \nocite{vakbib1}%
    \nocite{vakbib2}%
    %
    %% authorwos
    \nocite{wosbib1}%
    %
    %% authorscopus
    \nocite{scbib1}%
    %
    %% authorpathent
    \nocite{patbib1}%
    %
    %% authorprogram
    \nocite{progbib1}%
    %
    %% authorconf
    \nocite{confbib1}%
    \nocite{confbib2}%
    %
    %% authorother
    \nocite{bib1}%
    \nocite{bib2}%

    \ifnumgreater{\value{usefootcite}}{0}{
        \begin{refcontext}[labelprefix={}]
            \ifnum \value{bibgrouped}>0
                \insertbiblioauthorgrouped    % Вывод всех работ автора, сгруппированных по источникам
            \else
                \insertbiblioauthor      % Вывод всех работ автора
            \fi
        \end{refcontext}
    }{
        \ifnum \totvalue{citeexternal}>0
            \begin{refcontext}[labelprefix=A]
                \ifnum \value{bibgrouped}>0
                    \insertbiblioauthorgrouped    % Вывод всех работ автора, сгруппированных по источникам
                \else
                    \insertbiblioauthor      % Вывод всех работ автора
                \fi
            \end{refcontext}
        \else
            \ifnum \value{bibgrouped}>0
                \insertbiblioauthorgrouped    % Вывод всех работ автора, сгруппированных по источникам
            \else
                \insertbiblioauthor      % Вывод всех работ автора
            \fi
        \fi
        %  \insertbiblioauthorimportant  % Вывод наиболее значимых работ автора (определяется в файле characteristic во второй section)
        \begin{refcontext}[labelprefix={}]
            \insertbiblioexternal            % Вывод списка литературы, на которую ссылались в тексте автореферата
        \end{refcontext}
        % Невидимый библиографический список для подсчёта количества внешних публикаций
        % Используется, чтобы убрать приставку "А" у работ автора, если в автореферате нет
        % цитирований внешних источников.
        \printbibliography[heading=nobibheading, section=0, env=countexternal, keyword=biblioexternal, resetnumbers=true]%
    }
}
\ifdefmacro{\microtypesetup}{\microtypesetup{protrusion=true}}{}
\urlstyle{tt}                               % возвращаем установки шрифта ссылок URL
