\begin{frame}[noframenumbering,plain]
    \setcounter{framenumber}{1}
    \maketitle
\end{frame}

\begin{frame}
    \frametitle{Положения, выносимые на защиту}
    \begin{itemize}
		\item Доказательство теоремы об изоморфизме пространств корреляции, определяющей для них классы эквивалентности, в которые входят только неразличимые с теоретико-игровой точки зрения пространства.
		\item Доказательство теоремы о пространствах заговоров одной структуры, благодаря которой структуру пространства можно считать его исчерпывающим описанием.
		\item Доказательство чувствительности к дополнительной информационной асимметрии симметричной проблемы планирования заданий с немонотонной отдачей.
		\item Решение трёхсторонней симметричной проблемы планирования с немонотонной функцией оплаты за срочность в ассиметричном пространстве заговоров, удовлетворяющее критерию структурной согласованности.
		\item Модель повторяющихся игр с учётом стоимости вычислений, необходимых для выбора хода на очередной итерации.
		\item Механизм для криптографической корреляции стратегий в повторяющихся играх с учётом стоимости вычислений.
%		\item Криптографические стратегии наказания для повторяющегося трёхстороннего чёт-нечета, пополняющие множество совершенных подыгровых равновесий точками, не достижимыми без принятия во внимание сложности алгоритмов.
    \end{itemize}
\end{frame}
\note{
    Проговариваются вслух положения, выносимые на защиту
}

\begin{frame}
    \frametitle{Содержание}
    \tableofcontents
\end{frame}
\note{
    Работа состоит из трёх глав.

    \medskip
    В первой главе \dots

    Во второй главе \dots

    Третья глава посвящена \dots
}
