\chapter{Вычислительная сложность стратегий в повторяющихся играх с дисконтированием}\label{ch:ch3}

%\section{Повторяющийся трёхсторонний чёт-нечет}\label{sec:ch3/sect1}

%Чувствительность к дополнительной информационной асимметрии интересна не только тем, как на исходы проявляющих её игр влияет собственно информационная асимметрия. Далее будет показано, что в тех случаях, когда этим свойством обладают отдельные розыгрыши повторяющихся игр, необычные эффекты могут возникать даже при отсутствии приватных механизмов корреляции, в рамках классического смешанного расширения и совершенных подыгровых равновесий. Разберём этот феномен на примере игры в <<трёхсторонний чёт-нечет>>, описанной в разделе \ref{sec:ch1/sec4}. Сперва проанализируем повторяющуюся версию данного конфликта с классической точки зрения, применяя <<народную>> теорему.
%
%\begin{table} [htbp]
%	\centering
%	\begin{threeparttable}
%		\caption{Трёхсторонний чёт"~нечет}
%		\label{tab:coin3b}
%		\begin{tabular}{ |c|c|c|c|c| }
%			\cline{1-2} \cline{4-5}
%			\rule[-7pt]{0pt}{2em}$4, 4, 4$ &
%			\rule[-7pt]{0pt}{2em}$6, 0, 6$ & \qquad\qquad\qquad &
%			\rule[-7pt]{0pt}{2em}$6, 6, 0$ &
%			\rule[-7pt]{0pt}{2em}$0, 6, 6$ \\
%			\cline{1-2} \cline{4-5}
%			\rule[-7pt]{0pt}{2em}$0, 6, 6$ &
%			\rule[-7pt]{0pt}{2em}$6, 6, 0$ & \qquad\qquad\qquad &
%			\rule[-7pt]{0pt}{2em}$6, 0, 6$ &
%			\rule[-7pt]{0pt}{2em}$4, 4, 4$ \\
%			\cline{1-2} \cline{4-5}
%		\end{tabular}
%	\end{threeparttable}
%\end{table}
%
%В соответствии с формулировкой теоремы для получения множества равновесий по Нэшу в повторяющейся игре необходимо построить минимаксные стратегии наказания в отдельной её итерации. Однако трёхсторонний чёт-нечет любопытен тем, что в смешанных стратегиях никакая попытка двух игроков минимизировать выигрыш третьего не может опустить его результат ниже того, что он получает в точках равновесия по Нэшу (т.е. менее $4$). Действительно, пусть первый игрок избрал стратегию $(p, 1 - p)$, а второй "--- $(q, 1 - q)$. Тогда выигрыши третьего при выборе чистых стратегий составляют:
%\begin{align*}
%	u^3(p, q, 1) &= 6 (p + q) - 8 p q; \\
%	u^3(p, q, 0) &= 2 (p + q) - 8 p q + 4.
%\end{align*}
%
%Несложно заметить, что
%\begin{align*}
%	p \ge &q \ge \frac{1}{2} \Rightarrow u^3(p, q, 1) \ge 4; \\
%	p \ge 1 - &q \ge \frac{1}{2} \Rightarrow u^3(p, q, 1) \ge 4; \\
%	q \ge &p \ge \frac{1}{2} \Rightarrow u^3(p, q, 1) \ge 4; \\
%	q \ge 1 - &p \ge \frac{1}{2} \Rightarrow u^3(p, q, 1) \ge 4; \\
%	p \le &q \le \frac{1}{2} \Rightarrow u^3(p, q, 0) \ge 4; \\
%	p \le 1 - &q \le \frac{1}{2} \Rightarrow u^3(p, q, 0) \ge 4; \\
%	q \le &p \le \frac{1}{2} \Rightarrow u^3(p, q, 0) \ge 4; \\
%	q \le 1 - &p \le \frac{1}{2} \Rightarrow u^3(p, q, 0) \ge 4. \\
%\end{align*}
%
%Эти варианты исчерпывают всё пространство возможных ситуаций, так что никакое сочетание смешанных стратегий двух игроков не может быть эффективным наказанием для третьего. В классическом случае это означало бы, что повторяющаяся игра имеет в качестве решения то же самое множество равновесий по Нэшу, что и отдельный розыгрыш. Однако, вывод этот можно интересным образом уточнить, перейдя к модели, учитывающей стоимость вычислений, необходимых для выбора стратегии в очередной итерации.

\section{<<Народная>> теорема в пространствах заговоров}\label{sec:ch3/sect1}

Для того, чтобы понять, как многократное повторение влияет на конфликты, чувствительные к дополнительной информационной асимметрии, необходимо сперва проанализировать то, как <<народная>> теорема может быть обобщена на игры в пространствах заговоров. Ранее эту теорему уже доказывали в различных формах для коррелированного расширения игр \cite{Fudenberg}, сознательно ограничиваясь случаем публичных механизмов корреляции. Нам же необходимо сделать ещё один шаг, отказавшись от этого ограничения. По сути, главным изменением оказывается то, что использование приватных механизмов корреляции зачастую позволяет дополнительно уменьшить резервные выигрыши игроков, не имеющих к ним доступа. Напомним, что резервным выигрышем игрока $a$ называется
\begin{equation*}
	u^a_* = \min_{s \in \overline{S}_a} u^a(s), \overline{S}_a = \{\overline{s} \in S \mid \overline{s}^a \in \arg\max_{s^a \in S^a} u^a(\overline{s} | s^a)\},
\end{equation*}
то есть наименьший его выигрыш среди всевозможных исходов, в которых он использует стратегию наилучшего ответа. По сути, резервный выигрыш обозначает границу полезности, ниже которой для соответствующего игрока невозможно опустить ожидаемый выигрыш, даже если все остальные игроки объединятся для достижения этой цели, невзирая на ущерб себе. Набор стратегий, приводящий к такому исходу, называется минимаксным для игрока $a$. Вектор $u_* = (u^1_*, \ldots, u^m_*)$, составленный из резервных выигрышей каждого игрока, называется точкой минимакса игры. Аналогичным образом можно определить резервный равновесный выигрыш игрока $a$:
\begin{equation*}
	\tilde{u}^a_* = \min_{s \in \tilde{S}} u^a(s), \text{где $\tilde{S}$ "--- множество равновесных по Нэшу наборов},
\end{equation*}
из которых составляется вектор $\tilde{u}_* = (\tilde{u}^1_*, \ldots, \tilde{u}^m_*)$, называемый точкой равновесного минимакса игры. Очевидно при этом, что $\tilde{u}_* \succeq u_*$.

Центральной идеей, стоящей за доказательством большинства <<народных>> теорем, является использование минимаксных наборов стратегий для наказания игроков, отклоняющихся от запланированной цепочки действий, ожидаемых от них соперниками. Любая последовательность розыгрышей $(s_i)$ с выплатами, сходящимися (в смысле среднего или с дисконтированием, в зависимости от версии теоремы) к вектору $u_0 = (u^1_0, \ldots, u^m_0)$, строго доминирующему точку минимакса игры (т.е. $u^a_0 > u^a_*, a = \overline{1,m}$), может быть равновесным исходом (в смысле обычного Нэша или совершенного по подыграм, в зависимости от версии теоремы) повторяющейся игры. Если на итерации $i$ игрок $a$ отклоняется от ожидаемой стратегии $s^a_i$, то начиная со следующей итерации остальные игроки переходят к применению стратегии из минимаксного для этого игрока набора, продолжая делать это достаточно долго (в некоторых версиях доказательств бесконечно) для того, чтобы нанесённый ими игроку $a$ ущерб превзошёл его прибыль от отклонения.

Коррелированные стратегии в контексте повторяющихся игр понимаются обычно в ограниченном смысле "--- рассматриваются только публичные сигналы, что с точки зрения нашей модели соответствует пространствам заговоров структуры $\{A\}$, то есть состоящим из одного заговора, охватывающего всех игроков. Множество платёжных векторов, достижимых в игре $\Gamma | \{A\}$, представляет собой выпуклую оболочку множества векторов, входящих в матрицу выплат игры $\Gamma$. Множество коррелированных равновесий по Нэшу игры $\Gamma | \{A\}$ представляет собой выпуклую оболочку множества смешанных равновесий игры $\Gamma$. При этом, как было показано в предыдущих главах, добавление к пространству заговоров групп, в которые входят не все игроки, может пополнять множество коррелированных равновесий по Нэшу точками, лежащими за пределами выпуклой оболочки множества смешанных равновесий. Кроме того, благодаря использованию тайных сигналов могут быть снижены резервные выигрыши игроков, не имеющих возможности их наблюдать.

Для доказательств большинства версий <<народной>> теоремы переход к более сложным пространствам заговоров, вероятно, не представляет большой проблемы "--- логика рассуждений остаётся прежней, достаточно при необходимости учитывать новые значения минимаксов и коррелированные равновесия за пределами выпуклой оболочки множества смешанных. К счастью, в контексте данной работы у нас даже нет необходимости в современных её формулировках "--- вполне достаточно классического, относительно слабого утверждения:
\begin{theorem}\label{the:folk}
	Пусть $\Gamma = \langle A, S^a, u^a(s), a \in A \rangle$ "--- игра в нормальной форме с конечным множеством исходов, $V$ "--- выпуклая оболочка множества платёжных векторов её матрицы, а $\mathfrak{A} \subseteq 2^A$ "--- произвольное пространство заговоров. Если вектор выплат $v \in V$ строго доминирует точку минимакса игры $\Gamma | \mathfrak{A}$, то найдётся такой коэффициент дисконтирования $0 < \delta < 1$, что в бесконечно повторяющейся игре $\Gamma | \mathfrak{A}$ будет существовать равновесие Нэша с выплатами, сходящимися к $v$. Если же вектор $v$ доминирует ещё и точку равновесного минимакса той же игры, то в бесконечно повторяющейся игре с достаточно большим коэффициентом дисконтирования будет существовать совершенное подыгровое равновесие с выплатами, сходящимися к $v$.
\end{theorem}
\begin{proof}
	Если $A \in \mathfrak{A}$, то последовательность наборов с выплатами, сходящимися к $v$, строится из коррелированных стратегий, опирающихся на публичный сигнал и напрямую смешивающих наборы чистых стратегий в пропорциях, обеспечивающих необходимый вектор платежей. Если же $A \notin \mathfrak{A}$, то можно использовать последовательность наборов чистых стратегий, сходящуюся в пределе к той же точке. При отклонении любого игрока от предписанной стратегий остальные переключаются на его наказание соответствующими минимаксными наборами стратегий. Прибыль игрока $a$, отклоняющегося на $i$-й итерации в пользу стратегии $\hat{s}^a_i$, конечна и составляет $u^a(s_i | \hat{s}^a_i) - u^a(s_i)$, так что при достаточно большом $\delta$ ущерб от вечной кары минимаксом на все последующие итерации будет очевидно больше. Наказание, использующее обычный минимакс, может включать стратегии неоптимальные с точки зрения карающих игроков, так что получающиеся при этом точки равновесия в общем случае не являются совершенными по подыграм. Наказание с использованием равновесного минимакса лишено этого недостатка, а значит равновесия на его основе уже будут совершенными по подыграм.	
\end{proof}

Для наших целей этого хватает, поскольку рассматривать мы будем повторение игры, у которой точки обычного и равновесного минимакса совпадают во всех пространствах заговоров. Тем не менее, если возникнет такая необходимость, ничего не мешает подобному обобщению и более современных, усиленных формулировок \cite{Vasin}. Чтобы не перегружать работу более громоздкими рассуждениями, которые по факту были бы почти дословным цитированием доказательств за авторством Васина А.А., ограничимся кратким пересказом их центральной идеи. Сделать участие в наказании обычным минимаксом первого отклонившегося от предписанной стратегии игрока оптимальной стратегией для наказывающих можно при помощи чуть более сложного формата угрозы. Игрокам достаточно договориться о том, что наказание первого отклонившегося будет не вечным, а прерывающимся в тот момент, когда кто-либо отказывается принимать в нём участие. Как только один из наказывающих отклоняется от предписываемой минимаксным набором стратегии, предыдущий отклонившийся прощается и все переходят к такому же условно вечному наказанию последнего <<уклониста>>, в котором участвуют все, включая только что прощённого игрока. Таким образом процесс превращается в что-то наподобие <<салочек>>, создавая эффективную угрозу оказаться последним наказанным, что расширяет множество совершенных по подыграм равновесий до всех точек, доминирующих обычный минимакс. Добавление пространства заговоров здесь, опять же, не создаёт особых проблем "--- рассуждение продолжает работать даже при использовании в наказаниях синхронизации тайными сигналами.

\section{Повторяющийся трёхсторонний чёт-нечет}\label{sec:ch3/sect2}

%\begin{table} [htbp]
%	\centering
%	\begin{threeparttable}
%		\caption{Трёхсторонний чёт"~нечет}
%		\label{tab:coin3b}
%		\begin{tabular}{ |c|c|c|c|c| }
%			\cline{1-2} \cline{4-5}
%			\rule[-7pt]{0pt}{2em}$4, 4, 4$ &
%			\rule[-7pt]{0pt}{2em}$6, 0, 6$ & \qquad\qquad\qquad &
%			\rule[-7pt]{0pt}{2em}$6, 6, 0$ &
%			\rule[-7pt]{0pt}{2em}$0, 6, 6$ \\
%			\cline{1-2} \cline{4-5}
%			\rule[-7pt]{0pt}{2em}$0, 6, 6$ &
%			\rule[-7pt]{0pt}{2em}$6, 6, 0$ & \qquad\qquad\qquad &
%			\rule[-7pt]{0pt}{2em}$6, 0, 6$ &
%			\rule[-7pt]{0pt}{2em}$4, 4, 4$ \\
%			\cline{1-2} \cline{4-5}
%		\end{tabular}
%	\end{threeparttable}
%\end{table}

Продемонстрировать применение обобщения <<народной>> теоремы на игры с заговорами можно на примере всё того же трёхстороннего чёт-нечета (см. таблицу \ref{tab:coin3}). Здесь $V$ представляет собой треугольник с вершинами $(6, 6, 0)$, $(6, 0, 6)$ и $(0, 6, 6)$. У игры две точки равновесия по Нэшу в чистых стратегиях (синхронный выбор орла или решки всеми игроками) и одна дополнительная в смешанных (равновероятный независимый выбор между орлом и решкой всеми игроками), влекущие один и тот же вектор платежей $(4, 4, 4)$, являющийся, к тому же, ещё и точкой минимакса игры. Действительно, пусть первый игрок избрал стратегию $(p, 1 - p)$, а второй "--- $(q, 1 - q)$. Тогда выигрыши третьего при выборе чистых стратегий составляют:
\begin{align*}
	u^3(p, q, 1) &= 6 (p + q) - 8 p q; \\
	u^3(p, q, 0) &= 2 (p + q) - 8 p q + 4.
\end{align*}

Несложно заметить, что
\begin{align*}
	p \ge &q \ge \frac{1}{2} \Rightarrow u^3(p, q, 1) \ge 4; \\
	p \ge 1 - &q \ge \frac{1}{2} \Rightarrow u^3(p, q, 1) \ge 4; \\
	q \ge &p \ge \frac{1}{2} \Rightarrow u^3(p, q, 1) \ge 4; \\
	q \ge 1 - &p \ge \frac{1}{2} \Rightarrow u^3(p, q, 1) \ge 4; \\
	p \le &q \le \frac{1}{2} \Rightarrow u^3(p, q, 0) \ge 4; \\
	p \le 1 - &q \le \frac{1}{2} \Rightarrow u^3(p, q, 0) \ge 4; \\
	q \le &p \le \frac{1}{2} \Rightarrow u^3(p, q, 0) \ge 4; \\
	q \le 1 - &p \le \frac{1}{2} \Rightarrow u^3(p, q, 0) \ge 4. \\
\end{align*}

Эти варианты исчерпывают всё пространство возможных ситуаций, так что никакое сочетание смешанных стратегий двух игроков не может быть эффективным наказанием для третьего. Таким образом в отсутствие приватных механизмов корреляции народная теорема никак не расширяет множество решений повторяющегося чёт-нечета. Однако, добавление в пространство заговоров, например, пары $\{1, 2\}$ меняет картину "--- в игре появляются коррелированные равновесия с выплатами $(5, 5, 2)$ в ситуации, когда все игроки снова делают равновероятный выбор между орлом и решкой, но при этом благодаря секретному механизму корреляции выбор игроков 1 и 2 всегда синхронен. Это уменьшает резервные (в обычном и равновесном смыслах) выигрыши игрока 3, давая нам новую точку минимакса "--- $(4, 4, 2)$. В соответствии с сформулированной выше версией <<народной>> теоремы, в пространстве заговоров $\{\{1, 2\}\}$ у повторяющегося трёхстороннего чёт-нечета появляются новые совершенные по подыграм равновесия в треугольнике с вершинами $(6, 4, 2)$, $(4, 6, 2)$ и $(4, 4, 4)$.

Аналогичным образом, в пространстве заговоров $\{\{1, 2\}, \{2, 3\}\}$ точка минимакса игры перемещается в $(2, 4, 2)$, что расширяет множество решений до плоской трапеции с вершинами $(4, 6, 2)$, $(2, 6, 4)$, $(2, 4, 6)$ и $(6, 4, 2)$. Наконец, пространство $\{\{1, 2\}, \{2, 3\}, \{1, 3\}\}$, включающее все парные заговоры, даёт точку минимакса $(2, 2, 2)$, превращая множество выплат, достижимых в совершенных подыгровых равновесиях, в плоский шестиугольник с вершинами $(6, 4, 2)$, $(6, 2, 4)$, $(4, 2, 6)$, $(2, 4, 6)$, $(2, 6, 4)$ и $(4, 6, 2)$. Эта иллюстрация неплохо согласуется с интуитивным представлением о том, что группы агентов, имеющие возможность координировать свои действия втайне от остальных, могут использовать это как угрозу в адрес аутсайдеров, принуждая тех соглашаться на исходы конфликтов, которые в отсутствие тайных сговоров были бы отвергнуты как невыгодные. Однако, этим влияние чувствительности к дополнительной информационной асимметрии на повторяющиеся игры не исчерпывается. Далее будет показано, что даже при отсутствии априорной информационной асимметрии (т.е. внешних по отношению к конфликту событий, о которых его участники осведомлены по-разному) игроки, ограниченные в сложности производимых ими для выбора стратегий вычислений, могут использовать угрозу искусственного создания информационной асимметрии при помощи специальным образом организованных совместных публичных действий.

%Расширение отдельных итераций повторяющейся игры пространством заговоров влияет на формулировки и доказательства утверждений следующим образом:
%\begin{itemize}
%	\item \emph{Теоремы A и B}: формулировки остаются прежними, под точкой минимакса понимается вектор резервных выигрышей, достижимых при использовании против соответствующего игрока частично коррелированных стратегий, опирающихся на сокрытые от него в рамках структуры заговоров события;
%	\item \emph{Теорема C}: то же самое, что для предыдущих теорем, плюс в качестве равновесия по Нэшу отдельного розыгрыша может пониматься коррелированное равновесие с использованием событий из структуры заговоров;
%	\item \emph{Теорема D}: полунепрерывность сверху рассматриваемого отображения сохраняется и для точки минимакса в частично коррелированных стратегиях;
%	\item \emph{Теорема 1}: сохраняется неизменной, поскольку сформулирована для случая двух игроков, не имеющего дополнительных решений в пространствах заговоров;
%	\item \emph{Теорема 2}: 
%\end{itemize}

\section{Модель повторяющихся игр с учётом стоимости вычислений}\label{sec:ch3/sect3}

Построение искомой модели подразумевает конкретизацию способа, при помощи которого рациональные агенты вычисляют стратегию поведения. Для этой цели подойдёт любой формализм, позволяющий алгоритмически полные вычисления с вероятностным ветвлением. Кроме того, необходимы возможность сохранения произвольного внутреннего состояния в памяти для использования на последующих итерациях и, очевидно, метод численной оценки сложности произведённого на каждой итерации вычисления.
\begin{figure}[ht]
	\centerfloat{
		\ifdefmacro{\tikzsetnextfilename}{\tikzsetnextfilename{tikz_repeat_compiled}}{}
		\begin{tikzpicture}[scale=2]
			\node[circle,draw] (game1) at (1, 0) {$\Gamma$};
			\node[circle,draw] (game2) at (3, 0) {$\Gamma$};
			\node[circle,draw] (game3) at (5, 0) {$\Gamma$};
			\node[rectangle,draw] (calc11) at (0,  1) {$M^1$};
			\node[rectangle,draw] (calc12) at (2,  1) {$M^1$};
			\node[rectangle,draw] (calc13) at (4,  1) {$M^1$};
			\node[] (calc1i) at (6, 1) {$\cdots$};
			\node[rectangle,draw] (calc21) at (0, -1) {$M^2$};
			\node[rectangle,draw] (calc22) at (2, -1) {$M^2$};
			\node[rectangle,draw] (calc23) at (4, -1) {$M^2$};
			\node[] (calc2i) at (6, -1) {$\cdots$};
			\draw [->,thick,decorate,decoration={snake,post length=1mm}] (calc11) -- (game1) node[midway,sloped,below] {$s^1_0$};
			\draw [->,thick,decorate,decoration={snake,post length=1mm}] (calc21) -- (game1) node[midway,sloped,above] {$s^2_0$};
			\draw [->,thick,decorate,decoration={snake,post length=1mm}] (calc12) -- (game2) node[midway,sloped,below] {$s^1_1$};
			\draw [->,thick,decorate,decoration={snake,post length=1mm}] (calc22) -- (game2) node[midway,sloped,above] {$s^2_1$};
			\draw [->,thick,decorate,decoration={snake,post length=1mm}] (calc13) -- (game3) node[midway,sloped,below] {$s^1_2$};
			\draw [->,thick,decorate,decoration={snake,post length=1mm}] (calc23) -- (game3) node[midway,sloped,above] {$s^2_2$};
			\draw [->,thick,decorate,decoration={snake,post length=1mm}] (game1) -- (calc12) node[midway,sloped,below] {$s_0$};
			\draw [->,thick,decorate,decoration={snake,post length=1mm}] (game1) -- (calc22) node[midway,sloped,above] {$s_0$};
			\draw [->,thick,decorate,decoration={snake,post length=1mm}] (game2) -- (calc13) node[midway,sloped,below] {$s_1$};
			\draw [->,thick,decorate,decoration={snake,post length=1mm}] (game2) -- (calc23) node[midway,sloped,above] {$s_1$};
			\draw [->,thick,decorate,decoration={snake,post length=1mm}] (game3) -- (calc1i) node[midway,sloped,below] {$s_2$};
			\draw [->,thick,decorate,decoration={snake,post length=1mm}] (game3) -- (calc2i) node[midway,sloped,above] {$s_2$};
			\node[diamond,draw] (sum1) at (-1, 2) {$\Sigma^1$};
			\node[] (sum10) at (0,  2) [label=$-\mathfrak{w}^1_0$] {};
			\node[] (sum11) at (1,  2) [label=$+u^1(s_0)$] {};
			\node[] (sum12) at (2,  2) [label=$-\delta \mathfrak{w}^1_1$] {};
			\node[] (sum13) at (3,  2) [label=$+\delta u^1(s_1)$] {};
			\node[] (sum14) at (4,  2) [label=$-\delta^2 \mathfrak{w}^1_2$] {};
			\node[] (sum15) at (5,  2) [label=$+\delta^2 u^1(s_2)$] {};
			\node[] (sum1i) at (6,  2) {$\cdots$};
			\node[diamond,draw] (sum2) at (-1, -2) {$\Sigma^2$};
			\node[] (sum20) at (0, -2) [label=below:$-\mathfrak{w}^2_0$] {};
			\node[] (sum21) at (1, -2) [label=below:$+u^2(s_0)$] {};
			\node[] (sum22) at (2, -2) [label=below:$-\delta \mathfrak{w}^2_1$] {};
			\node[] (sum23) at (3, -2) [label=below:$+\delta u^2(s_1)$] {};
			\node[] (sum24) at (4, -2) [label=below:$-\delta^2 \mathfrak{w}^2_2$] {};
			\node[] (sum25) at (5, -2) [label=below:$+\delta^2 u^2(s_2)$] {};
			\node[] (sum2i) at (6, -2) {$\cdots$};
			\draw [->,thick] (sum1i) -- (sum1);
			\draw [->,thick] (calc11) -- (sum10);
			\draw [->,thick] (game1) -- (sum11);
			\draw [->,thick] (calc12) -- (sum12);
			\draw [->,thick] (game2) -- (sum13);
			\draw [->,thick] (calc13) -- (sum14);
			\draw [->,thick] (game3) -- (sum15);
			\draw [->,thick] (sum2i) -- (sum2);
			\draw [->,thick] (calc21) -- (sum20);
			\draw [->,thick] (game1) -- (sum21);
			\draw [->,thick] (calc22) -- (sum22);
			\draw [->,thick] (game2) -- (sum23);
			\draw [->,thick] (calc23) -- (sum24);
			\draw [->,thick] (game3) -- (sum25);
			\fill [white] (1,  1) circle (2pt);
			\fill [white] (3,  1) circle (2pt);
			\fill [white] (5,  1) circle (2pt);
			\fill [white] (1, -1) circle (2pt);
			\fill [white] (3, -1) circle (2pt);
			\fill [white] (5, -1) circle (2pt);
			\draw [->,double,thick] (-1,  1) -- (calc11) node[midway,above] {$\psi^1_0$};
			\draw [->,double,thick] (calc11) -- (calc12) node[near end,above] {$\psi^1_1$};
			\draw [->,double,thick] (calc12) -- (calc13) node[near end,above] {$\psi^1_2$};
			\draw [->,double,thick] (calc13) -- (calc1i) node[near end,above] {$\psi^1_3$};
			\draw [->,double,thick] (-1, -1) -- (calc21) node[midway,below] {$\psi^2_0$};
			\draw [->,double,thick] (calc21) -- (calc22) node[near end,below] {$\psi^2_1$};
			\draw [->,double,thick] (calc22) -- (calc23) node[near end,below] {$\psi^2_2$};
			\draw [->,double,thick] (calc23) -- (calc2i) node[near end,below] {$\psi^2_3$};
		\end{tikzpicture}
	}
	\legend{}
	\caption[Повторяющаяся игра с учётом стоимости вычислений]{Повторяющаяся игра с учётом стоимости вычислений}\label{fig:repeat}
\end{figure}

Диаграмма на рисунке \ref{fig:repeat} изображает общую схему взаимодействия агентов и среды для двух игроков (естественным образом обобщающуюся на любое конечное их число). В узлах, помеченных буквой $\Gamma$, происходят последовательные розыгрыши произвольной игры $\Gamma = \langle A, S^a, u^a(s), a \in A \rangle$. В $i$-м розыгрыше игрок $a$ выбирает свою стратегию $s^a_i$, применяя вероятностный алгоритм $M^a$, находящийся в состоянии $\psi^a_i$, к результату предыдущего розыгрыша $s_{i-1}$ (если таковой был) и запоминая необходимые для будущих итераций результаты вычислений в новом состоянии $\psi^a_{i+1}$. Узлы $\Sigma^a$ изображают последовательное суммирование разностей выигрыша $u^a(s_i)$ и затрат на произведённое вычисление $\mathfrak{w}^a_i$, с учётом экспоненциально уменьшающегося коэффициента дисконтирования $\delta^i$.

Поскольку мы говорим о конфликтах рациональных агентов, имеет смысл рассматривать только универсальные алгоритмы $M^a$, позволяющие закодировать любой набор вычислимых стратегий повторяющейся игры в начальных состояниях памяти $\psi^a_0$. Нотация $M^a[g]$ будет обозначать среднюю стоимость вычисления произвольной функции $g$ при её оптимальной реализации посредством $M^a$. Кроме того, если обозначить символом $\mathfrak{s}$ набор стратегий повторяющейся игры с дисконтированием, то нотация $M^a[\mathfrak{s}]$ подходит для обозначения стоимости полного объёма вычислений, необходимых игроку $a$ для выбора каждого хода с учётом того же коэффициента $\delta$. Таким образом, в повторяющейся игре с учётом стоимости вычислений каждый игрок $a$ оптимизирует не просто $u^a(\mathfrak{s})$, но
\begin{equation*}
	\hat{u}^a(\mathfrak{s}) = u^a(\mathfrak{s}) - M^a[\mathfrak{s}].	
\end{equation*}

\section{Криптографическое согласование стратегий}\label{sec:ch3/sect4}

Хотя, как было показано ранее, в отсутствие дополнительной информационной асимметрии трёхсторонний чёт-нечет не даёт возможности двум игрокам наказывать третьего, используя только лишь смешанные стратегии, даже в том случае, когда игроки не могут использовать тайные механизмы корреляции, учёт стоимости вычислений позволяет в повторяющихся играх применять стратегии наказания, опирающиеся на достижения современной криптографии. Для демонстрации этого нам понадобятся два распространённых криптографических примитива.

Во-первых, необходим \emph{протокол совместной выработки ключа} \cite{Boyd}. В криптографии этим термином называют механизм, при помощи которого Алиса и Боб могут создать общую секретную последовательность битов, априорно обладая лишь публичным знанием об устройстве и параметрах самого механизма, и обмениваясь сообщениями через незащищённый от прослушивания канал связи. При этом Кэрол, обладая тем же априорным знанием и имея возможность читать их сообщения, не может вычислить искомую секретную последовательность битов, поскольку это требует решения алгоритмически трудной задачи (такой, для которой необходимо количество операций, зависящее от длинны ключа экспоненциально). В качестве такого механизма может выступать, например, семейство протоколов Диффи-Хеллмана (далее DH) на основе задач факторизации целых чисел или дискретного логарифмирования (в конечной мультипликативной группе или на эллиптической кривой). Опишем общую схему произвольного протокола DH, не вдаваясь в технические детали.

Пусть имеется семейство биекций $f_n: \mathbb{N}_{<2^n} \leftrightarrow \mathbb{N}_{<2^n}, n \in \mathbb{N}$, обладающих свойством односторонности, т.е. для любого универсального вычислителя $M^*$ одновременно выполняется $M^*[f_n] \in o(2^n)$ (стоимость вычисления самой функции растёт с ростом $n$ полиномиально) и $M^*[f_n^{-1}] \in \Theta(2^n)$ (стоимость вычисления обратной функции растёт с ростом $n$ экспоненциально). Кроме этого, пусть имеется семейство двухместных функций $h_n : \mathbb{N}_{<2^n} \times \mathbb{N}_{<2^n} \rightarrow \mathbb{N}_{<2^n}$ таких, что
\begin{itemize}
	\item $\forall x, y \in \mathbb{N}_{<2^n}, h_n(f_n(x), y) = h_n(x, f_n(y))$;
	\item $\forall x, y, z \in \mathbb{N}_{<2^n}, h_n(h_n(x, y), z) = h_n(x, h_n(y, z))$;
	\item $h_n(x_1, y_1) = h_n(x_2, y_2) \Rightarrow x_1 = x_2 \cap y_1 = y_2 \cup x_1 \neq x_2 \cap y_1 \neq y_2$.;
	\item $M^*[h_n] \in o(2^n)$.
\end{itemize}

Алиса выбирает случайное натуральное число $0 \le x < 2^n$, выполняющее роль её закрытого ключа, и вычисляет $X = f_n(x)$, выполняющее роль её открытого ключа. На другом конце Боб аналогичным образом генерирует пару ключей $y$ и $Y = f_n(y)$. Алиса и Боб обмениваются открытыми ключами через прослушиваемый Кэрол канал связи. Теперь Алиса, зная свой закрытый ключ $x$ и открытый ключ Боба $Y$, может вычислить $h_n(x, Y)$, а Боб, соответственно, $h_n(X, y)$. В силу свойств функции $h_n$, вычисленные ими значения можно считать искомым общим секретным ключом $K = h_n(x, Y) = h_n(X, y)$. При этом для Кэрол, знающей только открытые ключи $X$ и $Y$, вычисление общего секретного ключа требует вычисления либо $f_n^{-1}(X)$, либо $f_n^{-1}(Y)$. Благодаря разнице между асимптотической сложностью прямой и обратной функции всегда можно подобрать такое $n$, что стоимости вычисления $f_n$ и $h_n$ оказываются приемлемы, тогда как вычисление $f_n^{-1}$ "--- непозволительно дорогой процесс. Также заметим, что вследствие ассоциативности функции $h_n$ можно считать многоместными, и члены группы агентов любого размера могут комбинировать с их помощью общие секретные ключи, если каждый опубликовал свой открытый ключ "--- например, $K = h_n(x, Y, Z) = h_n(X, y, Z) = h_n(X, Y, z)$ для трёх сторон.

Вторым криптографическим примитивом, необходимым для стратегии наказания, является \emph{криптографически стойкий генератор псевдослучайных чисел} \cite{Gutmann}, далее называемый CSPRNG. Его можно представить в виде семейства функций $G_n : \mathbb{N}_{<2^n} \times \mathbb{N} \rightarrow \{0, 1\}$, первый аргумент которых называется зерном (или seed), а второй "--- позицией. Программа, вычисляющая для заданного $K \in \mathbb{N}_{<2^n}$ последовательные значения $G_n(K, i), i = 1, 2, \ldots$, должна совершать каждый шаг за полиномиальное от $n$ число операций. При этом генератор обязан проходить тест на следующий бит, т.е. не должно существовать полиномиально сложного от $n$ алгоритма, способного без знания $K$ по первым $i$ битам генерируемой последовательности угадать $G_n(K, i+1)$ с вероятностью, отличной от $\frac{1}{2}$.

Теперь, используя вышеописанные примитивы, можно сконструировать три новых типа стратегий для повторяющегося трёхстороннего чёт-нечета. Представим, что игроки сидят за круглым столом так, что игрок 1 сидит справа от игрока 2, игрок 2 "--- справа от игрока 3, а игрок 3 "--- справа от игрока 1. Назовём первую из новых стратегией $\mathfrak{s}^L_n$ <<рукопожатие влево>>:
\begin{enumerate}
	\item Выбрать случайное число $x \in \mathbb{N}_{<2^n}$.
	\item Вычислить $X = f_n(x)$ и представить в виде битовой последовательности $(X_i) \in \{0, 1\}^n$.
	\item Для каждого $i = 1 \ldots n$ совершить один ход игры, выбирая решку при $X_i = 1$ и орла в противном случае. Выбранную сидящим слева игроком стратегию (с тем же сопоставлением) запомнить в качестве очередного элемента битовой последовательности $(Y_i) \in \{0, 1\}^n$, соответствующей числу $Y \in \mathbb{N}_{<2^n}$.
	\item Вычислить $K = h_n(x, Y)$.
	\item Все последующие ходы совершать, выбирая стратегию в соответствии с последовательно генерируемыми CSPRNG значениями $G_n(K, i), i = 1, 2, \ldots$.
\end{enumerate}

Стратегию $\mathfrak{s}^R_n$ <<рукопожатие вправо>> строим аналогичным образом, заменяя <<слева>> на <<справа>> и меняя местами $x$ с $y$, $X$ с $Y$ и $h_n(x, Y)$ с $h_n(X, y)$. Такие парные стратегии с равной длиной ключа позволяют любым двум игрокам превратить первые $n$ ходов игры в своеобразный <<танец синхронизации>>, вырабатывая общее секретное зерно для генератора псевдослучайных битов, чей вывод на последующих ходах используется в качестве механизма корреляции. Третьему игроку, при этом, для присоединения к согласованному выбору приходится применять стратегию $\mathfrak{s}^*_n$ <<взлом>>:
\begin{enumerate}
	\item Первые $n$ ходов играть смешанную стратегию равновероятного выбора и запоминать ходы оппонентов для получения их открытых ключей $X$ и $Y$.
	\item Вычислить $K = h_n(f_n^{-1}(X), Y)$ или $K = h_n(X, f_n^{-1}(Y))$.
	\item Все последующие ходы совершать, выбирая стратегию в соответствии с последовательно генерируемыми CSPRNG значениями $G_n(K, i), i = 1, 2, \ldots$.
\end{enumerate}

Обозначим также стратегию $\mathfrak{s}^{\varnothing}$ <<пас>>, при использовании которой игрок на каждой итерации просто выбирает между орлом и решкой случайно и равновероятно. Несложно заметить, что если мы ограничимся рассмотрением только четырёх вышеперечисленных классов стратегий, то почти все комбинации любых из них по выплатам неотличимы от классической точки равновесия в смешанных стратегиях $u(\mathfrak{s}^{\varnothing}, \mathfrak{s}^{\varnothing}, \mathfrak{s}^{\varnothing}) = (4, 4, 4)$. Единственным исключением оказываются ситуации, в которой любые два игрока применяют друг на друга соответствующие <<рукопожатия>> с одним и тем же $n$, а третий игрок не применяет <<взлома>> с той же длиной ключа "--- например, $u(\mathfrak{s}^L_n, \mathfrak{s}^R_n, \mathfrak{s}^{\varnothing}) = (4 + \delta^n, 4 + \delta^n, 4 - 2 \delta^n)$. Здесь первые два игрока первые $n$ ходов тратят на обмен ключами, что с точки зрения выплат неотличимо от случайного выбора, а после этого пользуются общим CSPRNG в качестве механизма корреляции, забирая у третьего половину его выигрыша. Если бы третий не мог ответить им стратегией <<взлома>>, то, применяя народную теорему, наборы с двумя <<рукопожатиями>> можно было бы использовать для его равновесного наказания. Поскольку взаимное рукопожатие возможно в любой паре игроков, запрет взломов обеспечил бы этой игре точку равновесного минимакса $(4 - 2 \delta^n, 4 - 2 \delta^n, 4 - 2 \delta^n)$, что при $\delta^n > \frac{1}{2}$ позволяло бы, например, конструировать совершенные подыгровые равновесия с вектором выплат $(6, 3, 3)$, недостижимым ни в одном равновесии по Нэшу однократного трёхстороннего чёт-нечета.

Покажем теперь, как учёт стоимости вычислений позволяет правильным подбором длины ключа $n$ достичь необходимого запрета на стратегию <<взлома>>. Выпишем для каждой из предложенных стратегий дисконтированные затраты на выбор ходов:
\begin{itemize}
	\item $M^a[\mathfrak{s}^{\varnothing}] = 0$, поскольку для любого разумно устроенного вычислительного устройства обычные смешанные стратегии, очевидно, можно считать бесплатными или почти бесплатными;
	\item $M^a[\mathfrak{s}^L_n] = M^a[\mathfrak{s}^R_n] = (1 - \delta) M^a[f_n] + \delta ^ n ((1 - \delta) M^a[h_n] + M_a[G_n])$, поскольку для стратегий рукопожатия необходимо один раз перед первым ходом создать пару ключей, по прошествии $n$ ходов вычислить общий секретный ключ, а потом каждый ход генерировать по одному биту CSPRNG;
	\item $M^a[\mathfrak{s}^*_n] = \delta ^ n ((1 - \delta) (M^a[f_n^{-1}] + M^a[h_n]) + M_a[G_n])$, поскольку для стратегии взлома необходимо один раз по прошествии $n$ ходов, имея только пару публичных ключей, вычислить секретный ключ, а потом каждый ход генерировать по одному биту CSPRNG.
\end{itemize}

Проверим теперь ситуацию $(\mathfrak{s}^L_n, \mathfrak{s}^R_n, \mathfrak{s}^{\varnothing})$ на равновесие по Нэшу с учётом расходов на вычисления. Для первых двух игроков требуется, чтобы затраты на криптографическую синхронизацию не превысили доход от хвоста розыгрыша, т.е. $\frac{1 - \delta}{\delta^n} M^a[f_n] + (1 - \delta) M^a[h_n] + M^a[G_n] \le 1$. Несложно заметить, что до тех пор, пока $M^a[G_n] < 1$, всегда можно подобрать $\delta$ достаточно большое для нивелирования разовых подготовительных расходов. Для третьего же игрока наоборот, приходится подбирать достаточно большую битовую длину ключа, чтобы процедура его взлома оказалась дороже, чем потенциальный доход в хвосте розыгрыша, т.е. $(1 - \delta) (M^a[f_n^{-1}] + M^a[h_n]) + M_a[G_n] \ge 2$. Здесь уже в качестве главного компонента выступает экспоненциально растущая стоимость обращения односторонней функции "--- при $M^a[f_n^{-1}] \ge \frac{2}{1 - \delta}$ присоединение к коррелированной стратегии полностью теряет смысл. Таким образом, чтобы рассматриваемая точка была равновесием по Нэшу, достаточно выполнения следующего набора условий:% для некоторого $0 < \gamma < 1$:
\begin{equation*}
	\begin{cases}
		(1 - \delta)(\delta^{-n} M^1[f_n] + M^1[h_n]) + M^1[G_n] < 1; \\
		(1 - \delta)(\delta^{-n} M^2[f_n] + M^2[h_n]) + M^2[G_n] < 1; \\
		(1 - \delta) M^3[f_n^{-1}] \ge 2.
	\end{cases}
\end{equation*}
%\begin{equation*}
%	\begin{cases}
%		M^{1,2}[G_n] < 1 - \gamma; \\
%		\frac{M^{1,2}[f_n]}{\delta^n} + M^{1,2}[h_n] \le \frac{\gamma}{1 - \delta}; \\
%		M^3[f_n^{-1}] \ge \frac{2}{1 - \delta}.
%	\end{cases}
%\end{equation*}

Поскольку речь идёт о сравнении производительности абстрактных вычислительных устройств с безразмерными величинами, характеризующими предпочтения рациональных агентов, попытки доказательства каких-либо формальных утверждений относительно совместимости этой системы неравенств представляются малопродуктивными. Тем не менее, мы вполне можем попробовать отобразить выделенную неравенствами область допустимых величин $n$ и $\delta$ на соответствующие объекты реального мира. На практике криптосистемы, использующие дискретное логарифмирование на эллиптических кривых, считаются надёжными уже при 256-битных ключах (Curve25519 \cite{Bernstein}, например), т.е. стоимость их взлома заведомо превосходит возможности, доступные человеческой цивилизации на текущем этапе технологического развития. Одновременно с этим существуют и широко используются генераторы псевдослучайных чисел с 256-битным зерном, считающиеся криптографически стойкими (CTR-DRBG \cite{Hoang}, например). Это задаёт для дисконтирующего коэффициента диапазон приемлемости $0 < \epsilon \le 1 - \delta \le \frac{1}{370}$, причём, поскольку вычисление $f_{256}^{-1}$ на данный момент считается невозможным, $\epsilon$ можно считать бесконечно малым. Таким образом, если игроки, в распоряжении которых находятся современные вычислительные устройства, будут играть с реальными ставками в повторяющийся трёхсторонний чёт-нечет, то протяжённости серий от нескольких сотен розыгрышей окажется достаточно, чтобы применение криптографического согласования стратегий стало реальным способом получить преимущество.

\section{Народная теорема для игр с учётом стоимости вычислений}\label{sec:ch3/sect5}

При помощи уточнённого варианта <<народной>> теоремы продемонстрированный в предыдущем разделе фокус можно обобщить для пополнения множества совершенных подыгровых равновесий любой повторяющейся игры, отдельные итерации которой чувствительны к дополнительной информационной асимметрии. С этой целью в качестве наказаний для игрока, отклоняющегося от предписанной стратегии, могут использоваться наборы, коррелированные в пространстве заговоров, состоящем из единственной группы, включающей всех игроков кроме самого наказываемого. Поскольку при этом стратегии зависят не более чем от одного корреляционного механизма, можно упростить рассуждения, перейдя к вероятностным распределениям на платёжной матрице. Рассматривая игру $\Gamma = \langle A, S^a, u^a(s), a \in A \rangle$ с множествами участников $A = \{1, \ldots, m\}$ и исходов $S = S^1 \times \ldots \times S^m$ соответственно, введём обозначения:
\begin{align*}
	\mathbf{P}_S &= \{\mu : S \rightarrow [0, 1] \mid \sum_{s \in S} \mu(s) = 1\}; \\
	\mathbf{P}_S^{\setminus a} &= \{\mu \in \mathbf{P}_S \mid \forall s_1, s_2 \in S, s^a \in S^a, \mu(s_1) \mu(s_2 | s^a) = \mu(s_2) \mu(s_1 | s^a)\}.
\end{align*}

Здесь $\mathbf{P}_S$ представляет собой всевозможные вероятностные меры на множестве исходов $S$, а $\mathbf{P}_S^{\setminus a}$ "--- все вероятностные меры, гарантирующие попарную независимость выбора стратегии игроком $a$ с действиями всех остальных. Очевидно, выплаты при этом вычисляются по формуле математического ожидания:
\begin{equation*}
	u^a(\mu) = \sum_{s \in S} \mu(s) u^a(s).
\end{equation*}

Кроме того, для удобства обозначим девиацию игрока $a$ в пользу чистой стратегии $s^a_0$ аналогичным классической нотации образом:
\begin{equation*}
	\mu | s^a_0(s) =
	\begin{cases}
		\sum_{s^a_* \in S^a} \mu(s | s^a_*), & s^a = s^a_0; \\
		0, & s^a \neq s^a_0.
	\end{cases}
\end{equation*}

Теперь можно наконец ввести уточнённое для рассматриваемой модели понятие о резервном выигрыше:
\begin{definition}
	Слепым резервным выигрышем игрока $a$ в игре $\Gamma$ называется его резервный выигрыш $u^a(\check{\mu}^a)$ в игре $\Gamma | \{A \setminus \{a\}\}$, т.е. с заговором, объединяющим всех игроков кроме него:
	\begin{equation*}
		\check{\mu}^a \in \arg\min_{\mu \in \check{\mathbf{P}}} u^a(\mu),
	\end{equation*}
	где $\check{\mathbf{P}} = \{\mu \in \mathbf{P}_S^{\setminus a} \mid u^a(\mu) \ge u^a(\mu | s^a), \forall s^a \in S^a\}$.
\end{definition}
%\begin{definition}
%	Равновесным слепым резервным выигрышем игрока $a$ в игре $\Gamma$ называется его равновесный резервный выигрыш $u^a(\bar{\mu}^a)$ в игре с тем же заговором:
%	\begin{equation*}
%		\bar{\mu}^a \in \arg\min_{\mu \in \bar{\mathbf{P}}} u^a(\mu),
%	\end{equation*}
%	где $\bar{\mathbf{P}}$ "--- множество распределений, достижимых в равновесиях по Нэшу игры $\Gamma | \{A \setminus \{a\}\}$.
%\end{definition}

%Поскольку для наказания слепыми резервными выигрышами используется незнание наказываемым игроком общего секретного ключа, необходимо также учитывать его возможный доход при <<взломе>> процедуры DH путём обращения односторонней функции. Введём дополнительные обозначения:
%\begin{align*}
%	\Delta \check{u}^a &= \max_{s \in S, \check{\mu}^a(s) > 0} u^a(s) - u^a(\check{\mu}^a); \\
%	\Delta \bar{u}^a &= \max_{s \in S, \bar{\mu}^a(s) > 0} u^a(s) - u^a(\bar{\mu}^a);
%\end{align*}
%для наибольшего увеличения выигрыша, которого может достигнуть наказываемый игрок, взламывая ключи обычного и равновесного слепого наказания.

Для классической народной теоремы резервные выигрыши напрямую определяют соответствующую точку минимакса, однако в нашем случае всё немного сложнее. Поскольку для синхронизации как предписанных стратегий, так и наказаний используются CSPRNG, необходимо учитывать стоимость вычисления необходимой для выбора стратегического набора последовательности псевдослучайных битов. Обозначим символом $\mathfrak{b}(\mu) \in \mathbb{R}_{\ge 0}$ среднее количество бит, необходимое для выбора коррелированного набора стратегий по следующей процедуре. Занумеруем все исходы $\{s_1, \ldots, s_k\} \subseteq S$, участвующие в $\mu$ с ненулевой вероятностью, и построим в единичном интервале $\left[0, 1\right)$ сетку неубывающих чисел $\rho = (\rho_0 = 0, \rho_1, \ldots, \rho_{k - 1}, \rho_k = 1)$ разбивающую его таким образом, что $\rho_j - \rho_{j-1} = \mu(s_j), j = \overline{1,k}$.

\begin{enumerate}
	\item Положим $\mathcal{X}_{min} = 0$ и $\mathcal{X}_{max} = 1$.
	\item Проверим, нет ли такого $j$, что $\rho_{j-1} \le \mathcal{X}_{min} < \mathcal{X}_{max} \le \rho_j$. Если есть, остановим алгоритм, приняв решение о выборе стратегического набора $s_j$.
	\item Сгенерируем очередной бит псевдослучайной последовательности. Значение $\frac{\mathcal{X}_{min} + \mathcal{X}_{max}}{2}$ в случае $0$ присвоим переменной $\mathcal{X}_{max}$, а в случае $1$ "--- $\mathcal{X}_{min}$.
	\item Перейдём к шагу 2.
\end{enumerate}

Если все игроки используют CSPRNG с одним и тем же зерном, очевидно, что они выберут один и тот же набор стратегий, использовав при этом одно и то же количество псевдослучайных битов, в среднем зависящее только от распределения вероятностей $\mu$. Следовательно, ожидаемый доход игрока $a$ с учётом стоимости вычислений можно представить как $u_{M,n}^a(\mu) = u^a(\mu) - \mathfrak{b}(\mu) M^a[G_n]$, где $M^a[G_n]$ "--- стоимость генерации одного псевдослучайного бита. Заметим, что полученное значение зависит ещё и от $n$, т.е. битовой длины используемого зерна CSPRNG, что приводит к необходимости заменять понятие о точке минимакса более сложным определением:

\begin{definition}
%	Для каждого распределения $\mu$ на множестве исходов игры в нормальной форме $\Gamma = \langle A, S^a, u^a(s), a \in A \rangle$ 
	Распределение вероятностей $\mu$ на множестве исходов игры в нормальной форме $\Gamma = \langle A, S^a, u^a(s), a \in A \rangle$ называется $M,n$-приемлемым, если $u_{M,n}^a(\mu) > u^a(\check{\mu}^a)$ для каждого игрока $a \in A$. %В случае $u_{M,n}^a(\mu) > u^a(\bar{\mu}^a)$, его можно называть равновесно $M,n$-приемлемым.
\end{definition}

Заметим, что мы пока говорили о том, как игроки используют для синхронизации действий CSPRNG с общим зерном, не уточняя, как именно это общее зерно вырабатывается. Аналогично примеру из предыдущего раздела в этом качестве будем использовать общие секретные ключи, вычисляемые по протоколу DH. Опишем процедуру для генерации и публикации ключей. Сперва каждый игрок $a$ выбирает случайное вещественное число $\chi^a$, равномерно распределённое в диапазоне $\left[0, 2^n\right)$. Округление этого числа $x^a = \lfloor\chi^a\rfloor \in \mathbb{N}_{<2^n}$ используется им далее в качестве закрытого ключа. При помощи односторонней функции игрок вычисляет открытый ключ $X^a = f(x^a) \in \mathbb{N}_{<2^n}$. Для обмена значениями $X^a$ игроки могут использовать любой условленный \emph{тотально смешанный} набор стратегий:

\begin{definition}
	Набор смешанных стратегий назовём тотально смешанным, когда в его составе нет ни одной чистой стратегии.
\end{definition}

Поскольку $f$ "--- биекция, случайная величина $\mathcal{X}^a = \chi^a - x^a + X^a$ также равномерно распределена в диапазоне $\left[0, 2^n\right)$. Для того, чтобы раскрыть другим игрокам своё значение $X^a$, каждый игрок использует метод <<фрактального>> кодирования числа $\mathcal{X}^a$ в последовательности совершаемых им ходов, неотличимых по вероятностному распределению от смешанной стратегии $s_0^a = (p^a_1, \ldots, p^a_{\lvert S^a \rvert})$ из условленного тотально смешанного набора $s_0$. Построим в единичном интервале $\left[0, 1\right)$ сетку неубывающих чисел $\rho^a = (\rho^a_0 = 0, \rho^a_1, \ldots, \rho^a_{\lvert S^a \rvert - 1}, \rho^a_{\lvert S^a \rvert} = 1)$ разбивающую его таким образом, что $\rho^a_j - \rho^a_{j-1} = p^a_j, j = \overline{1,\lvert S^a \rvert}$. Это позволяет применить следующий алгоритм кодирования:

\begin{enumerate}
	\item До совершения первого хода другие игроки знают, что $\mathcal{X}^a$ равномерно распределено в интервале $\left[0, 2^n\right)$. Положим $\mathcal{X}^a_{min} = 0$ и $\mathcal{X}^a_{max} = 2^n$.
	\item Отмасштабируем $\rho^a$ на интервал $\left[\mathcal{X}^a_{min}, \mathcal{X}^a_{max}\right)$, получив вложенную в него сетку $\mathcal{P}^a = (\mathcal{X}^a_{min} (1 - \rho^a_j) + \mathcal{X}^a_{max} \rho^a_j, j = \overline{0, \lvert S^a \rvert})$.
	\item Выберем на очередной итерации $s^a_j$ такое, что $\mathcal{P}^a_{j-1} \le \mathcal{X}^a < \mathcal{P}^a_j$.
	\item Теперь другие игроки знают, что $\mathcal{X}^a$ равномерно распределено в интервале $\left[\mathcal{P}^a_{j-1}, \mathcal{P}^a_j\right)$. Вернёмся к шагу 2, положив $\mathcal{X}^a_{min} = \mathcal{P}^a_{j-1}$ и $\mathcal{X}^a_{max} = \mathcal{P}^a_j$.
\end{enumerate}

Этот алгоритм может выполняться бесконечно, постепенно уменьшая меру незнания других игроков относительно величины $\mathcal{X}^a$. При этом на каждой отдельной итерации распределение вероятностей исходов неотличимо от набора смешанных стратегий $s_0$. В тот момент когда начинает выполняться неравенство $X^a \le \mathcal{X}^a_{min} < \mathcal{X}^a_{max} \le X^a + 1$, другие игроки получают уверенность относительно значения $X^a$, и открытый ключ игрока $a$ можно считать опубликованным. Если все игроки начинают публикацию ключей одновременно, то ожидаемое количество итераций до завершения публикации последнего из них зависит только от используемого стратегического набора $s_0$ и величины $n$, так что его можно обозначить символом $\mathfrak{t}(s_0, n) \in \mathbb{N}$.

Прежде чем переходить к аналогу народной теоремы для повторяющихся игр с учётом стоимости вычислений, необходимо ввести ещё одно обозначение. Вектор $\Delta u = (\Delta u^a, a \in A) \in \mathbb{R}_{\ge 0}^m$, где
\begin{equation*}
	\Delta u^a = \max_{s \in S, s^a_* \in S^a} (u^a(s | s^a_*) - u^a(s)),
\end{equation*}
состоит из максимальных доходов, которые может получить каждый игрок, отклоняясь от стратегии из произвольного предписанного набора.

\begin{theorem}
	Пусть $\Gamma = \langle A, S^a, u^a(s), a \in A \rangle$ "--- произвольная матричная игра $m$ участников, имеющая хотя бы одно тотально смешанное равновесие $s_0$. Пусть $M = (M^1, \ldots, M^m)$ "--- набор универсальных вычислительных устройств, которые соответствующие игроки могут использовать для выбора очередного шага стратегии в повторяющейся игре. Пусть $(\Gamma)_M^{\delta}$ "--- бесконечно повторяющаяся игра с итерацией $\Gamma$, дисконтирующим коэффициентом $\delta$ и учётом стоимости эксплуатации вычислительных устройств $M$. Любому $M,n$-приемлемому распределению вероятностей $\mu$ на множестве исходов игры $\Gamma$ можно поставить в соответствие набор стратегий игры $(\Gamma)_M^{\delta}$ с выплатами каждого игрока $a$, равными %$v^a_{SYNC} + \delta^{\mathfrak{t}(s_0, n)} u_{M,n}^a(\mu)$, где
	\begin{equation*}
		(1 - \delta^{\mathfrak{t}(s_0, n)}) u^a(s_0) + \delta^{\mathfrak{t}(s_0, n)} u_{M,n}^a(\mu) - (1 - \delta) (M^a[f_n] + \delta^{\mathfrak{t}(s_0, n)} (m - 1) M^a[h_n]).
	\end{equation*}
	
	При этом для того, чтобы этот набор оказался равновесием по Нэшу, достаточно выполнения следующих условий для каждого $a \in A$:
	\begin{enumerate}
		\item $(1 - \delta) (M^a[f_n] + \delta^{\mathfrak{t}(s_0, n)} (m - 1) M^a[h_n]) < \delta^{\mathfrak{t}(s_0, n)}(u_{M,n}^a(\mu) - u^a(\check{\mu}^a))$;
		\item $(1 - \delta) \Delta u^a < \delta (u_{M,n}^a(\mu) - u^a(\check{\mu}^a))$;
		\item $(1 - \delta) M^a[f_n^{-1}] > \delta \Delta u^a$.
	\end{enumerate}
%
%	При замене в условиях обычного слепого наказания $\check{\mu}^a$ на равновесное слепое наказание $\bar{\mu}^a$, результирующее равновесие становится совершенным подыгровым.
\end{theorem}

\begin{proof}
	Равновесие, соответствующее $M,n$-приемлемому распределению вероятностей $\mu$ строится как трёхэтапный процесс:
	\begin{itemize}
		\item \emph{Синхронизация.} Сперва каждый игрок $a$ создаёт пару ключей длины $n$, терпя при этом убыток равный $(1 - \delta) M^a[f_n]$. Затем, используя вышеописанный алгоритм фрактального кодирования в тотально смешанном равновесии $s_0$, публикует свой открытый ключ. Эта фаза продолжается до тех пор, пока не будут опубликованы все ключи, что в среднем займёт $\mathfrak{t}(s_0, n)$ ходов, давая суммарный выигрыш $(1 - \delta^{\mathfrak{t}(s_0, n)}) u^a(s_0)$. Завершается стадия вычислением общего секретного ключа, для чего каждому игроку необходимо $m - 1$ раз вычислить функцию $h_n$, что соответствует убытку размером $(1 - \delta) \delta^{\mathfrak{t}(s_0, n)} (m - 1) M^a[h_n]$.
		\item \emph{Розыгрыш.} Используя CSPRNG, инициируемое общим секретным ключом, игроки на каждой итерации псевдослучайно выбирают в соответствии с распределением $\mu$ новый набор чистых стратегий, так что каждый игрок $a$ получает при бесконечном повторении ожидаемый суммарный выигрыш $\delta^{\mathfrak{t}(s_0, n)} u_{M,n}^a(\mu)$. Если ни один из игроков не отклонился от предписанного поведения ни на предыдущей стадии, ни на этой, то итоговые выплаты в пределе соответствуют предсказанным в формулировке теоремы.
		\item \emph{Наказание.} Если любой из игроков на стадии розыгрыша выбирает чистую стратегию, не соответствующую предписанной CSPRNG, остальные игроки переключаются в режим его наказания. То же самое происходит сразу после стадии синхронизации, если на ней кто-то из игроков игнорирует процедуру создания ключей. Для этого они вычисляют новый общий секретный ключ, на этот раз исключая публичный ключ наказываемого игрока $a$. Инициировав CSPRNG этим новым ключом, они на каждой итерации псевдослучайно выбирают новый набор чистых стратегий в соответствии с распределением $\check{\mu}^{a}$. %или $\bar{\mu}^{a_*}$, в зависимости от необходимости получить обычное или совершенное подыгровое равновесие по Нэшу.
	\end{itemize}

	От этой схемы предписанных действий возможны несколько видов индивидуальных отклонений. Во-первых, на стадии синхронизации любой игрок $a$ может попытаться сэкономить путём отказа от создания пары ключей и вычисления общего секретного ключа. Поскольку набор $s_0$ равновесен по Нэшу, улучшить выигрыш по сравнению с $(1 - \delta^{\mathfrak{t}(s_0, n)}) u^a(s_0)$ не получится, но можно используя простую смешанную стратегию избежать расходов на криптографию в размере $(1 - \delta) (M^a[f_n] + \delta^{\mathfrak{t}(s_0, n)} (m - 1) M^a[h_n])$. Реакцией на это решение становится переход остальных игроков к стадии наказания игрока $a$ сразу по завершении публикации ключей, так что вместо $\delta^{\mathfrak{t}(s_0, n)} u_{M,n}^a(\mu)$ в хвосте розыгрыша игрок получает $\delta^{\mathfrak{t}(s_0, n)} u^a(\check{\mu}^a)$. Такое отклонение делает невыгодным ограничение на $\delta$, накладываемое условием 1 формулировки теоремы.
	
	Во-вторых, на основной стадии розыгрыша любой игрок $a$ может, вычислив очередной псевдослучайный набор $s_i$ отказаться играть предписанную стратегию $s_i^a$ в пользу более выгодной, получив разовый доход, не превосходящий $(1 - \delta) \delta^i \Delta u^a$. Заметив это, остальные игрокаи переходят к стадии наказания, что уменьшает его доход в хвосте розыгрыша с $\delta^{i+1} u_{M,n}^a(\mu)$ до $\delta^{i+1} u^a(\check{\mu}^a)$. Такое отклонение делает невыгодным ограничение на $\delta$, накладываемое условием 2 формулировки теоремы.
	
	В-третьих, поскольку слепые наказания опираются на то, что наказываемый игрок $a$ не может предсказать биты CSPRNG, инициированного общим секретным ключом, вычисление которого производилось без использования его собственной пары ключей, он может попытаться ослабить наказание, <<взломав>> секретный ключ путём обращения односторонней функции на одном из открытых ключей. Если это происходит на $i$-й итерации игры, то он должен потратить на это $(1 - \delta) \delta^i M^a[f_n^{-1}]$, получив в результате не более чем $\delta^{i+1} \Delta u^a$ дохода от хвоста розыгрыша. Такое отклонение делает невыгодным ограничение на $\delta$, накладываемое условием 3 формулировки теоремы.
	
	Таким образом предписанная процедура розыгрыша бесконечной повторяющейся игры $(\Gamma)_M^{\delta}$ действительно оказывается равновесием по Нэшу с искомыми ожидаемыми выигрышами. %Для того, чтобы она оказалась ещё и совершенной по подыграм достаточно дополнить стадию наказания, предписав, что любой из наказывающих игроков, отклонившись от вычисленной в соответствии с CSPRNG %Остаётся заметить, что использование в качестве наказания равновесных по Нэшу в каждой итерации распределений $\bar{\mu}^a$ не даёт также и наказывающим игрокам получать дополнительную выгоду путём отказа участвовать в предписанном 
\end{proof}

Несложно заметить, что доказанное утверждение является аналогом скорее самых первых формулировок <<народной>> теоремы, в которых речь ещё не шла о равновесиях, совершенных по подыграм. Увы, но методы, при помощи которых строятся совершенные подыгровые равновесия в классическом случае, по объективным причинам сложно перенести на игры с учётом стоимости вычислений. Как уже было упомянуто в начале главы, обычно для этого используется схема бесконечного наказания последнего отклонившегося от предписанной стратегии, применяющаяся в том числе и в процессе наказания предыдущего отклонившегося. Здесь же такой наивный подход натыкается на препятствие "--- рассмотрим ситуацию, в которой игрок $a$ решил сэкономить на создании ключей и вместо предписанной процедуры на стадии синхронизации просто играл смешанную стратегию $s_0^a$. Заметив это, остальные игроки по завершении обмена ключами начинают его слепое наказание коррелированным набором стратегий $\check{\mu}^a$. Представим теперь, что будет, если один из наказывающих решит на очередной итерации использовать стратегию более выгодную для себя, чем предписывает в качестве наказания CSPRNG. По классической схеме такой игрок должен быть сам назначен наказываемым начиная со следующей итерации, причём в его наказании должен принимать участие в том числе и игрок $a$. Однако, поскольку игрок $a$ сэкономил на создании своей пары ключей, он просто не имеет возможности вычислить общий секретный ключ, требующийся для этого, так что любое наказание, в котором он должен участвовать, перестаёт быть эффективной угрозой.

Ещё одним заметным недостатком доказанной теоремы можно считать то, что она неявно опирается на наблюдаемость стратегий, используемых игроками. Подразумевается, что единственная тайна, которую игроки скрывают от своих оппонентов - это конкретные значения ключей, тогда как применяемые ими алгоритмы известны публично. Однако, утверждение могло бы стать намного сильнее и убедительнее, если бы мы считали публично известными только конкретные чистые стратегии, играемые участниками. Это имеет значение, например, когда мы говорим, что игрок, решивший сэкономить на создании ключей, назначается наказываемым по окончании стадии синхронизации "--- ведь если игроки могут судить о том, создавал кто-то ключи или нет, только по совершаемым ходам, то в течение некоторого количества первых итераций стадии розыгрыша отклонившийся мог бы случайно угадывать, какую стратегию предписывает CSPRNG, даже без общего секретного ключа, просто выбирая наиболее вероятную в соответствии с распределением $\mu$ стратегию. Причём, если распределение $\mu$ таково, что в нём кто-то должен играть одну и ту же чистую стратегию с вероятностью $1$, то ему вообще не имеет смысла создавать ключи, поскольку отказ от их создания никогда не будет раскрыт.

Эту же уязвимость можно использовать ещё более тонким образом, если доход игрока $a$ от синхронизационного набора смешанных стратегий $s_0$ превышает его доход от целевого распределения $\mu$. Если вместо случайного выбора значения $\chi^a$ при создании ключей игрок будет <<подкручивать>> его дробную часть так, чтобы она была близка к $0$ или $1$, то тем самым он может произвольно увеличивать продолжительность стадии синхронизации и, соответственно, свой итоговый доход. Описанные проблемы, возникающие при отказе от наблюдаемости используемых игроками алгоритмов, вряд ли можно назвать непреодолимыми, однако попытки их решения в рамках данной работы значительно усложнили бы формальные рассуждения, не привнеся при этом ничего ценного для понимания центральных её идей.

Кроме того следует обратить внимание на то, что, поскольку условия теоремы ограничивают значение дисконтирующего коэффициента $\delta$ как снизу, так и сверху, её невозможно сформулировать в более изящной форме для сколь угодно малых $\epsilon$-приближений к вектору выплат $u_{M,n}(\mu)$. Однако мы можем отобразить её параметры на объекты реального мира, чтобы иметь возможность судить о её практических следствиях. Представим себе, что используемый игроками набор вычислительных устройств $M$ "--- это современные компьютеры, а $n = 256$, что совпадает с наиболее распространённой в современной криптографии длинной ключей. Для удачно подобранных\footnote{Здесь подразумевается, что никто из игроков не выбирает одну из своих чистых стратегий с вероятностью близкой к $1$.} смешанных равновесий $s_0$ можно ожидать, что средняя продолжительность стадии обмена ключами $\mathfrak{t}(s_0, n)$ по порядку величины будет совпадать со значением $n$, то есть измеряться сотнями итераций. Это значит, что в сериях розыгрышей с хотя бы ста тысячами значимых итераций вклад стадии синхронизации уже заведомо не будет превышать одного процента от итогового результата. Учитывая, что, к примеру, системы для автоматизированного трейдинга на биржах вполне могут совершать сотни тысяч транзакций в день, для многих практических применений это можно считать неплохим приближением.

Далее, стоимость вычисления функций, использующихся при создании общих секретных ключей, т.е. $M^a[f_{256}]$ и $M^a[h_{256}]$ хотя и нельзя считать пренебрежимо малой, но для её оценки можно заметить, что каждое открытие интернет-страницы современным браузером в подавляющем большинстве случаев подразумевает генерацию нескольких ключевых пар и соответствующих общих секретных ключей. Наконец, стоимость генерации псевдослучайных последовательностей современными CSPRNG, т.е. $M^a[G_{256}]$ на масштабе в миллионы битов уже можно считать пренебрежимо малой, так как, скажем, передача изображения с видеокарты на монитор современного компьютера по цифровому интерфейсу HDMI подразумевает шифрование потока, измеряющегося десятками гигабит в секунду, а алгоритм, использующийся в процессе этого шифрования, может использоваться и для генерации псевдослучайных последовательностей.

Наконец, остаётся заметить, что обращение односторонней функции с 256-битным аргументом считается на данный момент невозможным и предполагается, что оно останется таковым вплоть до создания функционирующих квантовых компьютеров. Это означает, что значение $M^a[f_{256}^{-1}]$ для практических применений можно считать почти бесконечным, что позволяет $\delta$ приближаться к $1$ на сколь угодно малое (в смысле реальных конфликтов) расстояние.

%\begin{proof}
%	Логика доказательства во многом повторяет рассуждения теоремы \ref{the:folk}. Игроки придерживаются предписанной последовательности стратегий, пока на итерации $i$ игрок $a$ не решит отклониться, получив конечную прибыль $u^a(s_i | \hat{s}^a_i) - u^a(s_i)$. В ответ на это остальные игроки переключаются на его наказание соответствующим минимаксным набором стратегий. Проблема заключается в том, что наказание игрока $a$ в пространстве заговоров $\mathfrak{A}$, может использовать недоступность ему тайн заговоров $\mathfrak{A}^{\setminus a} = \{A_* \in \mathfrak{A} \mid a \notin A_*\}$, тогда как в повторяющейся игре $[\Gamma, M]^{\infty}_{\delta}$ отдельные итерации разыгрываются без пространств корреляции вообще. Для решения этой проблемы алгоритм поведения игроков немного усложняется и проходит в три этапа, предваряясь дополнительной стадией синхронизации.
%	
%	В начале первой стадии каждый игрок $a$ выбирает случайное вещественное число $\chi^a$, равномерно распределённое в диапазоне $\left[0, 2^n\right)$. Округление этого числа $x^a = \lfloor\chi^a\rfloor \in \mathbb{N}_{<2^n}$ используется им далее в качестве закрытого ключа. При помощи односторонней функции игрок вычисляет открытый ключ $X^a = f(x^a) \in \mathbb{N}_{<2^n}$. Поскольку $f$ "--- биекция, случайная величина $\mathcal{X}^a = \chi^a - x^a + X^a$ также равномерно распределена в диапазоне $\left[0, 2^n\right)$. Для того, чтобы раскрыть другим игрокам своё значение $X^a$, игрок использует метод <<фрактального>> кодирования числа $\mathcal{X}^a$ в последовательности совершаемых им ходов, неотличимых по вероятностному распределению от стратегии участия в смешанном равновесии $s_0$, где ни один из игроков не играет чистой стратегии (см. требование существования такого равновесия для игры $\Gamma$ в условии теоремы). Пусть $(p^a_1, \ldots, p^a_{\lvert S^a \rvert})$ "--- вероятности выбора в этом равновесии соответствующих стратегий из $S^a = \{s^a_j, j = \overline{1, \lvert S^a \rvert}\}$. Начнём с построения в единичном интервале $\left[0, 1\right)$ сетки неубывающих чисел $\rho^a = (\rho^a_0 = 0, \rho^a_1, \ldots, \rho^a_{\lvert S^a \rvert - 1}, \rho^a_{\lvert S^a \rvert} = 1)$ разбивающей его таким образом, что $\rho^a_j - \rho^a_{j-1} = p^a_j, j = \overline{1,\lvert S^a \rvert}$. Это позволяет применить следующий алгоритм кодирования:
%	
%	\begin{enumerate}
%		\item До совершения первого хода другие игроки знают, что $\mathcal{X}^a$ равномерно распределено в интервале $\left[0, 2^n\right)$. Положим $\mathcal{X}^a_{min} = 0$ и $\mathcal{X}^a_{max} = 2^n$.
%		\item Отмасштабируем $\rho^a$ на интервал $\left[\mathcal{X}^a_{min}, \mathcal{X}^a_{max}\right)$ получив вложенную в него сетку $\mathcal{P}^a = (\mathcal{X}^a_{min} (1 - \rho^a_j) + \mathcal{X}^a_{max} \rho^a_j, j = \overline{0, \lvert S^a \rvert})$.
%		\item Выберем на очередной итерации $s^a_j$ такое, что $\mathcal{P}^a_{j-1} \le \mathcal{X}^a < \mathcal{P}^a_j$.
%		\item Теперь другие игроки знают, что $\mathcal{X}^a$ равномерно распределено в интервале $\left[\mathcal{P}^a_{j-1}, \mathcal{P}^a_j\right)$. Вернёмся к шагу 2, положив $\mathcal{X}^a_{min} = \mathcal{P}^a_{j-1}$ и $\mathcal{X}^a_{max} = \mathcal{P}^a_j$.
%	\end{enumerate}
%
%	Этот алгоритм может выполняться бесконечно, постепенно уменьшая меру незнания других игроков относительно величины $\mathcal{X}^a$. На каждой итерация вероятность выбора всех стратегий совпадает с соответствующими вероятностями из одного и того же смешанного равновесия по Нэшу игры $\Gamma$, причём решения из разных итераций попарно независимы, так что единственный способ улучшить свой выигрыш для любого игрока на этом этапе "--- экономить на вычислениях, используя вместо генерации пары ключей простую смешанную стратегию. Однако, такая экономия предотвращается угрозой наказания на следующих стадиях игры. В тот момент, когда начинает выполняться неравенство $X^a \le \mathcal{X}^a_{min} < \mathcal{X}^a_{max} \le X^a + 1$ другие игроки получают уверенность относительно значения $X^a$ и открытый ключ игрока $a$ можно считать опубликованным. Когда все игроки завершили процесс публикации открытых ключей, каждый игрок $a$ вычисляет общий секретный ключ $K^A = h(X^1, \ldots, x^a, \ldots, X^m)$, комбинируя свой закрытый ключ с открытыми ключами других игроков, на чём стадия синхронизации завершается.
%	
%	В начале второй стадии каждый игрок инициирует один CSPRNG зерном $K^A$. С использованием следующего детерминированного псевдослучайного алгоритма игроки вычисляют последовательность ходов, которая в игре $(\Gamma)^{\infty}_{\delta}$ сошлась бы к вектору выплат $v$. Пусть $v^a = p_1 u^a(s_1) + \ldots + p_k u^a(s_k), a = \overline{1,m}$, т.е. $v$ представляет собой линейную комбинацию выплат в наборах чистых стратегий $s_1, \ldots, s_k$. Построим в единичном интервале $\left[0, 1\right)$ сетку неубывающих чисел $\rho = (\rho_0 = 0, \rho_1, \ldots, \rho_{k - 1}, \rho_k = 1)$ разбивающую его аналогично предыдущей фазе таким образом, что $\rho_j - \rho_{j-1} = p_j, j = \overline{1,k}$.
%	
%	\begin{enumerate}
%		\item Положим $\mathcal{X}_{min} = 0$ и $\mathcal{X}_{max} = 1$.
%		\item Проверим, нет ли такого $j$, что $\rho_{j-1} \le \mathcal{X}_{min} < \mathcal{X}_{max} \le \rho_j$. Если есть, остановим алгоритм, приняв решение о выборе стратегического набора $s_j$.
%		\item Сгенерируем очередной бит псевдослучайной последовательности. Значение $\frac{\mathcal{X}_{min} + \mathcal{X}_{max}}{2}$ в случае $0$ присвоим переменной $\mathcal{X}_{max}$, а в случае $1$ "--- $\mathcal{X}_{min}$.
%		\item Перейдём к шагу 2.
%	\end{enumerate}
%	
%	Выбор набора стратегий в соответствии с этим алгоритмом детерминирован для всех, знающих секретный ключ $K^A$, и в силу свойств CSPRNG без знания зерна неотличим от случайного выбора в соответствии с необходимым распределением $(p_1, \ldots, p_k)$. Если один из игроков отклоняется от предписанной таким образом стратегии (из желания увеличить выигрыш или вследствие экономии на генерации ключей во время фазы синхронизации), остальные игроки немедленно это замечают и переходят к применению против него стратегии наказания.
%	
%	На этой стадии против наказываемого игрока $\check{a}$ играют минимаксный набор коррелированных стратегий в семействе заговоров $\mathfrak{A}^{\setminus \check{a}} = \{A_* \in \mathfrak{A} \mid \check{a} \notin A_*\}$ (поскольку тайны заговоров, которые наказываемый игрок знает, очевидно, не могут быть использованы для уменьшения его гарантированного выигрыша). Каждому заговору $A_* = \{a_1, \ldots, a_{\lvert A_* \rvert}\} \in \mathfrak{A}^{\setminus \check{a}}$ ставится в соответствие собственный секретный ключ $K^{A_*} = h(X_{a_1}, \ldots, x_{a_j}, \ldots, X_{a_{\lvert A_* \rvert}})$, который может быть скомбинирован каждым его членом из собственного закрытого ключа и открытых ключей остальных заговорщиков. Каждый игрок инициализирует соответствующими зёрнами по одному CSPRNG на каждый заговор из $\mathfrak{A}^{\setminus \check{a}}$, в который он сам входит, и применяет ту же процедуру псевдослучайного выбора стратегии с помощью детерминированного алгоритма. Из этого складывается набор стратегий, который с точки зрения наказываемого игрока $\check{a}$, не знающего секретных ключей, неотличим от минимаксного набора в пространстве заговоров $\mathfrak{A}^{\setminus \check{a}}$.
%	
%	Единственным свободным параметром в вышеописанной схеме является битовый размер ключа $n$. Для игрока $a$ от него зависят:
%	\begin{itemize}
%		\item стоимость вычислений при создании игроком пары закрытого и открытого ключей $M^a[f_n] \in o(2^n)$;
%		\item среднее количество раундов публикации ключа $\mathfrak{t}(n) \in \Theta(n)$;
%		\item стоимость комбинирования одной пары ключей $M^a[h_n] \in o(2^n)$;
%		\item стоимость генерации одного бита псевдослучайной последовательности $M^a[G_n] \in o(2^n)$;
%		\item стоимость обращения односторонней функции $M^a[f_n^{-1}] \in \Theta(2^n)$.
%	\end{itemize}
%
%	Как и ранее, под $o(2^n)$ здесь понимается множество функций полиномиальной сложности, а под $\Theta(n)$ и $\Theta(2^n)$ "--- множества линейно и экспоненциально сложных соответственно. Таким образом общий доход игрока $a$ при следовании предписанной стратегии можно записать так:
%	\begin{align*}
%		u^a_{SYNC} &= (1 - \delta^{\mathfrak{t}(n)}) u^a(s_0) - (1 - \delta) (M^a[f_n] + \delta^{\mathfrak{t}(n)} (m - 1) M^a[h_n]); \\
%		u^a_{PLAY} &= v^a - M^a[G_n]; \\
%		u^a_{TOTAL} &= u^a_{SYNC} + \delta^{\mathfrak{t}(n)} u^a_{PLAY}.
%	\end{align*}
%
%	Здесь $u^a_{SYNC}$ "--- доход от стадии синхронизации, а $u^a_{PLAY}$ "--- недисконтированный доход от <<рабочего хвоста>> розыгрышей. Рассмотрим возможные девиации, которые могли бы принести игроку дополнительный доход. На стадии синхронизации следование процедуре выбора очередной стратегии приводит к равновесию по Нэшу в смешанных стратегиях, так что из доступных способов оптимизации остаётся экономия на $\mathfrak{w}^a_{KG}(n)$ "--- можно, не создавая ключей, использовать простую смешанную стратегию $s_0^a$, что краткосрочно даёт тот же выигрыш. Это, однако, приводит к невозможности присоединиться к псевдослучайной коррелированной стратегии без траты $\delta^{\mathfrak{t}^a(n)} (1 - \delta) \mathfrak{w}^a_{IR}(n)$ на <<взлом>> общего секретного ключа. Игроку также может оказаться выгодно намеренное затягивание фазы синхронизации путём специального подбора $\mathcal{X}^a$ вблизи от целых чисел, если $u^a(s_0) > v^a - \mathfrak{w}^a_{SG}$, то есть если его доход от смешанного равновесия используемого в синхронизации превосходит целевой доход. Однако, такое отклонение может быть предотвращено при помощи следующей угрозы "--- в тот момент, когда все игроки кроме $a$ завершают публикацию своих ключей, они вычисляют секретный ключ $K^{A \setminus \{a\}}$ и используют CSPRNG с ним в качестве зерна для того, чтобы, до тех пор пока он не завершит публикацию, на каждой итерации с некоторой заданной вероятностью $p^a_{SU}$ назначить отстающего наказываемым и перейти к его вечному минимаксу.
%	
%	В хвосте розыгрышей игрок $a$ может получить дополнительный доход, не превосходящий константы $u^a_{GM}$, однократным отклонением от предписанного псевдослучайным алгоритмом выбора, что автоматически переводит игру на стадию его наказания. На этой стадии он может либо смириться со своим наказанием и получать до конца игры соответствующий минимакс, либо тратить $(1 - \delta) \mathfrak{w}^a_{IR}(n)$ на <<взлом>> каждого ключа, необходимого для предсказания играемого против него коррелированного набора. При этом, очевидно, для любого $\delta$ можно подобрать достаточно большое $n$, чтобы ограниченный константой выигрыш от предсказания ходов оппонентов был перекрыт стоимостью обращения односторонней функции. Сам же $\delta$ следует выбирать в соответствии с тем же принципом, что и в классической <<народной>> теореме "--- так, чтобы ущерб от вечного минимакса превысил верхнюю границу возможного дохода от однократного отклонения.
%	
%	Для доказательства теоремы достаточно показать, что с при $\delta$, стремящемся к $1$, выигрыш $u^a_{TOTAL}$ будет стремиться к $v^a - \mathfrak{w}^a_{SG}$. Заметим, что при увеличении $n$ на $1$, стоимость обращения односторонней функции $\mathfrak{w}^a_{IR}(n)$ увеличивается вдвое. Это значит, что при уменьшении $1 - \delta$ вдвое, продолжительность стадии синхронизации достаточно увеличить не более чем на константу, и следовательно $\delta^{\mathfrak{t}^a(n)}$ также стремится к $1$. При этом, очевидно, пренебрежимо малыми становится $(1 - \delta^{\mathfrak{t}^a(n)}) (u^a(s_0) - \mathfrak{w}^a_{KP})$ и 
%	
%	а значит асимптотически $u^a_{TOTAL}$ ведёт себя так же как $u^a_{PLAY}$. Остаётся указать на то, что $\mathfrak{w}^a_{KG}(n)$
%	
%	основные : с одной стороны увеличение веса <<хвоста>> позволяет окупаться взлому ключей большей длины, так что $n$ растёт, а вслед за ним растут и 
%	
%%	\begin{enumerate}
%%		\item Если в игре есть наказываемый игрок, перейти к шагу 4.
%%		\item Сравнить стратегии, сыгранные игроками на предыдущем раунде с $s_{i-1}$. Если какой-либо игрок отклонился от предписанной стратегии, назначить его наказываемым и перейти к шагу 4.
%%		\item Вычислить набор $s_i$, запомнить его и завершить ход сыграв стратегию $s_i^a$.
%%		\item Вычислить и сыграть стратегию наказания.
%%	\end{enumerate}
%%
%%	Прежде чем полноценно разворачивать шаг 4 остановимся заметить, что в случае, когда угроза, создаваемая стратегией наказания для каждого из игроков, достаточно велика, чтобы её ущерб превысил прибыль однократного уклонения от предписанной стратегии, фактически шаг 4 не исполняется никогда, так что под $\mathfrak{w}^a$ мы понимаем дисконтированную сумму стоимости вычислений на шагах 1, 2 и 3. Обозначим также символом $s_*$ равновесие по Нэшу игры $\Gamma$, не включающее чистых стратегий.
%%	
%%	\begin{enumerate}
%%		\item[4.] Вычислить и сыграть стратегию наказания:
%%		\begin{enumerate}
%%			\item Если это первый ход после первого отклонения, сгенерировать случайный закрытый ключ $x^a \in \{0,1\}^n$, вычислить при помощи односторонней функции открытый ключ $X^a = f(x^a) \in \{0,1\}^n$ и построить для открытого ключа $X^a$ кодирующую последовательность стратегий $\check{s}^a = \operatorname{enc}(s_*^a, X^a) \in S^a \times \ldots \times S^a$. Начать процедуру обмена ключами со счётчиком $j = 0$.
%%			\item Иначе, если процедура обмена ключами не закончена, запомнить стратегии, сыгранные другими игроками в предыдущем раунде, в качестве $\check{s}^1_{j-1}, \ldots, \check{s}^m_{j-1}$.
%%			\item Если процедура обмена ключами не закончена
%%		\end{enumerate}
%%	\end{enumerate}
%\end{proof}

%Заметим, что с ростом $n_{PK}^a$, растёт и минимальный выигрыш игрока, при котором последовательность розыгрышей $(s_i)$ образует совершенное подыгровое равновесие. То есть, чем дешевле игроку обходятся вычисления, тем сложнее остальным принудить его к принятию невыгодного исхода повторяющейся игры. Это создаёт картину, неожиданную для казалось бы элементарного конфликта с матрицей платежей размером $2 \times 2 \times 2$ "--- поскольку из известных моделей асимметричной криптографии наименьшую длину ключа при равной стойкости обеспечивает дискретное логарифмирование на эллиптической кривой, получается, что для реализации успешных стратегий приходится применять математический аппарат переднего края теории групп. Более того, доход игроков напрямую зависит от их способности производить сложные вычисления на пределе любых доступных возможностей.

%\begin{theorem}
%	Пусть $\Gamma$ "--- игра в нормальной форме, чувствительная к дополнительной информационной асимметрии. Пусть $\mathbf{s}$ "--- равновесие по Нэшу в коррелированных стратегиях игры $\Gamma | \Phi$, где $\Phi = \{A_*\}$ "--- невырожденное семейство заговоров из одного элемента. Пусть $s$ "--- такое равновесие по Нэшу в смешанных стратегиях, что вероятности реализации отдельных стратегий в $\mathbf{s}^a$ и $s^a$ попарно совпадают для каждого игрока $a$. Тогда в повторяющейся игре с коэффициентом дисконтирования $\delta$ найдётся совершенное подыгровое равновесие по Нэшу в смешанных стратегиях
%\end{theorem}

\section{Обобщение результатов, перспективы и гипотезы}\label{sec:ch3/sect6}

Для того, чтобы оценить значение этого феномена, следует отступить на пару шагов от конкретики описанной игры и попытаться охватить взглядом более широкую картину. Во-первых, благодаря обобщению народной теоремы становится очевидно, что сам по себе трёхсторонний чёт-нечет выступает в роли не более чем относительно произвольно сконструированного примера игры, чувствительной к дополнительной информационной асимметрии. Вышеописанные криптографические стратегии наказания не используют фактически никаких других свойств данного конфликта и нет причин думать, что они не могут быть обобщены на множество других игр, проявляющих то же свойство. К примеру, если мы возьмём в качестве отдельной итерации игру $\Gamma^3_n$ из второй главы этой работы, то окажется, что при её повторении можно аналогичным образом кодировать открытые ключи алфавитом, состоящим из $n$ символов по числу компьютеров в вычислительном центре, а потом использовать полученный общий секретный ключ в качестве зерна генератора, псевдослучайно выбирающего из опять же $n$ элементов.

Во-вторых, имеет смысл задаться вопросом о том, исчерпывает ли описанная схема криптографических стратегий наказания все возможности для пополнения множества совершенных подыгровых равновесий. Здравый смысл подсказывает, что нет хотя бы потому, что оба использованных здесь криптографических примитива (и протокол совместной выработки ключа, и криптографически стойкий генератор псевдослучайных чисел) имеют немало реализаций, опирающихся на самые разные математические формализмы, список которых год от года пополняется благодаря бурному развитию соответствующих областей знания. Более того, построение стратегии наказания из протокола DH с последующим использованием полученного ключа в качестве зерна CSPRNG само по себе достаточно произвольно "--- мы использовали инструменты, изначально создававшиеся в совершенно ином контексте для других целей, просто потому, что они уже есть и имеют доказанные свойства, удобные для решения нашей задачи.

Эти рассуждения позволяют обоснованно предположить, что и сам трёхсторонний чёт-нечет, и представленные криптографические стратегии наказания "--- всего лишь наиболее очевидные представители более широкого класса пока ещё не исследованных математических формализмов. Говоря максимально общим языком, рациональные агенты, участвуя в повторяющихся конфликтах, могут вырабатывать секретные коррелированные стратегии поведения, используя лишь специальным образом выбираемые публичные действия и наблюдая за аналогично действующим контрагентом. Секретность при этом обеспечивается за счёт того, что присоединение к корреляции требует от наблюдающей ту же последовательность публичных действий третьей стороны когнитивных усилий, превосходящих её возможности. Кроме того, обобщая связь между удельной стоимостью вычислений и требуемой длиной ключей, можно предположить, что чем большие когнитивные усилия способна приложить сторона, от которой заговорщики пытаются скрыть свою общую стратегию, тем сложнее этот процесс и тем меньший доход (за счёт растущего дисконтирования в хвосте розыгрышей) приносит такая секретность.

В самом деле, если представить себе обычных людей, играющих в чувствительную к дополнительной информационной асимметрии игру без применения специальных технических средств, мы вряд ли можем ожидать, что они будут производить в уме вычисления, необходимые для связки DH+CSPRNG. При этом, вполне вероятно, что повсеместно распространены примеры того, как люди могут добиваться необходимой тайной синхронизации неосознанно, воспринимая результат как самоочевидный, не требующий объяснений факт. В данном случае имеются в виду опытные картёжники, специализирующиеся на сложных интеллектуальных играх, таких как бридж или преферанс. В их среде считается неоспоримым, что помимо индивидуальных навыков на исход розыгрышей сильно влияет опыт именно совместной игры "--- пара игроков, по отдельности не хватающих звёзд с небес, могут оказаться грозными соперниками, если у них за плечами много партий за одним столом. Если верны вышеизложенные предположения о достаточно общем характере построенной нами модели, то феномен такой <<сыгранности>> может найти удовлетворительное объяснение в её рамках.

Следует заметить, что если мы захотим исследовать с этой точки зрения стратегии игроков в уже существующие салонные игры, на нашем пути могут встать затруднения, связанные с тем, что правила тех из них, что можно заподозрить в чувствительности к дополнительной информационной асимметрии, сложны даже без учёта этого их свойства. Скажем, бридж или преферанс останутся, очевидно, весьма нетривиальными, даже если играть в них отдельными партиями, каждый раз подбирая состав анонимных участников случайно из большого пула кандидатов (удалённо по сети, например). Для облегчения задачи будущих исследователей неплохо было бы сконструировать специальную карточную игру, в которой использование эффекта <<сыгранности>> было бы необходимым элементом любой успешной стратегии. Попытка создания такой новой игры под названием <<Тессеракт>> вынесена в Приложение~\cref{app:D} "--- будем надеяться, что в будущем она окажется полезна как в научных, так и в развлекательных целях.

Завершить данную работу хотелось бы формулировкой достаточно смелой неформальной гипотезы, продолжающей линию рассуждений последней главы в русле популяционных игр:
\begin{conjecture}
	Если в популяции периодически (много раз на протяжении жизни одной особи) возникают конфликтные ситуации, модель которых чувствительна к дополнительной информационной асимметрии, и вероятность продолжения рода отдельными особями существенно зависит от их успеха в этих конфликтах, то давление отбора закрепляет в популяции признаки, способствующие увеличению когнитивного потенциала следующих поколений (понимаемого в общем смысле, как способность производить Тьюринг-полные вычисления над произвольными данными).
\end{conjecture}

Если эта гипотеза верна, то чувствительность игр к дополнительной информационной асимметрии может оказаться <<Святым Граалем>> эволюционной теории игр "--- фактором, порождающим неограниченную гонку вооружений в сфере способностей к сложному поведению. Любые игры с этим свойством, даже будучи очень просты сами по себе, поощряют когнитивный потенциал участников необходимостью строить и раскрывать заговоры, так что его изучение может как обогатить наше понимание эволюции интеллекта наших предков, так и стать инструментом совершенствования технологий искусственного разума.

\clearpage