\chapter{Вычислительная сложность стратегий в повторяющихся играх с дисконтированием}\label{ch:ch3}

\section{Повторяющийся трёхсторонний чёт-нечет с памятью}\label{sec:ch3/sect1}

Чувствительность к дополнительной информационной асимметрии интересна не только тем, как на исходы проявляющих её игр влияет собственно информационная асимметрия. Далее будет показано, что в тех случаях, когда этим свойством обладают отдельные розыгрыши повторяющихся игр, необычные эффекты могут возникать даже при отсутствии приватных механизмов корреляции, в рамках классического смешанного расширения и совершенных подыгровых равновесий. Для демонстрации феномена, однако, нам потребуется использовать модель рациональных агентов с памятью, учитывающую стоимость вычислений, необходимых для реализации очередного шага стратегии. Построим такую модель для повторяющейся игры с произвольной нормальной формой $\Gamma = \langle A, S^a, u^a(s), a \in A \rangle$ отдельного розыгрыша.
\begin{figure}[ht]
	\centerfloat{
		\ifdefmacro{\tikzsetnextfilename}{\tikzsetnextfilename{tikz_repeat_compiled}}{}
		\begin{tikzpicture}[scale=2]
			\node[circle,draw] (game 1) at (1, 0) {$\Gamma$};
			\node[circle,draw] (game 2) at (3, 0) {$\Gamma$};
			\node[circle,draw] (game 3) at (5, 0) {$\Gamma$};
			\node[rectangle,draw] (calc 1.1) at (0,  1) {$M^1$};
			\node[rectangle,draw] (calc 1.2) at (2,  1) {$M^1$};
			\node[rectangle,draw] (calc 1.3) at (4,  1) {$M^1$};
			\node[rectangle,draw] (calc 2.1) at (0, -1) {$M^2$};
			\node[rectangle,draw] (calc 2.2) at (2, -1) {$M^2$};
			\node[rectangle,draw] (calc 2.3) at (4, -1) {$M^2$};
		\end{tikzpicture}
	}
	\legend{}
	\caption[Повторяющаяся игра с учётом стоимости вычислений]{Повторяющаяся игра с учётом стоимости вычислений}\label{fig:repeat}
\end{figure}


%Тогда стратегию игрока $a$ можно представить в следующем виде:
%\begin{equation*}
%	\overbrace{s}^a = \langle \psi^a \in \Psi^a, f^a : \Psi^a \rightarrow \mathcal{M}(X^a), g^a :  / \rangle
%\end{equation*}

\clearpage
