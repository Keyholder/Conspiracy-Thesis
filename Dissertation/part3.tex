\chapter{Вычислительная сложность стратегий в повторяющихся играх с дисконтированием}\label{ch:ch3}

\section{Повторяющийся трёхсторонний чёт-нечет с памятью}\label{sec:ch3/sect1}

Чувствительность к дополнительной информационной асимметрии интересна не только тем, как на исходы проявляющих её игр влияет собственно информационная асимметрия. Далее будет показано, что в тех случаях, когда этим свойством обладают отдельные розыгрыши повторяющихся игр, необычные эффекты могут возникать даже при отсутствии приватных механизмов корреляции, в рамках классического смешанного расширения и совершенных подыгровых равновесий. Для демонстрации феномена, однако, нам потребуется использовать модель рациональных агентов с памятью, учитывающую стоимость вычислений, необходимых для реализации очередного шага стратегии. Построим такую модель для повторяющейся игры с произвольной нормальной формой $\Gamma = \langle A, S^a, u^a(s), a \in A \rangle$ отдельного розыгрыша.
\begin{figure}[ht]
	\centerfloat{
		\ifdefmacro{\tikzsetnextfilename}{\tikzsetnextfilename{tikz_repeat_compiled}}{}
		\begin{tikzpicture}[scale=2]
			\node[circle,draw] (game1) at (1, 0) {$\Gamma$};
			\node[circle,draw] (game2) at (3, 0) {$\Gamma$};
			\node[circle,draw] (game3) at (5, 0) {$\Gamma$};
			\node[rectangle,draw] (calc11) at (0,  1) {$M^1$};
			\node[rectangle,draw] (calc12) at (2,  1) {$M^1$};
			\node[rectangle,draw] (calc13) at (4,  1) {$M^1$};
			\node[] (calc1i) at (6, 1) {$\cdots$};
			\node[rectangle,draw] (calc21) at (0, -1) {$M^2$};
			\node[rectangle,draw] (calc22) at (2, -1) {$M^2$};
			\node[rectangle,draw] (calc23) at (4, -1) {$M^2$};
			\node[] (calc2i) at (6, -1) {$\cdots$};
			\draw [->,thick,decorate,decoration={snake,post length=1mm}] (calc11) -- (game1);
			\draw [->,thick,decorate,decoration={snake,post length=1mm}] (calc21) -- (game1);
			\draw [->,thick,decorate,decoration={snake,post length=1mm}] (calc12) -- (game2);
			\draw [->,thick,decorate,decoration={snake,post length=1mm}] (calc22) -- (game2);
			\draw [->,thick,decorate,decoration={snake,post length=1mm}] (calc13) -- (game3);
			\draw [->,thick,decorate,decoration={snake,post length=1mm}] (calc23) -- (game3);
			\draw [->,thick,decorate,decoration={snake,post length=1mm}] (game1) -- (calc12);
			\draw [->,thick,decorate,decoration={snake,post length=1mm}] (game1) -- (calc22);
			\draw [->,thick,decorate,decoration={snake,post length=1mm}] (game2) -- (calc13);
			\draw [->,thick,decorate,decoration={snake,post length=1mm}] (game2) -- (calc23);
			\draw [->,thick,decorate,decoration={snake,post length=1mm}] (game3) -- (calc1i);
			\draw [->,thick,decorate,decoration={snake,post length=1mm}] (game3) -- (calc2i);
			\node[diamond,draw] (sum1) at (-1, 2) {$\Sigma^1$};
			\node[] (sum10) at (0,  2) {};
			\node[] (sum11) at (1,  2) {};
			\node[] (sum12) at (2,  2) {};
			\node[] (sum13) at (3,  2) {};
			\node[] (sum14) at (4,  2) {};
			\node[] (sum15) at (5,  2) {};
			\node[] (sum1i) at (6,  2) {$\cdots$};
			\node[diamond,draw] (sum2) at (-1, -2) {$\Sigma^2$};
			\node[] (sum20) at (0, -2) {};
			\node[] (sum21) at (1, -2) {};
			\node[] (sum22) at (2, -2) {};
			\node[] (sum23) at (3, -2) {};
			\node[] (sum24) at (4, -2) {};
			\node[] (sum25) at (5, -2) {};
			\node[] (sum2i) at (6, -2) {$\cdots$};
			\draw [->,thick] (sum1i) -- (sum1);
			\draw [->,thick] (calc11) -- (sum10);
			\draw [->,thick] (game1) -- (sum11);
			\draw [->,thick] (calc12) -- (sum12);
			\draw [->,thick] (game2) -- (sum13);
			\draw [->,thick] (calc13) -- (sum14);
			\draw [->,thick] (game3) -- (sum15);
			\draw [->,thick] (sum2i) -- (sum2);
			\draw [->,thick] (calc21) -- (sum20);
			\draw [->,thick] (game1) -- (sum21);
			\draw [->,thick] (calc22) -- (sum22);
			\draw [->,thick] (game2) -- (sum23);
			\draw [->,thick] (calc23) -- (sum24);
			\draw [->,thick] (game3) -- (sum25);
			\fill [white] (1,  1) circle (4pt);
			\fill [white] (3,  1) circle (4pt);
			\fill [white] (5,  1) circle (4pt);
			\fill [white] (1, -1) circle (4pt);
			\fill [white] (3, -1) circle (4pt);
			\fill [white] (5, -1) circle (4pt);
			\draw [->,double,thick] (-1,  1) -- (calc11);
			\draw [->,double,thick] (calc11) -- (calc12);
			\draw [->,double,thick] (calc12) -- (calc13);
			\draw [->,double,thick] (calc13) -- (calc1i);
			\draw [->,double,thick] (-1, -1) -- (calc21);
			\draw [->,double,thick] (calc21) -- (calc22);
			\draw [->,double,thick] (calc22) -- (calc23);
			\draw [->,double,thick] (calc23) -- (calc2i);
		\end{tikzpicture}
	}
	\legend{}
	\caption[Повторяющаяся игра с учётом стоимости вычислений]{Повторяющаяся игра с учётом стоимости вычислений}\label{fig:repeat}
\end{figure}


%Тогда стратегию игрока $a$ можно представить в следующем виде:
%\begin{equation*}
%	\overbrace{s}^a = \langle \psi^a \in \Psi^a, f^a : \Psi^a \rightarrow \mathcal{M}(X^a), g^a :  / \rangle
%\end{equation*}

\clearpage
