\chapter{Вычислительная сложность стратегий в повторяющихся играх с дисконтированием}\label{ch:ch3}

\section{Повторяющийся трёхсторонний чёт-нечет}\label{sec:ch3/sect1}

Чувствительность к дополнительной информационной асимметрии интересна не только тем, как на исходы проявляющих её игр влияет собственно информационная асимметрия. Далее будет показано, что в тех случаях, когда этим свойством обладают отдельные розыгрыши повторяющихся игр, необычные эффекты могут возникать даже при отсутствии приватных механизмов корреляции, в рамках классического смешанного расширения и совершенных подыгровых равновесий. Разберём этот феномен на примере игры в <<трёхсторонний чёт-нечет>>, описанной в разделе \ref{sec:ch1/sec4}. Сперва проанализируем повторяющуюся версию данного конфликта с классической точки зрения, применяя <<народную>> теорему.

\begin{table} [htbp]
	\centering
	\begin{threeparttable}
		\caption{Трёхсторонний чёт"~нечет}
		\label{tab:coin3b}
		\begin{tabular}{ |c|c|c|c|c| }
			\cline{1-2} \cline{4-5}
			\rule[-7pt]{0pt}{2em}$4, 4, 4$ &
			\rule[-7pt]{0pt}{2em}$6, 0, 6$ & \qquad\qquad\qquad &
			\rule[-7pt]{0pt}{2em}$6, 6, 0$ &
			\rule[-7pt]{0pt}{2em}$0, 6, 6$ \\
			\cline{1-2} \cline{4-5}
			\rule[-7pt]{0pt}{2em}$0, 6, 6$ &
			\rule[-7pt]{0pt}{2em}$6, 6, 0$ & \qquad\qquad\qquad &
			\rule[-7pt]{0pt}{2em}$6, 0, 6$ &
			\rule[-7pt]{0pt}{2em}$4, 4, 4$ \\
			\cline{1-2} \cline{4-5}
		\end{tabular}
	\end{threeparttable}
\end{table}

В соответствии с формулировкой теоремы для получения множества равновесий по Нэшу в повторяющейся игре необходимо построить минимаксные стратегии наказания в отдельной её итерации. Однако трёхсторонний чёт-нечет любопытен тем, что в смешанных стратегиях никакая попытка двух игроков минимизировать выигрыш третьего не может опустить его результат ниже того, что он получает в точках равновесия по Нэшу (т.е. менее $4$). Действительно, пусть первый игрок избрал стратегию $(p, 1 - p)$, а второй "--- $(q, 1 - q)$. Тогда выигрыши третьего при выборе чистых стратегий составляют:
\begin{align*}
	u^3(p, q, 1) &= 6 (p + q) - 8 p q; \\
	u^3(p, q, 0) &= 2 (p + q) - 8 p q + 4.
\end{align*}

Несложно заметить, что
\begin{align*}
	p \ge &q \ge \frac{1}{2} \Rightarrow u^3(p, q, 1) \ge 4; \\
	p \ge 1 - &q \ge \frac{1}{2} \Rightarrow u^3(p, q, 1) \ge 4; \\
	q \ge &p \ge \frac{1}{2} \Rightarrow u^3(p, q, 1) \ge 4; \\
	q \ge 1 - &p \ge \frac{1}{2} \Rightarrow u^3(p, q, 1) \ge 4; \\
	p \le &q \le \frac{1}{2} \Rightarrow u^3(p, q, 0) \ge 4; \\
	p \le 1 - &q \le \frac{1}{2} \Rightarrow u^3(p, q, 0) \ge 4; \\
	q \le &p \le \frac{1}{2} \Rightarrow u^3(p, q, 0) \ge 4; \\
	q \le 1 - &p \le \frac{1}{2} \Rightarrow u^3(p, q, 0) \ge 4. \\
\end{align*}

Эти варианты исчерпывают всё пространство возможных ситуаций, так что никакое сочетание смешанных стратегий двух игроков не может быть эффективным наказанием для третьего. В классическом случае это означало бы, что повторяющаяся игра имеет в качестве решения то же самое множество равновесий по Нэшу, что и отдельный розыгрыш. Однако, вывод этот можно интересным образом уточнить, перейдя к модели, учитывающей стоимость вычислений, необходимых для выбора стратегии в очередной итерации.

\section{Модель повторяющихся игр с учётом стоимости вычислений}\label{sec:ch3/sect2}

Построение искомой модели подразумевает конкретизацию способа, при помощи которого рациональные агенты вычисляют стратегию поведения. Для этой цели подойдёт любой формализм, позволяющий алгоритмически полные вычисления с вероятностным ветвлением. Кроме того, необходимы возможность сохранения произвольного внутреннего состояния в памяти для использования на последующих итерациях и, очевидно, метод численной оценки сложности произведённого на каждой итерации вычисления. Общая схема взаимодействия агентов и среды может выглядеть так:
\begin{figure}[ht]
	\centerfloat{
		\ifdefmacro{\tikzsetnextfilename}{\tikzsetnextfilename{tikz_repeat_compiled}}{}
		\begin{tikzpicture}[scale=2]
			\node[circle,draw] (game1) at (1, 0) {$\Gamma$};
			\node[circle,draw] (game2) at (3, 0) {$\Gamma$};
			\node[circle,draw] (game3) at (5, 0) {$\Gamma$};
			\node[rectangle,draw] (calc11) at (0,  1) {$M^1$};
			\node[rectangle,draw] (calc12) at (2,  1) {$M^1$};
			\node[rectangle,draw] (calc13) at (4,  1) {$M^1$};
			\node[] (calc1i) at (6, 1) {$\cdots$};
			\node[rectangle,draw] (calc21) at (0, -1) {$M^2$};
			\node[rectangle,draw] (calc22) at (2, -1) {$M^2$};
			\node[rectangle,draw] (calc23) at (4, -1) {$M^2$};
			\node[] (calc2i) at (6, -1) {$\cdots$};
			\draw [->,thick,decorate,decoration={snake,post length=1mm}] (calc11) -- (game1) node[midway,sloped,below] {$s^1_1$};
			\draw [->,thick,decorate,decoration={snake,post length=1mm}] (calc21) -- (game1) node[midway,sloped,above] {$s^2_1$};
			\draw [->,thick,decorate,decoration={snake,post length=1mm}] (calc12) -- (game2) node[midway,sloped,below] {$s^1_2$};
			\draw [->,thick,decorate,decoration={snake,post length=1mm}] (calc22) -- (game2) node[midway,sloped,above] {$s^2_2$};
			\draw [->,thick,decorate,decoration={snake,post length=1mm}] (calc13) -- (game3) node[midway,sloped,below] {$s^1_3$};
			\draw [->,thick,decorate,decoration={snake,post length=1mm}] (calc23) -- (game3) node[midway,sloped,above] {$s^2_3$};
			\draw [->,thick,decorate,decoration={snake,post length=1mm}] (game1) -- (calc12) node[midway,sloped,below] {$s_1$};
			\draw [->,thick,decorate,decoration={snake,post length=1mm}] (game1) -- (calc22) node[midway,sloped,above] {$s_1$};
			\draw [->,thick,decorate,decoration={snake,post length=1mm}] (game2) -- (calc13) node[midway,sloped,below] {$s_2$};
			\draw [->,thick,decorate,decoration={snake,post length=1mm}] (game2) -- (calc23) node[midway,sloped,above] {$s_2$};
			\draw [->,thick,decorate,decoration={snake,post length=1mm}] (game3) -- (calc1i) node[midway,sloped,below] {$s_3$};
			\draw [->,thick,decorate,decoration={snake,post length=1mm}] (game3) -- (calc2i) node[midway,sloped,above] {$s_3$};
			\node[diamond,draw] (sum1) at (-1, 2) {$\Sigma^1$};
			\node[] (sum10) at (0,  2) [label=$-\delta \mathfrak{w}^1_1$] {};
			\node[] (sum11) at (1,  2) [label=$+\delta u^1(s_1)$] {};
			\node[] (sum12) at (2,  2) [label=$-\delta^2 \mathfrak{w}^1_2$] {};
			\node[] (sum13) at (3,  2) [label=$+\delta^2 u^1(s_2)$] {};
			\node[] (sum14) at (4,  2) [label=$-\delta^3 \mathfrak{w}^1_3$] {};
			\node[] (sum15) at (5,  2) [label=$+\delta^3 u^1(s_3)$] {};
			\node[] (sum1i) at (6,  2) {$\cdots$};
			\node[diamond,draw] (sum2) at (-1, -2) {$\Sigma^2$};
			\node[] (sum20) at (0, -2) [label=below:$-\delta \mathfrak{w}^2_1$] {};
			\node[] (sum21) at (1, -2) [label=below:$+\delta u^2(s_1)$] {};
			\node[] (sum22) at (2, -2) [label=below:$-\delta^2 \mathfrak{w}^2_2$] {};
			\node[] (sum23) at (3, -2) [label=below:$+\delta^2 u^2(s_2)$] {};
			\node[] (sum24) at (4, -2) [label=below:$-\delta^3 \mathfrak{w}^2_3$] {};
			\node[] (sum25) at (5, -2) [label=below:$+\delta^3 u^2(s_3)$] {};
			\node[] (sum2i) at (6, -2) {$\cdots$};
			\draw [->,thick] (sum1i) -- (sum1);
			\draw [->,thick] (calc11) -- (sum10);
			\draw [->,thick] (game1) -- (sum11);
			\draw [->,thick] (calc12) -- (sum12);
			\draw [->,thick] (game2) -- (sum13);
			\draw [->,thick] (calc13) -- (sum14);
			\draw [->,thick] (game3) -- (sum15);
			\draw [->,thick] (sum2i) -- (sum2);
			\draw [->,thick] (calc21) -- (sum20);
			\draw [->,thick] (game1) -- (sum21);
			\draw [->,thick] (calc22) -- (sum22);
			\draw [->,thick] (game2) -- (sum23);
			\draw [->,thick] (calc23) -- (sum24);
			\draw [->,thick] (game3) -- (sum25);
			\fill [white] (1,  1) circle (2pt);
			\fill [white] (3,  1) circle (2pt);
			\fill [white] (5,  1) circle (2pt);
			\fill [white] (1, -1) circle (2pt);
			\fill [white] (3, -1) circle (2pt);
			\fill [white] (5, -1) circle (2pt);
			\draw [->,double,thick] (-1,  1) -- (calc11) node[midway,above] {$\psi^1_0$};
			\draw [->,double,thick] (calc11) -- (calc12) node[near end,above] {$\psi^1_1$};
			\draw [->,double,thick] (calc12) -- (calc13) node[near end,above] {$\psi^1_2$};
			\draw [->,double,thick] (calc13) -- (calc1i) node[near end,above] {$\psi^1_3$};
			\draw [->,double,thick] (-1, -1) -- (calc21) node[midway,below] {$\psi^2_0$};
			\draw [->,double,thick] (calc21) -- (calc22) node[near end,below] {$\psi^2_1$};
			\draw [->,double,thick] (calc22) -- (calc23) node[near end,below] {$\psi^2_2$};
			\draw [->,double,thick] (calc23) -- (calc2i) node[near end,below] {$\psi^2_3$};
		\end{tikzpicture}
	}
	\legend{}
	\caption[Повторяющаяся игра с учётом стоимости вычислений]{Повторяющаяся игра с учётом стоимости вычислений}\label{fig:repeat}
\end{figure}

Диаграмма на рисунке \ref{fig:repeat} схематично изображает моделируемый процесс для двух игроков (естественным образом обобщающийся на любое конечное их число). В узлах, помеченных буквой $\Gamma$, происходят последовательные розыгрыши произвольной игры $\Gamma = \langle A, S^a, u^a(s), a \in A \rangle$. В $i$-м розыгрыше игрок $a$ выбирает свою стратегию $s^a_i$, применяя вероятностный алгоритм $M^a$ к результату предыдущей итерации $(\psi^a_{i-1}, s_{i-1})$, включающему сохранённое состояние памяти самого алгоритма и набор сыгранных на итерации $i-1$ стратегий. Узлы $\Sigma^a$ изображают последовательное суммирование разностей выигрыша $u^a(s_i)$ и затрат на произведённое вычисление $\mathfrak{w}^a_i$, с учётом экспоненциально уменьшающегося коэффициента дисконтирования $\delta^i$. При такой схеме взаимодействий имеет смысл рассматривать только универсальные алгоритмы $M^a$, что позволяет закодировать любой набор вычислимых стратегий повторяющейся игры в начальных состояниях памяти $\psi^a_0$.

\section{Криптографические стратегии наказания}

Хотя трёхсторонний чёт-нечет не даёт возможности двум игрокам наказывать третьего, используя только лишь смешанные стратегии, ещё в первой главе мы установили, что, например, наличие механизма корреляции, которым игроки 1 и 2 могут пользоваться втайне от игрока 3, добавляет решения с выплатами до $(5, 5, 2)$. Таким образом, если бы мы рассматривали повторяющуюся игру в пространстве заговоров структуры $\{\{1, 2\}, \{2, 3\}, \{3, 1\}\}$, то в соответствии с <<народной>> теоремой её множество совершенных подыгровых равновесий включало бы все стратегические наборы с выплатами, обеспечивающие каждому игроку выигрыш не менее $2$. Однако, даже в том случае, когда игроки не могут использовать тайные механизмы корреляции, учёт стоимости вычислений позволяет в повторяющихся играх применять стратегии наказания, опирающиеся на достижения современной криптографии. Для демонстрации этого нам понадобятся два распространённых криптографических примитива.

Во-первых, необходим \emph{протокол совместной выработки ключа} \cite{Boyd}. В криптографии этим термином называют механизм, при помощи которого Алиса и Боб могут создать общую секретную последовательность битов, априорно обладая лишь публичным знанием об устройстве и параметрах самого механизма, и обмениваясь сообщениями через незащищённый от прослушивания канал связи. При этом Кэрол, обладая тем же априорным знанием и имея возможность читать их сообщения, не может вычислить искомую секретную последовательность битов, поскольку это требует решения алгоритмически трудной задачи (такой, для которой необходимо количество операций, зависящее от длинны ключа экспоненциально). В качестве такого механизма может выступать, например, семейство протоколов Диффи-Хеллмана (далее DH) на основе задач факторизации целых чисел или дискретного логарифмирования (в конечной мультипликативной группе или на эллиптической кривой). Опишем общую схему произвольного протокола DH, не вдаваясь в технические детали.

Пусть имеется биекция $f: \mathbb{N} \rightarrow \mathbb{N}$, обладающая свойством односторонности, т.е. вычислимая за полиномиальное от битовой длины входа число операций при том, что обратная к ней $f^{-1}$ сложна уже экспоненциально. Кроме этого, пусть имеется двухместная функция $h : \mathbb{N} \times \mathbb{N} \rightarrow \mathbb{N}$ такая, что $\forall x, y \in \mathbb{N}, h(f(x), y) = h(x, f(y))$. Алиса выбирает случайное число $x$, выполняющее роль её закрытого ключа, и за приемлемое время вычисляет $X = f(x)$, выполняющее роль её открытого ключа. На другом конце Боб аналогичным образом генерирует пару ключей $y$ и $Y = f(y)$. Алиса и Боб обмениваются открытыми ключами через прослушиваемый Кэрол канал связи. Теперь Алиса, зная свой закрытый ключ $x$ и открытый ключ Боба $Y$, может вычислить $h(x, Y)$, а Боб, соответственно, $h(X, y)$. В силу вышеупомянутого свойства функции $h$, вычисленные ими значения можно считать искомым общим секретным ключом $K = h(x, Y) = h(X, y)$. При этом для Кэрол, знающей только открытые ключи $X$ и $Y$, вычисление общего секретного ключа требует вычисления либо $f^{-1}(X)$, либо $f^{-1}(Y)$, что при достаточной битовой длине закрытых ключей оказывается непозволительно дорогим процессом.

Вторым криптографическим примитивом, необходимым для стратегии наказания, является \emph{криптографически стойкий генератор псевдослучайных чисел} \cite{Gutmann}, далее называемый CSPRNG. Его можно представить в виде предиката $G : \mathbb{N} \times \mathbb{N} \rightarrow \{0, 1\}$, первый аргумент которого называется зерном (или seed), а второй "--- позицией. Программа, вычисляющая для заданного $K \in \mathbb{N}$ последовательные значения $G(K, i), i = 1, 2, \ldots$, должна совершать каждый шаг за полиномиальное от битовой длинны $K$ число операций. При этом генератор обязан проходить тест на следующий бит, т.е. не должно существовать полиномиального по сложности алгоритма, способного без знания $K$ по первым $n$ битам генерируемой последовательности угадать $G(K, n+1)$ с вероятностью, отличной от $\frac{1}{2}$.

Теперь из этих примитивов соберём следующий алгоритм поведения для первого игрока в повторяющемся трёхстороннем чёт-нечете, параметризующийся размером ключей $n_{PK}$\footnote{Для простоты будем считать, что использующаяся в протоколе DH односторонняя функция $f$ имеет домен и кодомен одной битовой длины, и что при случайном выборе закрытого ключа все биты открытого оказываются распределены равномерно и независимо. Это верно, например, при использовании дискретного логарифмирования на эллиптической кривой.}:
\begin{enumerate}
	\item Создать случайную битовую последовательность $x = (x_1, \ldots, x_{n_{PK}})$.
	\item Вычислить битовую последовательность $X = f(x)$.
	\item Для каждого $i = 1 \ldots n_{PK}$ совершить один ход игры, выбирая решку при $X_i = 1$ и орла в противном случае. Выбранную вторым игроком стратегию (с тем же сопоставлением) запомнить в качестве элемента $Y_i$ последовательности $Y = (Y_1, \ldots, Y_{n_{PK}})$.
	\item Вычислить $K = h(x, Y)$.
	\item Все последующие ходы совершать, выбирая стратегию в соответствии с последовательно генерируемыми CSPRNG значениями $G(K, i), i = 1, 2, \ldots$.
\end{enumerate}

Стратегию второго игрока строим аналогичным образом, меняя местами $x$ с $y$, $X$ с $Y$ и $h(x, Y)$ с $h(X, y)$. Таким образом игроки 1 и 2 используют первые $n_{PK}$ ходов игры в качестве своеобразного <<танца синхронизации>>, вырабатывая общее секретное зерно для генератора псевдослучайных битов, чей вывод на последующих ходах фактически используется в качестве тайного механизма корреляции. В зависимости от коэффициента дисконтирования $\delta$, размера ключей $n_{PK}$ и удельной стоимости вычислений наилучшей ответом для третьего игрока оказывается одна из двух стратегий: взлом или игнорирование. Стратегия взлома подразумевает, как это следует из названия, вычисление $f^{-1}(X)$ или $f^{-1}(Y)$ и присоединение к использованию CSPRNG в качестве механизма корреляции. Выигрыш третьего игрока в этом случае составляет $4 \frac{\delta}{1-\delta} - \delta^{n_{PK} + 1} \mathfrak{w}_{f^{-1}}(n_{PK})$, где первое слагаемое представляет собой суммарный выигрыш в классическом равновесии Нэша с учётом коэффициента дисконтирования, а второе "--- стоимость вычисления функции $f^{-1}$ для ключа размером $n_{PK}$ бит на $n + 1$ итерации игры. Стратегия игнорирования также вполне соответствует своему названию и подразумевает использование любой смешанной стратегии без учёта ходов оппонентов. Поскольку первые $n_{PK}$ ходов игроки 1 и 2 тратят на синхронизацию, то выигрыш третьего на протяжении этих розыгрышей совпадает с выигрышем в классическом равновесии Нэша, но затем его доход ополовинивается так, как если бы игра происходила в пространстве заговоров структуры $\{\{1, 2\}\}$. Таким образом общий выигрыш для стратегии игнорирования составляет $4 \frac{\delta}{1-\delta} - 2 \delta^{n_{PK}} \frac{\delta}{1-\delta}$.

Сравнивая выигрыши в имеющихся у третьего игрока альтернативах можно прийти к выводу, что стратегия взлома оказывается выгодна только в случае $\mathfrak{w}_{f^{-1}}(n_{PK}) < \frac{2}{1 - \delta}$. Это значит, что с одной стороны, чем больше коэффициент дисконтирования $\delta$, тем выгоднее стратегия взлома, обменивающая стоимость однократного обращения односторонней функции $f$ на сохранение полного дохода от всего <<хвоста>> розыгрышей. С другой же стороны, для любого $\delta$ игроки 1 и 2 могут подобрать достаточно большой размер ключа $n_{PK}$, вынуждая третьего удовлетвориться стратегией игнорирования. Следует заметить, что на практике криптосистемы, использующие дискретное логарифмирование на эллиптических кривых, считаются надёжными уже при 256-битных ключах (Curve25519 \cite{Bernstein}, например), т.е. стоимость их взлома заведомо превосходит возможности, доступные человеческой цивилизации на текущем этапе технологического развития. Возможно с появлением прикладного квантового криптоанализа эта ситуация может измениться, однако пока можно считать, что в повторяющемся трёхстороннем чёт-нечете для эффективной криптографической стратегии наказания достаточно всего нескольких десятков итераций, потраченных на <<танец синхронизации>>. 

%Тогда стратегию игрока $a$ можно представить в следующем виде:
%\begin{equation*}
%	\overbrace{s}^a = \langle \psi^a \in \Psi^a, f^a : \Psi^a \rightarrow \mathcal{M}(X^a), g^a :  / \rangle
%\end{equation*}

\clearpage
