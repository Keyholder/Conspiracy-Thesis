\chapter{Вычислительная сложность стратегий в повторяющихся играх с дисконтированием}\label{ch:ch3}

\section{Повторяющийся трёхсторонний чёт-нечет}\label{sec:ch3/sect1}

Чувствительность к дополнительной информационной асимметрии интересна не только тем, как на исходы проявляющих её игр влияет собственно информационная асимметрия. Далее будет показано, что в тех случаях, когда этим свойством обладают отдельные розыгрыши повторяющихся игр, необычные эффекты могут возникать даже при отсутствии приватных механизмов корреляции, в рамках классического смешанного расширения и совершенных подыгровых равновесий. Разберём этот феномен на примере игры в <<трёхсторонний чёт-нечет>>, описанной в разделе \ref{sec:ch1/sec4}. Сперва проанализируем повторяющуюся версию данного конфликта с классической точки зрения, применяя <<народную>> теорему.

\begin{table} [htbp]
	\centering
	\begin{threeparttable}
		\caption{Трёхсторонний чёт"~нечет}
		\label{tab:coin3b}
		\begin{tabular}{ |c|c|c|c|c| }
			\cline{1-2} \cline{4-5}
			\rule[-7pt]{0pt}{2em}$4, 4, 4$ &
			\rule[-7pt]{0pt}{2em}$6, 0, 6$ & \qquad\qquad\qquad &
			\rule[-7pt]{0pt}{2em}$6, 6, 0$ &
			\rule[-7pt]{0pt}{2em}$0, 6, 6$ \\
			\cline{1-2} \cline{4-5}
			\rule[-7pt]{0pt}{2em}$0, 6, 6$ &
			\rule[-7pt]{0pt}{2em}$6, 6, 0$ & \qquad\qquad\qquad &
			\rule[-7pt]{0pt}{2em}$6, 0, 6$ &
			\rule[-7pt]{0pt}{2em}$4, 4, 4$ \\
			\cline{1-2} \cline{4-5}
		\end{tabular}
	\end{threeparttable}
\end{table}

В соответствии с формулировкой теоремы для получения множества равновесий по Нэшу в повторяющейся игре необходимо построить минимаксные стратегии наказания в отдельной её итерации. Однако трёхсторонний чёт-нечет любопытен тем, что в смешанных стратегиях никакая попытка двух игроков минимизировать выигрыш третьего не может опустить его результат ниже того, что он получает в точках равновесия по Нэшу (т.е. менее $4$). Действительно, пусть первый игрок избрал стратегию $(p, 1 - p)$, а второй "--- $(q, 1 - q)$. Тогда выигрыши третьего при выборе чистых стратегий составляют:
\begin{align*}
	u^3(p, q, 1) &= 6 (p + q) - 8 p q; \\
	u^3(p, q, 0) &= 2 (p + q) - 8 p q + 4.
\end{align*}

Несложно заметить, что
\begin{align*}
	p \ge &q \ge \frac{1}{2} \Rightarrow u^3(p, q, 1) \ge 4; \\
	p \ge 1 - &q \ge \frac{1}{2} \Rightarrow u^3(p, q, 1) \ge 4; \\
	q \ge &p \ge \frac{1}{2} \Rightarrow u^3(p, q, 1) \ge 4; \\
	q \ge 1 - &p \ge \frac{1}{2} \Rightarrow u^3(p, q, 1) \ge 4; \\
	p \le &q \le \frac{1}{2} \Rightarrow u^3(p, q, 0) \ge 4; \\
	p \le 1 - &q \le \frac{1}{2} \Rightarrow u^3(p, q, 0) \ge 4; \\
	q \le &p \le \frac{1}{2} \Rightarrow u^3(p, q, 0) \ge 4; \\
	q \le 1 - &p \le \frac{1}{2} \Rightarrow u^3(p, q, 0) \ge 4. \\
\end{align*}

Эти варианты исчерпывают всё пространство возможных ситуаций, так что никакое сочетание смешанных стратегий двух игроков не может быть эффективным наказанием для третьего. В классическом случае это означало бы, что повторяющаяся игра имеет в качестве решения то же самое множество равновесий по Нэшу, что и отдельный розыгрыш. Однако, вывод этот можно интересным образом уточнить, перейдя к модели, учитывающей стоимость вычислений, необходимых для выбора стратегии в очередной итерации.

\section{Модель повторяющихся игр с учётом стоимости вычислений}\label{sec:ch3/sect2}

Построение искомой модели подразумевает конкретизацию способа, при помощи которого рациональные агенты вычисляют стратегию поведения. Для этой цели подойдёт любой формализм, позволяющий алгоритмически полные вычисления с вероятностным ветвлением. Кроме того, необходимы возможность сохранения произвольного внутреннего состояния в памяти для использования на последующих итерациях и, очевидно, метод численной оценки сложности произведённого на каждой итерации вычисления. Общая схема взаимодействия агентов и среды может выглядеть так:
\begin{figure}[ht]
	\centerfloat{
		\ifdefmacro{\tikzsetnextfilename}{\tikzsetnextfilename{tikz_repeat_compiled}}{}
		\begin{tikzpicture}[scale=2]
			\node[circle,draw] (game1) at (1, 0) {$\Gamma$};
			\node[circle,draw] (game2) at (3, 0) {$\Gamma$};
			\node[circle,draw] (game3) at (5, 0) {$\Gamma$};
			\node[rectangle,draw] (calc11) at (0,  1) {$M^1$};
			\node[rectangle,draw] (calc12) at (2,  1) {$M^1$};
			\node[rectangle,draw] (calc13) at (4,  1) {$M^1$};
			\node[] (calc1i) at (6, 1) {$\cdots$};
			\node[rectangle,draw] (calc21) at (0, -1) {$M^2$};
			\node[rectangle,draw] (calc22) at (2, -1) {$M^2$};
			\node[rectangle,draw] (calc23) at (4, -1) {$M^2$};
			\node[] (calc2i) at (6, -1) {$\cdots$};
			\draw [->,thick,decorate,decoration={snake,post length=1mm}] (calc11) -- (game1) node[midway,sloped,below] {$s^1_1$};
			\draw [->,thick,decorate,decoration={snake,post length=1mm}] (calc21) -- (game1) node[midway,sloped,above] {$s^2_1$};
			\draw [->,thick,decorate,decoration={snake,post length=1mm}] (calc12) -- (game2) node[midway,sloped,below] {$s^1_2$};
			\draw [->,thick,decorate,decoration={snake,post length=1mm}] (calc22) -- (game2) node[midway,sloped,above] {$s^2_2$};
			\draw [->,thick,decorate,decoration={snake,post length=1mm}] (calc13) -- (game3) node[midway,sloped,below] {$s^1_3$};
			\draw [->,thick,decorate,decoration={snake,post length=1mm}] (calc23) -- (game3) node[midway,sloped,above] {$s^2_3$};
			\draw [->,thick,decorate,decoration={snake,post length=1mm}] (game1) -- (calc12) node[midway,sloped,below] {$s_1$};
			\draw [->,thick,decorate,decoration={snake,post length=1mm}] (game1) -- (calc22) node[midway,sloped,above] {$s_1$};
			\draw [->,thick,decorate,decoration={snake,post length=1mm}] (game2) -- (calc13) node[midway,sloped,below] {$s_2$};
			\draw [->,thick,decorate,decoration={snake,post length=1mm}] (game2) -- (calc23) node[midway,sloped,above] {$s_2$};
			\draw [->,thick,decorate,decoration={snake,post length=1mm}] (game3) -- (calc1i) node[midway,sloped,below] {$s_3$};
			\draw [->,thick,decorate,decoration={snake,post length=1mm}] (game3) -- (calc2i) node[midway,sloped,above] {$s_3$};
			\node[diamond,draw] (sum1) at (-1, 2) {$\Sigma^1$};
			\node[] (sum10) at (0,  2) [label=$-\delta \mathfrak{w}^1_1$] {};
			\node[] (sum11) at (1,  2) [label=$+\delta u^1(s_1)$] {};
			\node[] (sum12) at (2,  2) [label=$-\delta^2 \mathfrak{w}^1_2$] {};
			\node[] (sum13) at (3,  2) [label=$+\delta^2 u^1(s_2)$] {};
			\node[] (sum14) at (4,  2) [label=$-\delta^3 \mathfrak{w}^1_3$] {};
			\node[] (sum15) at (5,  2) [label=$+\delta^3 u^1(s_3)$] {};
			\node[] (sum1i) at (6,  2) {$\cdots$};
			\node[diamond,draw] (sum2) at (-1, -2) {$\Sigma^2$};
			\node[] (sum20) at (0, -2) [label=below:$-\delta \mathfrak{w}^2_1$] {};
			\node[] (sum21) at (1, -2) [label=below:$+\delta u^2(s_1)$] {};
			\node[] (sum22) at (2, -2) [label=below:$-\delta^2 \mathfrak{w}^2_2$] {};
			\node[] (sum23) at (3, -2) [label=below:$+\delta^2 u^2(s_2)$] {};
			\node[] (sum24) at (4, -2) [label=below:$-\delta^3 \mathfrak{w}^2_3$] {};
			\node[] (sum25) at (5, -2) [label=below:$+\delta^3 u^2(s_3)$] {};
			\node[] (sum2i) at (6, -2) {$\cdots$};
			\draw [->,thick] (sum1i) -- (sum1);
			\draw [->,thick] (calc11) -- (sum10);
			\draw [->,thick] (game1) -- (sum11);
			\draw [->,thick] (calc12) -- (sum12);
			\draw [->,thick] (game2) -- (sum13);
			\draw [->,thick] (calc13) -- (sum14);
			\draw [->,thick] (game3) -- (sum15);
			\draw [->,thick] (sum2i) -- (sum2);
			\draw [->,thick] (calc21) -- (sum20);
			\draw [->,thick] (game1) -- (sum21);
			\draw [->,thick] (calc22) -- (sum22);
			\draw [->,thick] (game2) -- (sum23);
			\draw [->,thick] (calc23) -- (sum24);
			\draw [->,thick] (game3) -- (sum25);
			\fill [white] (1,  1) circle (2pt);
			\fill [white] (3,  1) circle (2pt);
			\fill [white] (5,  1) circle (2pt);
			\fill [white] (1, -1) circle (2pt);
			\fill [white] (3, -1) circle (2pt);
			\fill [white] (5, -1) circle (2pt);
			\draw [->,double,thick] (-1,  1) -- (calc11) node[midway,above] {$\psi^1_0$};
			\draw [->,double,thick] (calc11) -- (calc12) node[near end,above] {$\psi^1_1$};
			\draw [->,double,thick] (calc12) -- (calc13) node[near end,above] {$\psi^1_2$};
			\draw [->,double,thick] (calc13) -- (calc1i) node[near end,above] {$\psi^1_3$};
			\draw [->,double,thick] (-1, -1) -- (calc21) node[midway,below] {$\psi^2_0$};
			\draw [->,double,thick] (calc21) -- (calc22) node[near end,below] {$\psi^2_1$};
			\draw [->,double,thick] (calc22) -- (calc23) node[near end,below] {$\psi^2_2$};
			\draw [->,double,thick] (calc23) -- (calc2i) node[near end,below] {$\psi^2_3$};
		\end{tikzpicture}
	}
	\legend{}
	\caption[Повторяющаяся игра с учётом стоимости вычислений]{Повторяющаяся игра с учётом стоимости вычислений}\label{fig:repeat}
\end{figure}

Диаграмма на рисунке \ref{fig:repeat} схематично изображает моделируемый процесс для двух игроков (естественным образом обобщающийся на любое конечное их число). В узлах, помеченных буквой $\Gamma$, происходят последовательные розыгрыши произвольной игры $\Gamma = \langle A, S^a, u^a(s), a \in A \rangle$. В $i$-м розыгрыше игрок $a$ выбирает свою стратегию $s^a_i$, применяя вероятностный алгоритм $M^a$ к результату предыдущей итерации $(\psi^a_{i-1}, s_{i-1})$, включающему сохранённое состояние памяти самого алгоритма и набор сыгранных на итерации $i-1$ стратегий. Узлы $\Sigma^a$ изображают последовательное суммирование разностей выигрыша $u^a(s_i)$ и затрат на произведённое вычисление $\mathfrak{w}^a_i$, с учётом экспоненциально уменьшающегося коэффициента дисконтирования $\delta^i$. При такой схеме взаимодействий имеет смысл рассматривать только универсальные алгоритмы $M^a$, что позволяет закодировать любой набор вычислимых стратегий повторяющейся игры в начальных состояниях памяти $\psi^a_0$.

\section{Криптографические стратегии наказания и расширение пространства решений}\label{sec:ch3/sect3}

Хотя трёхсторонний чёт-нечет не даёт возможности двум игрокам наказывать третьего, используя только лишь смешанные стратегии, ещё в первой главе мы установили, что, например, наличие механизма корреляции, которым игроки 1 и 2 могут пользоваться втайне от игрока 3, добавляет решения с выплатами до $(5, 5, 2)$. Таким образом, если бы мы рассматривали повторяющуюся игру в пространстве заговоров структуры $\{\{1, 2\}, \{2, 3\}, \{3, 1\}\}$, то в соответствии с <<народной>> теоремой её множество совершенных подыгровых равновесий включало бы все стратегические наборы с выплатами, обеспечивающие каждому игроку выигрыш не менее $2$. Однако, даже в том случае, когда игроки не могут использовать тайные механизмы корреляции, учёт стоимости вычислений позволяет в повторяющихся играх применять стратегии наказания, опирающиеся на достижения современной криптографии. Для демонстрации этого нам понадобятся два распространённых криптографических примитива.

Во-первых, необходим \emph{протокол совместной выработки ключа} \cite{Boyd}. В криптографии этим термином называют механизм, при помощи которого Алиса и Боб могут создать общую секретную последовательность битов, априорно обладая лишь публичным знанием об устройстве и параметрах самого механизма, и обмениваясь сообщениями через незащищённый от прослушивания канал связи. При этом Кэрол, обладая тем же априорным знанием и имея возможность читать их сообщения, не может вычислить искомую секретную последовательность битов, поскольку это требует решения алгоритмически трудной задачи (такой, для которой необходимо количество операций, зависящее от длинны ключа экспоненциально). В качестве такого механизма может выступать, например, семейство протоколов Диффи-Хеллмана (далее DH) на основе задач факторизации целых чисел или дискретного логарифмирования (в конечной мультипликативной группе или на эллиптической кривой). Опишем общую схему произвольного протокола DH, не вдаваясь в технические детали.

Пусть имеется биекция $f: \mathbb{N} \rightarrow \mathbb{N}$, обладающая свойством односторонности, т.е. вычислимая за полиномиальное от битовой длины входа число операций при том, что обратная к ней $f^{-1}$ сложна уже экспоненциально. Кроме этого, пусть имеется двухместная функция $h : \mathbb{N} \times \mathbb{N} \rightarrow \mathbb{N}$ такая, что $\forall x, y \in \mathbb{N}, h(f(x), y) = h(x, f(y))$. Алиса выбирает случайное число $x$, выполняющее роль её закрытого ключа, и за приемлемое время вычисляет $X = f(x)$, выполняющее роль её открытого ключа. На другом конце Боб аналогичным образом генерирует пару ключей $y$ и $Y = f(y)$. Алиса и Боб обмениваются открытыми ключами через прослушиваемый Кэрол канал связи. Теперь Алиса, зная свой закрытый ключ $x$ и открытый ключ Боба $Y$, может вычислить $h(x, Y)$, а Боб, соответственно, $h(X, y)$. В силу вышеупомянутого свойства функции $h$, вычисленные ими значения можно считать искомым общим секретным ключом $K = h(x, Y) = h(X, y)$. При этом для Кэрол, знающей только открытые ключи $X$ и $Y$, вычисление общего секретного ключа требует вычисления либо $f^{-1}(X)$, либо $f^{-1}(Y)$, что при достаточной битовой длине закрытых ключей оказывается непозволительно дорогим процессом.

Вторым криптографическим примитивом, необходимым для стратегии наказания, является \emph{криптографически стойкий генератор псевдослучайных чисел} \cite{Gutmann}, далее называемый CSPRNG. Его можно представить в виде предиката $G : \mathbb{N} \times \mathbb{N} \rightarrow \{0, 1\}$, первый аргумент которого называется зерном (или seed), а второй "--- позицией. Программа, вычисляющая для заданного $K \in \mathbb{N}$ последовательные значения $G(K, i), i = 1, 2, \ldots$, должна совершать каждый шаг за полиномиальное от битовой длинны $K$ число операций. При этом генератор обязан проходить тест на следующий бит, т.е. не должно существовать полиномиального по сложности алгоритма, способного без знания $K$ по первым $n$ битам генерируемой последовательности угадать $G(K, n+1)$ с вероятностью, отличной от $\frac{1}{2}$.

Теперь из этих примитивов соберём следующий алгоритм поведения для первого игрока в повторяющемся трёхстороннем чёт-нечете, параметризующийся размером ключей $n_{PK}$\footnote{Для простоты будем считать, что использующаяся в протоколе DH односторонняя функция $f$ имеет домен и кодомен одной битовой длины, и что при случайном выборе закрытого ключа все биты открытого оказываются распределены равномерно и независимо. Это верно, например, при использовании дискретного логарифмирования на эллиптической кривой.}:
\begin{enumerate}
	\item Создать случайную битовую последовательность $x = (x_1, \ldots, x_{n_{PK}})$.
	\item Вычислить битовую последовательность $X = f(x)$.
	\item Для каждого $i = 1 \ldots n_{PK}$ совершить один ход игры, выбирая решку при $X_i = 1$ и орла в противном случае. Выбранную вторым игроком стратегию (с тем же сопоставлением) запомнить в качестве элемента $Y_i$ последовательности $Y = (Y_1, \ldots, Y_{n_{PK}})$.
	\item Вычислить $K = h(x, Y)$.
	\item Все последующие ходы совершать, выбирая стратегию в соответствии с последовательно генерируемыми CSPRNG значениями $G(K, i), i = 1, 2, \ldots$.
\end{enumerate}

Стратегию второго игрока строим аналогичным образом, меняя местами $x$ с $y$, $X$ с $Y$ и $h(x, Y)$ с $h(X, y)$. Таким образом игроки 1 и 2 используют первые $n_{PK}$ ходов игры в качестве своеобразного <<танца синхронизации>>, вырабатывая общее секретное зерно для генератора псевдослучайных битов, чей вывод на последующих ходах фактически используется в качестве тайного механизма корреляции. В зависимости от коэффициента дисконтирования $\delta$, размера ключей $n_{PK}$ и удельной стоимости вычислений наилучшей ответом для третьего игрока оказывается одна из двух стратегий: взлом или игнорирование. Стратегия взлома подразумевает, как это следует из названия, вычисление $f^{-1}(X)$ или $f^{-1}(Y)$ и присоединение к использованию CSPRNG в качестве механизма корреляции. Выигрыш третьего игрока в этом случае составляет $4 \frac{\delta}{1-\delta} - \delta^{n_{PK} + 1} \mathfrak{w}_{f^{-1}}(n_{PK})$, где первое слагаемое представляет собой суммарный выигрыш в классическом равновесии Нэша с учётом коэффициента дисконтирования, а второе "--- стоимость вычисления функции $f^{-1}$ для ключа размером $n_{PK}$ бит на $n + 1$ итерации игры. Стратегия игнорирования также вполне соответствует своему названию и подразумевает использование любой смешанной стратегии без учёта ходов оппонентов. Поскольку первые $n_{PK}$ ходов игроки 1 и 2 тратят на синхронизацию, то выигрыш третьего на протяжении этих розыгрышей совпадает с выигрышем в классическом равновесии Нэша, но затем его доход ополовинивается так, как если бы игра происходила в пространстве заговоров структуры $\{\{1, 2\}\}$. Таким образом общий выигрыш для стратегии игнорирования составляет $4 \frac{\delta}{1-\delta} - 2 \delta^{n_{PK}} \frac{\delta}{1-\delta}$.

Сравнивая выигрыши в имеющихся у третьего игрока альтернативах можно прийти к выводу, что стратегия взлома оказывается выгодна только в случае $\mathfrak{w}_{f^{-1}}(n_{PK}) < \frac{2}{1 - \delta}$. Это значит, что с одной стороны, чем больше коэффициент дисконтирования $\delta$, тем выгоднее стратегия взлома, обменивающая стоимость однократного обращения односторонней функции $f$ на сохранение полного дохода от всего <<хвоста>> розыгрышей. С другой же стороны, для любого $\delta$ игроки 1 и 2 могут подобрать достаточно большой размер ключа $n_{PK}$, вынуждая третьего удовлетвориться стратегией игнорирования. Следует заметить, что на практике криптосистемы, использующие дискретное логарифмирование на эллиптических кривых, считаются надёжными уже при 256-битных ключах (Curve25519 \cite{Bernstein}, например), т.е. стоимость их взлома заведомо превосходит возможности, доступные человеческой цивилизации на текущем этапе технологического развития. Возможно, с появлением прикладного квантового криптоанализа эта ситуация может измениться, однако пока можно считать, что в повторяющемся трёхстороннем чёт-нечете для эффективной криптографической стратегии наказания достаточно всего нескольких десятков итераций, потраченных на <<танец синхронизации>>. 

Этот результат можно при помощи <<народной>> теоремы использовать для пополнения множества совершенных подыгровых равновесий повторяющегося трёхстороннего чёт-нечета. Как уже было замечено в начале главы, с классической точки зрения в игре нет эффективных стратегий наказания, а значит её решениями оказываются только такие наборы стратегий, в которых исход каждой итерации равновесен по Нэшу в обычном смысле. При коэффициенте дисконтирования $\delta$ это даёт множество достижимых векторов платежей, состоящее из одной точки "--- $(4 \frac{\delta}{1-\delta}, 4 \frac{\delta}{1-\delta}, 4 \frac{\delta}{1-\delta})$. Однако, если мы допускаем применение криптографических стратегий наказания, ситуация становится намного интереснее.

Пусть в распоряжении игроков находятся универсальные вычислительные устройства с различными параметрами, влияющими на удельную стоимость вычислений. Абстрагируясь от деталей внутренней архитектуры, мы можем выделить главную характеристику каждой из них, интересующую нас в контексте задачи "--- минимальную длину ключа, при которой стоимость его взлома превосходит ущерб от игнорирования построенной на нём стратегии наказания. Это даёт нам дополнительный параметр игры "--- целочисленный вектор $(n_{PK}^1, n_{PK}^2, n_{PK}^3)$, составленный из соответствующих величин для каждого из игроков. Таким образом, <<народная>> теорема применяется здесь фактически в обычном виде "--- игроки договариваются о последовательности ходов $(s_i)$, если игрок $a$ на итерации $i$ отклоняется от стратегии $s^a_i$, то остальные применяют против него криптографическую стратегию наказания, начинающуюся с <<танца синхронизации>> продолжительностью в $n_{PK}^a$ итераций. Это обеспечивает игрокам следующий вектор гарантированных выигрышей:
\begin{equation*}
	((4 - 2 \delta^{n_{PK}^1}) \frac{\delta}{1-\delta}, (4 - 2 \delta^{n_{PK}^2}) \frac{\delta}{1-\delta}, (4 - 2 \delta^{n_{PK}^3}) \frac{\delta}{1-\delta})
\end{equation*}

Заметим, что с ростом $n_{PK}^a$, растёт и минимальный выигрыш игрока, при котором последовательность розыгрышей $(s_i)$ образует совершенное подыгровое равновесие. То есть, чем дешевле игроку обходятся вычисления, тем сложнее остальным принудить его к принятию невыгодного исхода повторяющейся игры. Это создаёт картину, неожиданную для казалось бы элементарного конфликта с матрицей платежей размером $2 \times 2 \times 2$ "--- поскольку из известных моделей асимметричной криптографии наименьшую длину ключа при равной стойкости обеспечивает дискретное логарифмирование на эллиптической кривой, получается, что для реализации успешных стратегий приходится применять математический аппарат переднего края теории групп. Более того, доход игроков напрямую зависит от их способности производить сложные вычисления на пределе любых доступных возможностей.

\section{Обобщение результатов, перспективы и гипотезы}\label{sec:ch3/sect4}

Для того, чтобы оценить значение этого феномена, следует отступить на пару шагов от конкретики описанной игры и попытаться охватить взглядом более широкую картину. Во-первых, достаточно очевидно, что сам по себе трёхсторонний чёт-нечет выступает здесь в роли не более чем относительно произвольно сконструированного примера игры, чувствительной к дополнительной информационной асимметрии. Вышеописанные криптографические стратегии наказания не используют фактически никаких других свойств данного конфликта и нет причин думать, что они не могут быть обобщены на множество других игр, проявляющих то же свойство. К примеру, если мы возьмём в качестве отдельной итерации игру $\Gamma^3_n$ из второй главы этой работы, то окажется, что при её повторении можно аналогичным образом кодировать открытые ключи алфавитом, состоящим из $n$ символов по числу компьютеров в вычислительном центре, а потом использовать полученный общий секретный ключ в качестве зерна генератора, псевдослучайно выбирающего из опять же $n$ элементов.

Во-вторых, имеет смысл задаться вопросом о том, исчерпывает ли описанная схема криптографических стратегий наказания все возможности для пополнения множества совершенных подыгровых равновесий. Здравый смысл подсказывает, что нет хотя бы потому, что оба использованных здесь криптографических примитива (и протокол совместной выработки ключа, и криптографически стойкий генератор псевдослучайных чисел) имеют немало реализаций, опирающихся на самые разные математические формализмы, список которых год от года пополняется благодаря бурному развитию соответствующих областей знания. Более того, построение стратегии наказания из протокола DH с последующим использованием полученного ключа в качестве зерна CSPRNG само по себе достаточно произвольно "--- мы использовали инструменты, изначально создававшиеся в совершенно ином контексте для других целей, просто потому, что они уже есть и имеют доказанные свойства, удобные для решения нашей задачи.

Эти рассуждения позволяют обоснованно предположить, что и сам трёхсторонний чёт-нечет, и представленные криптографические стратегии наказания "--- всего лишь наиболее очевидные представители более широкого класса пока ещё не исследованных математических формализмов. Говоря максимально общим языком, рациональные агенты, участвуя в повторяющихся конфликтах, могут вырабатывать секретные коррелированные стратегии поведения, используя лишь специальным образом выбираемые публичные действия и наблюдая за аналогично действующим контрагентом. Секретность при этом обеспечивается за счёт того, что присоединение к корреляции требует от наблюдающей ту же последовательность публичных действий третьей стороны когнитивных усилий, превосходящих её возможности. Кроме того, обобщая связь между удельной стоимостью вычислений и требуемой длиной ключей, можно предположить, что чем большие когнитивные усилия способна приложить сторона, от которой заговорщики пытаются скрыть свою общую стратегию, тем сложнее этот процесс и тем меньший доход (за счёт растущего дисконтирования в хвосте розыгрышей) приносит такая секретность.

В самом деле, если представить себе обычных людей, играющих в чувствительную к дополнительной информационной асимметрии игру без применения специальных технических средств, мы вряд ли можем ожидать, что они будут производить в уме вычисления, необходимые для связки DH+CSPRNG. При этом, вполне вероятно, что повсеместно распространены примеры того, как люди могут добиваться необходимой тайной синхронизации неосознанно, воспринимая результат как самоочевидный, не требующий объяснений факт. В данном случае имеются в виду опытные картёжники, специализирующиеся на сложных интеллектуальных играх, таких как бридж или преферанс. В их среде считается неоспоримым, что помимо индивидуальных навыков на исход розыгрышей сильно влияет опыт именно совместной игры "--- пара игроков, по отдельности не хватающих звёзд с небес, могут оказаться грозными соперниками, если у них за плечами много партий за одним столом. Если верны вышеизложенные предположения о достаточно общем характере построенной нами модели, то феномен такой <<сыгранности>> может найти удовлетворительное объяснение в её рамках.

Следует заметить, что если мы захотим исследовать с этой точки зрения стратегии игроков в уже существующие салонные игры, на нашем пути могут встать затруднения, связанные с тем, что правила тех из них, что можно заподозрить в чувствительности к дополнительной информационной асимметрии, сложны даже без учёта этого их свойства. Скажем, бридж или преферанс останутся, очевидно, весьма нетривиальными, даже если играть в них отдельными партиями, каждый раз подбирая состав анонимных участников случайно из большого пула кандидатов (удалённо по сети, например). Для облегчения задачи будущих исследователей неплохо было бы сконструировать специальную карточную игру, в которой использование эффекта <<сыгранности>> было бы необходимым элементом любой успешной стратегии. Попытка создания такой новой игры под названием <<Тессеракт>> вынесена в Приложение~\cref{app:D} "--- будем надеяться, что в будущем она окажется полезна как в научных, так и в развлекательных целях.

Завершить данную работу хотелось бы формулировкой достаточно смелой неформальной гипотезы, продолжающей линию рассуждений последней главы в русле популяционных игр:
\begin{conjecture}
	Если в популяции периодически (много раз на протяжении жизни одной особи) возникают конфликтные ситуации, модель которых чувствительна к дополнительной информационной асимметрии, и вероятность продолжения рода отдельными особями существенно зависит от их успеха в этих конфликтах, то давление отбора закрепляет в популяции признаки, способствующие увеличению когнитивного потенциала следующих поколений (понимаемого в общем смысле, как способность производить Тьюринг-полные вычисления над произвольными данными).
\end{conjecture}

Если эта гипотеза верна, то чувствительность игр к дополнительной информационной асимметрии может оказаться <<Святым Граалем>> эволюционной теории игр "--- фактором, порождающим неограниченную гонку вооружений в сфере способностей к сложному поведению. Любые игры с этим свойством, даже будучи очень просты сами по себе, поощряют когнитивный потенциал участников необходимостью строить и раскрывать заговоры, так что его изучение может как обогатить наше понимание эволюции интеллекта наших предков, так и стать инструментом совершенствования технологий искусственного разума.

\clearpage