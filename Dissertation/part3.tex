\chapter{Вычислительная сложность стратегий в повторяющихся играх с дисконтированием}\label{ch:ch3}

\section{Повторяющийся трёхсторонний чёт-нечет с памятью}\label{sec:ch3/sect1}

Чувствительность к дополнительной информационной асимметрии интересна не только тем, как на исходы проявляющих её игр влияет собственно информационная асимметрия. Далее будет показано, что в тех случаях, когда этим свойством обладают отдельные розыгрыши повторяющихся игр, необычные эффекты могут возникать даже при отсутствии приватных механизмов корреляции, в рамках классического смешанного расширения и совершенных подыгровых равновесий. Для демонстрации феномена, однако, нам потребуется использовать модель рациональных агентов с памятью, учитывающую стоимость вычислений, необходимых для реализации очередного шага стратегии. Пусть каждый розыгрыш повторяющейся игры описывается нормальной формой $\Gamma = \langle A, S^a, u^a(s), a \in A \rangle$. Тогда стратегию игрока $a$ можно представить в следующем виде:
\begin{equation*}
	\overbrace{s}^a = \langle \psi^a \in \Psi^a, f^a : \Psi^a \rightarrow \mathcal{M}(X^a), g^a :  / \rangle
\end{equation*}

\clearpage
