\chapter{Модель заговоров}\label{ch:ch1}

\section{Коррелированное расширение игр в нормальной форме}\label{sec:ch1/sec1}

Базовым формализмом для описания дополнительной информационной асимметрии в сложившейся научной традиции выступает коррелированное расширение нормальной формы игр, предложенное Робертом Ауманом в \cite{Aumann74}. Для удобства его центральные элементы будут изложены здесь в нотации, адаптированной к русскоязычной среде. Рассмотрим игру в нормальной форме $\Gamma = \langle A, S^a, u^a(s), a \in A \rangle$. Конечное множество игроков здесь и далее везде обозначается как $A = \{1, \ldots, m\}$, а конечное множество наборов чистых стратегий "--- $S = S^1 \times \ldots \times S^m$. Помимо множества стратегий $S^a$ каждый игрок определяется платёжной функцией $u^a : S \rightarrow \mathbb{R}$.

Также рассмотрим вероятностное пространство\cite{Kolmogorov} $\langle \Omega, \mathfrak{B}, \mathbb{P} \rangle$, в котором реализуется наблюдаемое игроками состояние природы. Здесь $\Omega$ "--- множество всевозможных таких состояний, $\mathfrak{B}$ "--- $\sigma$"~алгебра подмножеств $\Omega$, а $\mathbb{P} : \mathfrak{B} \rightarrow \mathbb{R}_{\ge 0}$ "--- вероятностная мера. Каждому игроку $a \in A$ поставим в соответствие \emph{собственное подпространство} $\langle \Omega, \mathfrak{I}^a, \mathbb{P} \rangle$ такое, что $\mathfrak{I}^a \subseteq \mathfrak{B}$. При этом набор $\sigma$"~алгебр $\mathfrak{I} = (\mathfrak{I}^a, a \in A)$ отражает информированность игроков о состоянии природы. В описываемой ситуации это состояние не влияет на функции выигрышей непосредственно, выступая исключительно как способ синхронизации действий игроков. Это значит, что $\sigma$"~алгебра $\mathfrak{B}$ сама по себе не является существенным параметром модели, и измеримость по ней для $\mathbb{P}$ можно заменить измеримостью по $\mathfrak{I}^a, \forall a \in A$.

Отдельно следует заметить, что в оригинале для собственных подпространств игроков Аумановский формализм предполагал индивидуальность не только $\sigma$"~алгебр, но и соответствующих им мер, учитывая тем самым возможную субъективность оценок вероятности наступления тех или иных событий, что немаловажно в случаях, когда в качестве механизма корреляции выступают процессы, слишком сложные для объективного анализа (например спортивные соревнования). Однако, для целей данного исследования этот аспект не имеет большого смысла, поскольку в модели заговоров подразумевается, что их участники могут произвольным образом выбирать механизм корреляции, а в такой ситуации разумно ожидать, что люди будут использовать простые источники случайности с известным распределением (рулетки, кости, жребий и т.д.). По этой причине здесь и далее модель коррелированных стратегий используется в её упрощённой форме, с общей для всех игроков объективной вероятностной мерой в пространстве состояний природы.

Таким образом, получается $\Phi = \langle A, \Omega, \mathfrak{I}^a, \mathbb{P}, a \in A \rangle$ "--- набор параметров, характеризующих некоторое \emph{пространство корреляции} для произвольной игры с множеством игроков $A$. Отметим, что в играх с одним множеством игроков, но различными множествами чистых стратегий и функциями выигрыша можно применять одно и то же пространство корреляции. Полностью же \emph{коррелированное расширение игры} определяет пара $\Gamma | \Phi$. Опишем полученную новую игру в терминах нормальной формы:
\begin{equation*}
	\Gamma | \Phi = \langle A, \mathbf{S}^a, u^a(\mathbf{s}), a \in A \rangle.
\end{equation*}

Здесь множество $\mathbf{S}^a$ доступных игроку $a$ коррелированных стратегий состоит из всех $\mathfrak{I}^a$"~измеримых функций $\mathbf{s}^a : \Omega \rightarrow S^a$, отображающих множество возможных состояний природы на множество доступных ему чистых стратегий. Соответственно, функция выигрыша вычисляется по формуле математического ожидания случайной величины
\begin{equation*}
	u^a(\mathbf{s}) = \sum\limits_{s \in S} \mathbb{P}(\mathbf{s}^{-1}(s)) u^a(s),\;\mathbf{s}^{-1}(s) = \bigcap_{a \in A} (\mathbf{s}^a)^{-1}(s^a),
\end{equation*}
где $\mathbb{P}(\mathbf{s}^{-1}(s))$ выступают в роли коэффициентов распределения на матрице игры.

В своей статья Ауман демонстрирует силу вводимого формализма, показывая на примерах, как с его помощью в играх можно получать новые точки равновесия по Нэшу. Подбирая параметры пространств корреляции, можно конструировать не только решения с любыми выплатами из выпуклой оболочки векторов выплат в точках классического смешанного равновесия Нэша, но для некоторых игр даже решения, лежащие за пределами такой выпуклой оболочки. Это позволяет сформулировать ключевое для данной работы понятие:
\begin{definition}
	Пусть $\Gamma$ --- игра в нормальной форме c $m$ участниками, а $U \subseteq \mathbb{R}^m$ --- множество всех векторов выплат, достижимых в её смешанных равновесиях по Нэшу. Игра $\Gamma$ называется \emph{чувствительной к дополнительной информационной асимметрии}, когда существует пространство корреляции $\Phi$ такое, что в игре $\Gamma | \Phi$ найдётся коррелированное равновесие по Нэшу с вектором выплат, не принадлежащим выпуклой оболочке множества $U$.
\end{definition}

Кроме того в этой же статье доказывается, что наличие в пространстве корреляции публичной вещественной рулетки влечёт выпуклость как множества достижимых выплат, так и множества равновесий Нэша в любой игре. Касательно коррелированного расширения игр в нормальной форме вышеизложенного вполне достаточно для понимания идей данной работы, более подробно же современное состояние знаний на эту тему можно проследить по публикациям, упомянутым в Приложении~\cref{app:A}.

\section{Изоморфизм пространств корреляции}\label{sec:ch1/sec2}

Следует отметить, что модель пространств корреляции в некотором смысле существенно избыточна, поскольку как таковые события из состояния природы значения не имеют и используются лишь в качестве сигналов для синхронизации стратегий игроков. Чтобы осмысленным образом рассуждать о влиянии, оказываемым дополнительной информационной асимметрией на исходы конфликтов, неизбежно требуется умение абстрагироваться от конкретных её источников, фокусируя внимание на структурных различиях в осведомлённости оппонентов. Если формально различные пространства корреляции оказываются полностью взаимозаменимы с теоретико"~игровой точки зрения, они должны быть отнесены к общему классу эквивалентности, чьё описание и является в действительности существенным параметром модели. Причём сразу следует заметить, что это касается не только тривиальных замен множества состояний природы на другое множество той же мощности с соответствующей биекцией остальных параметров пространства, но и более сложных случаев. Например, если в контексте некоторой игры группа игроков наблюдает общий сигнал в виде колеса вещественной рулетки, будет ли иметь значение наблюдение ими ещё и броска монетки? Здравый смысл подсказывает что любую общую стратегию с использованием рулетки и монетки можно легко превратить в эквивалентную для одной только рулетки, для чего достаточно поделить колесо пополам и отобразить отдельные варианты для орла и решки на полученные два сектора. Опишем этот феномен в виде изоморфизма:
\begin{definition}
	Разбиением пространства корреляции $\Phi = \langle A, \Omega, \mathfrak{I}^a, \mathbb{P}, a \in A \rangle$ в произвольное конечное множество исходов (кодомен) $X = X^1 \times ... \times X^m$ называется отображение $f : \Omega \rightarrow X$, состоящее из набора функций $(f^1, ..., f^m)$, где каждая $f^a : \Omega \rightarrow X^a$ измерима в $\mathfrak{I}^a$. Далее <<разбиение $f$ пространства корреляции $\Phi$>> будем сокращённо обозначать $f \models \Phi$.
\end{definition}

В контексте коррелированного расширения множествам исходов $X^a$ соответствуют множества чистых стратегий $S^a$, а элементам разбиения $f^a$ "--- коррелированные стратегии $\mathbf{s}^a$. Далее также будут использоваться отображения $f^{-1} : X \rightarrow 2^\Omega$, обратные к разбиениям пространств корреляции:
\begin{equation*}
	f^{-1}(x) = \bigcap_{a \in A} (f^a)^{-1}(x^a).
\end{equation*}
\begin{definition}
	Пространство $\Phi_1$ с мерой $\mathbb{P}_1$ называется отобразимым на $\Phi_2$ с мерой $\mathbb{P}_2$ (далее $\Phi_1 \precsim \Phi_2$), если их множества игроков совпадают и для любого разбиения $f_1 \models \Phi_1$ существует разбиение $f_2 \models \Phi_2$ с тем же кодоменом такое, что $\mathbb{P}_1 \circ f_1^{-1} = \mathbb{P}_2 \circ f_2^{-1}$. Взаимно отобразимые друг на друга пространства корреляции называются изоморфными (далее $\Phi_1 \sim \Phi_2$).
\end{definition}

Это определение легко проиллюстрировать упомянутым выше примером "--- для любого разбиения $f_1 : \left[0, 1\right) \times \{0, 1\} \rightarrow X$ пространства, состоящего из вещественной рулетки и симметричной монетки, можно построить соответствующий образ $f_2 : \left[0, 1\right) \rightarrow X$ в пространстве из одной только рулетки:
\begin{equation*}
	f_2(\alpha) = \begin{cases}
		f_1(2 \alpha, 0), & 0 \le \alpha < \frac{1}{2} \\
		f_1(2 \alpha - 1, 1), & \frac{1}{2} \le \alpha < 1
	\end{cases}.
\end{equation*}

Рефлексивность, симметричность и транзитивность вводимого при помощи данного определения изоморфизма очевидны, а значит это действительно отношение эквивалентности на множестве пространств корреляции. При этом, хотя определение изоморфизма дано в отрыве от коррелированного расширения игр, можно сформулировать следующую теорему:
\begin{definition}
	Для игры $\Gamma = \langle A, S^a, u^a(s), a \in A \rangle$ множеством достижимых выплат по отклонениям группы игроков $A_*$ от профиля стратегий $s$ будем называть
	\begin{equation*}
		U_\Gamma^{A_*}(s) = \{\bar{u} \mid \exists s_* \in S : u(s_*) = \bar{u}, \forall a \in A \setminus A_*, s^a = s_*^a\}.
	\end{equation*}
\end{definition}
\begin{theorem}[Об изоморфных пространствах]
	Пусть $\Phi_1 \sim \Phi_2$. Тогда для любой игры в нормальной форме $\Gamma$ с конечными множествами стратегий игроков её коррелированные расширения $\Gamma | \Phi_1$ и $\Gamma | \Phi_2$ обладают следующим свойством. Пусть $\mathbf{s}_1$ "--- некоторый профиль стратегий игры $\Gamma | \Phi_1$. Тогда существует $\mathbf{s}_2$ "--- профиль стратегий игры $\Gamma | \Phi_2$ такой, что $U_{\Gamma | \Phi_1}^{A_*}(\mathbf{s}_1) = U_{\Gamma | \Phi_2}^{A_*}(\mathbf{s}_2)$ для любой группы игроков $A_*$.
\end{theorem}

Эта теорема позволяет считать изоморфные пространства корреляции неразличимыми в контексте поиска равновесий, устойчивых к как индивидуальным, так и групповым отклонениям. Доказательство, представляющее собой упражнение в топологии без тесной связи с центральными идеями работы, вынесено в Приложение~\cref{app:B}.

\section{Пространства заговоров}\label{sec:ch1/sec3}

Получив осмысленный изоморфизм для пространств корреляции, можно выделить из всевозможных классов эквивалентности те, что представляют интерес с точки зрения моделирования дополнительной информационной асимметрии. Как показал Ауманн, выход за пределы выпуклой оболочки множества решений в смешанных стратегиях возможен если часть игроков (не менее двух) использует коррелированную стратегию, зависящую от события, о котором не проинформирован хотя бы один из остальных игроков. Подобную форму взаимовыгодного тайного согласования действий естественно называть \emph{заговором}, а используемый для синхронизации сигнал "--- \emph{тайной}. Пусть в пространстве корреляции $\Phi = \langle A, \Omega, \mathfrak{I}^a, \mathbb{P}, a \in A \rangle$ имеется тайна, т.е. вероятностное подпространство $\langle \Omega, \mathfrak{S}, \mathbb{P} \rangle, \mathfrak{S} \subseteq \sigma(\mathfrak{I}^1 \cup \ldots \cup \mathfrak{I}^m)$. Искомая асимметрия информированности предполагает, что некоторые игроки наблюдают события из $\mathfrak{S}$ (или другие, коррелирующие с ними), а некоторые "--- нет. Хотя теоретически можно представить заговор, степень вовлеченности в который варьируется от игрока к игроку (кто-то может наблюдать события из $\mathfrak{S}$ частично или опосредованно, через наблюдение других коррелирующих с ними событий), имеет смысл в первую очередь рассмотреть простейший случай "--- деление всех игроков на <<заговорщиков>> $A_{\mathfrak{S}} \subseteq A$, наблюдающих $\mathfrak{S}$ целиком, и <<аутсайдеров>> $A \setminus A_{\mathfrak{S}}$, в чьём поле зрения только события, независимые с элементами $\mathfrak{S}$.

Следующий логичный вопрос "--- о структуре самой тайны. Вполне можно представить, как заговорщики используют в её роли самые разные источники случайности: броски игральных костей, тасование колод карт, лотерейные розыгрыши и т.д., так что на первый взгляд неочевидно, можно ли ограничиться рассмотрением какого-то одного, естественного в контексте происходящего механизма. Положительный ответ можно получить, используя введённые выше отображения пространств корреляции. Если мы будем сравнивать всевозможные пространства, различающиеся только тайнами группы заговорщиков $A_* \subseteq A$, отношение $\precsim$ вводит на их множестве частичный порядок. Нижней гранью этого порядка будет вырожденное пространство корреляции, в котором тайна заговора состоит из единственного атомарного события с вероятностью $1$ "--- такое пространство отобразимо на любое другое и, очевидно, вообще не может быть использовано для корреляции стратегий. Верхняя грань интереснее "--- в её типичном представителе тайна заговора представляет собой произвольное безатомическое \cite[с.~81]{Bogachev} пространство. Проще всего представить такой механизм корреляции в виде вещественной рулетки, вращение которой генерирует равномерно распределённую случайную величину в единичном полуинтервале $\left[0, 1\right)$. Наблюдающие рулетку заговорщики могут, разделяя её на сектора необходимых размеров, согласовать любой набор коррелированных стратегий в играх с конечным множеством исходов. Именно такой универсальный источник случайности и имеет смысл рассматривать в первую очередь, как предоставляющий максимум свободы выбора.

Наконец, следует подумать о пространствах корреляции с множественными тайнами. По сути, ничего не мешает игрокам наблюдать сразу несколько рулеток, выбирая в зависимости от ситуации с кем из оппонентов коррелировать свою стратегию. Более того, стратегия игрока может одновременно существенно зависеть от более чем одной тайны. Таким образом, естественным предметом рассмотрения можно считать пространства корреляции, состоящие из наборов независимых вещественных рулеток, каждая из которых характеризуется подмножеством игроков, имеющих возможность её наблюдать. Остаётся заметить, что при наличии в одном пространстве корреляции двух и более вещественных рулеток, принадлежащих одному и тому же кругу заговорщиков, все кроме одной можно без ущерба для модели выкинуть, так как в играх с конечными множествами исходов такое дублирование очевидно бесполезно. Перейдём теперь к более формальному определению предложенной концепции. Для этого рассмотрим произвольное пространство корреляции $\Phi = \langle A, \Omega, \mathfrak{I}^a, \mathbb{P}, a \in A \rangle$. В этом пространстве для каждой непустой группы игроков $A_* \subseteq A$ определим следующее семейство событий:
\begin{equation*}
	\mathfrak{S}_\Phi^{A_*} = \{U \in \bigcap\limits_{a \in A_*} \mathfrak{I}^a \mid \mathbb{P}(U \cap V) = \mathbb{P}(U) \mathbb{P}(V), \forall V \in \sigma(\bigcup\limits_{a \in A \setminus A_*} \mathfrak{I}^a)\}.
\end{equation*}

Таким образом, $\mathfrak{S}_\Phi^{A_*}$ "--- множество всех таких событий, что о них осведомлены все члены $A_*$, и каждое событие попарно независимо со всеми событиями, известными не членам $A_*$ даже при объединении их знаний. Поскольку пересечение $\sigma$"~алгебр образует $\sigma$"~алгебру, и так как подмножество независимых с некоторым событием событий $\sigma$"~алгебры также образует $\sigma$"~алгебру, то $\mathfrak{S}_\Phi^{A_*}$ "--- $\sigma$"~алгебра. Это позволяет говорить о вероятностном подпространстве $\langle \Omega, \mathfrak{S}_\Phi^{A_*}, \mathbb{P} \rangle$, которое и называется тайной группы $A_*$. Выделим два случая: тайны с безатомическими мерами мы будем называть \emph{полными}, а тайны с тривиальными атомическими мерами с единственным атомом $\Omega$ "--- \emph{пустыми}. Это даёт возможность сформулировать следующее:
\begin{definition}
	Пространство корреляции $\langle A, \Omega, \mathfrak{I}^a, \mathbb{P}, a \in A \rangle$ назовём пространством заговоров структуры $\mathfrak{A} = \{A_1, ..., A_n\} \subseteq 2^A$, когда
	\begin{itemize}
		\item $\bigcup\limits_{i=0}^n A_i = A$;
		\item $\forall A_* \in \mathfrak{A}$, тайна $A_*$ полна;
		\item $\forall A_* \notin \mathfrak{A}$, тайна $A_*$ пуста;
		\item $\mathfrak{I}^a = \sigma(\bigcup\limits_{a \in A_* \in \mathfrak{A}}\mathfrak{S}_\Phi^{A_*})$, т.е. $\mathfrak{I}^a$ "--- наименьшая $\sigma$"~алгебра, включающая все $\sigma$"~алгебры тайн групп, в которые входит игрок $a$.
	\end{itemize}
\end{definition}

Проще говоря, пространствами заговоров называются такие пространства корреляции, в которых а) тайна любой группы игроков либо полна, либо пуста; б) каждый игрок входит хотя бы в одну группу с полной тайной и в) у игроков нет никаких знаний о состоянии природы, которые не порождались бы тайнами групп, которым они принадлежат. Построим для иллюстрации простейший пример такого пространства:
\begin{itemize}
	\item $A = \{1, 2, 3\}$,
	\item $\Omega = \left[0, 1\right)^2$,
	\item $\mathfrak{I}^1 = \sigma(\{\left[ 0, p_1 \right) \times \left[ 0, 1 \right) \mid 0 < p_1 \leq 1 \})$,
	\item $\mathfrak{I}^2 = \sigma(\{\left[ 0, 1 \right) \times \left[ 0, p_2 \right) \mid 0 < p_2 \leq 1 \})$,
	\item $\mathfrak{I}^3 = \sigma(\{\left[ 0, p_1 \right) \times \left[ 0, p_2 \right) \mid 0 < p_1 \leq 1, 0 < p_2 \leq 1 \})$,
	\item $\mathbb{P}$ "--- мера Лебега.
\end{itemize}

В этом примере пространство корреляции состоит из двух независимых вещественных рулеток, первый и второй игроки наблюдают по одной из них, а третий наблюдает обе. При этом выходит, что $\mathfrak{S}_\Phi^{\{1,3\}}$ совпадает с $\mathfrak{I}^1$, $\mathfrak{S}_\Phi^{\{2,3\}}$ совпадает с $\mathfrak{I}^2$, а для остальных групп $A_* \subseteq A$ соответствующая $\mathfrak{S}_\Phi^{A_*}$ тривиальна. Структурой пространства (или \emph{семейством заговоров}) называется множество всех групп игроков с полными тайнами. В вышеприведённом примере структура пространства $\mathfrak{A} = \{\{1,3\},\{2,3\}\}$. С точки зрения допустимых профилей стратегий это означает, что любая группа игроков, входящая в семейство заговоров, может использовать общую тайну для формирования коррелированной стратегии, причём игроки не входящие в эту группу не могут присоединиться к согласованному таким образом выбору стратегий. Напротив, группы игроков, не входящие в семейство заговоров, вышеописанной возможностью не располагают. Структуру пространства можно считать его исчерпывающим конечным описанием, поскольку
\begin{theorem} \label{the:struct}
	Все пространства заговоров одной структуры изоморфны.
\end{theorem}

Доказательство этой теоремы снова представляет собой упражнение в топологии без тесной связи с основными идеями работы и вынесено в приложение \ref{app:C}. Теперь, когда установленно, что множество всех пространств заговоров разбивается на классы эквивалентности, нетрудно предложить способ конструирования стандартного представителя каждого класса по соответствующему семейству заговоров:
\begin{definition}
	Стандартным пространством структуры $\mathfrak{A} = \{A_1, A_2, ..., A_n\}$ называется пространство корреляции $\Phi_{\mathfrak{A}} = \langle A, \Omega, \mathfrak{I}^a, \mathbb{P}, a \in A \rangle$ со следующими параметрами:
	\begin{itemize}
		\item $A = \bigcup\limits_{i=1}^n A_i$,
		\item $\Omega = \left[0, 1\right)^n$,
		\item $\mathfrak{I}^a = \sigma(\{\prod\limits_{i=1}^n\left[ 0, p_i \right) \mid \operatorname{\mathbf{if}} \: a \in A_i \: \operatorname{\mathbf{then}} \: 0 < p_i \leq 1 \: \operatorname{\mathbf{else}} \: p_i = 1 \})$,
		\item $\mathbb{P}$ "--- мера Лебега.
	\end{itemize}
\end{definition}

Множество состояний природы представляет собой $n$"~мерный (по числу заговоров, входящих в семейство) единичный куб, а вероятностная мера соответствует непрерывному равномерному распределению. При этом $\sigma$"~алгебра каждого игрока борелева в проекциях на оси, соответствующие заговорам в которые он входит, и тривиальна в проекциях на остальные оси. Выделение стандартного представителя для любых семейств заговоров позволяет использовать нотацию $\Gamma | \mathfrak{A}$, под которой в дальнейшем будет пониматься $\Gamma | \Phi_{\mathfrak{A}}$. Эта нотация подчёркивает тот факт, что выбор конкретного пространства корреляции среди всех пространств заговоров необходимой структуры для нас значения не имеет, а стандартное пространство выступает в роли простейшего представителя, пригодного для практических вычислений.

\section{Трёхсторонний чёт-нечет}\label{sec:ch1/sec4}

В качестве элементарного примера конфликта, чувствительного к дополнительной информационной асимметрии, может выступать <<трёхсторонний чёт"~нечет>>. В этой игре каждый из трёх участников тайно выбирает <<орла>> или <<решку>> на своей монете и прижимает её к столу соответствующей стороной вверх, после чего все одновременно поднимают ладони и в зависимости от сложившейся комбинации делят фиксированный банк. Когда все три монеты лежат одной и той же стороной, раунд считается сыгранным вничью и игроки делят банк поровну. Если же совпали только две из них, то оказавшийся в меньшинстве игрок считается проигравшим и не получает доли при дележе банка. В матричной форме это можно описать так:
\begin{table} [htbp]
	\centering
	\begin{threeparttable}
		\caption{Трёхсторонний чёт"~нечет}
		\label{tab:coin3}
		\begin{tabular}{ |c|c|c|c|c| }
			\cline{1-2} \cline{4-5}
			\rule[-7pt]{0pt}{2em}$4, 4, 4$ &
			\rule[-7pt]{0pt}{2em}$6, 0, 6$ & \qquad\qquad\qquad &
			\rule[-7pt]{0pt}{2em}$6, 6, 0$ &
			\rule[-7pt]{0pt}{2em}$0, 6, 6$ \\
			\cline{1-2} \cline{4-5}
			\rule[-7pt]{0pt}{2em}$0, 6, 6$ &
			\rule[-7pt]{0pt}{2em}$6, 6, 0$ & \qquad\qquad\qquad &
			\rule[-7pt]{0pt}{2em}$6, 0, 6$ &
			\rule[-7pt]{0pt}{2em}$4, 4, 4$ \\
			\cline{1-2} \cline{4-5}
		\end{tabular}
	\end{threeparttable}
\end{table}

В таблице \ref{tab:coin3} первый игрок выбирает строку, второй "--- столбец, а третий "--- матрицу. Решением этой игры в чистых стратегиях являются два равновесия Нэша, соответствующие синхронным выборам одинаковых сторон всеми игроками. В смешанных стратегиях добавляется ещё одно вырожденное решение, когда каждый игрок делает случайный выбор между орлом и решкой с равными вероятностями. Все эти решения, очевидно, дают математическое ожидание платежей равное $(4,4,4)$. В рамках классической теории игр этим анализ конфликта и исчерпывается, однако добавление фактора информационной асимметрии делает ситуацию интереснее. Рассмотрим эту же игру в пространстве заговоров структуры $\{\{1,2\}\}$, т.е. в ситуации, когда игроки 1 и 2 имеют возможность использовать согласованные втайне от игрока 3 коррелированные стратегии. Пусть $\alpha \in \left[0, 1\right)$ "--- значение соответствующей секретной рулетки. Заговорщики могут использовать стратегии вида $\mathbf{s}^1, \mathbf{s}^2 : \left[0, 1\right) \rightarrow P_{\{\mathbf{head}, \mathbf{tail}\}}$, где под $P_{\{\mathbf{head}, \mathbf{tail}\}}$ понимаются всевозможные вероятностные меры на множестве $\{\mathbf{head}, \mathbf{tail}\}$, т.е. множество классических смешанных стратегий рассматриваемой игры. Аутсайдер же вынужден довольствоваться стратегиями $\mathbf{s}^3 \in P_{\{\mathbf{head}, \mathbf{tail}\}}$, поскольку не имеет доступа к рулетке заговора. Для того, чтобы оказаться в более выгодной по сравнению с равным дележом точке равновесия игроки 1 и 2 могут выбрать любые стратегии, обеспечивающие им равновероятный синхронный выбор стороны монетки:
\begin{equation*}
	\mathbf{s}^1(\alpha) = \mathbf{s}^2(\alpha) = \begin{cases}
		(1, 0), & \alpha < \frac{1}{2},\\
		(0, 1), & \alpha \ge \frac{1}{2}.
	\end{cases}
\end{equation*}

При этом, любая смешанная стратегия игрока 3 в силу независимости с рулеткой заговора обеспечивает ему совпадение с остальными ровно в половине случаев. Выплаты в сложившейся ситуации равны $(5,5,2)$, причём ни для одного из игроков нет выгодного индивидуального отклонения.

\FloatBarrier
