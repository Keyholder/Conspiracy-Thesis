\chapter{Модель заговоров}\label{ch:ch1}

\section{Коррелированное расширение игр в нормальной форме}\label{sec:ch1/sec1}

Базовым формализмом для описания дополнительной информационной асимметрии в сложившейся научной традиции выступает коррелированное расширение нормальной формы игр, предложенное Робертом Ауманом в \cite{Aumann74}. Для удобства его центральные элементы будут изложены здесь в нотации, адаптированной к русскоязычной среде. Рассмотрим игру в нормальной форме $\Gamma = \langle A, S^a, u^a(s), a \in A \rangle$. Конечное множество игроков здесь и далее везде обозначается как $A = \{1, \ldots, m\}$, а конечное множество наборов чистых стратегий "--- $S = S^1 \times \ldots \times S^m$. Помимо множества стратегий $S^a$ каждый игрок определяется платёжной функцией $u^a : S \rightarrow \mathbb{R}$.

Также рассмотрим вероятностное пространство\cite{Kolmogorov} $\langle \Omega, \mathfrak{B}, \mathbb{P} \rangle$, в котором реализуется наблюдаемое игроками состояние природы. Здесь $\Omega$ "--- множество всевозможных таких состояний, $\mathfrak{B}$ "--- $\sigma$"~алгебра подмножеств $\Omega$, а $\mathbb{P} : \mathfrak{B} \rightarrow \mathbb{R}_{\ge 0}$ "--- вероятностная мера. Каждому игроку $a \in A$ поставим в соответствие \emph{собственное подпространство} $\langle \Omega, \mathfrak{I}^a, \mathbb{P} \rangle$ такое, что $\mathfrak{I}^a \subseteq \mathfrak{B}$. При этом набор $\sigma$"~алгебр $\mathfrak{I} = (\mathfrak{I}^a, a \in A)$ отражает информированность игроков о состоянии природы. В описываемой ситуации это состояние не влияет на функции выигрышей непосредственно, выступая исключительно как способ синхронизации действий игроков. Это значит, что $\sigma$"~алгебра $\mathfrak{B}$ сама по себе не является существенным параметром модели, и измеримость по ней для $\mathbb{P}$ можно заменить измеримостью по $\mathfrak{I}^a, \forall a \in A$.

Отдельно следует заметить, что в оригинале для собственных подпространств игроков Аумановский формализм предполагал индивидуальность не только $\sigma$"~алгебр, но и соответствующих им мер, учитывая тем самым возможную субъективность оценок вероятности наступления тех или иных событий, что немаловажно в случаях, когда в качестве механизма корреляции выступают процессы, слишком сложные для объективного анализа (например спортивные соревнования). Однако, для целей данного исследования этот аспект не имеет большого смысла, поскольку в модели заговоров подразумевается, что их участники могут произвольным образом выбирать механизм корреляции, а в такой ситуации разумно ожидать, что люди будут использовать простые источники случайности с известным распределением (рулетки, кости, жребий и т.д.). По этой причине здесь и далее модель коррелированных стратегий используется в её упрощённой форме, с общей для всех игроков объективной вероятностной мерой в пространстве состояний природы.

Таким образом, получается $\Phi = \langle A, \Omega, \mathfrak{I}^a, \mathbb{P}, a \in A \rangle$ "--- набор параметров, характеризующих некоторое \emph{пространство корреляции} для произвольной игры с множеством игроков $A$. Отметим, что в играх с одним множеством игроков, но различными множествами чистых стратегий и функциями выигрыша можно применять одно и то же пространство корреляции. Полностью же \emph{коррелированное расширение игры} определяет пара $\Gamma | \Phi$. Опишем полученную новую игру в терминах нормальной формы:
\begin{equation*}
	\Gamma | \Phi = \langle A, \mathbf{S}^a, u^a(\mathbf{s}), a \in A \rangle.
\end{equation*}

Здесь множество $\mathbf{S}^a$ доступных игроку $a$ коррелированных стратегий состоит из всех $\mathfrak{I}^a$"~измеримых функций $\mathbf{s}^a : \Omega \rightarrow S^a$, отображающих множество возможных состояний природы на множество доступных ему чистых стратегий. Соответственно, функция выигрыша вычисляется по формуле математического ожидания случайной величины
\begin{equation*}
	u^a(\mathbf{s}) = \sum\limits_{s \in S} \mathbb{P}(\mathbf{s}^{-1}(s)) u^a(s),\;\mathbf{s}^{-1}(s) = \bigcap_{a \in A} (\mathbf{s}^a)^{-1}(s^a),
\end{equation*}
где $\mathbb{P}(\mathbf{s}^{-1}(s))$ выступают в роли коэффициентов распределения на матрице игры.

В своей статья Ауман демонстрирует силу вводимого формализма, показывая на примерах, как с его помощью в играх можно получать новые точки равновесия по Нэшу. Подбирая параметры пространств корреляции, можно конструировать не только решения с любыми выплатами из выпуклой оболочки векторов выплат в точках классического смешанного равновесия Нэша, но для некоторых игр даже решения, лежащие за пределами такой выпуклой оболочки. Это позволяет сформулировать ключевое для данной работы понятие:
\begin{definition}
	Пусть $\Gamma$ --- игра в нормальной форме c $m$ участниками, а $U \subseteq \mathbb{R}^m$ --- множество всех векторов выплат, достижимых в её смешанных равновесиях по Нэшу. Игра $\Gamma$ называется чувствительной к дополнительной информационной асимметрии, когда существует пространство корреляции $\Phi$ такое, что в игре $\Gamma | \Phi$ найдётся коррелированное равновесие по Нэшу с вектором выплат, не принадлежащим выпуклой оболочке множества $U$.
\end{definition}

Кроме того в этой же статье доказывается, что наличие в пространстве корреляции публичной вещественной рулетки влечёт выпуклость как множества достижимых выплат, так и множества равновесий Нэша в любой игре. Касательно коррелированного расширения игр в нормальной форме вышеизложенного вполне достаточно для понимания идей данной работы, более подробно же современное состояние знаний на эту тему можно проследить по публикациям, упомянутым в Приложении~\cref{app:A}.

\section{Изоморфизм пространств корреляции}\label{sec:ch1/sec2}

Следует отметить, что модель пространств корреляции в некотором смысле существенно избыточна, поскольку как таковые события из состояния природы значения не имеют и используются лишь в качестве сигналов для синхронизации стратегий игроков. Чтобы осмысленным образом рассуждать о влиянии, оказываемым дополнительной информационной асимметрией на исходы конфликтов, неизбежно требуется умение абстрагироваться от конкретных её источников, фокусируя внимание на структурных различиях в осведомлённости оппонентов. Если формально различные пространства корреляции оказываются полностью взаимозаменимы с теоретико"~игровой точки зрения, они должны быть отнесены к общему классу эквивалентности, чьё описание и является в действительности существенным параметром модели. Причём сразу следует заметить, что это касается не только тривиальных замен множества состояний природы на другое множество той же мощности с соответствующей биекцией остальных параметров пространства, но и более сложных случаев. Например, если в контексте некоторой игры группа игроков наблюдает общий сигнал в виде колеса вещественной рулетки, будет ли иметь значение наблюдение ими ещё и броска монетки? Здравый смысл подсказывает что любую общую стратегию с использованием рулетки и монетки можно легко превратить в эквивалентную для одной только рулетки, для чего достаточно поделить колесо пополам и отобразить отдельные варианты для орла и решки на полученные два сектора. Опишем этот феномен в виде изоморфизма:
\begin{definition}
	Разбиением пространства корреляции $\Phi = \langle A, \Omega, \mathfrak{I}^a, \mathbb{P}, a \in A \rangle$ в произвольное конечное множество исходов (кодомен) $X = X^1 \times ... \times X^m$ называется отображение $f : \Omega \rightarrow X$, состоящее из набора функций $(f^1, ..., f^m)$, где каждая $f^a : \Omega \rightarrow X^a$ измерима в $\mathfrak{I}^a$. Далее <<разбиение $f$ пространства корреляции $\Phi$>> будем сокращённо обозначать $f \models \Phi$.
\end{definition}

В контексте коррелированного расширения множествам исходов $X^a$ соответствуют множества чистых стратегий $S^a$, а элементам разбиения $f^a$ "--- коррелированные стратегии $\mathbf{s}^a$. Далее также будут использоваться отображения $f^{-1} : X \rightarrow 2^\Omega$, обратные к разбиениям пространств корреляции:
\begin{equation*}
	f^{-1}(x) = \bigcap_{a \in A} (f^a)^{-1}(x^a).
\end{equation*}
\begin{definition}
	Пространство $\Phi_1$ с мерой $\mathbb{P}_1$ называется отобразимым на $\Phi_2$ с мерой $\mathbb{P}_2$ (далее $\Phi_1 \precsim \Phi_2$), если их множества игроков совпадают и для любого разбиения $f_1 \models \Phi_1$ существует разбиение $f_2 \models \Phi_2$ с тем же кодоменом такое, что $\mathbb{P}_1 \circ f_1^{-1} = \mathbb{P}_2 \circ f_2^{-1}$. Взаимно отобразимые друг на друга пространства корреляции называются изоморфными (далее $\Phi_1 \sim \Phi_2$).
\end{definition}

Это определение легко проиллюстрировать упомянутым выше примером "--- для любого разбиения $f_1 : \left[0, 1\right) \times \{0, 1\} \rightarrow X$ пространства, состоящего из вещественной рулетки и симметричной монетки, можно построить соответствующий образ $f_2 : \left[0, 1\right) \rightarrow X$ в пространстве из одной только рулетки:
\begin{equation*}
	f_2(\alpha) = \begin{cases}
		f_1(2 \alpha, 0), & 0 \le \alpha < \frac{1}{2} \\
		f_1(2 \alpha - 1, 1), & \frac{1}{2} \le \alpha < 1
	\end{cases}.
\end{equation*}

Рефлексивность, симметричность и транзитивность вводимого при помощи данного определения изоморфизма очевидны, а значит это действительно отношение эквивалентности на множестве пространств корреляции. При этом, хотя определение изоморфизма дано в отрыве от коррелированного расширения игр, можно сформулировать следующую теорему:
\begin{definition}
	Для игры $\Gamma = \langle A, S^a, u^a(s), a \in A \rangle$ множеством достижимых выплат по отклонениям группы игроков $A_*$ от профиля стратегий $s$ будем называть
	\begin{equation*}
		U_\Gamma^{A_*}(s) = \{\bar{u} \mid \exists s_* \in S : u(s_*) = \bar{u}, \forall a \in A \setminus A_*, s^a = s_*^a\}.
	\end{equation*}
\end{definition}
\begin{theorem}[Об изоморфных пространствах]
	Пусть $\Phi_1 \sim \Phi_2$. Тогда для любой игры в нормальной форме $\Gamma$ с конечными множествами стратегий игроков её коррелированные расширения $\Gamma | \Phi_1$ и $\Gamma | \Phi_2$ обладают следующим свойством. Пусть $\mathbf{s}_1$ "--- некоторый профиль стратегий игры $\Gamma | \Phi_1$. Тогда существует $\mathbf{s}_2$ "--- профиль стратегий игры $\Gamma | \Phi_2$ такой, что $U_{\Gamma | \Phi_1}^{A_*}(\mathbf{s}_1) = U_{\Gamma | \Phi_2}^{A_*}(\mathbf{s}_2)$ для любой группы игроков $A_*$.
\end{theorem}

Эта теорема позволяет считать изоморфные пространства корреляции неразличимыми в контексте поиска равновесий, устойчивых к как индивидуальным, так и групповым отклонениям. Доказательство, представляющее собой упражнение в топологии без тесной связи с центральными идеями работы, вынесено в Приложение~\cref{app:B}.

\section{Пространства заговоров}\label{sec:ch1/sec3}

Получив осмысленный изоморфизм для пространств корреляции, можно выделить из всевозможных классов эквивалентности те, что представляют интерес с точки зрения моделирования дополнительной информационной асимметрии. Как показал Ауманн, выход за пределы выпуклой оболочки множества решений в смешанных стратегиях возможен если часть игроков (не менее двух) использует коррелированную стратегию, зависящую от события, о котором не проинформирован хотя бы один из остальных игроков. Подобную форму взаимовыгодного тайного согласования действий естественно называть <<заговором>>, а используемый для синхронизации сигнал "--- <<тайной>>. Пусть в пространстве корреляции $\Phi = \langle A, \Omega, \mathfrak{I}^a, \mathbb{P}, a \in A \rangle$ имеется тайна, т.е. вероятностное подпространство $\langle \Omega, \mathfrak{S}, \mathbb{P} \rangle, \mathfrak{S} \subseteq \sigma(\mathfrak{I}^1 \cup \ldots \cup \mathfrak{I}^m)$. Искомая асимметрия информированности предполагает, что некоторые игроки наблюдают события из $\mathfrak{S}$ (или другие, коррелирующие с ними), а некоторые "--- нет. Хотя теоретически можно представить заговор, степень вовлеченности в который варьируется от игрока к игроку (кто-то может наблюдать события из $\mathfrak{S}$ частично или опосредованно, через наблюдение других коррелирующих с ними событий), имеет смысл в первую очередь рассмотреть простейший случай "--- деление всех игроков на <<заговорщиков>> $A_{\mathfrak{S}} \subseteq A$, наблюдающих $\mathfrak{S}$ целиком, и <<аутсайдеров>> $A \setminus A_{\mathfrak{S}}$, в чьём поле зрения только события, независимые с элементами $\mathfrak{S}$.

Следующий логичный вопрос "--- о структуре самой тайны. Вполне можно представить, как заговорщики используют в её роли самые разные источники случайности: броски игральных костей, тасование колод карт, лотерейные розыгрыши и т.д., так что на первый взгляд неочевидно, можно ли ограничиться рассмотрением какого-то одного, естественного в контексте происходящего механизма. Положительный ответ можно получить, используя введённые выше отображения пространств корреляции. Если мы будем сравнивать всевозможные пространства, различающиеся только тайнами группы заговорщиков $A_* \subseteq A$, отношение $\precsim$ вводит на их множестве частичный порядок. Нижней гранью этого порядка будет вырожденное пространство корреляции $\Phi_{\emptyset}$, в котором тайна заговора состоит из единственного атомарного события с вероятностью $1$ "--- такое пространство отобразимо на любое другое и, очевидно, вообще не может быть использовано для корреляции стратегий. Верхняя грань интереснее "--- в её типичном представителе $\Phi_{\infty}$ тайна заговора представляет собой произвольное безатомическое пространство. Проще всего представить такой механизм корреляции в виде вещественной рулетки, вращение которой генерирует равномерно распределённую случайную величину в единичном полуинтервале $\left[0, 1\right)$. Наблюдающие рулетку заговорщики могут, разделяя её на сектора необходимых размеров, согласовать любой набор коррелированных стратегий в играх с конечным множеством исходов. Именно такой универсальный источник случайности и имеет смысл рассматривать в первую очередь, как предоставляющий максимум свободы выбора.

Наконец, следует подумать о пространствах корреляции с множественными тайнами. По самой природе коррелированного расширения, 

соблюдении двух условий допускают простое конечное описание и охватывают интересные случаи тайного согласования действий игроками. Для этого рассмотрим произвольное пространство корреляции $\Phi = \langle A, \Omega, \mathfrak{I}^a, \mathbb{P}, a \in A \rangle$. В этом пространстве для каждой непустой группы игроков $A_* \subseteq A$ определим следующее семейство событий:
\begin{equation*}
	\mathfrak{S}_\Phi^{A_*} = \{U \in \bigcap\limits_{a \in A_*} \mathfrak{I}^a \mid \mathbb{P}(U \cap V) = \mathbb{P}(U) \mathbb{P}(V), \forall V \in \sigma(\bigcup\limits_{a \in A \setminus A_*} \mathfrak{I}^a)\}.
\end{equation*}

Таким образом, $\mathfrak{S}_\Phi^{A_*}$ "--- множество всех таких событий, что о них осведомлены все члены $A_*$, и каждое событие попарно независимо со всеми событиями, известными не членам $A_*$ даже при объединении их знаний. Поскольку пересечение $\sigma$"~алгебр образует $\sigma$"~алгебру, и так как подмножество независимых с некоторым событием событий $\sigma$"~алгебры также образует $\sigma$"~алгебру, то $\mathfrak{S}_\Phi^{A_*}$ "--- $\sigma$"~алгебра. Это позволяет говорить о вероятностном подпространстве $\langle \Omega, \mathfrak{S}_\Phi^{A_*}, \mathbb{P} \rangle$, которое логично называть \emph{тайной} группы $A_*$. Важным для нас свойством этого подпространства является тип его меры \cite[с.~81]{Bogachev}. Особо выделим два случая: тайны с безатомическими мерами мы будем называть \emph{полными}, а тайны с тривиальными атомическими мерами с единственным атомом $\Omega$ "--- \emph{пустыми}. Это даёт возможность сформулировать следующее:
\begin{definition}
	Пространство корреляции $\langle A, \Omega, \mathfrak{I}^a, \mathbb{P}, a \in A \rangle$ назовём пространством заговоров структуры $\mathfrak{A} = \{A_1, ..., A_n\} \subseteq 2^A$, когда
	\begin{itemize}
		\item $\bigcup\limits_{i=0}^n A_i = A$;
		\item $\forall A_* \in \mathfrak{A}$, тайна $A_*$ полна;
		\item $\forall A_* \notin \mathfrak{A}$, тайна $A_*$ пуста;
		\item $\mathfrak{I}^a = \sigma(\bigcup\limits_{a \in A_* \in \mathfrak{A}}\mathfrak{S}_\Phi^{A_*})$, т.е. $\mathfrak{I}^a$ "--- наименьшая $\sigma$"~алгебра, включающая все $\sigma$"~алгебры тайн групп, в которые входит игрок $a$.
	\end{itemize}
\end{definition}

Проще говоря, пространствами заговоров называются такие пространства корреляции, в которых а) тайна любой группы игроков либо полна, либо пуста; б) каждый игрок входит хотя бы в одну группу с полной тайной и в) у игроков нет никаких знаний о состоянии природы, которые не порождались бы тайнами групп, которым они принадлежат. Построим для иллюстрации простейший пример такого пространства:
\begin{itemize}
	\item $A = \{1, 2, 3\}$,
	\item $\Omega = \left[0, 1\right)^2$,
	\item $\mathfrak{I}^1 = \sigma(\{\left[ 0, p_1 \right) \times \left[ 0, 1 \right) \mid 0 < p_1 \leq 1 \})$,
	\item $\mathfrak{I}^2 = \sigma(\{\left[ 0, 1 \right) \times \left[ 0, p_2 \right) \mid 0 < p_2 \leq 1 \})$,
	\item $\mathfrak{I}^3 = \sigma(\{\left[ 0, p_1 \right) \times \left[ 0, p_2 \right) \mid 0 < p_1 \leq 1, 0 < p_2 \leq 1 \})$,
	\item $\mathbb{P}$ "--- мера Лебега.
\end{itemize}

В этом примере пространство корреляции состоит из двух независимых вещественных рулеток, первый и второй игроки наблюдают по одной из них, а третий наблюдает обе. При этом выходит, что $\mathfrak{S}_\Phi^{\{1,3\}}$ совпадает с $\mathfrak{I}^1$, $\mathfrak{S}_\Phi^{\{2,3\}}$ совпадает с $\mathfrak{I}^2$, а для остальных групп $A_* \subseteq A$ соответствующая $\mathfrak{S}_\Phi^{A_*}$ тривиальна. Структурой пространства (или \emph{семейством заговоров}) называется множество всех групп игроков с полными тайнами. В вышеприведённом примере структура пространства $\mathfrak{A} = \{\{1,3\},\{2,3\}\}$. С точки зрения допустимых профилей стратегий это означает, что любая группа игроков, входящая в семейство заговоров, может использовать общую тайну для формирования коррелированной стратегии, причём игроки не входящие в эту группу не могут присоединиться к согласованному таким образом выбору стратегий. Напротив, группы игроков, не входящие в семейство заговоров, вышеописанной возможностью не располагают. Структуру пространства можно считать его исчерпывающим конечным описанием, поскольку
\begin{theorem} \label{the:struct}
	Все пространства заговоров одной структуры изоморфны.
\end{theorem}

\section{Форматирование текста}

Мы можем сделать \textbf{жирный текст} и \textit{курсив}.

\section{Ссылки}

Сошлёмся на библиографию.
Одна ссылка: \cite[с.~54]{Sokolov}\cite[с.~36]{Gaidaenko}.
Две ссылки: \cite{Sokolov,Gaidaenko}.
Ссылка на собственные работы: \cite{vakbib1, confbib2}.
Много ссылок: %\cite[с.~54]{Lermontov,Management,Borozda} % такой «фокус»
%вызывает biblatex warning относительно опции sortcites, потому что неясно, к
%какому источнику относится уточнение о страницах, а bibtex об этой проблеме
%даже не предупреждает
\cite{Lermontov, Management, Borozda, Marketing, Constitution, FamilyCode,
    Gost.7.0.53, Razumovski, Lagkueva, Pokrovski, Methodology, Berestova,
    Kriger}%
\ifnumequal{\value{bibliosel}}{0}{% Примеры для bibtex8
    \cite{Sirotko, Lukina, Encyclopedia, Nasirova}%
}{% Примеры для biblatex через движок biber
    \cite{Sirotko2, Lukina2, Encyclopedia2, Nasirova2}%
}%
.
И~ещё немного ссылок:~\cite{Article,Book,Booklet,Conference,Inbook,Incollection,Manual,Mastersthesis,
    Misc,Phdthesis,Proceedings,Techreport,Unpublished}
% Следует обратить внимание, что пробел после запятой внутри \cite{}
% обрабатывается ожидаемо, а пробел перед запятой, может вызывать проблемы при
% обработке ссылок.
\cite{medvedev2006jelektronnye, CEAT:CEAT581, doi:10.1080/01932691.2010.513279,
    Gosele1999161,Li2007StressAnalysis, Shoji199895, test:eisner-sample,
    test:eisner-sample-shorted, AB_patent_Pomerantz_1968, iofis_patent1960}%
\ifnumequal{\value{bibliosel}}{0}{% Примеры для bibtex8
}{% Примеры для biblatex через движок biber
    \cite{patent2h, patent3h, patent2}%
}%
.

\ifnumequal{\value{bibliosel}}{0}{% Примеры для bibtex8
Попытка реализовать несколько ссылок на конкретные страницы
для \texttt{bibtex} реализации библиографии:
[\citenum{Sokolov}, с.~54; \citenum{Gaidaenko}, с.~36].
}{% Примеры для biblatex через движок biber
Несколько источников (мультицитата):
% Тут специально написано по-разному тире, для демонстрации, что
% применение специальных тире в настоящий момент в biblatex приводит к непоказу
% "с.".
\cites[vii--x, 5, 7]{Sokolov}[v"--~x, 25, 526]{Gaidaenko}[vii--x, 5, 7]{Techreport},
работает только в \texttt{biblatex} реализации библиографии.
}%

Ссылки на собственные работы:~\cite{vakbib1, confbib1}.

Сошлёмся на приложения: Приложение~\cref{app:A}, Приложение~\cref{app:B2}.

Сошлёмся на формулу: формула~\cref{eq:equation1}.

Сошлёмся на изображение: рисунок~\cref{fig:knuth}.

Стандартной практикой является добавление к ссылкам префикса, характеризующего тип элемента.
Это не является строгим требованием, но~позволяет лучше ориентироваться в документах большого размера.
Например, для ссылок на~рисунки используется префикс \textit{fig},
для ссылки на~таблицу "--- \textit{tab}.

В таблице \cref{tab:tab_pref} приложения~\cref{app:B4} приведён список рекомендуемых
к использованию стандартных префиксов.

\section{Формулы}

Благодаря пакету \textit{icomma}, \LaTeX~одинаково хорошо воспринимает
в~качестве десятичного разделителя и запятую (\(3,1415\)), и точку (\(3.1415\)).

\subsection{Ненумерованные одиночные формулы}\label{subsec:ch1/sec3/sub1}

Вот так может выглядеть формула, которую необходимо вставить в~строку
по~тексту: \(x \approx \sin x\) при \(x \to 0\).

А вот так выглядит ненумерованная отдельностоящая формула c подстрочными
и надстрочными индексами:
\[
    (x_1+x_2)^2 = x_1^2 + 2 x_1 x_2 + x_2^2
\]

Формула с неопределенным интегралом:
\[
    \int f(\alpha+x)=\sum\beta
\]

При использовании дробей формулы могут получаться очень высокие:
\[
    \frac{1}{\sqrt{2}+
        \displaystyle\frac{1}{\sqrt{2}+
            \displaystyle\frac{1}{\sqrt{2}+\cdots}}}
\]

В формулах можно использовать греческие буквы:
%Все \original... команды заранее, ради этого примера, определены в Dissertation\userstyles.tex
\[
    \alpha\beta\gamma\delta\originalepsilon\epsilon\zeta\eta\theta%
    \vartheta\iota\kappa\varkappa\lambda\mu\nu\xi\pi\varpi\rho\varrho%
    \sigma\varsigma\tau\upsilon\originalphi\phi\chi\psi\omega\Gamma\Delta%
    \Theta\Lambda\Xi\Pi\Sigma\Upsilon\Phi\Psi\Omega
\]
\[%https://texfaq.org/FAQ-boldgreek
    \boldsymbol{\alpha\beta\gamma\delta\originalepsilon\epsilon\zeta\eta%
        \theta\vartheta\iota\kappa\varkappa\lambda\mu\nu\xi\pi\varpi\rho%
        \varrho\sigma\varsigma\tau\upsilon\originalphi\phi\chi\psi\omega\Gamma%
        \Delta\Theta\Lambda\Xi\Pi\Sigma\Upsilon\Phi\Psi\Omega}
\]

Для добавления формул можно использовать пары \verb+$+\dots\verb+$+ и \verb+$$+\dots\verb+$$+,
но~они считаются устаревшими.
Лучше использовать их функциональные аналоги \verb+\(+\dots\verb+\)+ и \verb+\[+\dots\verb+\]+.

\subsection{Ненумерованные многострочные формулы}\label{subsec:ch1/sec3/sub2}

Вот так можно написать две формулы, не нумеруя их, чтобы знаки <<равно>> были
строго друг под другом:
\begin{align}
    f_W & =  \min \left( 1, \max \left( 0, \frac{W_{soil} / W_{max}}{W_{crit}} \right)  \right), \nonumber \\
    f_T & =  \min \left( 1, \max \left( 0, \frac{T_s / T_{melt}}{T_{crit}} \right)  \right), \nonumber
\end{align}

Выровнять систему ещё и по переменной \( x \) можно, используя окружение
\verb|alignedat| из пакета \verb|amsmath|. Вот так:
\[
|x| = \left\{
\begin{alignedat}{2}
    &&x, \quad &\text{eсли } x\geqslant 0 \\
    &-&x, \quad & \text{eсли } x<0
\end{alignedat}
\right.
\]
Здесь первый амперсанд (в исходном \LaTeX\ описании формулы) означает
выравнивание по~левому краю, второй "--- по~\( x \), а~третий "--- по~слову
<<если>>. Команда \verb|\quad| делает большой горизонтальный пробел.

Ещё вариант:
\[
    |x|=
    \begin{cases}
        \phantom{-}x, \text{если } x \geqslant 0 \\
        -x, \text{если } x<0
    \end{cases}
\]

Кроме того, для  нумерованных формул \verb|alignedat| делает вертикальное
выравнивание номера формулы по центру формулы. Например, выравнивание
компонент вектора:
\begin{equation}
    \label{eq:2p3}
    \begin{alignedat}{2}
        {\mathbf{N}}_{o1n}^{(j)} = \,{\sin} \phi\,n\!\left(n+1\right)
        {\sin}\theta\,
        \pi_n\!\left({\cos} \theta\right)
        \frac{
        z_n^{(j)}\!\left( \rho \right)
        }{\rho}\,
        &{\boldsymbol{\hat{\mathrm e}}}_{r}\,+   \\
        +\,
        {\sin} \phi\,
        \tau_n\!\left({\cos} \theta\right)
        \frac{
        \left[\rho z_n^{(j)}\!\left( \rho \right)\right]^{\prime}
        }{\rho}\,
        &{\boldsymbol{\hat{\mathrm e}}}_{\theta}\,+   \\
        +\,
        {\cos} \phi\,
        \pi_n\!\left({\cos} \theta\right)
        \frac{
        \left[\rho z_n^{(j)}\!\left( \rho \right)\right]^{\prime}
        }{\rho}\,
        &{\boldsymbol{\hat{\mathrm e}}}_{\phi}\:.
    \end{alignedat}
\end{equation}

Ещё об отступах. Иногда для лучшей <<читаемости>> формул полезно
немного исправить стандартные интервалы \LaTeX\ с учётом логической
структуры самой формулы. Например в формуле~\cref{eq:2p3} добавлен
небольшой отступ \verb+\,+ между основными сомножителями, ниже
результат применения всех вариантов отступа:
\begin{align*}
    \backslash!             & \quad f(x) = x^2\! +3x\! +2         \\
    \mbox{по-умолчанию}     & \quad f(x) = x^2+3x+2               \\
    \backslash,             & \quad f(x) = x^2\, +3x\, +2         \\
    \backslash{:}           & \quad f(x) = x^2\: +3x\: +2         \\
    \backslash;             & \quad f(x) = x^2\; +3x\; +2         \\
    \backslash \mbox{space} & \quad f(x) = x^2\ +3x\ +2           \\
    \backslash \mbox{quad}  & \quad f(x) = x^2\quad +3x\quad +2   \\
    \backslash \mbox{qquad} & \quad f(x) = x^2\qquad +3x\qquad +2
\end{align*}

Можно использовать разные математические алфавиты:
\begin{align}
    \mathcal{ABCDEFGHIJKLMNOPQRSTUVWXYZ} \nonumber  \\
    \mathfrak{ABCDEFGHIJKLMNOPQRSTUVWXYZ} \nonumber \\
    \mathbb{ABCDEFGHIJKLMNOPQRSTUVWXYZ} \nonumber
\end{align}

Посмотрим на систему уравнений на примере аттрактора Лоренца:

\[
\left\{
\begin{array}{rl}
    \dot x = & \sigma (y-x)  \\
    \dot y = & x (r - z) - y \\
    \dot z = & xy - bz
\end{array}
\right.
\]

А для вёрстки матриц удобно использовать многоточия:
\[
    \left(
        \begin{array}{ccc}
            a_{11} & \ldots & a_{1n} \\
            \vdots & \ddots & \vdots \\
            a_{n1} & \ldots & a_{nn} \\
        \end{array}
    \right)
\]

\subsection{Нумерованные формулы}\label{subsec:ch1/sec3/sub3}

А вот так пишется нумерованная формула:
\begin{equation}
    \label{eq:equation1}
    e = \lim_{n \to \infty} \left( 1+\frac{1}{n} \right) ^n
\end{equation}

Нумерованных формул может быть несколько:
\begin{equation}
    \label{eq:equation2}
    \lim_{n \to \infty} \sum_{k=1}^n \frac{1}{k^2} = \frac{\pi^2}{6}
\end{equation}

Впоследствии на формулы~\cref{eq:equation1, eq:equation2} можно ссылаться.

Сделать так, чтобы номер формулы стоял напротив средней строки, можно,
используя окружение \verb|multlined| (пакет \verb|mathtools|) вместо
\verb|multline| внутри окружения \verb|equation|. Вот так:
\begin{equation} % \tag{S} % tag - вписывает свой текст
    \label{eq:equation3}
    \begin{multlined}
        1+ 2+3+4+5+6+7+\dots + \\
        + 50+51+52+53+54+55+56+57 + \dots + \\
        + 96+97+98+99+100=5050
    \end{multlined}
\end{equation}

Уравнения~\cref{eq:subeq_1,eq:subeq_2} демонстрируют возможности
окружения \verb|\subequations|.
\begin{subequations}
    \label{eq:subeq_1}
    \begin{gather}
        y = x^2 + 1 \label{eq:subeq_1-1} \\
        y = 2 x^2 - x + 1 \label{eq:subeq_1-2}
    \end{gather}
\end{subequations}
Ссылки на отдельные уравнения~\cref{eq:subeq_1-1,eq:subeq_1-2,eq:subeq_2-1}.
\begin{subequations}
    \label{eq:subeq_2}
    \begin{align}
        y & = x^3 + x^2 + x + 1 \label{eq:subeq_2-1} \\
        y & = x^2
    \end{align}
\end{subequations}

\subsection{Форматирование чисел и размерностей величин}\label{sec:units}

Числа форматируются при помощи команды \verb|\num|:
\num{5,3};
\num{2,3e8};
\num{12345,67890};
\num{2,6 d4};
\num{1+-2i};
\num{.3e45};
\num[exponent-base=2]{5 e64};
\num[exponent-base=2,exponent-to-prefix]{5 e64};
\num{1.654 x 2.34 x 3.430}
\num{1 2 x 3 / 4}.
Для написания последовательности чисел можно использовать команды \verb|\numlist| и \verb|\numrange|:
\numlist{10;30;50;70}; \numrange{10}{30}.
Значения углов можно форматировать при помощи команды \verb|\ang|:
\ang{2.67};
\ang{30,3};
\ang{-1;;};
\ang{;-2;};
\ang{;;-3};
\ang{300;10;1}.

Обратите внимание, что ГОСТ запрещает использование знака <<->> для обозначения отрицательных чисел
за исключением формул, таблиц и~рисунков.
Вместо него следует использовать слово <<минус>>.

Размерности можно записывать при помощи команд \verb|\si| и \verb|\SI|:
\si{\farad\squared\lumen\candela};
\si{\joule\per\mole\per\kelvin};
\si[per-mode = symbol-or-fraction]{\joule\per\mole\per\kelvin};
\si{\metre\per\second\squared};
\SI{0.10(5)}{\neper};
\SI{1.2-3i e5}{\joule\per\mole\per\kelvin};
\SIlist{1;2;3;4}{\tesla};
\SIrange{50}{100}{\volt}.
Список единиц измерений приведён в таблицах~\cref{tab:unit:base,
    tab:unit:derived,tab:unit:accepted,tab:unit:physical,tab:unit:other}.
Приставки единиц приведены в~таблице~\cref{tab:unit:prefix}.

С дополнительными опциями форматирования можно ознакомиться в~описании пакета \texttt{siunitx};
изменить или добавить единицы измерений можно в~файле \texttt{siunitx.cfg}.

\begin{table}
    \centering
    \captionsetup{justification=centering} % выравнивание подписи по-центру
    \caption{Основные величины СИ}\label{tab:unit:base}
    \begin{tabular}{llc}
        \toprule
        Название  & Команда                 & Символ         \\
        \midrule
        Ампер     & \verb|\ampere| & \si{\ampere}   \\
        Кандела   & \verb|\candela| & \si{\candela}  \\
        Кельвин   & \verb|\kelvin| & \si{\kelvin}   \\
        Килограмм & \verb|\kilogram| & \si{\kilogram} \\
        Метр      & \verb|\metre| & \si{\metre}    \\
        Моль      & \verb|\mole| & \si{\mole}     \\
        Секунда   & \verb|\second| & \si{\second}   \\
        \bottomrule
    \end{tabular}
\end{table}

\begin{table}
    \small
    \centering
    \begin{threeparttable}% выравнивание подписи по границам таблицы
        \caption{Производные единицы СИ}\label{tab:unit:derived}
        \begin{tabular}{llc|llc}
            \toprule
            Название       & Команда                 & Символ              & Название & Команда & Символ \\
            \midrule
            Беккерель      & \verb|\becquerel| & \si{\becquerel}     &
            Ньютон         & \verb|\newton| & \si{\newton}                                      \\
            Градус Цельсия & \verb|\degreeCelsius| & \si{\degreeCelsius} &
            Ом             & \verb|\ohm| & \si{\ohm}                                         \\
            Кулон          & \verb|\coulomb| & \si{\coulomb}       &
            Паскаль        & \verb|\pascal| & \si{\pascal}                                      \\
            Фарад          & \verb|\farad| & \si{\farad}         &
            Радиан         & \verb|\radian| & \si{\radian}                                      \\
            Грей           & \verb|\gray| & \si{\gray}          &
            Сименс         & \verb|\siemens| & \si{\siemens}                                     \\
            Герц           & \verb|\hertz| & \si{\hertz}         &
            Зиверт         & \verb|\sievert| & \si{\sievert}                                     \\
            Генри          & \verb|\henry| & \si{\henry}         &
            Стерадиан      & \verb|\steradian| & \si{\steradian}                                   \\
            Джоуль         & \verb|\joule| & \si{\joule}         &
            Тесла          & \verb|\tesla| & \si{\tesla}                                       \\
            Катал          & \verb|\katal| & \si{\katal}         &
            Вольт          & \verb|\volt| & \si{\volt}                                        \\
            Люмен          & \verb|\lumen| & \si{\lumen}         &
            Ватт           & \verb|\watt| & \si{\watt}                                        \\
            Люкс           & \verb|\lux| & \si{\lux}           &
            Вебер          & \verb|\weber| & \si{\weber}                                       \\
            \bottomrule
        \end{tabular}
    \end{threeparttable}
\end{table}

\begin{table}
    \centering
    \begin{threeparttable}% выравнивание подписи по границам таблицы
        \caption{Внесистемные единицы}\label{tab:unit:accepted}

        \begin{tabular}{llc}
            \toprule
            Название        & Команда                 & Символ          \\
            \midrule
            День            & \verb|\day| & \si{\day}       \\
            Градус          & \verb|\degree| & \si{\degree}    \\
            Гектар          & \verb|\hectare| & \si{\hectare}   \\
            Час             & \verb|\hour| & \si{\hour}      \\
            Литр            & \verb|\litre| & \si{\litre}     \\
            Угловая минута  & \verb|\arcminute| & \si{\arcminute} \\
            Угловая секунда & \verb|\arcsecond| & \si{\arcsecond} \\ %
            Минута          & \verb|\minute| & \si{\minute}    \\
            Тонна           & \verb|\tonne| & \si{\tonne}     \\
            \bottomrule
        \end{tabular}
    \end{threeparttable}
\end{table}

\begin{table}
    \centering
    \captionsetup{justification=centering}
    \caption{Внесистемные единицы, получаемые из эксперимента}\label{tab:unit:physical}
    \begin{tabular}{llc}
        \toprule
        Название                & Команда                 & Символ                 \\
        \midrule
        Астрономическая единица & \verb|\astronomicalunit| & \si{\astronomicalunit} \\
        Атомная единица массы   & \verb|\atomicmassunit| & \si{\atomicmassunit}   \\
        Боровский радиус        & \verb|\bohr| & \si{\bohr}             \\
        Скорость света          & \verb|\clight| & \si{\clight}           \\
        Дальтон                 & \verb|\dalton| & \si{\dalton}           \\
        Масса электрона         & \verb|\electronmass| & \si{\electronmass}     \\
        Электрон Вольт          & \verb|\electronvolt| & \si{\electronvolt}     \\
        Элементарный заряд      & \verb|\elementarycharge| & \si{\elementarycharge} \\
        Энергия Хартри          & \verb|\hartree| & \si{\hartree}          \\
        Постоянная Планка       & \verb|\planckbar| & \si{\planckbar}        \\
        \bottomrule
    \end{tabular}
\end{table}

\begin{table}
    \centering
    \begin{threeparttable}% выравнивание подписи по границам таблицы
        \caption{Другие внесистемные единицы}\label{tab:unit:other}
        \begin{tabular}{llc}
            \toprule
            Название                  & Команда                 & Символ             \\
            \midrule
            Ангстрем                  & \verb|\angstrom| & \si{\angstrom}     \\
            Бар                       & \verb|\bar| & \si{\bar}          \\
            Барн                      & \verb|\barn| & \si{\barn}         \\
            Бел                       & \verb|\bel| & \si{\bel}          \\
            Децибел                   & \verb|\decibel| & \si{\decibel}      \\
            Узел                      & \verb|\knot| & \si{\knot}         \\
            Миллиметр ртутного столба & \verb|\mmHg| & \si{\mmHg}         \\
            Морская миля              & \verb|\nauticalmile| & \si{\nauticalmile} \\
            Непер                     & \verb|\neper| & \si{\neper}        \\
            \bottomrule
        \end{tabular}
    \end{threeparttable}
\end{table}

\begin{table}
    \small
    \centering
    \begin{threeparttable}% выравнивание подписи по границам таблицы
        \caption{Приставки СИ}\label{tab:unit:prefix}
        \begin{tabular}{llcc|llcc}
            \toprule
            Приставка & Команда                  & Символ      & Степень &
            Приставка & Команда                  & Символ      & Степень   \\
            \midrule
            Иокто     & \verb|\yocto|  & \si{\yocto} & -24     &
            Дека      & \verb|\deca|  & \si{\deca}  & 1         \\
            Зепто     & \verb|\zepto|  & \si{\zepto} & -21     &
            Гекто     & \verb|\hecto|  & \si{\hecto} & 2         \\
            Атто      & \verb|\atto|  & \si{\atto}  & -18     &
            Кило      & \verb|\kilo|  & \si{\kilo}  & 3         \\
            Фемто     & \verb|\femto|  & \si{\femto} & -15     &
            Мега      & \verb|\mega|  & \si{\mega}  & 6         \\
            Пико      & \verb|\pico|  & \si{\pico}  & -12     &
            Гига      & \verb|\giga|  & \si{\giga}  & 9         \\
            Нано      & \verb|\nano|  & \si{\nano}  & -9      &
            Терра     & \verb|\tera|  & \si{\tera}  & 12        \\
            Микро     & \verb|\micro|  & \si{\micro} & -6      &
            Пета      & \verb|\peta|  & \si{\peta}  & 15        \\
            Милли     & \verb|\milli|  & \si{\milli} & -3      &
            Екса      & \verb|\exa|  & \si{\exa}   & 18        \\
            Санти     & \verb|\centi|  & \si{\centi} & -2      &
            Зетта     & \verb|\zetta|  & \si{\zetta} & 21        \\
            Деци      & \verb|\deci| & \si{\deci}  & -1      &
            Иотта     & \verb|\yotta| & \si{\yotta} & 24        \\
            \bottomrule
        \end{tabular}
    \end{threeparttable}
\end{table}

\subsection{Заголовки с формулами: \texorpdfstring{\(a^2 + b^2 = c^2\)}{%
        a\texttwosuperior\ + b\texttwosuperior\ = c\texttwosuperior},
    \texorpdfstring{\(\left\vert\textrm{{Im}}\Sigma\left(
            \protect\varepsilon\right)\right\vert\approx const\)}{|ImΣ (ε)| ≈ const},
    \texorpdfstring{\(\sigma_{xx}^{(1)}\)}{σ\_\{xx\}\textasciicircum\{(1)\}}
}\label{subsec:with_math}

Пакет \texttt{hyperref} берёт текст для закладок в pdf-файле из~аргументов
команд типа \verb|\section|, которые могут содержать математические формулы,
а~также изменения цвета текста или шрифта, которые не отображаются в~закладках.
Чтобы использование формул в заголовках не вызывало в~логе компиляции появление
предупреждений типа <<\texttt{Token not allowed in~a~PDF string
    (Unicode):(hyperref) removing...}>>, следует использовать конструкцию
\verb|\texorpdfstring{}{}|, где в~первых фигурных скобках указывается
формула, а~во~вторых "--- запись формулы для закладок.

\section{Рецензирование текста}\label{sec:markup}

В шаблоне для диссертации и автореферата заданы команды рецензирования.
Они видны при компиляции шаблона в режиме черновика или при установке
соответствующей настройки (\verb+showmarkup+) в~файле \verb+common/setup.tex+.

Команда \verb+\todo+ отмечает текст красным цветом.
\todo{Например, так.}

Команда \verb+\note+ позволяет выбрать цвет текста.
\note{Чёрный, } \note[red]{красный, } \note[green]{зелёный, }
\note[blue]{синий.} \note[orange]{Обратите внимание на ширину и расстановку
    формирующихся пробелов, в~результате приведённой записи (зависит также
    от~применяемого компилятора).}

Окружение \verb+commentbox+ также позволяет выбрать цвет.

\begin{commentbox}[red]
    Красный текст.

    Несколько параграфов красного текста.
\end{commentbox}

\begin{commentbox}[blue]
    Синяя формула.

    \begin{equation}
        \alpha + \beta = \gamma
    \end{equation}
\end{commentbox}

\verb+commentbox+ позволяет закомментировать участок кода в~режиме чистовика.
Чтобы убрать кусок кода для всех режимов, можно использовать окружение
\verb+comment+.

\begin{comment}
Этот текст всегда скрыт.
\end{comment}

\section{Работа со списком сокращений и~условных обозначений}\label{sec:acronyms}

С помощью пакета \texttt{nomencl} можно создавать удобный сортированный список
сокращений и условных обозначений во время написания текста. Вызов
\verb+\nomenclature+ добавляет нужный символ или сокращение с~описанием
в~список, который затем печатается вызовом \verb+\printnomenclature+
в~соответствующем разделе.
Для того, чтобы эти операции прошли, потребуется дополнительный вызов
\verb+makeindex -s nomencl.ist -o %.nls %.nlo+ в~командной строке, где вместо
\verb+%+ следует подставить имя главного файла проекта (\verb+dissertation+
для этого шаблона).
Затем потребуется один или два дополнительных вызова компилятора проекта.
\begin{equation}
    \omega = c k,
\end{equation}
где \( \omega \) "--- частота света, \( c \) "--- скорость света, \( k \) "---
модуль волнового вектора.
\nomenclature{\(\omega\)}{частота света\nomrefeq}
\nomenclature{\(c\)}{скорость света\nomrefpage}
\nomenclature{\(k\)}{модуль волнового вектора\nomrefeqpage}
Использование
\begin{verbatim}
\nomenclature{\(\omega\)}{частота света\nomrefeq}
\nomenclature{\(c\)}{скорость света\nomrefpage}
\nomenclature{\(k\)}{модуль волнового вектора\nomrefeqpage}
\end{verbatim}
после уравнения добавит в список условных обозначений три записи.
Ссылки \verb+\nomrefeq+ на последнее уравнение, \verb+\nomrefpage+ "--- на
страницу, \verb+\nomrefeqpage+ "--- сразу на~последнее уравнение и~на~страницу,
можно опускать и~не~использовать.

Группировкой и сортировкой пунктов в списке можно управлять с~помощью указания
дополнительных аргументов к команде \verb+nomenclature+.
Например, при вызове
\begin{verbatim}
\nomenclature[03]{\( \hbar \)}{постоянная Планка}
\nomenclature[01]{\( G \)}{гравитационная постоянная}
\end{verbatim}
\( G \) будет стоять в списке выше, чем \( \hbar \).
Для корректных вертикальных отступов между строками в описании лучше
не~использовать многострочные формулы в~списке обозначений.

\nomenclature{%
    \( \begin{rcases}
        a_n \\
        b_n
    \end{rcases} \)%
}{коэффициенты разложения Ми в дальнем поле соответствующие электрическим и
    магнитным мультиполям}
\nomenclature[a\( e \)]{\( {\boldsymbol{\hat{\mathrm e}}} \)}{единичный вектор}
\nomenclature{\( E_0 \)}{амплитуда падающего поля}
\nomenclature{\( j \)}{тип функции Бесселя}
\nomenclature{\( k \)}{волновой вектор падающей волны}
\nomenclature{%
    \( \begin{rcases}
        a_n \\
        b_n
    \end{rcases} \)%
}{и снова коэффициенты разложения Ми в дальнем поле соответствующие
    электрическим и магнитным мультиполям. Добавлено много текста, так что
    описание группы условных обозначений значительно превысило высоту этой
    группы...}
\nomenclature{\( L \)}{общее число слоёв}
\nomenclature{\( l \)}{номер слоя внутри стратифицированной сферы}
\nomenclature{\( \lambda \)}{длина волны электромагнитного излучения в вакууме}
\nomenclature{\( n \)}{порядок мультиполя}
\nomenclature{%
    \( \begin{rcases}
        {\mathbf{N}}_{e1n}^{(j)} & {\mathbf{N}}_{o1n}^{(j)} \\
        {\mathbf{M}_{o1n}^{(j)}} & {\mathbf{M}_{e1n}^{(j)}}
    \end{rcases} \)%
}{сферические векторные гармоники}
\nomenclature{\( \mu \)}{магнитная проницаемость в вакууме}
\nomenclature{\( r, \theta, \phi \)}{полярные координаты}
\nomenclature{\( \omega \)}{частота падающей волны}

С помощью \verb+nomenclature+ можно включать в~список сокращения,
не~используя их~в~тексте.
% запись сокращения в список происходит командой \nomenclature,
% а не употреблением самого сокращения
\nomenclature{FEM}{finite element method, метод конечных элементов}
\nomenclature{FIT}{finite integration technique, метод конечных интегралов}
\nomenclature{FMM}{fast multipole method, быстрый метод многополюсника}
\nomenclature{FVTD}{finite volume time-domain, метод конечных объёмов
    во~временной области}
\nomenclature{MLFMA}{multilevel fast multipole algorithm, многоуровневый
    быстрый алгоритм многополюсника}
\nomenclature{BEM}{boundary element method, метод граничных элементов}
\nomenclature{CST MWS}{Computer Simulation Technology Microwave Studio
    программа для компьютерного моделирования уравнен Максвелла}
\nomenclature{DDA}{discrete dipole approximation, приближение дискретиных
    диполей}
\nomenclature{FDFD}{finite difference frequency domain, метод конечных
    разностей в~частотной области}
\nomenclature{FDTD}{finite difference time domain, метод конечных разностей
    во~временной области}
\nomenclature{MoM}{method of moments, метод моментов}
\nomenclature{MSTM}{multiple sphere T-Matrix, метод Т-матриц для множества
    сфер}
\nomenclature{PSTD}{pseudospectral time domain method, псевдоспектральный метод
    во~временной области}
\nomenclature{TLM}{transmission line matrix method, метод матриц линий передач}

\FloatBarrier
