\chapter{Коллективная рациональность в играх с заговорами}\label{ch:ch2}

\section{Проблема планирования заданий}\label{sec:ch2/sec1}

Концепция равновесия по Нэшу, будучи фундаментом классических моделей теории игр, сама по себе зачастую оказывается недостаточно сильным формализмом. При анализе многосторонних конфликтов нередко возникают ситуации, когда множество равновесий по Нэшу слишком велико, чтобы считать его полноценным решением игры. В этом случае на помощь исследователям приходят критерии коллективной рациональности "--- выбирая среди равновесных точек оптимальные по Парето или Слейтеру, иногда можно значительно сузить пространство решений за счёт вполне естественного исключения заведомо невыгодных для всех участников. Присутствие дополнительной информационной асимметрии при этом создаёт дополнительные затруднения, поскольку предлагаемая модель подразумевает неизбежный элемент антагонизма интересов "--- в новых точках равновесия увеличение выигрышей участвующих в корреляции происходит за счёт сокращения выплат тех, кто не может к ней присоединиться, тем самым делая непродуктивными обычные критерии коллективной рациональности. Для демонстрации этого эффекта рассмотрим тривиальное обобщение классической проблемы планирования заданий \cite{Koutsoupias}, занимающей важное место в концептуальном ландшафте теории игр. При помощи этого конфликта хорошо иллюстрируются понятия \emph{цены анархии} и \emph{цены стабильности}, давая вероятно самый выразительный пример того, насколько равновесия Нэша в одной и той же игре могут различаться в смысле их глобальной оптимальности. Нам, однако, необходимо взглянуть на эту игру под другим углом, при котором понятие цен анархии и стабильности теряет смысл, уступая место чувствительности к дополнительной информационной асимметрии.

Начнём с классической модели планирования заданий. В вычислительном центре работают $m$ сотрудников, каждому из которых поручено произвести некое вычисление. В их распоряжении находятся $n$ компьютеров, на каждом из которых может быть запущена одна или несколько программ, производящих вычисления сотрудников. Машины отличаются архитектурными особенностями, что задаётся матрицей констант $t_i^a \ge 0$, обозначающих время выполнения программы сотрудника $a=\overline{1,\ldots,m}$ на компьютере $i=\overline{1,\ldots,n}$. Каждое вычисление может производиться только одним устройством. Несколько программ на одном компьютере выполняются последовательно, но результаты их работы выводятся одновременно после остановки последней из них. Таким образом, выгода сотрудника заключается в выборе такого компьютера для своего вычисления, что для него оказывается минимальным суммарное время выполнения всех запущенных программ. Опишем происходящее в терминах игры нормальной формы
\begin{equation}\label{intro:game}
	\Gamma = \langle A, S^a, u^a(s), a \in A \rangle
\end{equation}
с параметрами:
\begin{itemize}
	\item $A = \{ 1, \ldots, m \}$;
	\item $S^1 = \ldots = S^m = \{ 1, \ldots, n \}$;
	\item $u^a(s) = -t_{s^a}(s)$, где $t_i(s) = \sum\limits_{a \in A, s^a = i} t_i^a$.
\end{itemize}

Здесь следует заострить внимание на определении функции выплат. Выбрав $u^a(s) = -t_{s^a}(s)$, мы моделируем ситуацию, когда все вычисления должны были быть готовы ещё вчера, и сотрудники напрямую штрафуются за каждую лишнюю секунду, пока их результаты не лягут на стол начальству. Однако, можно смоделировать и менее напряжённый рабочий момент, взяв, например, ступенчатую функцию выплат:
\begin{equation*}%\label{intro:stairs}
	u^a(s) = \begin{cases}
		u^a_{GOOD}, &t_{s^a}(s) < t^a_{DEADLINE};\\
		u^a_{LATE}, &t_{s^a}(s) \ge t^a_{DEADLINE}.
	\end{cases}
\end{equation*}

При этом получается, что для каждого игрока $a$ назначен срок успешного выполнения задания $t^a_{DEADLINE}$, уложившись в который, он получает фиксированную выплату $u^a_{GOOD}$ (с учётом премии), а не уложившись "--- $u^a_{LATE}$ (обычную ставку). Можно придумать и более сложные схемы поощрения сотрудников, так что сформулируем сразу в общем виде:
\begin{equation}\label{intro:rewards}
	u^a(s) = v^a(t_{s^a}(s)),
\end{equation}
где $v^a(t)$ "--- монотонно невозрастающая функция оплаты за срочность по заданию сотрудника $a$. Любая игра в нормальной форме, построенная по схеме \ref{intro:game} с платёжной функцией вида \ref{intro:rewards}, в сущности является проблемой планирования задач. При этом условие монотонного невозрастания $v^a(t)$ необходимо, поскольку на него в явном виде опирается доказательство считающегося важным свойства этой игры "--- равенство $1$ цены стабильности \cite{Agussurja}. Напомним, в оптимизационных задачах с эгоистичными агентами цена стабильности представляет собой соотношение $\frac{t_{NASH}}{t_{BEST}}$, где $t_{NASH}$ "--- значение наилучшего из равновесий Нэша, а $t_{BEST}$ "--- значение глобально оптимального решения. Это означает, что в проблеме планирования задач среди всех ситуаций, минимизирующих время до остановки последнего компьютера, обязательно найдутся равновесия по Нэшу. Однако, в данной работе предлагается на время забыть о минимизации общей продолжительности вычислений и вместо этого проанализировать, какие новые свойства модели могут проявиться в отсутствие такого ограничения на монотонность.

\section{Штраф за индивидуализм}\label{sec:ch2/sec2}

Представим вычислительный центр с компьютерами, требующими сложного техобслуживания после смены, если на них запускали хотя бы одну задачу. Если поощрять укладывающихся в дедлайны сотрудников невзирая на это, несложно представить ситуацию, когда они, в стремлении любой ценой гарантировать себе премию, будут раскидывать задачи по неразумно большому количеству компьютеров. Перед лицом такой перспективы у руководства может возникнуть соблазн стимулировать своих сотрудников избегать систематической недозагрузки машин при помощи штрафов. Это может быть смоделировано ступенчатой функцией оплаты за срочность следующего вида:
\begin{equation*}
	v^a(t) = \begin{cases}
		\begin{aligned}
			u^a_{HAST}, \qquad & & & \ t < t^a_{BREAKAWAY} \hspace{-10pt} & ;\\
			u^a_{GOOD}, \qquad & t^a_{BREAKAWAY} \hspace{-8pt} & \le & \ t < t^a_{DEADLINE} & ;\\
			u^a_{LATE}, \qquad & t^a_{DEADLINE} & \le & \ t & .
		\end{aligned}
	\end{cases}
\end{equation*}

Здесь для каждого сотрудника $a$ зафиксирован не только дедлайн $t^a_{DEADLINE}$, к которому требуется успеть, чтобы получить премию, но и минимальная загруженность используемого компьютера $t^a_{BREAKAWAY}$, которой нужно достигнуть, чтобы не нарваться на штраф за нерациональное расходование вычислительных ресурсов (причём $u^a_{HAST} < u^a_{LATE} < u^a_{GOOD}$). Размер минимальной загруженности может устанавливаться, например, в зависимости от важности соответствующей задачи "--- если срочное получение результата окупает использование дополнительных машин, то его можно сделать пониже или вовсе приравнять к нулю. Если же наоборот задача не так уж и важна, то большая минимальная загруженность заставит соответствующего сотрудника вспомнить об интересах фирмы и скооперироваться с коллегами.

Как видно, сотрудникам в данном случае приходится руководствоваться немонотонной функцией оплаты за срочность, что создаёт некоторые эффекты, несвойственные для классической формулировки проблемы планирования заданий. Во-первых, при такой постановке ожидаемо не во всякой игре цена стабильности обязана равняться $1$. Достаточно рассмотреть игру с $2$ сотрудниками и $2$ одинаковыми компьютерами:
\begin{itemize}
	\item $t_1^1 = t_2^1 = 8,\\ t_1^2 = t_2^2 = 2$;
	\item $t^1_{DEADLINE} = t^2_{DEADLINE} = 9,\\ t^1_{BREAKAWAY} = t^2_{BREAKAWAY} = 3$;
	\item $u^1_{HAST} < u^1_{LATE} < u^1_{GOOD},\\ u^2_{HAST} < u^2_{LATE} < u^2_{GOOD}$.
\end{itemize}

Поскольку первый игрок, заведомо не имеющий проблем с недозагрузкой, укладывается в дедлайн только если компьютер будет в его безраздельном пользовании, то очевидно, что сочетания стратегий, в которых оба игрока выбирают одну машину, равновесиями Нэша быть не могут. Похожим образом, поскольку второму игроку выгоднее опоздать с расчётами нежели быть наказанным за использование отдельного компьютера для недостаточно большой задачи ($u^a_{HAST} < u^a_{LATE}$), равновесиями Нэша не могут быть и те ситуации, где каждый из сотрудников использует свою машину. По структуре выплат игра оказывается неотличима от игры в чёт-нечет, вовсе не имеющей решений в чистых стратегиях и с единственным равновесием Нэша в точке независимого равновероятного выбора между альтернативами обоими игроками. При этом максимально загруженная машина с вероятностью $\frac{1}{2}$ проработает либо $10$ часов при совпадении их выбора, либо $8$ при несовпадении, что даёт математическое ожидание цены стабильности, равное $\frac{9}{8}$. По сути, при отказе от монотонности функции оплаты за срочность вряд ли имеет смысл вообще рассуждать о ценах стабильности и анархии, поскольку класс игр теперь начинает включать и такие, которые явно не имеют отношения к поискам минимума продолжительности вычислений.

\section{Смешанные равновесия игры $\Gamma^3_n$}\label{sec:ch2/sec3}

Планирование заданий с немонотонной отдачей представляет собой довольно обширный класс конфликтов, анализ которого в общем виде выходит за рамки данной работы. Для наглядной демонстрации необходимого эффекта вполне достаточно будет специально сконструированного примера.  Рассмотрим игру $\Gamma^3_n$ той же общей схемы, что в предыдущем разделе, но чуть боле сложную, с $3$ однотипными заданиями и $n \ge 2$ одинаковыми компьютерами:
\begin{itemize}
	\item $t_i^1 = t_i^2 = t_i^3 = 2, i = \overline{1,n}$;
	\item $t^1_{DEADLINE} = t^2_{DEADLINE} = t^3_{DEADLINE} = 5,\\ t^1_{BREAKAWAY} = t^2_{BREAKAWAY} = t^3_{BREAKAWAY} = 3$;
	\item $u^1_{HAST} = u^2_{HAST} = u^3_{HAST} = 0,\\ u^1_{GOOD} = u^2_{GOOD} = u^3_{GOOD} = 3,\\ u^1_{LATE} = u^2_{LATE} = u^3_{LATE} = 2$
	%	\item $u^1_{HAST} < u^1_{LATE} < u^1_{GOOD},\\ u^2_{HAST} < u^2_{LATE} < u^2_{GOOD},\\ u^3_{HAST} < u^3_{LATE} < u^3_{GOOD}$.
\end{itemize}

Проще говоря, в игре $\Gamma^3_n$ выгоднее всего использовать компьютер вдвоём "--- $4$ часа суммарной продолжительности работы оказываются как раз в оптимальном промежутке между границей недогруза и дедлайном. Следующий по выгоде вариант "--- использование одной машины втроём, что приводит к штрафу за опоздание. Наименее привлекателен выбор компьютера, на котором задача оказывается одна "--- за такое расходование общественного ресурса игрок не получает вообще ничего. На первый взгляд эта игра не выглядит слишком необычно. В ней без особого труда находятся равновесия Нэша в чистых стратегиях "--- все наборы $(i, i, i), i = \overline{1,n}$, причём очевидно и то, что других решений в чистых стратегиях быть не может. Со смешанными стратегиями дело становится чуть интереснее.

\begin{lemma}
	Пусть $S_* \subseteq \{1, \ldots, n\}$ "--- произвольное непустое подмножество компьютеров. Тогда в игре $\Gamma^3_n$ набор одинаковых смешанных стратегий $s^1 = s^2 = s^3 = \left(\frac{[1 \in S_*]}{\left| S_* \right|}, \ldots, \frac{[n \in S_*]}{\left| S_* \right|}\right)$, где каждый игрок независимо и равновероятно выбирает одну из машин множества $S_*$, является равновесием по Нэшу.\footnote{Здесь и далее для упрощения и сокращения записи используется нотация <<скобка Айверсона>>: $[\textsc{true}] = 1, [\textsc{false}] = 0$.}
\end{lemma}

\begin{proof}
	Воспользуемся тем, что достаточно проверить отклонения только в пользу чистых стратегий. Обозначим символом $\left. s \right|^a_i$ отклонение от набора $s$ игроком $a$ в пользу стратегии $i$ и заметим, что 
	\begin{align*}
		u^a(\left. s \right|^a_i) &= [i \in S_*] \left(\frac{(\left| S_* \right| - 1)^2}{\left| S_* \right|^2} u^a_{HAST} + 2 \frac{\left| S_* \right| - 1}{\left| S_* \right|^2} u^a_{GOOD} + \frac{1}{\left| S_* \right|^2} u^a_{LATE}\right)\\
		&= [i \in S_*] \frac{6 \left| S_* \right| - 4}{\left| S_* \right|^2}.
	\end{align*}
	
	Таким образом, для каждого игрока максимум ожидаемого выигрыша достигается при отклонении в пользу любого из компьютеров, входящих в $S_*$.
\end{proof}

Соответственно без отклонений математическое ожидание выигрышей предстаёт в виде $u^a(s) = \frac{6 \left| S_* \right| - 4}{\left| S_* \right|^2}, a = \overline{1,3}$. Для доказательства того, что других равновесий по Нэшу в смешанных стратегиях нет, нам понадобятся ещё пара лемм:

\begin{lemma}
	В игре $\Gamma^3_n$ набор смешанных стратегий $s = (s^1, s^2, s^3)$ может быть равновесием по Нэшу только в том случае, когда $s^1 = s^2 = s^3$.
\end{lemma}

\begin{proof}
	Пусть $s^a = (p_1^a, \ldots, p_n^a), a = \overline{1,3}$. Если стратегии игроков не совпадают, то найдётся компьютер $i$, для которого (без потери общности) вероятности выбора первым и вторым игроком $p_i^1 > p_i^2$. В силу элементарных свойств вероятностей, найдётся также и компьютер $j$, где $p_j^1 < p_j^2$. Выпишем математическое ожидание выигрышей тех же игроков при выборе ими $i$"~го компьютера (для $j$"~го всё совпадает с точностью до индекса, очевидно):
	\begin{align*}
		u^1(\left. s \right|^1_i) &= (1 - p_i^2) (1 - p_i^3) u^1_{HAST} + (p_i^2 + p_i^3 - 2 p_i^2 p_i^3) u^1_{GOOD} + p_i^2 p_i^3 u^1_{LATE}\\ &= 3 p_i^2 + 3 p_i^3 - 4 p_i^2 p_i^3;\\
		u^2(\left. s \right|^2_i) &= 3 p_i^1 + 3 p_i^3 - 4 p_i^1 p_i^3;\\
		u^3(\left. s \right|^3_i) &= 3 p_i^1 + 3 p_i^2 - 4 p_i^1 p_i^2.
	\end{align*}
	
	Про функцию $f(x,y) = 3x + 3y - 4xy$ на области определения $0 \le x \le 1, 0 \le y \le 1$ можно заметить следующее "--- если $f(x_0, y_0) < 2$, то для любых $x_1 > x_0, y_1 \ge y_0$ или $x_1 \ge x_0, y_1 > y_0$ выполняется $f(x_0, y_0) < f(x_1, y_1)$. Предположим, что $p_i^3 \le p_j^3$ (если $p_i^3 \ge p_j^3$, рассуждения аналогичны с точностью до индексов). Рассмотрим следующие случаи:
	\begin{itemize}
		\item $p_i^2 < p_j^2$. В силу элементарных свойств вероятностей мы можем быть уверены, что $p_i^2 < \frac{1}{2}$ и $p_i^3 \le \frac{1}{2}$, откуда $u^1(\left. s \right|^1_i) < 2$, а значит $u^1(\left. s \right|^1_i) < u^1(\left. s \right|^1_j)$. Поскольку $p_i^1 > p_i^2 \ge 0$, первый игрок выбирает неоптимальную стратегию с ненулевой вероятностью. $\bot$
		\item $p_i^2 \ge p_j^2$, из чего следует $p_i^1 > p_j^1$, а значит аналогично предыдущему пункту $u^3(\left. s \right|^3_j) < u^3(\left. s \right|^3_i)$. Если $p_j^3 > 0$, то третий игрок выбирает неоптимальную стратегию с ненулевой вероятностью. $\bot$
		\item $p_i^2 \ge p_j^2$, как и в предыдущем случае, но теперь $p_i^3 = p_j^3 = 0$. Из $p_j^1 < p_i^1$ аналогичным образом следует $u^2(\left. s \right|^2_j) < u^2(\left. s \right|^2_i)$, и поэтому $p_j^2 > p_j^1 \ge 0$ влечёт выбор неоптимальной стратегии с ненулевой вероятностью уже вторым игроком. $\bot$
	\end{itemize}
	
	Таким образом, предположение о существовании компьютера, для которого вероятность выбора его одним игроком отличается от вероятности выбора другим, в любом случае противоречит необходимому условию равновесия Нэша.
\end{proof}

\begin{lemma}
	В игре $\Gamma^3_n$ набор одинаковых смешанных стратегий может быть равновесием по Нэшу только в том случае, когда все компьютеры, выбираемые с ненулевой вероятностью, выбираются с равными вероятностями.
\end{lemma}

\begin{proof}
	Возьмём любой набор, состоящий из одинаковых стратегий $(p_1, \ldots, p_n)$, где $0 < p_i < p_j$. Пользуясь формулой выплат из предыдущей леммы, $u^a(\left. s \right|^a_i) = 2 p_i (3 - 2 p_i)$. Опять же, из $p_i < \frac{1}{2}$ следует $u^a(\left. s \right|^a_i) < u^a(\left. s \right|^a_j)$, а значит все игроки выбрали неоптимальную стратегию с ненулевой вероятностью, что противоречит необходимому условию равновесия по Нэшу.
\end{proof}

Доказав, что предложенные точки равновесия исчерпывают пространство решений в смешанных стратегиях, мы можем сконструировать выпуклую оболочку множества достижимых векторов выплат. Поскольку множество целиком лежит на прямой $(u, u, u)$, достаточно найти минимум и максимум ожидаемых выигрышей:

\begin{align*}
	\min_{\emptyset \subset S_* \subseteq \{1,\ldots,n\}} \frac{6 \left| S_* \right| - 4}{\left| S_* \right|^2} &= \frac{6 n - 4}{n^2};\\
	\max_{\emptyset \subset S_* \subseteq \{1,\ldots,n\}} \frac{6 \left| S_* \right| - 4}{\left| S_* \right|^2} &= 2.
\end{align*}

Таким образом искомая выпуклая оболочка представляет собой отрезок, соединяющий точки $(2, 2, 2)$ и $(\frac{6 n - 4}{n^2}, \frac{6 n - 4}{n^2}, \frac{6 n - 4}{n^2})$. Если бы рассматриваемая игра не была чувствительна к дополнительной информационной асимметрии, на этом её анализ можно было бы закончить "--- все игроки находятся в равном положении и, действуя оптимально, могут ожидать равных выигрышей из указанного промежутка. Однако, взгляд на этот конфликт через призму модели заговоров делает картину происходящего куда интереснее.

\section{Коррелированные равновесия игры $\Gamma^3_n$ в пространстве заговоров}\label{sec:ch2/sec4}

Проанализируем этот же конфликт с позиций теории заговоров, перейдя к игре $\Gamma^3_n | \{\{1,2\}\}$. При этом игра $\Gamma^3_n$ дополняется одной вещественной рулеткой, результат вращения которой перед выбором стратегии узнают игроки $1$ и $2$, но не $3$. Для получения равновесного набора коррелированных стратегий, выводящего платежи за выпуклую оболочку множества решений в смешанных стратегиях, заговорщикам достаточно взять любую из классических точек равновесия по Нэшу с $\left| S_* \right| \ge 2$, но вместо того, чтобы выбирать между элементами $S_* \subseteq \{1, \ldots, n\}$ независимо, они должны разделить тайную рулетку на $\left| S_* \right|$ равных секторов и делать свой выбор синхронно, в зависимости от выпавшего сектора. Опишем это более формально, используя пространство корреляции
\begin{equation*}
	\Phi = \langle A, \Omega, \mathfrak{I}^a, \mathbb{P}, a \in A \rangle
\end{equation*}

В данном случае множество состояний природы $\Omega = \left[0, 1\right)$, $\sigma$"~алгебры информированности игроков $\mathfrak{I}^1 = \mathfrak{I}^2$ "--- борелевские, $\mathfrak{I}^3 = \{\emptyset, \Omega\}$ и мера $\mathbb{P}(X) = \left| X \right|$. Вышеописанные стратегии в игре $\Gamma^3_n | \Phi$ можно представить в виде функций, отображающих множество состояний природы в пространство смешанных стратегий:
\begin{align*}
	\mathbf{s}^1(\omega) = \mathbf{s}^2(\omega) &= ([\zeta(\omega) = 1], \ldots, [\zeta(\omega) = n]); \\ \mathbf{s}^3(\omega) &= \left(\frac{[1 \in S_*]}{\left| S_* \right|}, \ldots, \frac{[n \in S_*]}{\left| S_* \right|}\right),
\end{align*}
где общая для игроков 1 и 2 функция $\zeta : \Omega \rightarrow S_*$ определяет разбиение рулетки на $\left| S_* \right|$ равных секторов.

Выплаты в этом наборе уже не симметричны:
\begin{align*}
	u^1(\mathbf{s}) = u^2(\mathbf{s}) &= \frac{\left| S_* \right| - 1}{\left| S_* \right|} u^a_{GOOD} + \frac{1}{\left| S_* \right|} u^a_{LATE} = \frac{3 \left| S_* \right| - 1}{\left| S_* \right|};\\
	u^3(\mathbf{s}) &= \frac{\left| S_* \right| - 1}{\left| S_* \right|} u^a_{HAST} + \frac{1}{\left| S_* \right|} u^a_{LATE} = \frac{2}{\left| S_* \right|}.
\end{align*}

При этом ситуация действительно является равновесием по Нэшу, поскольку первый и второй игроки могут в качестве стратегий использовать любые функции, отображающие $\Omega$ в пространство вероятностных мер на $\{1, \ldots, n\}$, а вот третий игрок вынужден довольствоваться только константными в виду того, что механизм корреляции не информирует его о состоянии природы. Заметив, что $\frac{3 \left| S_* \right| - 1}{\left| S_* \right|} > 2$ при $\left| S_* \right| \ge 2$, мы подтверждаем чувствительность игры $\Gamma^3_n$ к дополнительной информационной асимметрии "--- в новых решениях игроки, наблюдающие не связанный напрямую с выплатами случайный эксперимент, увеличивают свой выигрыш по сравнению с наилучшим результатом, достижимым в классическом смешанном случае.

\section{Коллективная рациональность решений}\label{sec:ch2/sec5}

При рассмотрении множества решений игры $\Gamma^3_n$ в обычных смешанных стратегиях следует обратить внимание на то, что при $\left| S_* \right| > 2$ получающиеся точки равновесия по Нэшу лишены оптимальности не только в смысле Парето, но и по Слейтеру. В самом деле, выплаты всем игрокам в точках, где $\left| S_* \right| = 1$ или $\left| S_* \right| = 2$, равны $2$, а вот при больших размерах множества $\frac{6 \left| S_* \right| - 4}{\left| S_* \right|^2} < 2$. Такие обстоятельствах наделяют решения с использованием одного или двух компьютеров особым статусом "--- можно ожидать, что агенты, знакомые с принципом коллективной рациональности, всё же сумеют договориться о том, чтобы не оказаться в ситуации, которую другое решение доминирует по всем платежам. Естественно сразу задаться вопросом, а нельзя ли и решения с учётом дополнительной информационной асимметрии отфильтровать похожим образом, выделив из них удовлетворяющие принципам коллективной рациональности хоть в каком-то смысле?

При добавлении в игру пространства заговоров, состоящего из группы $\{1, 2\}$, сразу бросается в глаза то, что классические принципы коллективной рациональности становится бесполезны. В новых точках равновесия выигрыши первых двух игроков теперь $u^1(\mathbf{s}) = u^2(\mathbf{s}) > 2$ и растут с ростом $k$, а третьего "--- $u^3(\mathbf{s}) < 2$ и падают, что говорит о прямом антагонизме интересов. Это делает невозможным коллективно рациональный выбор в обычном смысле между смешанным равновесием с $\left| S_* \right|$ равным $1$ или $2$ и коррелированными решениями с различными $k$. Тем не менее, можно попытаться применить более тонкий критерий оптимальности, опирающийся на слегка расширенную интерпретацию происходящего в игре "--- назовём этот формализм \emph{структурно согласованным равновесием по Нэшу}.

Идея структурной согласованности равновесий в играх с заговорами довольно проста "--- если допустить, что входящие в семейство заговоров группы игроков объединяет не только общий механизм корреляции, но и в целом большие возможности для согласования действий, то среди обычных равновесий по Нэшу в коррелированных стратегиях можно особо выделить обладающие устойчивостью не только к индивидуальным отклонениям, но и к групповым, имея в виду исключительно входящие в семейство заговоров группы. Для игры $\Gamma^3_n | \{\{1,2\}\}$, например, это могло бы означать, что структурно несогласованными окажутся те равновесия, для которых найдётся отклонение, в котором участвуют первый и второй игроки, обоюдно увеличивая при этом свои выигрыши. В наиболее общих терминах это можно выразить так:
\begin{definition}
	В игре с заговорами $\Gamma | \mathfrak{A}$ равновесие по Нэшу $\mathbf{s}$ называется структурно согласованным, если для всех заговоров $A_* \in \mathfrak{A}$ отсутствуют приемлемые отклонения от ситуации $\mathbf{s}$.
\end{definition}

Остаётся сформулировать, что в рамках данной модели можно считать отклонением, приемлемым для того или иного заговора. Если взглянуть на этот вопрос как на проблему многокритериальной оптимизации, где критериями являются выигрыши отдельных участников, то напрашиваются два варианта:
\begin{enumerate}
	\item Отклонение приемлемо, если оно увеличивает выигрыш всех участников заговора; (по Слейтеру)
	\item Отклонение приемлемо, если оно увеличивает выигрыш хотя бы одного участника заговора, при этом не уменьшая выигрыша остальных. (по Парето)
\end{enumerate}

Увы, оба варианта нельзя назвать подходящими для наших целей. Разумно было бы ожидать, что добавление в любой заговор <<болвана>>, т.е. игрока с единственной чистой стратегией и константным выигрышем, не должно ничего менять в решении игры. Однако, заговор с подобным <<болваном>> в составе вообще не может иметь приемлемых отклонений в смысле Слейтера, а значит его участники теряют возможность использовать коллективную рациональность. Это делает первый из предложенных вариантов, очевидно, слишком слабым. С другой же стороны, рассмотрение игры в трёхсторонний чёт"~нечет (см. \ref{sec:ch1/sec4}) с двумя заговорами порождает неприятную проблему и для второго варианта. Если 1"~й игрок находится в заговоре со 2"~м, а 2"~й с 3"~м, то можно ожидать исходов с платежами, образованными любым смешением $(5,5,2)$ и $(2,5,5)$, причём выбор пропорции происходит по воле 2"~го игрока. Загвоздка в том, что приемлемость в смысле Парето побуждает 2"~го игрока в ситуации $(5,5,2)$ отклоняться в рамках заговора с 3"~м для улучшения чужого выигрыша. Аналогично, в ситуации $(2,5,5)$ 2"~му игроку приходится спасать 1"~го. В промежуточных же ситуациях 2"~й игрок может помочь обоим, а значит подобный альтруизм вообще исключает существование структурно согласованного равновесия в задаче, делая второй вариант определения приемлемости слишком сильным. Для обхода обоих проблем следует предложить промежуточное определение отклонения, которое будет сильнее чем по Слейтеру, но слабее чем по Парето:
\begin{definition}
	В игре с заговорами $\Gamma | \mathfrak{A}$ ситуацию $\mathbf{s}_* \neq \mathbf{s}$ назовём отклонением от $\mathbf{s}$, приемлемым для заговора $A_* \in \mathfrak{A}$, если
	\begin{itemize}
		\item $\forall a \notin A_* \quad \mathbf{s}^a = \mathbf{s}_*^a$;
		\item $\forall a \in A_* \quad u^a(\mathbf{s}_*) \geq u^a(\mathbf{s})$;
		\item $\forall a : \mathbf{s}^a \neq \mathbf{s}_*^a \quad u^a(\mathbf{s}_*) > u^a(\mathbf{s})$.
	\end{itemize}
\end{definition}

Если бы мы говорили только о равновесиях Нэша в чистых и смешанных стратегиях, такой критерий оптимальности оказался бы слишком сильным "--- действительно, одновременный выбор первым и вторым игроками компьютера $i$, выбираемого третьим с вероятностью $p_i^3 < 1$, даёт обоим выигрыш $(1 - p_i^3) u^a_{GOOD} + p_i^3 u^a_{LATE} = 3 - p_i^3 > 2$, что отсеивает вообще все решения в игре $\Gamma^3_n$. Для игры с заговорами же можно сформулировать следующее:

\begin{theorem}
	В игре с заговорами $\Gamma^3_n | \mathfrak{A}$ структурно согласованное равновесие Нэша существует при любых $n$ и $\mathfrak{A}$. При невырожденном $\mathfrak{A}$, каждому заговору из двух участников соответствует единственное такое равновесие.
\end{theorem}

\begin{proof}
	Возможны три случая в зависимости от $\mathfrak{A}$:
	\begin{enumerate}
		\item $\mathfrak{A} = \emptyset$. При пустом семействе заговоров групповые отклонения невозможны, так что любое равновесие по Нэшу в чистых или смешанных стратегиях будет структурно согласованным.
		\item $\mathfrak{A} = \{\{1, 2, 3\}\}$. При вырожденном семействе заговоров с одним публичным механизмом корреляции множество решений представляет собой выпуклую оболочку векторов платежей смешанных равновесий, что, как было показано в разделе \ref{sec:ch2/sec3}, даёт отрезок, соединяющий $(2, 2, 2)$ и $(\frac{6 n - 4}{n^2}, \frac{6 n - 4}{n^2}, \frac{6 n - 4}{n^2})$. При этом критерий структурной согласованности, очевидно, отсеивает всё, кроме точки $(2, 2, 2)$, соответствующей чистым равновесиям с выбором одной любой общей стратегии на всех и смешанным равновесиям с независимым равновероятным выбором из двух любых одних и тех же стратегий каждым игроком.
		\item $\mathfrak{A}$ содержит $\{1, 2\}$, $\{1, 3\}$ или $\{2, 3\}$. Как было показано в разделе \ref{sec:ch2/sec4}, использование тайных механизмов корреляции даёт новые точки равновесия с выплатами $\frac{3k - 1}{k}$ участникам заговора и $\frac{2}{k}$ аутсайдеру, где $k$ "--- количество задействованных в наборе стратегий машин. Поскольку при $k \ge 2$ выигрыш каждого из заговорщиков превосходит наилучший классический результат, никакие точки смешанного равновесия структурно согласованными уже не будут. Поскольку с ростом $k$ растут и выигрыши заговорщиков, отсеиваются также и все коррелированные равновесия, кроме максимизирующих доход любой из пар заговорщиков при $k = n$. Таким образом, каждой группе, входящей в структуру заговоров, соответствует одна (с точностью до пермутаций рулеток) точка структурно согласованного равновесия, в которой игроки выбирают равновероятно из всего доступного парка машин, причём выбор членов группы всегда совпадает между собой и независим с выбором оставшегося игрока.
	\end{enumerate}
\end{proof}

Для новых структурно согласованных равновесий несложно подобрать вполне естественную интерпретацию. Если представить, что два сотрудника могут координировать свои действия втайне от третьего, то нет ничего неожиданного в их стремлении выбрать один компьютер на двоих, чтобы избежать штрафа за недозагруз, минимизируя при этом шанс для третьего игрока наткнуться на них по воле случая, лишив их премии за срочность. Достигается это логичным образом тогда, когда наибольшая вероятность выбора каждой из машин минимальна, т.е. при равновероятном выборе из всех. Аналогичным образом, для третьего игрока целью становится максимизация наименьшей вероятности выбора каждого из компьютеров, поскольку он понимает, что заговорщики действуют заодно и пытаются избежать встречи с ним, и это тоже достигается в ситуации равновероятного выбора из всего парка машин.

\section{Сохранение тайн заговоров в процессе выработки консенсуса}\label{sec:ch2/sec6}

Как известно, в матричных играх нередко наличествуют несколько точек равновесия по Нэшу (например, координационные игры), причём наборы, состоящие из стратегий, принадлежащих к разным точкам, сами равновесиями не являются. При интерпретации таких точек равновесия в качестве решений игры приходится оговаривать, что выбор игроками равновесных стратегий в сущности не является независимым "--- предпочтение одной точки равновесия другой должно носить характер консенсуса среди игроков. В классических играх с полной информацией это не создаёт больших проблем, поскольку процедура, приводящая к консенсусу, вполне может быть гласной. Однако, когда речь заходит о моделировании конфликтов с заговорами, данный вопрос начинает требовать гораздо более осторожного подхода. При появлении в игре информационной асимметрии, существенно влияющей на её исход, игрокам может становиться выгодно изменять картину этой асимметрии (оповещая, к примеру, о значении тайного сигнала игроков, которые не должны его знать в соответствии со структурой пространства корреляции), для чего при неправильном дизайне могут использоваться те самые механизмы выработки консенсуса. Для того чтобы понять, что может пойти не так, имеет смысл начать с классического случая. Если представить произвольную матричную игру в виде реального процесса с живыми игроками под управлением беспристрастного ведущего, следящего за соблюдением протокола игры, то для получения равновесия по Нэшу можно применить что-то вроде следующей процедуры:

\begin{enumerate}
	\item Ведущий объявляет матрицу выплат;
	\item Игроки гласно обсуждают выбор стратегий;
	\item Игроки втайне друг от друга извещают ведущего о своих ходах;
	\item Ведущий оглашает собранный набор стратегий;
	\item Ведущий может с ненулевой вероятностью выбрать любого из игроков и предложить ему переходить;
	\item Ведущий вычисляет и объявляет выигрыши.
\end{enumerate}

Этой процедуры вполне достаточно, если мы говорим о классических равновесиях Нэша в чистых и смешанных стратегиях. Во втором случае следует только уточнить, что реализация конкретного исхода, определяемого набором смешанных стратегий, либо не происходит вообще (ведущий объявляет математические ожидания выигрышей), либо происходит только на последнем этапе. Однако, при добавлении к модели пространств корреляции ситуация несколько усложняется. Поскольку коррелированными стратегиями могут быть любые функции, отображающие получаемые игроками сигналы в смешанные стратегии, кажется естественным представлять себе, как игроки сами вычисляют избранные ими же функции, получив все релевантные сигналы:

\begin{enumerate}
	\item Ведущий объявляет матрицу выплат;
	\item Игроки гласно обсуждают выбор коррелированных стратегий;
	\item Ведущий генерирует состояние природы и извещает игроков о соответствующих событиях из их $\sigma$"~алгебр информированности;
	\item Игроки втайне друг от друга вычисляют смешанные стратегии и извещают ведущего о своих ходах;
	\item Ведущий оглашает собранный набор смешанных стратегий;
	\item Ведущий может с ненулевой вероятностью выбрать любого из игроков и предложить ему переходить;
	\item Ведущий вычисляет и объявляет выигрыши.
\end{enumerate}

К сожалению, такой наивный подход обладает существенным недостатком "--- он работает ожидаемым образом только в симметричных пространствах корреляции, где $\sigma$"~алгебры информированности всех игроков совпадают. Если же мы говорим о пространствах заговоров, то возникают сразу две проблемы. Во-первых, между этапами 3 и 4 у кого-то из игроков может возникнуть искушение разгласить значение приватного сигнала, если это может сподвигнуть неосведомлённых о нём игроков на выбор более выгодной для разглашающего стратегии. Этот вопрос ещё можно было бы закрыть, добавив в алгоритм запрет на коммуникацию между игроками, начинающийся после этапа 2, но, увы, это только одна из проблем.

Во-вторых, что поправить несколько сложнее, при наличии информационной асимметрии возможность одного из игроков изменить выбранную стратегию на этапе 6 перестаёт соответствовать концепции равновесия по Нэшу. В симметричном случае между первоначальным выбором смешанной стратегии на этапе 4 и возможным отклонением от неё на этапе 6 игрок не получает никакой дополнительной информации, поскольку публичность сигнала и так позволяет вычислить избранные другими игроками смешанные стратегии "--- оглашая их, ведущий, фактически, только фиксирует результат публичной договорённости, достигнутой на этапе 2. Асимметричные пространства корреляции же содержат события, о которых на этапе 3 оповещается только часть игроков. При этом каждый игрок может достоверно вычислить свою смешанную стратегию, но не стратегии оппонентов, завязанные на скрытые от него сигналы. В этой ситуации оглашение ведущим собранных смешанных стратегий на этапе 5 увеличивает знание игроков перед принятием кем-то из них решения об отклонении, что противоречит идее равновесия по Нэшу. Для приведения вышеописанной процедуры в соответствие с моделируемым формализмом её необходимо изменить несколько противоречащим интуиции образом:

\begin{enumerate}
	\item Ведущий объявляет матрицу выплат;
	\item Игроки гласно обсуждают выбор коррелированных стратегий;
	\item Игроки втайне друг от друга извещают ведущего о выбранных коррелированных стратегиях;
	\item Ведущий оглашает собранный набор коррелированных стратегий;
	\item Ведущий генерирует состояние природы и вычисляет смешанные стратегии игроков;
	\item Ведущий может с ненулевой вероятностью выбрать любого из игроков, известить его обо всех реализовавшихся событиях из его $\sigma$"~алгебры информированности и предложить ему изменить вычисленную ведущим смешанную стратегию;
	\item Ведущий вычисляет и объявляет выигрыши.
\end{enumerate}

Таким образом, в отличие от симметричного случая модель заговоров не позволяет смотреть на коррелированные стратегии как на <<чёрные ящики>> в головах игроков, просто подсказывающие им синхронную реакцию на раздражители. Здесь наборы коррелированных стратегий приходится интерпретировать как проговариваемые и формально фиксируемые соглашения, поскольку концепция равновесия по Нэшу подразумевает возможность отклонения именно в тот момент, когда общим знанием являются только намерения игроков реагировать тем или иным образом на тайные сигналы, но ещё не конкретные их реакции.

Интересно то, что при попытке обобщить эту процедуру для получения структурно согласованных равновесий в пространствах заговоров мы снова сталкиваемся с похожей проблемой. На первый взгляд достаточно было бы уточнить только этап 6, чтобы ведущий мог предлагать отклониться от выбранных стратегий как отдельным игрокам, так и целым группам заговорщиков. Однако, это срабатывает ожидаемым образом только для наиболее простых семейств с непересекающимися заговорами. В том же случае, когда у двух заговоров могут быть общие участники, уже групповые отклонения перестают соответствовать формализму "--- ведь если перед их обсуждением ведущий оповестит каждого заговорщика о рулетках всех заговоров, в которые тот входит, то кто-то из них тогда мог бы разгласить тайну другого заговора, тем самым неправомерно увеличив знания не входящих в него игроков до принятия решения об отклонении. Проще всего поправить это с помощью выноса групповых отклонений в отдельный этап, предшествующий генерации состояния природы:

\begin{enumerate}
	\item Ведущий объявляет матрицу выплат;
	\item Игроки гласно обсуждают выбор коррелированных стратегий;
	\item Игроки втайне друг от друга извещают ведущего о выбранных коррелированных стратегиях;
	\item Ведущий оглашает собранный набор коррелированных стратегий;
	\item Ведущий может с ненулевой вероятностью выбрать любую из групп заговорщиков и предложить им изменить выбранные коррелированные стратегии;
	\item Ведущий генерирует состояние природы и вычисляет смешанные стратегии игроков;
	\item Ведущий может с ненулевой вероятностью выбрать любого из игроков, известить его обо всех реализовавшихся событиях из его $\sigma$"~алгебры информированности и предложить ему изменить вычисленную ведущим смешанную стратегию;
	\item Ведущий вычисляет и объявляет выигрыши.
\end{enumerate}

Заметим, что этап 5 (обсуждение и утверждение заговорщиками группового отклонения) "--- сам по себе многостадийный процесс, в котором те, кто хочет увеличить свой выигрыш сменой стратегии, предлагают проект отклонения, а остальные участники заговора имеют индивидуальное право вето, если этот проект приносит им убыток. Таким образом, структурная согласованность равновесия подразумевает, что группы заговорщиков могут отклоняться на стадии планирования, перед получением игроками информации о состоянии природы, а индивидуальные отклонения возможны уже после срабатывания механизмов корреляции, но до оглашения конкретный смешанных стратегий, сыгранных противниками.

\section{Немонотонная отдача в других конфликтах планирования}\label{sec:ch2/sec7}

При помощи игры $\Gamma^3_n$ мы продемонстрировали, во-первых, что планирование заданий с немонотонной функцией оплаты за срочность может быть чувствительно к дополнительной информационной асимметрии, и, во-вторых, что в играх с заговорами, несмотря на присущий им частичный антагонизм интересов, возможны решения, отвечающие принципу коллективной рациональности. Подчеркнём полезность этих результатов, заметив, что проблема планирования заданий гораздо шире рассмотренного нами примера, причём как в смысле возможных значений параметров (матрицы коэффициентов $(t_i^a) \in \mathbb{R}_{\ge 0}^{m \times n}$ и функций отдачи $v^a : \mathbb{R}_{\ge 0} \rightarrow \mathbb{R}, a = \overline{1,m}$), так и в смысле разнообразия практических приложений модели. В контексте этой работы нет смысла заниматься слишком подробным разбором более сложных случаев, но для того, чтобы показать возможную связь модели с реальным миром за пределами вычислительных центров со странными схемами поощрений сотрудников, попробуем построить пару примеров с более солидной предметной областью.

Начнём с экономики, представив себе, как $m$ компаний готовятся выйти на рынок с предложениями высокотехнологичного товара, и перед ними встаёт выбор между $n$ различными открытыми стандартами на один и тот же его важный аспект. К примеру, это могут быть разнообразные промышленные роботы и стандарты их интеграции в <<умный>> цех. Когда компания $a \in \{1, \ldots, m\}$ выходит на рынок стандарта $i \in \{1, \ldots, n\}$, она тем самым осуществляет вклад в его развитие, характеризующийся векторной константой $t_i^a \in \mathbb{R}_{\ge 0} \times \ldots \times \mathbb{R}_{\ge 0}$, компоненты которой соответствуют отдельным независимым аспектам (например, функциям, для исполнения которых приобретаются роботы). Если в ситуации $s$ одним стандартом $i$ пользуются несколько компаний, то простым суммированием их вкладов можно посчитать общий индекс развития $t_i(s) = [s^1 = i] t_i^1 + \ldots + [s^m = i] t_i^m$. На ожидаемый доход от инвестиций в каждый из стандартов существенным образом влияют два дисконтирующих фактора: сетевой эффект и насыщение рынка.

Под сетевым эффектом мы понимаем зависимость покупательского энтузиазма от общего индекса развития стандарта "--- функция $0 \le \alpha^a(t) \le 1$ характеризует долю покупателей, готовых приобретать роботов компании $a$, выполненных по стандарту с общим индексом развития $t$. Более развитый стандарт всегда привлекает больше потребителей, так что функции $\alpha^a(t)$ монотонно неубывающие, т.е. $\alpha^a(t) \le \alpha^a(t + \Delta), \forall t, \Delta \succeq (0, \ldots, 0)$. Насыщение рынка, с другой стороны, подразумевает ограниченность спроса "--- при избытке инвестиций в любой из стандартов, платёжеспособности покупателей перестаёт хватать на всех, цены приходится снижать, а с ними падают и доходы. Соответственно, ещё одна функция $0 \le \beta^a(t) \le 1$ характеризует, какой долей прибыли придётся ограничиться компании $a$ для сохранения конкурентоспособности своих роботов на рынке стандарта с общим индексом развития $t$. Эта функция, по понятным причинам, монотонно невозрастающая, т.е. $\beta^a(t) \ge \beta^a(t + \Delta), \forall t, \Delta \succeq (0, \ldots, 0)$. Целью компании $a$ при выборе стратегии $s^a$ является максимизация комбинации дисконтирующих факторов $u^a(s) = \alpha^a(t_{s^a}(s)) \beta^a(t_{s^a}(s))$.

Можно представить и политическую интерпретацию этой же игры. Пусть в некий коллегиальный выборный орган пытаются избираться $n$ кандидатов (самостоятельно, без партийных списков), а $m$ эффективных менеджеров выбирают, за кого из них развернуть агитацию в подведомственных учреждениях. Когда олигарх $a$ принимает решение о поддержке кандидата $i$, тем самым он вносит вклад в его популярность, характеризующийся векторной константой $t_i^a \in \mathbb{R}_{\ge 0} \times \ldots \times \mathbb{R}_{\ge 0}$, компоненты которой соответствуют электорально значимым демографическим группам. Если в ситуации $s$ кандидата $i$ поддерживают несколько олигархов, то простым суммированием их вкладов можно получить общий индекс популярности кандидата $t_i(s) = [s^1 = i] t_i^1 + \ldots + [s^m = i] t_i^m$. На ожидаемую выгоду от поддержки того или иного кандидата влияют два дисконтирующих фактора: политическое влияние и готовность к сотрудничеству.

Политическое влияние кандидата в вопросах, интересующих поддержавшего его олигарха, очевидно, растёт вместе с общим индексом его популярности, что выражается функцией $0 \le \alpha^a(t) \le \alpha^a(t + \Delta) \le 1, \forall t, \Delta \succeq (0, \ldots, 0)$. Готовность же кандидата к сотрудничеству с каждым из своих сторонников, наоборот, падает с ростом его суммарной популярности, что выражается функцией $1 \ge \beta^a(t) \ge \beta^a(t + \Delta) \ge 0, \forall t, \Delta \succeq (0, \ldots, 0)$. Целью олигарха $a$ при выборе стратегии $s^a$ является максимизация комбинации дисконтирующих факторов $u^a(s) = \alpha^a(t_{s^a}(s)) \beta^a(t_{s^a}(s))$.

Сведём описание обоих конфликтов к матричной игре в нормальной форме:
\begin{equation*}
	\Gamma = \langle A, S^a, u^a(s), a \in A \rangle;
\end{equation*}
\begin{equation*}
	A = \{1, \ldots, m\}, S^1 = \ldots = S^m = \{1, \ldots, n\};
\end{equation*}
\begin{equation*}
	u^a(s) = \alpha^a(t_{s^a}(s)) \beta^a(t_{s^a}(s)), a = \overline{1,m};
\end{equation*}
\begin{equation*}
	\alpha^a, \beta^a : \mathbb{R}_{\ge 0} \times \ldots \times \mathbb{R}_{\ge 0} \rightarrow [0, 1];
\end{equation*}
\begin{equation*}
	0 \le \alpha^a(t) \le \alpha^a(t + \Delta) \le 1, \forall t, \Delta \succeq (0, \ldots, 0);
\end{equation*}
\begin{equation*}
	1 \ge \beta^a(t) \ge \beta^a(t + \Delta) \ge 0, \forall t, \Delta \succeq (0, \ldots, 0);
\end{equation*}
\begin{equation*}
	t_i(s) = [s^1 = i] t_i^1 + \ldots + [s^m = i] t_i^m, i = \overline{1,n}.
\end{equation*}

Несложно заметить сходство этой игры с проблемой планирования заданий, что мы здесь постарались подчеркнуть использованием тех же символов для переменных и констант. Фактически, единственным значимым отличием оказывается то, что в планировании заданий время $t$ было скаляром, а не вектором. Функциям оплаты за срочность в новых формулировках соответствуют $v^a(t) = \alpha^a(t) \beta^a(t)$, форма которых и определяет наши ожидания от исхода конфликта. Ранее мы уже говорили, что анализ проблемы планирования заданий в общем виде, для произвольных $(t_i^a)$ и $(v^a)$ представляет собой чрезвычайно сложную задачу, и, конечно же, переход от скаляров к векторам в домене функций отдачи дело нисколько не упрощает. Лучшее, что тут можно сделать "--- дать качественный прогноз для некоторых неформально описанных подклассов конфликта с опорой на здравый смысл, интуицию и аналогию с разобранным выше частным случаем $\Gamma^3_n$.

Во избежание переусложнения ограничимся случаем квазивогнутых функций отдачи $v^a(t) = \alpha^a(t) \beta^a(t)$, естественным образом обобщив это понятие на многомерные области определения. Для начала обозначим символом $T^a_{\nearrow}$ множество всех таких $t$, что $t_* \prec t \Rightarrow v^a(t_*) \le v^a(t) \wedge t_* \in T^a_{\nearrow}$. Аналогично, символом $T^a_{\searrow}$ обозначим множество всех таких $t$, что $t_* \succ t \Rightarrow v^a(t_*) \le v^a(t) \wedge t_* \in T^a_{\searrow}$. Это будут области непрерывного неубывания и невозрастания $v^a(t)$ соответственно. Если эти два множества покрывают всю область определения, т.е. $T^a_{\nearrow} \cup T^a_{\searrow} = \mathbb{R}_{\ge 0} \times \ldots \times \mathbb{R}_{\ge 0}$, то функция $v^a(t)$ квазивогнута. Для подобных функций также можно обозначить <<гребень>> $T^a_{\sim} = T^a_{\nearrow} \cap T^a_{\searrow}$, в одномерном случае соответствующий максимуму.

Попробуем представить логику конфликта для простейшего случая с двухэлементным семейством заговоров $\mathfrak{A} = \{A_p, A_q\}, A_p \cap A_q = \emptyset, A_p \cup A_q = A$. Сократим запись с помощью следующих обозначений:
\begin{equation*}
	t_i^{A_*} = \sum_{a \in A_*} t_i^a, \forall A_* \subseteq A;
\end{equation*}
\begin{equation*}
	\check{u}^a = \min_{1 \le i \le n} v^a(t_i^A), \hat{u}^a = \max_{1 \le i \le n} v^a(t_i^A), \forall a \in A.
\end{equation*}

Здесь $t_i^{A_*}$ соответствует суммарному индексу развития стандарта (популярности кандидата) $i$ при его выборе группой игроков $A_* \subseteq A$. Так же, для каждого игрока $a$ базовым платёжным интервалом называется промежуток от $\check{u}^a$ до $\hat{u}^a$, т.е. от наименьшего до наибольшего возможных выигрышей при единогласном выборе общей стратегии всеми участниками конфликта. Рассмотрим игры, ограниченные следующими условиями:

\begin{itemize}
	\item $\forall a \in A, i = \overline{1,n}, v^a(t_i^{A_p \cup \{a\}}) < \check{u}^a$, т.е. ни один игрок не может достигнуть нижней границы своего базового платёжного интервала, если ту же стратегию выбирает только группа $A_p$;
	\item $\forall a \in A, i = \overline{1,n}, v^a(t_i^{A_q \cup \{a\}}) > \check{u}^a$, т.е. каждый игрок преодолевает нижнюю границу своего базового платёжного интервала при выборе любой стратегии совместно с группой $A_q$;
	\item $\forall a \in A_q, i = \overline{1,n}, t_i^{A_q} \in T^a_{\searrow}$, т.е. для всех членов группы $A_q$ выбор общей стратегии выводит суммарный индекс в область невозрастания функции отдачи.
\end{itemize}

Таким образом мы очертили круг ситуаций, в которых участники конфликта поделены на две непересекающиеся группы заговорщиков. Если рассматривать происходящее с точки зрения каждого заговора в предположении, что аутсайдеры вообще не участвует в игре, несложно заметить, что любой стратегический набор, в котором заговорщики выбирают одну стратегию на всех, будет хорошим кандидатом в равновесия по Нэшу. Это не означает, что других равновесий не может быть, но мы в целях наглядности сознательно ограничимся анализом наборов, характеризующихся двумя независимыми распределениями вероятностей $p = (p_1, \ldots, p_n)$ и $q = (q_1, \ldots, q_n)$, где участники заговоров $A_p$ и $A_q$ синхронно внутри групп но независимо между группами выбирают стратегии $i = \overline{1,n}$ c вероятностями $p_i$ и $q_i$ соответственно, используя соответствующий приватный механизм корреляции. Выплаты при этом считаются по формуле
\begin{equation*}
	u^a(p, q) = \sum_{i=1}^n p_i ((1 - q_i) v^a(t_i^{A_p}) + q_i v^a(t_i^A)), \forall a \in A_p,
\end{equation*}
\begin{equation*}
	u^a(p, q) = \sum_{i=1}^n q_i ((1 - p_i) v^a(t_i^{A_q}) + p_i v^a(t_i^A)), \forall a \in A_q,
\end{equation*}

В соответствии с условиями, заговор $A_p$ недостаточно велик для максимизации прибылей своих участников, так что каждый из них предпочёл бы присоединиться к стратегии, избранной заговором $A_q$. Однако, поскольку тайна чужого заговора игрокам не доступна, от стратегии предписанной механизмом корреляции они могут отклониться только в пользу другой чистой стратегии. Таким образом, для получения прибыли в результате индивидуального отклонения от пары распределений $(p, q)$, заговорщику $a \in A_p$ необходимо и достаточно найти такие индексы $i \neq j \in \{1, \ldots, n\}$, что при $p_i > 0$ выполняется неравенство
\begin{equation*}
	(1 - q_i) v^a(t_i^{A_p}) + q_i v^a(t_i^A) < (1 - q_j) v^a(t_j^a) + q_j v^a(t_j^{A_q \cup \{a\}}).
\end{equation*}

Несложно заметить, что с ростом любого $q_j$ постепенно увеличивается и кол-во индексов $i$, для которых выполняется это неравенство, а при приближении $q_j$ к $1$ оно рано или поздно начинает выполняться для всех $i \neq j$. Зафиксировав произвольное распределение $q$, можно для каждого заговорщика $a \in A_p$ вычислить множество $S^a_q \subseteq \{1, \ldots, n\}$, состоящее из стратегий, допускающих подобные продуктивные отклонения. При этом, поскольку участники заговора $A_q$, отклонившись от предписанной стратегии, неизбежно терпят убытки, для них проверять ничего не нужно. В результате, произвольная пара распределений $(p, q)$ описывает равновесие по Нэшу тогда и только тогда, когда
\begin{equation*}
	\forall i \in \bigcup_{a \in A_p} S^a_q, p_i = 0.
\end{equation*}

Таким образом, если мы говорим о равновесиях только в классическом Нэшевском смысле без учёта коллективной рациональности, то в описанном противостоянии участникам большого заговора вообще не приходится думать о возможном предательстве со стороны соратников, тогда как малый заговор должен внимательно выбирать общую стратегию так, чтобы у его участников не было искушения попытаться угадать стратегию, избранную большим. Стремление же удостовериться в структурной согласованности обозначенных решений дают чуть более интересную картину.

Сразу оговоримся, что в установленных ограничениях сложно точно убедиться в структурной согласованности даже для узкого множества рассматриваемых $(p, q)$"~наборов, поскольку вполне возможны, например, коллективные отклонения, разделяющие заговор $A_q$ на две группы, выбирающие разные стратегии. Одна при этом получает прибыль в результате избавления от лишних участников (см. ограничение $t_i^{A_q} \in T^a_{\searrow}$, т.е. принадлежность точек единогласного выбора к области невозрастания функции отдачи). Вторая же потенциально увеличивает доход, присоединившись к стратегии, избранной заговором $A_p$, если существует достаточно большое $p_i$. Можно, конечно, попытаться наложить на параметры конфликта дополнительные ограничения, предупреждающие подобные и даже более экзотические отклонения, однако это сильно усложнит постановку, не слишком добавляя иллюстративности.

Вместо этого мы ограничимся поиском только тех $(p, q)$"~наборов, от которых не существует успешных коллективных отклонений в пользу других $(p, q)$"~наборов. Найденные точки равновесия всё ещё можно будет подозревать в отсутствии структурной согласованности, однако мы хотя бы исключим большой класс заведомо несогласованных. Итак, для прибыльности коллективного отклонения малого заговора достаточно найти такие индексы $i \neq j \in \{1, \ldots, n\}$, что при $p_i > 0$ для каждого $a \in A_p$ выполняется неравенство
\begin{equation*}
	(1 - q_i) v^a(t_i^{A_p}) + q_i v^a(t_i^A) < (1 - q_j) v^a(t_j^{A_p}) + q_j v^a(t_j^A).
\end{equation*}

Аналогично, для большого заговора отклонение успешно, если есть такие индексы $i \neq j \in \{1, \ldots, n\}$, что при $q_i > 0$ для каждого $a \in A_q$ выполняется неравенство
\begin{equation*}
	(1 - p_i) v^a(t_i^{A_q}) + p_i v^a(t_i^A) < (1 - p_j) v^a(t_j^{A_q}) + p_j v^a(t_j^A).
\end{equation*}

Рассматривая эти неравенства в свете ограничений, наложенных нами на параметры конфликта, несложно заметить, что коллективная рациональность стимулирует обе группы игроков к минимизации наибольших вероятностей выбора отдельных стратегий, однако по противоположным причинам. Участникам малого заговора выгодно сообща присоединиться к стратегии, выбранной большим заговором, выплаты членам которого такое совпадение стратегий заведомо уменьшает. Переводя на язык условленных ранее интерпретаций, слабый картель с удовольствием воспользовался бы развитостью стандарта (или влиятельностью кандидата) избираемого крупным картелем, однако крупному картелю, напротив, совершенно не улыбается делить ограниченный спрос на и без того насыщенном рынке (или соревноваться за внимание и без того уверенного в избрании политика) с лишними конкурентами.

На уровне коллективно рациональных решений игра, таким образом, превращается в разновидность двухсторонних пряток, где одна группа ищет встречи с другой, пытающейся этого столкновения избежать, а заговоры служат для получения преимущества от объединения усилий при минимизации вероятности присоединения к дележу прибылей нежелательных попутчиков. По сути, формализм структурно согласованного равновесия в играх с заговорами является не какой-то сложной экономической концепцией, а всего лишь воплощением интуитивного принципа, применявшегося, вероятно, ещё в дописьменную эпоху. Вполне возможно, что накой-нибудь охотник, заметив раненного мамонта при обходе племенных угодий, прикинул: <<В одиночку я его, пожалуй, не завалю, но и племя всё звать смысла нет. Шепну-ка я лучше на ушко паре друзей "--- это ж сколько почёта и славы будет, втроём столько мяса добыть.>> С похожего рассуждения и могла начаться мировая история заговоров.

\FloatBarrier
