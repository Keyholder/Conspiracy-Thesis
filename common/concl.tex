%% Согласно ГОСТ Р 7.0.11-2011:
%% 5.3.3 В заключении диссертации излагают итоги выполненного исследования, рекомендации, перспективы дальнейшей разработки темы.
%% 9.2.3 В заключении автореферата диссертации излагают итоги данного исследования, рекомендации и перспективы дальнейшей разработки темы.
\begin{enumerate}
  \item На основе анализа понятия коррелированного равновесия в контексте многосторонних конфликтов было сформулировано свойство чувствительности игр к дополнительной информационной асимметрии.
  \item Исследование изоморфизма пространств корреляции позволило ввести сужающий их формализм пространства заговоров, с применением которого удобно рассуждать о влиянии дополнительной информационной асимметрии на решения игр.
  \item Моделирование проблемы планирования заданий при допущении немонотонности функций отдачи показало, что в пространствах заговоров концепция структурной согласованности равновесий может использоваться как функциональный аналог классических критериев коллективной рациональности по отношению к равновесиям Нэша в смешанных стратегиях. 
  \item Для демонстрации значимости феномена чувствительности игр к дополнительной информационной асимметрии была построена модель повторяющихся конфликтов с учётом стоимости вычисления очередного шага стратегии.
  \item В рамках построенной модели было показано, как в повторяющихся играх можно даже без дополнительной информационной асимметрии как таковой использовать современные криптографические примитивы для конструирования эффективных стратегий наказания, использующих чувствительность к ней.
\end{enumerate}
