
{\actuality} В теории игр конфликты с участием трёх и более преследующих собственные цели сторон по многим причинам считаются существенно более трудными для моделирования в сравнении с классическими парными противостояниями. Среди этих причин следует особо подчеркнуть влияние, оказываемое на их ход информационной асимметрией. В играх с двумя участниками её эффект, в частности, сводится к последствиям априорной неполноты их знаний о параметрах конфликта, и обычно описывается при помощи байесовских моделей. Платёжная функция в таких случаях не является общим знанием игроков "--- каждый из них действуют исходя из собственных, возможно различающихся предположений, выраженных в форме распределения вероятностей на пространстве всевозможных платёжных функций с заданным пространством стратегий.

Ещё одним источником информационной асимметрии в парном конфликте может выступать присутствие в действиях оппонентов тайной составляющей в тех случаях, когда он проходит в несколько стадий. При этом действия, совершённые игроком на ранних стадиях, позднее становятся полностью или частично известны его противнику, чья стратегия может варьироваться в зависимости от них. Это естественным образом формализуется при помощи информационных разбиений дерева ходов в развёрнутой форме игры. По сути, двумя упомянутыми аспектами исчерпывается влияние информационной асимметрии на конфликты с двумя сторонами. Однако, увеличение количества участников ещё хотя бы на одного порождает новый феномен, замеченный ещё Робертом Ауманом в статье \cite{Aumann74}, где впервые было сформулировано коррелированное расширение игр в нормальной форме.

Выражается этот феномен в том, как для одних и тех же игр соотносятся множества равновесий по Нэшу в смешанных стратегиях и с использованием внешних механизмов корреляции. Для случая двух игроков все вектора математических ожиданий выплат в точках коррелированных равновесий принадлежат выпуклой оболочке множества смешанных, т.\:е. механизмы корреляции можно рассматривать просто как способ получения линейных комбинаций классических решений. С появлением третьего игрока картина меняется "--- в некоторых играх присутствие непубличного корреляционного механизма позволяет достичь точек равновесия по Нэшу с выплатами за пределами выпуклой оболочки решений в смешанных стратегиях. Фактически, это означает, что асимметрия знаний может оказывать существенное влияние на исход многостороннего конфликта даже в тех случаях, когда собственно предмет знаний вообще не имеет к нему отношения. Назовём игры, подверженные этому эффекту, \emph{чувствительными к дополнительной информационной асимметрии}.

В качестве простейшего примера приведём игру <<трёхсторонний чёт"~нечет>>, где каждый из трёх участников тайно выбирает <<орла>> или <<решку>> на своей монете и прижимает её к столу соответствующей стороной вверх, после чего все одновременно поднимают ладони и в зависимости от сложившейся комбинации делят фиксированный банк. Когда все три монеты лежат одной и той же стороной, раунд считается сыгранным вничью и игроки делят банк поровну. Если же совпали только две из них, то оказавшийся в меньшинстве игрок считается проигравшим и не получает доли при дележе банка. В матричной форме это можно описать так:
\begin{table} [htbp]
	\centering
	\begin{threeparttable}
		\caption{Трёхсторонний чёт"~нечет}
		\label{tab:coin3}
		\begin{tabular}{ |c|c|c|c|c| }
			\cline{1-2} \cline{4-5}
			\rule[-7pt]{0pt}{2em}$4, 4, 4$ &
			\rule[-7pt]{0pt}{2em}$6, 0, 6$ & \qquad\qquad\qquad &
			\rule[-7pt]{0pt}{2em}$6, 6, 0$ &
			\rule[-7pt]{0pt}{2em}$0, 6, 6$ \\
			\cline{1-2} \cline{4-5}
			\rule[-7pt]{0pt}{2em}$0, 6, 6$ &
			\rule[-7pt]{0pt}{2em}$6, 6, 0$ & \qquad\qquad\qquad &
			\rule[-7pt]{0pt}{2em}$6, 0, 6$ &
			\rule[-7pt]{0pt}{2em}$4, 4, 4$ \\
			\cline{1-2} \cline{4-5}
		\end{tabular}
	\end{threeparttable}
\end{table}

В таблице \ref{tab:coin3} первый игрок выбирает строку, второй "--- столбец, а третий "--- матрицу. Решением этой игры в чистых стратегиях являются два равновесия Нэша, соответствующие синхронным выборам одинаковых сторон всеми игроками. В смешанных стратегиях добавляется ещё одно вырожденное решение, когда каждый игрок делает случайный выбор между орлом и решкой с равными вероятностями. Все эти решения, очевидно, дают математическое ожидание платежей равное $(4,4,4)$. В рамках классической теории игр этим анализ игры и исчерпывается, однако, добавление фактора информационной асимметрии делает ситуацию интереснее. Представим, что принятие решения игроками предваряется случайным событием, исход которого становится известен только двум игрокам из трёх "--- к примеру, некий доброжелатель подбрасывает симметричную монету и по секрету сообщает результат первому и второму игроку. Хотя это событие никак не влияет на платёжную матрицу, оно позволяет обоим узнавшим о нём игрокам использовать условные стратегии вида <<если выпала сторона $s$, выбрать сторону $f(s)$>>. Таким образом, если первый и второй игроки договорятся положить свои монеты на стол той же стороной, какой выпала монета доброжелателя, то их решения будут совпадать всегда, а вот любая стратегия третьего игрока даст совпадение с остальными только в половине случаев. Математическое ожидание платежей при этом составляет $(5,5,2)$, причём ни для одного из игроков нет выгодного индивидуального отклонения.

Хотя описанный феномен известен уже давно, при построении моделей большинство исследователей обходили его стороной, рассматривая скорее как курьёзное свойство некоторых игр многих игроков. Важным исключением при этом выступает, пожалуй, наиболее значимая из активно использующих формализм коррелированного расширения игр в нормальной форме область теории игр "--- <<дизайн механизмов>> \cite{Nikolenko} Леонида Гурвича, Эрика Маскина и Роджера Майерсона. Их подход ставит своей целью создание экономических инструментов, стимулирующих эгоистичных рациональных агентов к поведению, оптимальному с точки зрения общих целевых функций, формализующих различные социальные блага. Будучи чрезвычайно плодотворной областью исследований, дизайн механизмов породил множество направлений и ответвлений, объединённых тем не менее рядом неотъемлемых общих черт, проистекающих из информационной структуры игр, для которых доказываются его основные положения. Типичная схема взаимодействий выглядит так "--- игроки"~агенты, знающие свои предпочтения и возможности, но находящиеся в неведении относительно этих параметров у других участников, информируют о них центр, формирующий на основе этой информации набор коррелированных стратегий. Далее центр реализует его для конкретного случая в виде набора чистых стратегий и инструктирует каждого из агентов, которые в свою очередь и принимают окончательное решение о том или ином действии. При этом подразумевается, что агенты могут лгать на первом этапе и не подчиняться на последнем. Главной задачей дизайна механизмов в этой парадигме становится создание таких алгоритмов поведения центра, что стратегии правдивости и послушания образуют для агентов равновесие Нэша. Несложно заметить, что вышеописанная игра с доброжелателем вполне может быть сформулирована в терминах дизайна механизмов, если предположить, что по какой-то причине центр благоволит первому и второму игроку (в его целевой функции их выигрыш имеет больший вес).

Не вдаваясь в детали, можно сказать, что дизайн механизмов базируется на частном случае информационной асимметрии "--- своего рода звёздчатой структуре связей, где выделенный центральный агент может в своих интересах распоряжаться общим механизмом корреляции, а подчинённые ему агенты находятся в полной изоляции как друг от друга, так и от остального мира. Название здесь действительно неплохо отражает свойственный для модели взгляд на конфликты "--- через её призму стороны рассматриваются как взаимозаменимые детали единого рукотворного механизма, не связанные ничем кроме участия в нём. Хотя моделирование в рамках подобного упрощения вполне может быть полезно при конструировании формализованных способов решения конфликтов, оно никак не может помочь в тех случаях, когда на них оказывает существенное влияние информационная асимметрия, складывающаяся не в результате сознательного дизайна, а естественным образом, по мере спонтанного взаимодействия агентов в неконтролируемой, внешней с точки зрения модели среде. К примеру, сложно переоценить значимость влияния коррупции на политические и экономические институты, а ведь она складывается именно из таких незапланированных информационных связей, внешних по отношению к самим институтам.

По описанной причине исследование влияния дополнительной информационной асимметрии на решения игр многих игроков никак нельзя сводить только к конструктивным моделям. Увы, но за пределами дизайна механизмов сложилась традиция игнорировать этот феномен. К примеру, в статье \cite{Fudenberg} Дрю Фуденберга и Эрика Маскина можно найти следующую сноску: <<Actually, if $n \ge 3$, the other players may be able to keep player $j$'s payoff even lower by using a correlated strategy against $j$, where the outcome of the correlating device is not observed by $j$ (...). In keeping with the rest of the literature on repeated games, however, we shall rule out such correlated strategies.>> А ведь казалось бы, в контексте повторяющихся игр с дисконтированием проблематика использования секретности механизма корреляции для усиления стратегий наказания довольно любопытна "--- наверное в каждой области исследований, использующей народную теорему, от антропологии до международной политики несложно отыскать примеры того, как группы агентов усиливали свою коллективную долгосрочную позицию при помощи необходимо тайного согласования действий. Увы, но приходится констатировать, что теории игр до сих пор почти нечего предложить другим наукам в качестве инструмента анализа описанного феномена.

% {\progress}
% Этот раздел должен быть отдельным структурным элементом по
% ГОСТ, но он, как правило, включается в описание актуальности
% темы. Нужен он отдельным структурынм элемементом или нет ---
% смотрите другие диссертации вашего совета, скорее всего не нужен.

{\aim} данной работы является создание новой модели многосторонних конфликтов, учитывающей влияние на их ход дополнительной информационной асимметрии.

Для~достижения поставленной цели необходимо было решить следующие {\tasks}:
\begin{enumerate}[beginpenalty=10000] % https://tex.stackexchange.com/a/476052/104425
  \item Исследовать формализм коррелированного обобщения игр в нормальной форме с точки зрения проблематики работы.
  \item Разработать способ описания информационных структур, достаточно разнообразным образом связывающих участников произвольного конфликта.
  \item Исследовать влияние дополнительной информационной асимметрии на соответствие равновесий критериям коллективной рациональности.
  \item Разработать приемлемую концепцию решения с учётом связей между агентами для игр с дополнительной информационной асимметрией.
\end{enumerate}


{\novelty}
\begin{enumerate}[beginpenalty=10000] % https://tex.stackexchange.com/a/476052/104425
  \item Были впервые выделены в качестве самостоятельного объекта исследования игры многих игроков, проявляющие чувствительность к дополнительной информационной асимметрии.
  \item Был впервые предложен формализм пространства заговоров, специальным образом сужающий в целях моделирования дополнительной информационной асимметрии формализм пространства корреляции.
  \item Была впервые сформулирована концепция структурно согласованного равновесия, позволяющая во многих случаях выделять среди решений игр в пространствах заговоров отвечающие достаточно тонкому принципу коллективной рациональности.
\end{enumerate}

{\influence} работы проистекает из явной необходимости учитывать при моделировании многосторонних конфликтов тот факт, что состав их участников не является в большинстве случаев случайной выборкой никак не связанных друг с другом агентов. Классический формализм игр в нормальной форме опирается на неявное допущение, состоящее в том, что единственной значимой характеристикой каждого игрока является порядок его предпочтений относительно исхода розыгрыша, выражающийся в форме платёжной функции. Совершенно очевидно при этом, что реальных людей, вступающих в противостояние, зачастую связывают значимые для его исхода отношения, структура которых не может быть выражена простым сочетанием платёжных функций. В качестве наглядной иллюстрации такой связи можно сравнить две воображаемые партии в бридж или преферанс с участием одинаково сильных игроков, различающиеся тем, что в одном случае за столом сидят незнакомцы, а в другом часть из них играют вместе уже много лет. Любой достаточно опытный картёжник скажет, что при равных навыках фактор <<сыгранности>> с партнёром надёжно обеспечивает решающее преимущество. Естественным образом этот феномен можно обобщить и на более значимые конфликты: политика, бизнес, дипломатия "--- везде, где исход противостояния существенно зависит от согласованности и непредсказуемости действий, взаимопонимание не требующее коммуникации зачастую может превратить поражение в победу. Таким образом, для более точного предсказания исходов многосторонних конфликтов насущно необходимы модели, позволяющие учитывать этот фактор.

{\methods} В работе используются методы теории игр, теории вероятности, топологии и криптографии.

{\defpositions}
\begin{enumerate}[beginpenalty=10000] % https://tex.stackexchange.com/a/476052/104425
  \item Фактор дополнительной информационной асимметрии важен при исследовании многосторонних конфликтов, поскольку получаемые с его помощью новые решения отражают реальные преимущества, которые даёт агентам возможность тайной координации действий.
  \item Формализм игр с заговорами позволяет, оставаясь в рамках пространств с конечным описанием, моделировать широкий класс испытывающих влияние дополнительной информационной асимметрии конфликтов.
  \item Концепция структурно согласованного равновесия позволяет во многих случаях находить в играх с заговорами решения, удовлетворяющие более тонкому в сравнении с эффективностью по Парето критерию коллективной рациональности.
  \item В повторяющихся играх эффект влияния дополнительной информационной асимметрии может проявляться даже при ограничении пространства стратегий игроков классическим смешанным случаем.
\end{enumerate}
%В папке Documents можно ознакомиться с решением совета из Томского~ГУ
%(в~файле \verb+Def_positions.pdf+), где обоснованно даются рекомендации
%по~формулировкам защищаемых положений.

{\reliability} полученных результатов обеспечивается при помощи формальных доказательств с применением топологических инструментов. Результаты находятся в соответствии с результатами, полученными другими авторами.


{\probation}
Основные результаты работы докладывались~на: Ломоносовских чтениях (2017, 2021 гг.) \cite{ownlmr2017, ownlmr2021} и IX Московской международной конференции по исследованию операций \cite{ownorm2018}.

{\contribution} Автор принимал активное участие \ldots

\ifnumequal{\value{bibliosel}}{0}
{%%% Встроенная реализация с загрузкой файла через движок bibtex8. (При желании, внутри можно использовать обычные ссылки, наподобие `\cite{vakbib1,vakbib2}`).
    {\publications} Основные результаты по теме диссертации изложены
    в~XX~печатных изданиях,
    X из которых изданы в журналах, рекомендованных ВАК,
    X "--- в тезисах докладов.
}%
{%%% Реализация пакетом biblatex через движок biber
    \begin{refsection}[bl-author, bl-registered]
        % Это refsection=1.
        % Процитированные здесь работы:
        %  * подсчитываются, для автоматического составления фразы "Основные результаты ..."
        %  * попадают в авторскую библиографию, при usefootcite==0 и стиле `\insertbiblioauthor` или `\insertbiblioauthorgrouped`
        %  * нумеруются там в зависимости от порядка команд `\printbibliography` в этом разделе.
        %  * при использовании `\insertbiblioauthorgrouped`, порядок команд `\printbibliography` в нём должен быть тем же (см. biblio/biblatex.tex)
        %
        % Невидимый библиографический список для подсчёта количества публикаций:
        \printbibliography[heading=nobibheading, section=1, env=countauthorvak,          keyword=biblioauthorvak]%
        \printbibliography[heading=nobibheading, section=1, env=countauthorwos,          keyword=biblioauthorwos]%
        \printbibliography[heading=nobibheading, section=1, env=countauthorscopus,       keyword=biblioauthorscopus]%
        \printbibliography[heading=nobibheading, section=1, env=countauthorconf,         keyword=biblioauthorconf]%
        \printbibliography[heading=nobibheading, section=1, env=countauthorother,        keyword=biblioauthorother]%
        \printbibliography[heading=nobibheading, section=1, env=countregistered,         keyword=biblioregistered]%
        \printbibliography[heading=nobibheading, section=1, env=countauthorpatent,       keyword=biblioauthorpatent]%
        \printbibliography[heading=nobibheading, section=1, env=countauthorprogram,      keyword=biblioauthorprogram]%
        \printbibliography[heading=nobibheading, section=1, env=countauthor,             keyword=biblioauthor]%
        \printbibliography[heading=nobibheading, section=1, env=countauthorvakscopuswos, filter=vakscopuswos]%
        \printbibliography[heading=nobibheading, section=1, env=countauthorscopuswos,    filter=scopuswos]%
        %
        \nocite{*}%
        %
        {\publications} Основные результаты по теме диссертации изложены в~\arabic{citeauthor}~печатных изданиях,
        \arabic{citeauthorvak} из которых изданы в журналах, рекомендованных ВАК\sloppy%
        \ifnum \value{citeauthorscopuswos}>0%
            , \arabic{citeauthorscopuswos} "--- в~периодических научных журналах, индексируемых Web of~Science и Scopus\sloppy%
        \fi%
        \ifnum \value{citeauthorconf}>0%
            , \arabic{citeauthorconf} "--- в~тезисах докладов.
        \else%
            .
        \fi%
        \ifnum \value{citeregistered}=1%
            \ifnum \value{citeauthorpatent}=1%
                Зарегистрирован \arabic{citeauthorpatent} патент.
            \fi%
            \ifnum \value{citeauthorprogram}=1%
                Зарегистрирована \arabic{citeauthorprogram} программа для ЭВМ.
            \fi%
        \fi%
        \ifnum \value{citeregistered}>1%
            Зарегистрированы\ %
            \ifnum \value{citeauthorpatent}>0%
            \formbytotal{citeauthorpatent}{патент}{}{а}{}\sloppy%
            \ifnum \value{citeauthorprogram}=0 . \else \ и~\fi%
            \fi%
            \ifnum \value{citeauthorprogram}>0%
            \formbytotal{citeauthorprogram}{программ}{а}{ы}{} для ЭВМ.
            \fi%
        \fi%
        % К публикациям, в которых излагаются основные научные результаты диссертации на соискание учёной
        % степени, в рецензируемых изданиях приравниваются патенты на изобретения, патенты (свидетельства) на
        % полезную модель, патенты на промышленный образец, патенты на селекционные достижения, свидетельства
        % на программу для электронных вычислительных машин, базу данных, топологию интегральных микросхем,
        % зарегистрированные в установленном порядке.(в ред. Постановления Правительства РФ от 21.04.2016 N 335)
    \end{refsection}%
    \begin{refsection}[bl-author, bl-registered]
        % Это refsection=2.
        % Процитированные здесь работы:
        %  * попадают в авторскую библиографию, при usefootcite==0 и стиле `\insertbiblioauthorimportant`.
        %  * ни на что не влияют в противном случае
        \nocite{vakbib2}%vak
        \nocite{patbib1}%patent
        \nocite{progbib1}%program
        \nocite{bib1}%other
        \nocite{confbib1}%conf
    \end{refsection}%
        %
        % Всё, что вне этих двух refsection, это refsection=0,
        %  * для диссертации - это нормальные ссылки, попадающие в обычную библиографию
        %  * для автореферата:
        %     * при usefootcite==0, ссылка корректно сработает только для источника из `external.bib`. Для своих работ --- напечатает "[0]" (и даже Warning не вылезет).
        %     * при usefootcite==1, ссылка сработает нормально. В авторской библиографии будут только процитированные в refsection=0 работы.
}

%При использовании пакета \verb!biblatex! будут подсчитаны все работы, добавленные
%в файл \verb!biblio/author.bib!. Для правильного подсчёта работ в~различных
%системах цитирования требуется использовать поля:
%\begin{itemize}
%        \item \texttt{authorvak} если публикация индексирована ВАК,
%        \item \texttt{authorscopus} если публикация индексирована Scopus,
%        \item \texttt{authorwos} если публикация индексирована Web of Science,
%        \item \texttt{authorconf} для докладов конференций,
%        \item \texttt{authorpatent} для патентов,
%        \item \texttt{authorprogram} для зарегистрированных программ для ЭВМ,
%        \item \texttt{authorother} для других публикаций.
%\end{itemize}
%Для подсчёта используются счётчики:
%\begin{itemize}
%        \item \texttt{citeauthorvak} для работ, индексируемых ВАК,
%        \item \texttt{citeauthorscopus} для работ, индексируемых Scopus,
%        \item \texttt{citeauthorwos} для работ, индексируемых Web of Science,
%        \item \texttt{citeauthorvakscopuswos} для работ, индексируемых одной из трёх баз,
%        \item \texttt{citeauthorscopuswos} для работ, индексируемых Scopus или Web of~Science,
%        \item \texttt{citeauthorconf} для докладов на конференциях,
%        \item \texttt{citeauthorother} для остальных работ,
%        \item \texttt{citeauthorpatent} для патентов,
%        \item \texttt{citeauthorprogram} для зарегистрированных программ для ЭВМ,
%        \item \texttt{citeauthor} для суммарного количества работ.
%\end{itemize}
%% Счётчик \texttt{citeexternal} используется для подсчёта процитированных публикаций;
%% \texttt{citeregistered} "--- для подсчёта суммарного количества патентов и программ для ЭВМ.
%
%Для добавления в список публикаций автора работ, которые не были процитированы в
%автореферате, требуется их~перечислить с использованием команды \verb!\nocite! в
%\verb!Synopsis/content.tex!.
