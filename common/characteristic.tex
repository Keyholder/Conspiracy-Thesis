
{\actuality} В теории игр конфликты с участием трёх и более преследующих собственные цели сторон по многим причинам считаются существенно более трудными для моделирования в сравнении с классическими парными противостояниями. Среди этих причин следует особо подчеркнуть влияние, оказываемое на их ход информационной асимметрией. В играх с двумя участниками её эффект, в частности, сводится к последствиям априорной неполноты их знаний о параметрах конфликта, и обычно описывается при помощи байесовских моделей. Платёжная функция в таких случаях не является общим знанием игроков "--- каждый из них действуют исходя из собственных, возможно различающихся предположений, выраженных в форме распределения вероятностей на пространстве всевозможных платёжных функций с заданным пространством стратегий.

Ещё одним источником информационной асимметрии в парном конфликте может выступать присутствие в действиях оппонентов тайной составляющей в тех случаях, когда он проходит в несколько стадий. При этом действия, совершённые игроком на ранних стадиях, позднее становятся полностью или частично известны его противнику, чья стратегия может варьироваться в зависимости от них. Это естественным образом формализуется при помощи информационных разбиений дерева ходов в развёрнутой форме игры. По сути, двумя упомянутыми аспектами исчерпывается влияние информационной асимметрии на конфликты с двумя сторонами. Однако, увеличение количества участников ещё хотя бы на одного порождает новый феномен, замеченный ещё Робертом Ауманом в статье \fixme{[1]}, где впервые было сформулировано коррелированное расширение игр в нормальной форме.

Выражается этот феномен в том, как для одних и тех же игр соотносятся множества равновесий по Нэшу в смешанных стратегиях и с использованием внешних механизмов корреляции. Для случая двух игроков все вектора математических ожиданий выплат в точках коррелированных равновесий принадлежат выпуклой оболочке множества смешанных, т.\:е. механизмы корреляции можно рассматривать просто как способ получения линейных комбинаций классических решений. С появлением третьего игрока картина меняется "--- в некоторых играх присутствие корреляционного механизма позволяет достичь точек равновесия по Нэшу с выплатами за пределами выпуклой оболочки решений в смешанных стратегиях. Фактически, это означает, что асимметрия знаний может оказывать существенное влияние на исход многостороннего конфликта даже в тех случаях, когда собственно предмет знаний вообще не имеет к нему отношения.

Проиллюстрировать вышеописанное можно с помощью игры <<трёхсторонний чёт"~нечет>>, где каждый из трёх игроков тайно выбирает <<орла>> или <<решку>> на своей монете и прижимает её к столу соответствующей стороной вверх, после чего все одновременно поднимают ладони и в зависимости от сложившейся комбинации делят фиксированный банк. Когда все три монеты лежат одной и той же стороной, раунд считается сыгранным вничью и игроки делят банк поровну. Если же совпали только две из них, то оказавшийся в меньшинстве игрок считается проигравшим и не получает доли при дележе банка. В матричной форме это можно описать так:
\begin{table} [htbp]
	\centering
	\begin{threeparttable}
		\caption{Трёхсторонний чёт"~нечет}
		\label{tab:coin3}
		\begin{tabular}{ |c|c|c|c|c| }
			\cline{1-2} \cline{4-5}
			\rule[-7pt]{0pt}{2em}$4, 4, 4$ &
			\rule[-7pt]{0pt}{2em}$6, 0, 6$ & \qquad\qquad\qquad &
			\rule[-7pt]{0pt}{2em}$6, 6, 0$ &
			\rule[-7pt]{0pt}{2em}$0, 6, 6$ \\
			\cline{1-2} \cline{4-5}
			\rule[-7pt]{0pt}{2em}$0, 6, 6$ &
			\rule[-7pt]{0pt}{2em}$6, 6, 0$ & \qquad\qquad\qquad &
			\rule[-7pt]{0pt}{2em}$6, 0, 6$ &
			\rule[-7pt]{0pt}{2em}$4, 4, 4$ \\
			\cline{1-2} \cline{4-5}
		\end{tabular}
	\end{threeparttable}
\end{table}

В таблице \ref{tab:coin3} первый игрок выбирает строку, второй "--- столбец, а третий "--- матрицу. Решением этой игры в чистых стратегиях являются два равновесия Нэша, соответствующие синхронным выборам одинаковых сторон всеми игроками. В смешанных стратегиях добавляется ещё одно вырожденное решение, когда каждый игрок делает случайный выбор между орлом и решкой с равными вероятностями. Все эти решения, очевидно, дают математическое ожидание платежей равное $(4,4,4)$. В рамках классической теории игр этим анализ игры и исчерпывается, однако, добавление фактора информационной асимметрии делает ситуацию интереснее. Представим, что принятие решения игроками предваряется случайным событием, исход которого становится известен только двум игрокам из трёх "--- к примеру, некий доброжелатель подбрасывает симметричную монету и по секрету сообщает результат первому и второму игроку. Хотя это событие никак не влияет на платёжную матрицу, оно позволяет обоим узнавшим о нём игрокам использовать условные стратегии вида <<если выпала сторона $s$, выбрать сторону $f(s)$>>. Таким образом, если первый и второй игроки договорятся положить свои монеты на стол той же стороной, какой выпала монета доброжелателя, то их решения будут совпадать всегда, а вот любая стратегия третьего игрока даст совпадение с остальными только в половине случаев. Математическое ожидание платежей при этом составляет $(5,5,2)$, причём ни для одного из игроков нет выгодного индивидуального отклонения.

%Для большей формальности можно предложить следующее:
%\begin{definition}
%	Пусть $\Gamma = \langle A, S^a, u^a(s), a \in A \rangle$ "--- игра в нормальной форме c $m$ участниками, а $U \subseteq \mathbb{R}^m$ "--- множество всех векторов выплат, достижимых в её смешанных равновесиях по Нэшу. Игра $\Gamma$ называется \emph{чувствительной к дополнительной информационной асимметрии}, когда существует пространство корреляции $\Phi = \langle A, \Omega, \mathfrak{I}^a, \mathbb{P}, a \in A \rangle$ такое, что в игре $\Gamma | \Phi$ найдётся коррелированное равновесие по Нэшу с вектором выплат, не принадлежащим выпуклой оболочке множества $U$.
%\end{definition}

%Здесь коррелированное расширение игры в нормальной форме $\Gamma = \langle A, S^a, u^a(s), a \in A \rangle$ записывается в нотации, адаптированной к русскоязычным традициям. Далее, если не сказано иного, конечное множество игроков обозначается как $A = \{1, \ldots, m\}$, а конечное множество наборов чистых стратегий "--- $S = S^1 \times \ldots \times S^m$. Помимо множества стратегий $S^a$ каждый игрок определяется платёжной функцией $u^a : S \rightarrow \mathbb{R}$. В вероятностном пространстве\cite{kolmog74} $\langle \Omega, \mathfrak{B}, \mathbb{P} \rangle$, в реализуется наблюдаемое игроками состояние природы. Здесь $\Omega$ "--- множество всевозможных таких состояний, $\mathfrak{B}$ "--- $\sigma$"~алгебра подмножеств $\Omega$, а $\mathbb{P} : \mathfrak{B} \rightarrow \mathbb{R}_{\ge 0}$ "--- вероятностная мера. Каждому игроку $a \in A$ поставим в соответствие \emph{собственное подпространство} $\langle \Omega, \mathfrak{I}^a, \mathbb{P} \rangle$ такое, что $\mathfrak{I}^a \subseteq \mathfrak{B}$. При этом набор $\sigma$"~алгебр $\mathfrak{I} = (\mathfrak{I}^a, a \in A)$ отражает информированность игроков о состоянии природы. В описываемой ситуации это состояние не влияет на функции выигрышей непосредственно, выступая исключительно как способ синхронизации действий игроков. Это значит, что $\sigma$"~алгебра $\mathfrak{B}$ сама по себе не является существенным параметром модели, и измеримость по ней для $\mathbb{P}$ можно заменить измеримостью по $\mathfrak{I}^a, \forall a \in A$.

%Отдельно следует заметить, что в оригинале для собственных подпространств игроков формализм Аумана предполагал индивидуальность не только $\sigma$"~алгебр, но и соответствующих им мер, учитывая тем самым возможную субъективность оценок вероятности наступления тех или иных событий, что немаловажно в случаях, когда в качестве механизма корреляции выступают процессы, слишком сложные для объективного анализа (например спортивные соревнования). Однако, для целей данного исследования этот аспект не имеет большого смысла, поскольку предлагающаяся модель подразумевает, что участники конфликта могут произвольным образом выбирать механизм корреляции, а в такой ситуации разумно ожидать, что они будут использовать простые источники случайности с известным распределением (рулетки, кости, жребий и т.\:д.). По этой причине здесь и далее формализм коррелированных стратегий используется в его упрощённой форме, с общей для всех игроков объективной вероятностной мерой в пространстве состояний природы.

Хотя описанный феномен известен уже давно, при построении моделей большинство исследователей обходили его стороной, рассматривая скорее как курьёзное свойство некоторых игр многих игроков. Важным исключением при этом выступает, пожалуй, наиболее значимая из активно использующих формализм коррелированного расширения игр в нормальной форме область теории игр "--- <<дизайн механизмов>> \fixme{[2]} Леонида Гурвича, Эрика Маскина и Роджера Майерсона. Их подход ставит своей целью создание экономических инструментов, стимулирующих эгоистичных рациональных агентов к поведению, оптимальному с точки зрения общих целевых функций, формализующих различные социальные блага. Будучи чрезвычайно плодотворной областью исследований, дизайн механизмов породил множество направлений и ответвлений, объединённых тем не менее рядом неотъемлемых общих черт, проистекающих из информационной структуры игр, для которых доказываются его основные положения. Типичная схема взаимодействий выглядит так "--- игроки"~агенты, знающие свои предпочтения и возможности, но находящиеся в неведении относительно этих параметров у других участников, информируют о них центр, формирующий на основе этой информации набор коррелированных стратегий. Далее центр реализует его для конкретного случая в виде набора чистых стратегий и инструктирует каждого из агентов, которые в свою очередь и принимают окончательное решение о том или ином действии. При этом подразумевается, что агенты могут лгать на первом этапе и не подчиняться на последнем. Главной задачей дизайна механизмов в этой парадигме становится создание таких алгоритмов поведения центра, что стратегии правдивости и послушания образуют для агентов равновесие Нэша. Несложно заметить, что вышеописанная игра с доброжелателем вполне может быть сформулирована в терминах дизайна механизмов, если предположить, что по какой-то причине центр благоволит первому и второму игроку (в его целевой функции их выигрыш имеет больший вес).

%Опишем предложенную ситуацию в терминах коррелированного расширения, дополнив игру вероятностным пространством, в котором реализуется состояние природы, использующееся игроками в качестве сигнала для синхронизации их действий. Таким образом, стратегиями игроков становятся функции, отображающие множество всевозможных сигналов в соответствующие множества чистых стратегий. При этом информационная асимметрия достигается за счёт требования измеримости стратегии каждого игрока относительно его индивидуальной $\sigma$"~алгебры, вложенной в $\sigma$"~алгебру общего вероятностного пространства. То есть, некоторые состояния природы могут быть различимы с точки зрения одних игроков и идентичны для других.

Обзор, введение в тему, обозначение места данной работы в
мировых исследованиях и~т.\:п., можно использовать ссылки на~другие
работы~\autocite{Gosele1999161,Lermontov}
(если их~нет, то~в~автореферате
автоматически пропадёт раздел <<Список литературы>>). Внимание! Ссылки
на~другие работы в~разделе общей характеристики работы можно
использовать только при использовании \verb!biblatex! (из-за технических
ограничений \verb!bibtex8!. Это связано с тем, что одна
и~та~же~характеристика используются и~в~тексте диссертации, и в
автореферате. В~последнем, согласно ГОСТ, должен присутствовать список
работ автора по~теме диссертации, а~\verb!bibtex8! не~умеет выводить в~одном
файле два списка литературы).
При использовании \verb!biblatex! возможно использование исключительно
в~автореферате подстрочных ссылок
для других работ командой \verb!\autocite!, а~также цитирование
собственных работ командой \verb!\cite!. Для этого в~файле
\verb!common/setup.tex! необходимо присвоить положительное значение
счётчику \verb!\setcounter{usefootcite}{1}!.

Для генерации содержимого титульного листа автореферата, диссертации
и~презентации используются данные из файла \verb!common/data.tex!. Если,
например, вы меняете название диссертации, то оно автоматически
появится в~итоговых файлах после очередного запуска \LaTeX. Согласно
ГОСТ 7.0.11-2011 <<5.1.1 Титульный лист является первой страницей
диссертации, служит источником информации, необходимой для обработки и
поиска документа>>. Наличие логотипа организации на~титульном листе
упрощает обработку и~поиск, для этого разметите логотип вашей
организации в папке images в~формате PDF (лучше найти его в векторном
варианте, чтобы он хорошо смотрелся при печати) под именем
\verb!logo.pdf!. Настроить размер изображения с логотипом можно
в~соответствующих местах файлов \verb!title.tex!  отдельно для
диссертации и автореферата. Если вам логотип не~нужен, то просто
удалите файл с~логотипом.

\ifsynopsis
Этот абзац появляется только в~автореферате.
Для формирования блоков, которые будут обрабатываться только в~автореферате,
заведена проверка условия \verb!\!\verb!ifsynopsis!.
Значение условия задаётся в~основном файле документа (\verb!synopsis.tex! для
автореферата).
\else
Этот абзац появляется только в~диссертации.
Через проверку условия \verb!\!\verb!ifsynopsis!, задаваемого в~основном файле
документа (\verb!dissertation.tex! для диссертации), можно сделать новую
команду, обеспечивающую появление цитаты в~диссертации, но~не~в~автореферате.
\fi

% {\progress}
% Этот раздел должен быть отдельным структурным элементом по
% ГОСТ, но он, как правило, включается в описание актуальности
% темы. Нужен он отдельным структурынм элемементом или нет ---
% смотрите другие диссертации вашего совета, скорее всего не нужен.

{\aim} данной работы является \ldots

Для~достижения поставленной цели необходимо было решить следующие {\tasks}:
\begin{enumerate}[beginpenalty=10000] % https://tex.stackexchange.com/a/476052/104425
  \item Исследовать, разработать, вычислить и~т.\:д. и~т.\:п.
  \item Исследовать, разработать, вычислить и~т.\:д. и~т.\:п.
  \item Исследовать, разработать, вычислить и~т.\:д. и~т.\:п.
  \item Исследовать, разработать, вычислить и~т.\:д. и~т.\:п.
\end{enumerate}


{\novelty}
\begin{enumerate}[beginpenalty=10000] % https://tex.stackexchange.com/a/476052/104425
  \item Впервые \ldots
  \item Впервые \ldots
  \item Было выполнено оригинальное исследование \ldots
\end{enumerate}

{\influence} \ldots

{\methods} \ldots

{\defpositions}
\begin{enumerate}[beginpenalty=10000] % https://tex.stackexchange.com/a/476052/104425
  \item Первое положение
  \item Второе положение
  \item Третье положение
  \item Четвертое положение
\end{enumerate}
В папке Documents можно ознакомиться с решением совета из Томского~ГУ
(в~файле \verb+Def_positions.pdf+), где обоснованно даются рекомендации
по~формулировкам защищаемых положений.

{\reliability} полученных результатов обеспечивается \ldots \ Результаты находятся в соответствии с результатами, полученными другими авторами.


{\probation}
Основные результаты работы докладывались~на:
перечисление основных конференций, симпозиумов и~т.\:п.

{\contribution} Автор принимал активное участие \ldots

\ifnumequal{\value{bibliosel}}{0}
{%%% Встроенная реализация с загрузкой файла через движок bibtex8. (При желании, внутри можно использовать обычные ссылки, наподобие `\cite{vakbib1,vakbib2}`).
    {\publications} Основные результаты по теме диссертации изложены
    в~XX~печатных изданиях,
    X из которых изданы в журналах, рекомендованных ВАК,
    X "--- в тезисах докладов.
}%
{%%% Реализация пакетом biblatex через движок biber
    \begin{refsection}[bl-author, bl-registered]
        % Это refsection=1.
        % Процитированные здесь работы:
        %  * подсчитываются, для автоматического составления фразы "Основные результаты ..."
        %  * попадают в авторскую библиографию, при usefootcite==0 и стиле `\insertbiblioauthor` или `\insertbiblioauthorgrouped`
        %  * нумеруются там в зависимости от порядка команд `\printbibliography` в этом разделе.
        %  * при использовании `\insertbiblioauthorgrouped`, порядок команд `\printbibliography` в нём должен быть тем же (см. biblio/biblatex.tex)
        %
        % Невидимый библиографический список для подсчёта количества публикаций:
        \printbibliography[heading=nobibheading, section=1, env=countauthorvak,          keyword=biblioauthorvak]%
        \printbibliography[heading=nobibheading, section=1, env=countauthorwos,          keyword=biblioauthorwos]%
        \printbibliography[heading=nobibheading, section=1, env=countauthorscopus,       keyword=biblioauthorscopus]%
        \printbibliography[heading=nobibheading, section=1, env=countauthorconf,         keyword=biblioauthorconf]%
        \printbibliography[heading=nobibheading, section=1, env=countauthorother,        keyword=biblioauthorother]%
        \printbibliography[heading=nobibheading, section=1, env=countregistered,         keyword=biblioregistered]%
        \printbibliography[heading=nobibheading, section=1, env=countauthorpatent,       keyword=biblioauthorpatent]%
        \printbibliography[heading=nobibheading, section=1, env=countauthorprogram,      keyword=biblioauthorprogram]%
        \printbibliography[heading=nobibheading, section=1, env=countauthor,             keyword=biblioauthor]%
        \printbibliography[heading=nobibheading, section=1, env=countauthorvakscopuswos, filter=vakscopuswos]%
        \printbibliography[heading=nobibheading, section=1, env=countauthorscopuswos,    filter=scopuswos]%
        %
        \nocite{*}%
        %
        {\publications} Основные результаты по теме диссертации изложены в~\arabic{citeauthor}~печатных изданиях,
        \arabic{citeauthorvak} из которых изданы в журналах, рекомендованных ВАК\sloppy%
        \ifnum \value{citeauthorscopuswos}>0%
            , \arabic{citeauthorscopuswos} "--- в~периодических научных журналах, индексируемых Web of~Science и Scopus\sloppy%
        \fi%
        \ifnum \value{citeauthorconf}>0%
            , \arabic{citeauthorconf} "--- в~тезисах докладов.
        \else%
            .
        \fi%
        \ifnum \value{citeregistered}=1%
            \ifnum \value{citeauthorpatent}=1%
                Зарегистрирован \arabic{citeauthorpatent} патент.
            \fi%
            \ifnum \value{citeauthorprogram}=1%
                Зарегистрирована \arabic{citeauthorprogram} программа для ЭВМ.
            \fi%
        \fi%
        \ifnum \value{citeregistered}>1%
            Зарегистрированы\ %
            \ifnum \value{citeauthorpatent}>0%
            \formbytotal{citeauthorpatent}{патент}{}{а}{}\sloppy%
            \ifnum \value{citeauthorprogram}=0 . \else \ и~\fi%
            \fi%
            \ifnum \value{citeauthorprogram}>0%
            \formbytotal{citeauthorprogram}{программ}{а}{ы}{} для ЭВМ.
            \fi%
        \fi%
        % К публикациям, в которых излагаются основные научные результаты диссертации на соискание учёной
        % степени, в рецензируемых изданиях приравниваются патенты на изобретения, патенты (свидетельства) на
        % полезную модель, патенты на промышленный образец, патенты на селекционные достижения, свидетельства
        % на программу для электронных вычислительных машин, базу данных, топологию интегральных микросхем,
        % зарегистрированные в установленном порядке.(в ред. Постановления Правительства РФ от 21.04.2016 N 335)
    \end{refsection}%
    \begin{refsection}[bl-author, bl-registered]
        % Это refsection=2.
        % Процитированные здесь работы:
        %  * попадают в авторскую библиографию, при usefootcite==0 и стиле `\insertbiblioauthorimportant`.
        %  * ни на что не влияют в противном случае
        \nocite{vakbib2}%vak
        \nocite{patbib1}%patent
        \nocite{progbib1}%program
        \nocite{bib1}%other
        \nocite{confbib1}%conf
    \end{refsection}%
        %
        % Всё, что вне этих двух refsection, это refsection=0,
        %  * для диссертации - это нормальные ссылки, попадающие в обычную библиографию
        %  * для автореферата:
        %     * при usefootcite==0, ссылка корректно сработает только для источника из `external.bib`. Для своих работ --- напечатает "[0]" (и даже Warning не вылезет).
        %     * при usefootcite==1, ссылка сработает нормально. В авторской библиографии будут только процитированные в refsection=0 работы.
}

При использовании пакета \verb!biblatex! будут подсчитаны все работы, добавленные
в файл \verb!biblio/author.bib!. Для правильного подсчёта работ в~различных
системах цитирования требуется использовать поля:
\begin{itemize}
        \item \texttt{authorvak} если публикация индексирована ВАК,
        \item \texttt{authorscopus} если публикация индексирована Scopus,
        \item \texttt{authorwos} если публикация индексирована Web of Science,
        \item \texttt{authorconf} для докладов конференций,
        \item \texttt{authorpatent} для патентов,
        \item \texttt{authorprogram} для зарегистрированных программ для ЭВМ,
        \item \texttt{authorother} для других публикаций.
\end{itemize}
Для подсчёта используются счётчики:
\begin{itemize}
        \item \texttt{citeauthorvak} для работ, индексируемых ВАК,
        \item \texttt{citeauthorscopus} для работ, индексируемых Scopus,
        \item \texttt{citeauthorwos} для работ, индексируемых Web of Science,
        \item \texttt{citeauthorvakscopuswos} для работ, индексируемых одной из трёх баз,
        \item \texttt{citeauthorscopuswos} для работ, индексируемых Scopus или Web of~Science,
        \item \texttt{citeauthorconf} для докладов на конференциях,
        \item \texttt{citeauthorother} для остальных работ,
        \item \texttt{citeauthorpatent} для патентов,
        \item \texttt{citeauthorprogram} для зарегистрированных программ для ЭВМ,
        \item \texttt{citeauthor} для суммарного количества работ.
\end{itemize}
% Счётчик \texttt{citeexternal} используется для подсчёта процитированных публикаций;
% \texttt{citeregistered} "--- для подсчёта суммарного количества патентов и программ для ЭВМ.

Для добавления в список публикаций автора работ, которые не были процитированы в
автореферате, требуется их~перечислить с использованием команды \verb!\nocite! в
\verb!Synopsis/content.tex!.
