
{\actuality} В теории игр конфликты с участием трёх и более преследующих собственные цели сторон по многим причинам считаются существенно более трудными для моделирования в сравнении с классическими парными противостояниями. Среди этих причин следует особо подчеркнуть влияние, оказываемое на их ход информационной асимметрией. В играх с двумя участниками её эффект, в частности, сводится к последствиям априорной неполноты их знаний о параметрах конфликта и обычно описывается при помощи байесовских моделей. Платёжная функция в таких случаях не является общим знанием игроков, каждый из них действует исходя из собственных, возможно, различающихся предположений, выраженных в форме распределения вероятностей на пространстве всевозможных платёжных функций с заданным пространством стратегий.

Ещё одним источником информационной асимметрии в парном конфликте может выступать присутствие в действиях оппонентов тайной составляющей в тех случаях, когда он проходит в несколько стадий. При этом действия, совершённые игроком на ранних стадиях, позднее становятся полностью или частично известны его противнику, чья стратегия может варьироваться в зависимости от них. Это естественным образом формализуется при помощи информационных разбиений дерева ходов в развёрнутой форме игры. По сути, двумя упомянутыми аспектами исчерпывается влияние информационной асимметрии на конфликты с двумя сторонами. Однако, увеличение количества участников ещё хотя бы на одного порождает новый феномен, замеченный ещё Робертом Ауманом в статье \ifsynopsis\smartcite{Aumann74}\else\cite{Aumann74}\fi, где впервые было сформулировано коррелированное расширение игр в нормальной форме.

Выражается этот феномен в том, как для одних и тех же игр соотносятся множества равновесий по Нэшу в смешанных стратегиях и с использованием внешних механизмов корреляции. Для случая двух игроков все вектора математических ожиданий выплат в точках коррелированных равновесий принадлежат выпуклой оболочке множества смешанных, т.\:е. механизмы корреляции можно рассматривать просто как способ получения линейных комбинаций классических решений. С появлением третьего игрока картина меняется "--- в некоторых играх присутствие непубличного корреляционного механизма позволяет достичь точек равновесия по Нэшу с выплатами за пределами выпуклой оболочки решений в смешанных стратегиях. Фактически, это означает, что асимметрия знаний может оказывать существенное влияние на исход многостороннего конфликта даже в тех случаях, когда собственно предмет знаний вообще не имеет к нему отношения. Назовём игры, подверженные этому эффекту, \emph{чувствительными к дополнительной информационной асимметрии}.

Хотя описанный феномен известен уже давно, при построении моделей большинство исследователей обходили его стороной, рассматривая скорее как курьёзное свойство некоторых игр многих игроков. Важным исключением при этом выступает, пожалуй, наиболее значимая из активно использующих формализм коррелированного расширения игр в нормальной форме область теории игр "--- <<дизайн механизмов>> \ifsynopsis\smartcite{Nikolenko}\else\cite{Nikolenko}\fi Леонида Гурвича, Эрика Маскина и Роджера Майерсона. Их подход ставит своей целью создание экономических инструментов, стимулирующих эгоистичных рациональных агентов к поведению, оптимальному с точки зрения общих целевых функций, формализующих различные социальные блага. Будучи чрезвычайно плодотворной областью исследований, дизайн механизмов породил множество направлений и ответвлений, объединённых тем не менее рядом неотъемлемых общих черт, проистекающих из информационной структуры игр, для которых доказываются его основные положения. Типичная схема взаимодействий выглядит так: игроки"~агенты, знающие свои предпочтения и возможности, но находящиеся в неведении относительно этих параметров у других участников, информируют о них центр, формирующий на основе этой информации набор коррелированных стратегий. Далее центр реализует его для конкретного случая в виде набора чистых стратегий и инструктирует каждого из агентов, которые в свою очередь и принимают окончательное решение о том или ином действии. При этом подразумевается, что агенты могут лгать на первом этапе и не подчиняться на последнем. Главной задачей дизайна механизмов в этой парадигме становится создание таких алгоритмов поведения центра, что стратегии правдивости и послушания образуют для агентов равновесие Нэша.

Не вдаваясь в детали, можно сказать, что дизайн механизмов базируется на частном случае информационной асимметрии "--- своего рода звёздчатой структуре связей, где выделенный центральный агент может в своих интересах распоряжаться общим механизмом корреляции, а подчинённые ему агенты находятся в полной изоляции как друг от друга, так и от остального мира. Название здесь действительно неплохо отражает свойственный для модели взгляд на конфликты "--- через её призму стороны рассматриваются как взаимозаменимые детали единого рукотворного механизма, не связанные ничем кроме участия в нём. Хотя моделирование в рамках подобного упрощения вполне может быть полезно при конструировании формализованных способов решения конфликтов, оно никак не может помочь в тех случаях, когда на них оказывает существенное влияние информационная асимметрия, складывающаяся не в результате сознательного дизайна, а естественным образом, по мере спонтанного взаимодействия агентов в неконтролируемой, внешней с точки зрения модели среде. К примеру, сложно переоценить значимость влияния коррупции на политические и экономические институты, а ведь она складывается именно из таких незапланированных информационных связей, внешних по отношению к самим институтам.

По описанной причине исследование влияния дополнительной информационной асимметрии на решения игр многих игроков никак нельзя сводить только к конструктивным моделям. Увы, но за пределами дизайна механизмов сложилась традиция игнорировать этот феномен. К примеру, в статье \ifsynopsis\smartcite{Fudenberg}\else\cite{Fudenberg}\fi Дрю Фуденберга и Эрика Маскина можно найти следующую сноску: <<Actually, if $n \ge 3$, the other players may be able to keep player $j$'s payoff even lower by using a correlated strategy against $j$, where the outcome of the correlating device is not observed by $j$ (...). In keeping with the rest of the literature on repeated games, however, we shall rule out such correlated strategies.>> \ifsynopsis\else А ведь казалось бы, в контексте повторяющихся игр с дисконтированием проблематика использования секретности механизма корреляции для усиления стратегий наказания довольно любопытна "--- наверное в каждой области исследований, использующей народную теорему, от антропологии до международной политики, несложно отыскать примеры того, как группы агентов усиливали свою коллективную долгосрочную позицию при помощи необходимо тайного согласования действий. Увы, но приходится констатировать, что теории игр до сих пор почти нечего предложить другим наукам в качестве инструмента анализа описанного феномена. \fi

% {\progress}
% Этот раздел должен быть отдельным структурным элементом по
% ГОСТ, но он, как правило, включается в описание актуальности
% темы. Нужен он отдельным структурынм элемементом или нет ---
% смотрите другие диссертации вашего совета, скорее всего не нужен.

{\aim} данной работы является создание новой модели многосторонних конфликтов, учитывающей влияние на их ход дополнительной информационной асимметрии.

Для~достижения поставленной цели необходимо было решить следующие {\tasks}:
\begin{enumerate}[beginpenalty=10000] % https://tex.stackexchange.com/a/476052/104425
  \item Исследовать формализм коррелированного обобщения игр в нормальной форме с точки зрения проблематики работы.
  \item Разработать способ описания информационных структур, достаточно разнообразным образом связывающих участников произвольного конфликта.
  \item Исследовать влияние дополнительной информационной асимметрии на соответствие равновесий критериям коллективной рациональности.
  \item Разработать приемлемую концепцию решения с учётом связей между агентами для игр с дополнительной информационной асимметрией.
\end{enumerate}


{\novelty}
\begin{enumerate}[beginpenalty=10000] % https://tex.stackexchange.com/a/476052/104425
  \item Впервые выделены в качестве самостоятельного объекта исследования игры многих игроков, проявляющие чувствительность к дополнительной информационной асимметрии.
  \item Впервые предложен формализм пространства заговоров, специальным образом сужающий в целях моделирования дополнительной информационной асимметрии формализм пространства корреляции.
  \item Впервые сформулирована концепция структурно согласованного равновесия, позволяющая во многих случаях выделять среди решений игр в пространствах заговоров отвечающие принципу коллективной рациональности.
  \item Впервые показана возможность расширения множества совершенных подыгровых равновесий в повторяющихся играх, чувствительных к дополнительной информационной асимметрии, при помощи инструментов современной криптографии.
\end{enumerate}

{\influence} работы проистекает из явной необходимости учитывать при моделировании многосторонних конфликтов тот факт, что состав их участников не является в большинстве случаев случайной выборкой никак не связанных друг с другом агентов. Классический формализм игр в нормальной форме опирается на неявное допущение, состоящее в том, что единственной значимой характеристикой каждого игрока является порядок его предпочтений относительно исхода розыгрыша, выражающийся в форме платёжной функции. Совершенно очевидно при этом, что реальных людей, вступающих в противостояние, зачастую связывают значимые для его исхода отношения, структура которых не может быть выражена простым сочетанием платёжных функций. \ifsynopsis\else В качестве наглядной иллюстрации такой связи можно сравнить две воображаемые партии в бридж или преферанс с участием одинаково сильных игроков, различающиеся тем, что в одном случае за столом сидят незнакомцы, а в другом часть из них играют вместе уже много лет. Любой достаточно опытный картёжник скажет, что при равных навыках фактор <<сыгранности>> с партнёром надёжно обеспечивает решающее преимущество. Естественным образом этот феномен можно обобщить и на более значимые конфликты: политика, бизнес, дипломатия "--- везде, где исход противостояния существенно зависит от согласованности и непредсказуемости действий, взаимопонимание, не требующее коммуникации, зачастую может превратить поражение в победу. \fi Таким образом, для более точного предсказания исходов многосторонних конфликтов насущно необходимы модели, позволяющие учитывать этот фактор.

{\methods} В работе используются методы теории игр, теории вероятности, топологии и криптографии.

{\defpositions}
\begin{enumerate}[beginpenalty=10000] % https://tex.stackexchange.com/a/476052/104425
  \item Доказательство теоремы об изоморфизме пространств корреляции, определяющей для них классы эквивалентности, в которые входят только неразличимые с теоретико-игровой точки зрения пространства.
  \item Доказательство теоремы о пространствах заговоров одной структуры, благодаря которой структуру пространства можно считать его исчерпывающим описанием.
  \item Доказательство чувствительности к дополнительной информационной асимметрии симметричной проблемы планирования заданий с немонотонной отдачей.
  \item Решение трёхсторонней симметричной проблемы планирования с немонотонной функцией оплаты за срочность в ассиметричном пространстве заговоров, удовлетворяющее критерию структурной согласованности.
  \item Модель повторяющихся игр с учётом стоимости вычислений, необходимых для выбора хода на очередной итерации.
  \item Криптографические стратегии наказания для повторяющегося трёхстороннего чёт-нечета, пополняющие множество совершенных подыгровых равновесий точками, не достижимыми без принятия во внимание сложности алгоритмов.
\end{enumerate}

%{\reliability} полученных результатов обеспечивается при помощи формальных доказательств с применением топологических инструментов. Результаты находятся в соответствии с результатами, полученными другими авторами.

{\probation}
Основные результаты работы докладывались~на: Ломоносовских чтениях (2017, 2021 гг.) \cite{ownlmr2017, ownlmr2021}, IX Московской международной конференции по исследованию операций \cite{ownorm2018} и конференции молодых учёных по математической экономике и экономической теории (MEET-2021) \cite{meet2021}.

{\contribution} Результаты, представленные теоремами в диссертационной работе и автореферате, помимо имеющих ссылки на работы других авторов, положения, выносимые на защиту, получены автором самостоятельно. Подготовка к публикации полученных результатов проводилась без соавторов.

{\publications} Основные результаты по теме диссертации изложены в~7~печатных работах, 3 из которых \cite{ownconsp}\cite{owncards}\cite{ownsched} изданы в периодическом научном журнале, рекомендованном ВАК и индексируемом Web of Science и Scopus. Центральная работа имеет перевод на английский язык\cite{ownconsp_eng}. 3 работы изданы в тезисах докладов.
