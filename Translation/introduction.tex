\chapter*{Introduction}                         % Заголовок
\addcontentsline{toc}{chapter}{Introduction}    % Добавляем его в оглавление

\newcommand{\actuality}{}
\newcommand{\progress}{}
\newcommand{\aim}{{\textbf\aimTXT}}
\newcommand{\tasks}{\textbf{\tasksTXT}}
\newcommand{\novelty}{\textbf{\noveltyTXT}}
\newcommand{\influence}{\textbf{\influenceTXT}}
\newcommand{\methods}{\textbf{\methodsTXT}}
\newcommand{\defpositions}{\textbf{\defpositionsTXT}}
\newcommand{\reliability}{\textbf{\reliabilityTXT}}
\newcommand{\probation}{\textbf{\probationTXT}}
\newcommand{\contribution}{\textbf{\contributionTXT}}
\newcommand{\publications}{\textbf{\publicationsTXT}}


{\actuality} In game theory, modeling of conflicts with three or more parties independently pursuing own goals is, for many reasons, considered much more difficult comparing to modeling of classical bilateral confrontations. Among these reasons one plays very special part "--- the influence of information asymmetry exerted on course of a struggle. For games with two participants its effect generally comes down to the consequences of a priori incompleteness in their knowledge about the parameters of the conflict and is usually described with Bayesian models. Common knowledge in such cases leaves out players' payoff functions, so each of them acts according to their own, possibly different assumptions, expressed in the form of a probability distribution in the space of all payoff functions with a given set of strategies. %В теории игр конфликты с участием трёх и более преследующих собственные цели сторон по многим причинам считаются существенно более трудными для моделирования в сравнении с классическими парными противостояниями. Среди этих причин следует особо подчеркнуть влияние, оказываемое на их ход информационной асимметрией. В играх с двумя участниками её эффект, в целом, сводится к последствиям априорной неполноты их знаний о параметрах конфликта и обычно описывается при помощи байесовских моделей. Платёжная функция в таких случаях не является общим знанием игроков, каждый из них действует исходя из собственных, возможно, различающихся предположений, выраженных в форме распределения вероятностей на пространстве всевозможных платёжных функций с заданным множеством стратегий.

For multistage bilateral conflicts there can be another source of information asymmetry "--- the presence of a secret component in the actions of opponents. If this occurs, actions already performed by the player in the early stages may be completely or partially unknown to his opponent, who is forced to base the strategy of later stages on assumptions. Naturally, this is formalized through partitioning of the extensive-form game tree into information sets. Effectively, this two aspects exhaust the impact of information asymmetry on two-sided conflicts. However, addition of third participant induces a new phenomenon, noticed by Robert Aumann in his article \cite{Aumann74} introducing now-familiar new formalism "--- correlated extension of normal form games. %Ещё одним источником информационной асимметрии в парном конфликте может выступать присутствие в действиях оппонентов тайной составляющей в тех случаях, когда он проходит в несколько стадий. При этом действия, уже совершённые игроком на ранних стадиях, могут быть полностью или частично неизвестны его противнику, вынужденному основывать стратегию более поздних стадий на предположениях. Это естественным образом формализуется при помощи информационных разбиений дерева ходов в развёрнутой форме игры. По сути, двумя упомянутыми аспектами исчерпывается влияние информационной асимметрии на конфликты с двумя сторонами. Однако, увеличение количества участников ещё хотя бы на одного порождает новый феномен, замеченный ещё Робертом Ауманом в статье \ifsynopsis\smartcite{Aumann74}\else\cite{Aumann74}\fi, где впервые было сформулировано коррелированное расширение игр в нормальной форме.

This phenomenon reflects on comparison of the sets of Nash equilibria for the same games, calculated using mixed strategies on one level and involving external correlation mechanisms on the other. In case of two players, all vectors of expected utility in the correlated equilibria belong to the convex hull of the set of mixed equilibria utilities, i.e., correlation mechanisms can be thought of simply as a way to obtain linear combinations of classical solutions. With the advent of the third player, the picture changes "--- in some games, the presence of a non-public correlation mechanism allows reaching Nash equilibrium points with payoffs outside the convex hull of mixed solutions' utilities. Effectively, it means that asymmetry of knowledge can have a significant impact on the outcome of a multilateral conflict even in cases when its subject have no relation to the payout structure. Let's call the games prone to this effect \emph{sensitive to additional information asymmetry}. %Выражается этот феномен в том, как для одних и тех же игр соотносятся множества равновесий по Нэшу в смешанных стратегиях и с использованием внешних механизмов корреляции. Для случая двух игроков все вектора математических ожиданий выплат в точках коррелированных равновесий принадлежат выпуклой оболочке множества смешанных равновесий, т.\:е. механизмы корреляции можно рассматривать просто как способ получения линейных комбинаций классических решений. С появлением третьего игрока картина меняется "--- в некоторых играх присутствие непубличного корреляционного механизма позволяет достичь точек равновесия по Нэшу с выплатами за пределами выпуклой оболочки решений в смешанных стратегиях. Фактически, это означает, что асимметрия знаний может оказывать существенное влияние на исход многостороннего конфликта даже в тех случаях, когда она не касается существенных для структуры выплат факторов. Назовём игры, подверженные этому эффекту, \emph{чувствительными к дополнительной информационной асимметрии}.

While the described phenomenon has been known for a while, most of the models built by researchers bypassed it, as it is considered rather feature of curiosity in some games of many players. An important exception worth noting is, perhaps, the most significant area of game theory actively using the correlated extension formalism of games in normal form "--- <<mechanism design>> \cite{Nikolenko} by Leonid Gurvitch, Eric Maskin, and Roger Myerson. Their approach aims to create economic instruments that incentivize selfish rational agents to behave optimally in alignment with common goal functions that formalize various social goods. Being an extremely fruitful area of research, mechanism design has spawned many directions and branches, united nevertheless by a number of integral common features arising from the information structure of games for which its main provisions are proved. A typical interaction scheme looks like this: players"~agents, knowing their own preferences and capabilities, but being ignorant of these parameters for other participants, report their type to the central authority, which coins a set of correlated strategies based on this information. Next, the center actualizes it for an instance in the form of a pure strategies set and instructs each of the agents, who, in turn, make the final decision on a particular action. This implies that agents can lie at the first stage and disobey at the last. In paradigm of mechanism design the main goal is about the creation of such algorithms for the behavior of central authority that the strategies of truthfulness and obedience form a Nash equilibrium for agents. %Хотя описанный феномен известен уже давно, при построении моделей большинство исследователей обходили его стороной, рассматривая скорее как курьёзную особенность некоторых игр многих игроков. Важным исключением при этом выступает, пожалуй, наиболее значимая из активно использующих формализм коррелированного расширения игр в нормальной форме область теории игр "--- <<дизайн механизмов>> \ifsynopsis\smartcite{Nikolenko}\else\cite{Nikolenko}\fi Леонида Гурвича, Эрика Маскина и Роджера Майерсона. Их подход ставит своей целью создание экономических инструментов, стимулирующих эгоистичных рациональных агентов к поведению, оптимальному с точки зрения общих целевых функций, формализующих различные социальные блага. Будучи чрезвычайно плодотворной областью исследований, дизайн механизмов породил множество направлений и ответвлений, объединённых тем не менее рядом неотъемлемых общих черт, проистекающих из информационной структуры игр, для которых доказываются его основные положения. Типичная схема взаимодействий выглядит так: игроки"~агенты, знающие свои предпочтения и возможности, но находящиеся в неведении относительно этих параметров у других участников, информируют о них центр, формирующий на основе этой информации набор коррелированных стратегий. Далее центр реализует его для конкретного случая в виде набора чистых стратегий и инструктирует каждого из агентов, которые в свою очередь и принимают окончательное решение о том или ином действии. При этом подразумевается, что агенты могут лгать на первом этапе и не подчиняться на последнем. Главной задачей дизайна механизмов в этой парадигме становится создание таких алгоритмов поведения центра, что стратегии правдивости и послушания образуют для агентов равновесие Нэша.

Without diving into details, it can be said that mechanism design is based on a special case of information asymmetry "--- a kind of star connection structure, where a dedicated central agent controls the single correlation mechanism in his own interests, while subordinate agents are in complete isolation both from each other, and from the outside world. The name here accurately reflects the view of conflicts inherent for the model "--- through its lens, the players are seen as interchangeable parts of a single man-made mechanism, not connected by anything other than participation in it. Although modeling within the framework of this simplification can be useful in construction of formalized methods for conflict resolution, it doesn't cover cases where they are significantly affected by information asymmetries developing not as a result of calculated design, but naturally, as spontaneous interaction occurs between agents in an uncontrolled, external to the model environment. For example, it is difficult to overestimate the significance of corruption's influence on political and economic institutions, and yet it is made up of precisely such unplanned information links external to the institutions themselves. %Не вдаваясь в детали, можно сказать, что дизайн механизмов базируется на частном случае информационной асимметрии "--- своего рода звёздчатой структуре связей, где выделенный центральный агент может в своих интересах распоряжаться общим механизмом корреляции, а подчинённые ему агенты находятся в полной изоляции как друг от друга, так и от остального мира. Название здесь действительно неплохо отражает свойственный для модели взгляд на конфликты "--- через её призму стороны рассматриваются как взаимозаменимые детали единого рукотворного механизма, не связанные ничем кроме участия в нём. Хотя моделирование в рамках подобного упрощения вполне может быть полезно при конструировании формализованных способов решения конфликтов, оно никак не может помочь в тех случаях, когда на них оказывает существенное влияние информационная асимметрия, складывающаяся не в результате сознательного дизайна, а естественным образом, по мере спонтанного взаимодействия агентов в неконтролируемой, внешней с точки зрения модели среде. К примеру, сложно переоценить значимость влияния коррупции на политические и экономические институты, а ведь она складывается именно из таких незапланированных информационных связей, внешних по отношению к самим институтам.

For the reason described above, the study of the influence of additional information asymmetry on the solutions of multilateral games shouldn't be reduced to purely constructive models. Alas, outside of mechanism design disregard for this phenomenon became prevailing convention. For example, in the article \cite{Fudenberg} by Drew Fudenberg and Eric Maskin, you can find the following footnote: <<In essence, if $n \ge 3$, the rest of the players get the opportunity to omit player $j$ payouts even lower, using a correlated strategy against him, in which player $j$ cannot observe the signal of the correlation mechanism (...). Adhering to the tradition that has developed in publications devoted to repetitive games, however, we will not consider such correlated strategies.>> It would seem, however, that in the context of repeated games with discounting, the topic of using the secrecy of the correlation mechanism to strengthen punishment strategies is rather intriguing "--- examples of agent groups strengthening their collective long-term position through necessarily secret coordination of actions seemingly isn't hard to find in every field of research that uses the folk theorem, from anthropology to international politics. Disconcertingly, up to date game theory has very little to offer as an analysis tool for the described phenomenon to other sciences. %По описанной причине исследование влияния дополнительной информационной асимметрии на решения игр многих игроков никак нельзя сводить только к конструктивным моделям. Увы, но за пределами дизайна механизмов сложилась традиция игнорировать этот феномен. К примеру, в статье \cite{Fudenberg} Дрю Фуденберга и Эрика Маскина можно найти следующую сноску: <<Actually, if $n \ge 3$, the other players may be able to keep player $j$'s payoff even lower by using a correlated strategy against $j$, where the outcome of the correlating device is not observed by $j$ (...). In keeping with the rest of the literature on repeated games, however, we shall rule out such correlated strategies.>>\footnote{<<В сущности, если $n \ge 3$, остальные игроки получают возможность опустить выплаты игрока $j$ ещё ниже, используя против него коррелированную стратегию, в которой игрок $j$ не может наблюдать сигнала механизма корреляции (...). Придерживаясь сложившейся в посвящённых повторяющимся играм публикациях традиции, мы, впрочем, не будем рассматривать такие коррелированные стратегии.>>} А ведь казалось бы, в контексте повторяющихся игр с дисконтированием проблематика использования секретности механизма корреляции для усиления стратегий наказания довольно любопытна "--- наверное в каждой области исследований, использующей народную теорему, от антропологии до международной политики, несложно отыскать примеры того, как группы агентов усиливали свою коллективную долгосрочную позицию при помощи необходимо тайного согласования действий. Увы, но приходится констатировать, что теории игр до сих пор почти нечего предложить другим наукам в качестве инструмента анализа описанного феномена.

% {\progress}
% Этот раздел должен быть отдельным структурным элементом по
% ГОСТ, но он, как правило, включается в описание актуальности
% темы. Нужен он отдельным структурынм элемементом или нет ---
% смотрите другие диссертации вашего совета, скорее всего не нужен.

This study is \textbf{aim}ed toward development of new multilateral conflict model, that takes into account how its course is affected by additional informational asymmetry. %{\aim} данной работы является создание новой модели многосторонних конфликтов, учитывающей влияние на их ход дополнительной информационной асимметрии.

To approach stated goal following \textbf{tasks} had to be addressed: %Для~достижения поставленной цели необходимо было решить следующие {\tasks}:
\begin{enumerate}[beginpenalty=10000] % https://tex.stackexchange.com/a/476052/104425
  \item Analyze the correlated extension of normal form games model in scope of the research. %Исследовать формализм коррелированного обобщения игр в нормальной форме с точки зрения проблематики работы.
  \item Develop the notation aimed to describe the informational structures, that could be tying together participants of the arbitrary conflicts in a variety of fashions. %Разработать способ описания информационных структур, достаточно разнообразным образом связывающих участников произвольного конфликта.
  \item Analyze the impact of additional informational asymmetry on the solutions' conformity with the criteria of collective rationality. %Исследовать влияние дополнительной информационной асимметрии на соответствие равновесий критериям коллективной рациональности.
  \item Develop the reasonable solution concept for the games with additional informational asymmetry, taking into account the inter-agent relations. %Разработать приемлемую концепцию решения с учётом связей между агентами для игр с дополнительной информационной асимметрией.
\end{enumerate}


\textbf{Scientific novelty:}
\begin{enumerate}[beginpenalty=10000] % https://tex.stackexchange.com/a/476052/104425
  \item The games of many players, which are sensitive to additional information asymmetry, are singled out as an independent object of study. %Выделены в качестве самостоятельного объекта исследования игры многих игроков, проявляющие чувствительность к дополнительной информационной асимметрии.
  \item A formalism of the conspiracy space is proposed, purposefully narrowing the formalism of the correlation space in order to model additional information asymmetry. %Предложен формализм пространства заговоров, специальным образом сужающий в целях моделирования дополнительной информационной асимметрии формализм пространства корреляции.
  \item The concept of structurally consistent equilibrium is formulated, in many cases allowing to single out among the solutions of games in conspiracy spaces those that adhere to the principle of collective rationality. %Сформулирована концепция структурно согласованного равновесия, позволяющая во многих случаях выделять среди решений игр в пространствах заговоров отвечающие принципу коллективной рациональности.
  \item The possibility of extending the set of perfect subgame equilibria in repeated games sensitive to additional information asymmetry  with the help of modern cryptography tools is demonstrated. %Показана возможность расширения множества совершенных подыгровых равновесий в повторяющихся играх, чувствительных к дополнительной информационной асимметрии, при помощи инструментов современной криптографии.
\end{enumerate}

The \textbf{practical significance} of the work stems from the obvious need to take into account, when modeling multilateral conflicts, the fact that the composition of their participants is not, in most cases, a random sample of agents that are not related to each other in any way. The classical formalism of games in normal form relies on the implicit assumption that the only significant characteristic of each player is the order of preference regarding the outcome of the draw, expressed in the form of a payoff function. At the same time, it is quite obvious that real people entering into a confrontation are often connected by relationships, significant for its outcome, the structure of which cannot be expressed by a simple combination of payment functions. Such connection can be clearly illustrated by comparing two imaginary games of bridge or preference with equally strong players, differing in that one table is occupied by strangers, while at the other some have been playing together for many years. Any sufficiently experienced card-player can confirm that, given equal skill, the <<cohesion>> factor between partners reliably provides a decisive advantage. Naturally, this phenomenon can be generalized to more significant conflicts: politics, business, diplomacy "--- wherever the outcome of the confrontation depends significantly on the coordination and unpredictability of actions, mutual understanding that does not require communication can often turn defeat into victory. Thus, to improve accuracy of predictions for the outcomes of multilateral conflicts, there is a compelling need for models that take this factor into account. %{\influence} работы проистекает из явной необходимости учитывать при моделировании многосторонних конфликтов тот факт, что состав их участников не является в большинстве случаев случайной выборкой никак не связанных друг с другом агентов. Классический формализм игр в нормальной форме опирается на неявное допущение, состоящее в том, что единственной значимой характеристикой каждого игрока является порядок его предпочтений относительно исхода розыгрыша, выражающийся в форме платёжной функции. Совершенно очевидно при этом, что реальных людей, вступающих в противостояние, зачастую связывают значимые для его исхода отношения, структура которых не может быть выражена простым сочетанием платёжных функций. \ifsynopsis\else В качестве наглядной иллюстрации такой связи можно сравнить две воображаемые партии в бридж или преферанс с участием одинаково сильных игроков, различающиеся тем, что в одном случае за столом сидят незнакомцы, а в другом часть из них играют вместе уже много лет. Любой достаточно опытный картёжник скажет, что при равных навыках фактор <<сыгранности>> с партнёром надёжно обеспечивает решающее преимущество. Естественным образом этот феномен можно обобщить и на более значимые конфликты: политика, бизнес, дипломатия "--- везде, где исход противостояния существенно зависит от согласованности и непредсказуемости действий, взаимопонимание, не требующее коммуникации, зачастую может превратить поражение в победу. \fi Таким образом, для более точного предсказания исходов многосторонних конфликтов насущно необходимы модели, позволяющие учитывать этот фактор.

\textbf{Methodology and research methods.} The study utilizes frameworks of game theory, probability theory, topology and cryptography. %В работе используются методы теории игр, теории вероятности, топологии и криптографии.

\textbf{Defense positions:}
\begin{enumerate}[beginpenalty=10000] % https://tex.stackexchange.com/a/476052/104425
  \item Proof of the theorem on the isomorphism of correlation spaces, which defines equivalence classes for them, demarcating indistinguishable from the game-theoretic point of view spaces. %Доказательство теоремы об изоморфизме пространств корреляции, определяющей для них классы эквивалентности, в которые входят только неразличимые с теоретико-игровой точки зрения пространства.
  \item A proof of the theorem on conspiracy spaces of one structure, thanks to which the structure of a space describes it comprehensively. %Доказательство теоремы о пространствах заговоров одной структуры, благодаря которой структуру пространства можно считать его исчерпывающим описанием.
  \item Proof of sensitivity to additional information asymmetry of the symmetric task scheduling problem with nonmonotonic returns. %Доказательство чувствительности к дополнительной информационной асимметрии симметричной проблемы планирования заданий с немонотонной отдачей.
  \item A solution to a three-way symmetric scheduling problem with a nonmonotonic rush premium function in an asymmetric conspiracy space that satisfies the structural consistency criterion. %Решение трёхсторонней симметричной проблемы планирования с немонотонной функцией оплаты за срочность в ассиметричном пространстве заговоров, удовлетворяющее критерию структурной согласованности.
  \item A model of repeated games, that takes into account the cost of calculations required to choose a move at the each iteration. %Модель повторяющихся игр с учётом стоимости вычислений, необходимых для выбора хода на очередной итерации.
  \item Cryptographic punishment strategies for repeated three-player even-odd, widening the set of perfect subgame equilibria with points achievable only by taking into account the complexity of the algorithms. %Криптографические стратегии наказания для повторяющегося трёхстороннего чёт-нечета, пополняющие множество совершенных подыгровых равновесий точками, не достижимыми без принятия во внимание сложности алгоритмов.
\end{enumerate}

%{\reliability} полученных результатов обеспечивается при помощи формальных доказательств с применением топологических инструментов. Результаты находятся в соответствии с результатами, полученными другими авторами.

\textbf{Work approbation.}
The main results of the work were reported on: Lomonosov readings (2017, 2021) \cite{ownlmr2017, ownlmr2021}, IX Moscow International Conference on Operations Research \cite{ownorm2018} and Conference for Young Scientists in Mathematical Economics and Economic Theory (MEET-2021) \cite{meet2021}. %Основные результаты работы докладывались~на: Ломоносовских чтениях (2017, 2021 гг.) \cite{ownlmr2017, ownlmr2021}, IX Московской международной конференции по исследованию операций \cite{ownorm2018} и конференции молодых учёных по математической экономике и экономической теории (MEET-2021) \cite{meet2021}.

\textbf{Personal contribution.} The author independently obtained the results featured in the dissertation work in the form of theorems and other provisions submitted for defense. The obtained results were prepared for publication without co-authors. %Результаты, представленные в диссертационной работе теоремами и другими выносимыми на защиту положениями, получены автором самостоятельно. Подготовка к публикации полученных результатов проводилась без соавторов.

\textbf{Publications.} The main results on the topic of the dissertation are presented in~7~published papers, 3 of which \cite{ownconsp}\cite{owncards}\cite{ownsched} were published in a periodical scientific journal recommended by the HAC and indexed by Web of Science and Scopus. The central work has an English translation\cite{ownconsp_eng}. 3 papers were published in conference abstracts. %Основные результаты по теме диссертации изложены в~7~печатных работах, 3 из которых \cite{ownconsp}\cite{owncards}\cite{ownsched} изданы в периодическом научном журнале, рекомендованном ВАК и индексируемом Web of Science и Scopus. Центральная работа имеет перевод на английский язык\cite{ownconsp_eng}. 3 работы изданы в тезисах докладов.
 % Характеристика работы по структуре во введении и в автореферате не отличается (ГОСТ Р 7.0.11, пункты 5.3.1 и 9.2.1), потому её загружаем из одного и того же внешнего файла, предварительно задав форму выделения некоторым параметрам

\textbf{Volume and structure of work.} The dissertation consists of introduction, 3 chapters, conclusion and 4 appendices. The full volume of the dissertation is 86 pages, including 2 figures and 5 tables. The list of references contains 26 titles.
%\textbf{Объем и структура работы.} Диссертация состоит из~введения,
%\formbytotal{totalchapter}{глав}{ы}{}{},
%заключения и
%\formbytotal{totalappendix}{приложен}{ия}{ий}{}.
%%% на случай ошибок оставляю исходный кусок на месте, закомментированным
%%Полный объём диссертации составляет  \ref*{TotPages}~страницу
%%с~\totalfigures{}~рисунками и~\totaltables{}~таблицами. Список литературы
%%содержит \total{citenum}~наименований.
%%
%Полный объём диссертации составляет
%\formbytotal{TotPages}{страниц}{у}{ы}{}, включая
%\formbytotal{totalcount@figure}{рисун}{ок}{ка}{ков} и
%\formbytotal{totalcount@table}{таблиц}{у}{ы}{}.
%Список литературы содержит
%\formbytotal{citenum}{наименован}{ие}{ия}{ий}.
