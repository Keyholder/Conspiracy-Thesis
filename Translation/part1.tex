\chapter{Conspiracy model}\label{ch:ch1}

\section{Correlated extension of normal-form game}\label{sec:ch1/sec1}

In the established scientific tradition additional information asymmetry is usualy described by formalism of the correlated extension of the normal-form games proposed by Robert Aumann in \cite{Aumann74}. For convenience, its central elements will be presented here in a notation adapted to the Russian-speaking community. Consider the normal-form game $\Gamma = \langle A, S^a, u^a(s), a \in A \rangle$. The finite set of players from this point onward is denoted as $A = \{1, \ldots, m\}$, while the finite set of pure strategy profiles is denoted as $S = S^1 \times \ldots \times S^m$. In addition to the set of strategies $S^a$, each player is defined by the payoff function $u^a : S \rightarrow \mathbb{R}$. %Базовым формализмом для описания дополнительной информационной асимметрии в сложившейся научной традиции выступает коррелированное расширение нормальной формы игр, предложенное Робертом Ауманом в \cite{Aumann74}. Для удобства его центральные элементы будут изложены здесь в нотации, адаптированной к русскоязычной среде. Рассмотрим игру в нормальной форме $\Gamma = \langle A, S^a, u^a(s), a \in A \rangle$. Конечное множество игроков здесь и далее везде обозначается как $A = \{1, \ldots, m\}$, а конечное множество наборов чистых стратегий "--- $S = S^1 \times \ldots \times S^m$. Помимо множества стратегий $S^a$, каждый игрок определяется платёжной функцией $u^a : S \rightarrow \mathbb{R}$.

Consider also a probability space \cite{Kolmogorov} $\langle \Omega, \mathfrak{B}, \mathbb{P} \rangle$ in which the state of nature observed by the players is realized. Here $\Omega$ is the set of such states, $\mathfrak{B}$ is a $\sigma$"~algebra of subsets of $\Omega$, and $\mathbb{P} : \mathfrak{B} \rightarrow \mathbb{R}_{\ge 0}$ is a probability measure. To each player $a \in A$, we assign \emph{their own subspace} $\langle \Omega, \mathfrak{I}^a, \mathbb{P} \rangle$ such that $\mathfrak{I}^a \subseteq \mathfrak{B}$. The tuple of $\sigma$"~algebras $\mathfrak{I} = (\mathfrak{I}^a, a \in A)$ reflects the players’ awareness about the state of nature. In the situation described, the state of nature does not affect the payoff functions directly and serves solely as a means to synchronize the players’ actions. This means that the $\sigma$"~algebra $\mathfrak{B}$ itself is not a significant parameter of the model, and the measurability with respect to this algebra for $\mathbb{P}$ can be replaced by the measurability with respect to $\mathfrak{I}^a, \forall a \in A$. %Также рассмотрим вероятностное пространство\cite{Kolmogorov} $\langle \Omega, \mathfrak{B}, \mathbb{P} \rangle$, в котором реализуется наблюдаемое игроками состояние природы. Здесь $\Omega$ "--- множество всевозможных таких состояний, $\mathfrak{B}$ "--- $\sigma$"~алгебра подмножеств $\Omega$, а $\mathbb{P} : \mathfrak{B} \rightarrow \mathbb{R}_{\ge 0}$ "--- вероятностная мера. Каждому игроку $a \in A$ поставим в соответствие \emph{собственное подпространство} $\langle \Omega, \mathfrak{I}^a, \mathbb{P} \rangle$ такое, что $\mathfrak{I}^a \subseteq \mathfrak{B}$. При этом набор $\sigma$"~алгебр $\mathfrak{I} = (\mathfrak{I}^a, a \in A)$ отражает информированность игроков о состоянии природы. В описываемой ситуации это состояние не влияет на функции выигрышей непосредственно, выступая исключительно как способ синхронизации действий игроков. Это значит, что $\sigma$"~алгебра $\mathfrak{B}$ сама по себе не является существенным параметром модели, и измеримость по ней для $\mathbb{P}$ можно заменить измеримостью по $\mathfrak{I}^a, \forall a \in A$.

Separately, it is worth noting that for the players' own subspaces the Aumannian formalism historically assumed the individuality of not only $\sigma$"~algebras, but also their corresponding measures, thereby taking into account the possible subjectivity of estimates regarding the probability of certain events, which is important in cases where processes too complex for objective analysis (for example, sports competitions) act as a correlation mechanism. However, for the purposes of present study, this aspect does not make much sense, since the conspiracy model implies that conspirators can arbitrarily choose the mechanism of correlation, and in such situation, it is reasonable to expect that people will use simple sources of randomness with a known distribution (roulettes, dices, lots, etc.). For this reason, hereinafter, the model of correlated strategies is utilized in its simplified form, with an objective probability measure in the space of states of nature shared by all players. %Отдельно следует заметить, что в оригинале для собственных подпространств игроков Аумановский формализм предполагал индивидуальность не только $\sigma$"~алгебр, но и соответствующих им мер, учитывая тем самым возможную субъективность оценок вероятности наступления тех или иных событий, что немаловажно в случаях, когда в качестве механизма корреляции выступают процессы, слишком сложные для объективного анализа (например спортивные соревнования). Однако, для целей данного исследования этот аспект не имеет большого смысла, поскольку в модели заговоров подразумевается, что их участники могут произвольным образом выбирать механизм корреляции, а в такой ситуации разумно ожидать, что люди будут использовать простые источники случайности с известным распределением (рулетки, кости, жребий и т.д.). По этой причине здесь и далее модель коррелированных стратегий используется в её упрощённой форме, с общей для всех игроков объективной вероятностной мерой в пространстве состояний природы.

Thus, we obtain the parameter tuple $\Phi = \langle A, \Omega, \mathfrak{I}^a, \mathbb{P}, a \in A \rangle$ characterizing some correlation space for an arbitrary game with the set A of players. Note that one and the same \emph{correlation space} can be used in games with one set of players but with different sets of pure strategies and payoff functions. However, a \emph{correlated game extension} is completely determined by the pair $\Gamma | \Phi$. Let us describe the resulting new game in terms of normal form\footnote{Following the notation introduced in \cite{Aumann74}, the sets of strategies and outcome sets of a correlated extension of the game are denoted in bold ($\mathbf{s}$ and $\mathbf{S}$ where the underlying game has $s$ and $S$).}: %Таким образом, получается $\Phi = \langle A, \Omega, \mathfrak{I}^a, \mathbb{P}, a \in A \rangle$ "--- набор параметров, характеризующих некоторое \emph{пространство корреляции} для произвольной игры с множеством игроков $A$. Отметим, что в играх с одним множеством игроков, но различными множествами чистых стратегий и функциями выигрыша можно применять одно и то же пространство корреляции. Полностью же \emph{коррелированное расширение игры} определяет пара $\Gamma | \Phi$. Опишем полученную новую игру в терминах нормальной формы\footnote{Следуя нотации, введённой в \cite{Aumann74}, наборы стратегий и множества исходов коррелированного расширения игры обозначаются жирным шрифтом ($\mathbf{s}$ и $\mathbf{S}$ там, где в базовой игре $s$ и $S$).}:
\begin{equation*}
	\Gamma | \Phi = \langle A, \mathbf{S}^a, u^a(\mathbf{s}), a \in A \rangle.
\end{equation*}

Here the set $\mathbf{S}^a$ of correlated strategies available to a player $a$ consists of all $\mathfrak{I}^a$"~measurable functions $\mathbf{s}^a : \Omega \rightarrow S^a$ mapping the set of possible states of nature onto the set of pure strategies available to this player. Accordingly, the payoff function is calculated by the formula for the expectation of a random variable, %Здесь множество $\mathbf{S}^a$ доступных игроку $a$ коррелированных стратегий состоит из всех $\mathfrak{I}^a$"~измеримых функций $\mathbf{s}^a : \Omega \rightarrow S^a$, отображающих множество возможных состояний природы на множество доступных ему чистых стратегий. Соответственно, функция выигрыша вычисляется по формуле математического ожидания случайной величины
\begin{equation*}
	u^a(\mathbf{s}) = \sum\limits_{s \in S} \mathbb{P}(\mathbf{s}^{-1}(s)) u^a(s),\;\mathbf{s}^{-1}(s) = \{\omega \in \Omega \mid \mathbf{s}^a(\omega) = s^a, \forall a \in A\},
\end{equation*}
where the $\mathbb{P}(\mathbf{s}^{-1}(s))$ serve as the coefficients of the distribution on the game matrix. %где $\mathbb{P}(\mathbf{s}^{-1}(s))$ выступают в роли коэффициентов распределения на матрице игры.

In his article, Aumann demonstrates by examples how games can obtain new Nash equilibrium points using the power of the introduced formalism. Depending on the parameters of the correlation spaces, one can construct not only solutions with any payoffs from the convex hull of the payoff vectors at the points of the classical mixed Nash equilibrium, but for some games even solutions that lie outside such a convex hull. This allows us to formulate the key concept of this work: %В своей статье Ауман демонстрирует силу вводимого формализма, показывая на примерах, как с его помощью в играх можно получать новые точки равновесия по Нэшу. Подбирая параметры пространств корреляции, можно конструировать не только решения с любыми выплатами из выпуклой оболочки векторов выплат в точках классического смешанного равновесия Нэша, но для некоторых игр даже решения, лежащие за пределами такой выпуклой оболочки. Это позволяет сформулировать ключевое для данной работы понятие:
\begin{definition}
	Let $\Gamma$ be a game in normal form with $m$ players, and $U \subseteq \mathbb{R}^m$ be the set of all payoff vectors achievable in its mixed Nash equilibria. A game $\Gamma$ is said to be \emph{sensitive to additional information asymmetry} when there exists a correlation space $\Phi$ such that for the game $\Gamma | \Phi$ there is a correlated Nash equilibrium with a payoff vector not belonging to the convex hull of the set $U$. %Пусть $\Gamma$ --- игра в нормальной форме c $m$ участниками, а $U \subseteq \mathbb{R}^m$ --- множество всех векторов выплат, достижимых в её смешанных равновесиях по Нэшу. Игра $\Gamma$ называется \emph{чувствительной к дополнительной информационной асимметрии}, когда существует пространство корреляции $\Phi$ такое, что в игре $\Gamma | \Phi$ найдётся коррелированное равновесие по Нэшу с вектором выплат, не принадлежащим выпуклой оболочке множества $U$.
\end{definition}

Additionally, in the same article it is proved that the presence of a public real-valued roulette in the correlation space, i.e. subspaces of events with a uniformly distributed real outcome in the range $\left[0, 1\right)$, implies the convexity of both the set of attainable payoffs and the set of Nash equilibria in any game. Regarding the correlated expansion of normal form games, the above is quite enough to understand the ideas of this work, however the current state of knowledge on this topic can be traced in greater detail through the publications mentioned in Appendix~\cref{app:A}. %Кроме того, в этой же статье доказывается, что наличие в пространстве корреляции публичной вещественной рулетки, т.е. подпространства событий с равномерно распределённым вещественным исходом в диапазоне $\left[0, 1\right)$, влечёт выпуклость как множества достижимых выплат, так и множества равновесий Нэша в любой игре. Касательно коррелированного расширения игр в нормальной форме, вышеизложенного вполне достаточно для понимания идей данной работы, более подробно же современное состояние знаний на эту тему можно проследить по публикациям, упомянутым в Приложении~\cref{app:A}.

\section{Isomorphism of correlation spaces}\label{sec:ch1/sec2}

It should be noted that the model of correlation spaces is, in a sense, essentially redundant, because events from the state of nature as such do not matter and are used only as signals for synchronizing strategies. Talking about the impact of additional information asymmetry on the outcomes of conflicts in a meaningful way inevitably requires the ability to abstract from its specific sources, focusing on structural differences in the awareness of opponents. If formally different correlation spaces turn out to be completely interchangeable from a game-theoretic point of view, then they should be assigned to the shared equivalence class, whose description is the effectively essential parameter of the model. We note right away that this applies not only to trivial replacements of the set of states of nature by another set of the same size with the corresponding bijection of the rest of the space parameters, but also to more complex cases. For example, if in the context of a certain game a group of players observes a common signal in the form of a real-valued roulette wheel, will their observation of a coin toss also matter? Common sense dictates that any general roulette-and-coin strategy can easily be converted into a roulette-only equivalent by dividing the wheel in half and mapping the individual options for heads and tails into the resulting two sectors. We describe this phenomenon in the form of an isomorphism: %Следует отметить, что модель пространств корреляции в некотором смысле существенно избыточна, поскольку как таковые события из состояния природы значения не имеют и используются лишь в качестве сигналов для синхронизации стратегий игроков. Чтобы осмысленным образом рассуждать о влиянии, оказываемом дополнительной информационной асимметрией на исходы конфликтов, неизбежно требуется умение абстрагироваться от конкретных её источников, фокусируя внимание на структурных различиях в осведомлённости оппонентов. Если формально различные пространства корреляции оказываются полностью взаимозаменимы с теоретико"~игровой точки зрения, то они должны быть отнесены к общему классу эквивалентности, чьё описание и является в действительности существенным параметром модели. Сразу заметим, что это касается не только тривиальных замен множества состояний природы на другое множество той же мощности с соответствующей биекцией остальных параметров пространства, но и более сложных случаев. Например, если в контексте некоторой игры группа игроков наблюдает общий сигнал в виде колеса вещественной рулетки, будет ли иметь значение наблюдение ими ещё и броска монетки? Здравый смысл подсказывает, что любую общую стратегию с использованием рулетки и монетки можно легко превратить в эквивалентную для одной только рулетки, для чего достаточно поделить колесо пополам и отобразить отдельные варианты для орла и решки на полученные два сектора. Опишем этот феномен в виде изоморфизма:
\begin{definition}
	A \emph{partition} of a correlation space $\Phi = \langle A, \Omega, \mathfrak{I}^a, \mathbb{P}, a \in A \rangle$ \emph{into an arbitrary finite set of outcomes (codomain)} $X = X^1 \times ... \times X^m$ is a mapping $f : \Omega \rightarrow X$ consisting of the tuple of functions $(f^1, ..., f^m)$, where each $f^a : \Omega \rightarrow X^a$ is measurable with respect to $\mathfrak{I}^a$. In the sequel, a “partition $f$ of a correlation space $\Phi$” will be written in abbreviated form as $f \models \Phi$. %\emph{Разбиением} пространства корреляции $\Phi = \langle A, \Omega, \mathfrak{I}^a, \mathbb{P}, a \in A \rangle$ \emph{в произвольное конечное множество исходов (кодомен)} $X = X^1 \times ... \times X^m$ называется отображение $f : \Omega \rightarrow X$, состоящее из набора функций $(f^1, ..., f^m)$, где каждая $f^a : \Omega \rightarrow X^a$ измерима в $\mathfrak{I}^a$. Далее разбиение $f$ пространства корреляции $\Phi$ будем сокращённо обозначать $f \models \Phi$.
\end{definition}

In the context of correlated extension, the sets of outcomes $X^a$ are associated with the sets of pure strategies $S^a$, and the elements of the partition $f^a$ are associated with the correlated strategies $\mathbf{s}^a$. In what follows, we also use the mappings $f^{-1} : X \rightarrow 2^\Omega$ inverse to the partitions of correlation spaces, %В контексте коррелированного расширения множествам исходов $X^a$ соответствуют множества чистых стратегий $S^a$, а элементам разбиения $f^a$ "--- коррелированные стратегии $\mathbf{s}^a$. Далее также будут использоваться отображения $f^{-1} : X \rightarrow 2^\Omega$, обратные к разбиениям пространств корреляции:
\begin{equation*}
	f^{-1}(x) = \bigcap_{a \in A} (f^a)^{-1}(x^a).
\end{equation*}
\begin{definition}
	A space $\Phi_1$ with a measure $\mathbb{P}_1$ is said to be \emph{mappable onto} $\Phi_2$ \emph{with a measure} $\mathbb{P}_2$ (in the sequel, $\Phi_1 \precsim \Phi_2$) if their sets of players coincide and for each partition $f_1 \models \Phi_1$ there existsa partition $f_2 \models \Phi_2$ with the same codomain such that $\mathbb{P}_1 \circ f_1^{-1} = \mathbb{P}_2 \circ f_2^{-1}$. Mutually mappable correlation spaces are said to be \emph{isomorphic} (in the sequel, $\Phi_1 \sim \Phi_2$). %Пространство $\Phi_1$ с мерой $\mathbb{P}_1$ называется \emph{отобразимым на} $\Phi_2$ \emph{с мерой} $\mathbb{P}_2$ (далее $\Phi_1 \precsim \Phi_2$), если их множества игроков совпадают и для любого разбиения $f_1 \models \Phi_1$ существует разбиение $f_2 \models \Phi_2$ с тем же кодоменом такое, что $\mathbb{P}_1 \circ f_1^{-1} = \mathbb{P}_2 \circ f_2^{-1}$. Взаимно отобразимые друг на друга пространства корреляции называются \emph{изоморфными} (далее $\Phi_1 \sim \Phi_2$).
\end{definition}

This definition is easy to illustrate by the aforementioned example: for each partition $f_1 : \left[0, 1\right) \times \{0, 1\} \rightarrow X$ of the space consisting of a real roulette and a symmetric coin, one can construct the corresponding image $f_2 : \left[0, 1\right) \rightarrow X$ in the space of a roulette alone, %Это определение легко проиллюстрировать упомянутым выше примером "--- для любого разбиения $f_1 : \left[0, 1\right) \times \{0, 1\} \rightarrow X$ пространства, состоящего из вещественной рулетки и симметричной монетки, можно построить соответствующий образ $f_2 : \left[0, 1\right) \rightarrow X$ в пространстве из одной только рулетки:
\begin{equation*}
	f_2(\alpha) = \begin{cases}
		f_1(2 \alpha, 0), & 0 \le \alpha < \frac{1}{2} \\
		f_1(2 \alpha - 1, 1), & \frac{1}{2} \le \alpha < 1
	\end{cases}.
\end{equation*}

The reflexivity, symmetry, and transitivity of the isomorphism introduced using this definition are obvious, which means that this is indeed an equivalence relation on the set of correlation spaces. Moreover, although the definition of isomorphism was given in isolation from the correlated extension of games, the following theorem can be stated. %Рефлексивность, симметричность и транзитивность вводимого при помощи данного определения изоморфизма очевидны, а значит это действительно отношение эквивалентности на множестве пространств корреляции. При этом, хотя определение изоморфизма дано в отрыве от коррелированного расширения игр, можно сформулировать следующую теорему:
\begin{definition}
	For a game $\Gamma = \langle A, S^a, u^a(s), a \in A \rangle$, the set of achievable payoffs based on the deviations of a cabal $A_*$ of players from a profile $s$ of strategies is defined as %Для игры $\Gamma = \langle A, S^a, u^a(s), a \in A \rangle$ множеством достижимых выплат по отклонениям группы игроков $A_*$ от профиля стратегий $s$ будем называть
	\begin{equation*}
		U_\Gamma^{A_*}(s) = \{\bar{u} \mid \exists s_* \in S : u(s_*) = \bar{u}, \forall a \in A \setminus A_*, s^a = s_*^a\}.
	\end{equation*}
\end{definition}
\begin{theorem}[on isomorphic spaces]
	Let $\Phi_1 \sim \Phi_2$. Then for each normal-form game $\Gamma$ with finite sets of players’ strategies, its correlated extensions $\Gamma | \Phi_1$ and $\Gamma | \Phi_2$ possess the following property. Let $\mathbf{s}_1$ be some profile of strategies of the game $\Gamma | \Phi_1$. Then there exists a profile $\mathbf{s}_2$ of strategies of the game $\Gamma | \Phi_2$ such that $U_{\Gamma | \Phi_1}^{A_*}(\mathbf{s}_1) = U_{\Gamma | \Phi_2}^{A_*}(\mathbf{s}_2)$ for each cabal $A_*$ of players. %Пусть $\Phi_1 \sim \Phi_2$. Тогда для любой игры в нормальной форме $\Gamma$ с конечными множествами стратегий игроков её коррелированные расширения $\Gamma | \Phi_1$ и $\Gamma | \Phi_2$ обладают следующим свойством. Пусть $\mathbf{s}_1$ "--- некоторый профиль стратегий игры $\Gamma | \Phi_1$. Тогда существует $\mathbf{s}_2$ "--- профиль стратегий игры $\Gamma | \Phi_2$ такой, что $U_{\Gamma | \Phi_1}^{A_*}(\mathbf{s}_1) = U_{\Gamma | \Phi_2}^{A_*}(\mathbf{s}_2)$ для любой группы игроков $A_*$.
\end{theorem}

This theorem permits one to deem isomorphic correlation spaces indistinguishable in the context of searching of equilibria stable under both individual and group deviations. The proof, which is an exercise in topology without close connection to the central ideas of the study, is in Appendix~\cref{app:B}. %Эта теорема позволяет считать изоморфные пространства корреляции неразличимыми в контексте поиска равновесий, устойчивых к как индивидуальным, так и групповым отклонениям. Доказательство, представляющее собой упражнение в топологии без тесной связи с центральными идеями работы, вынесено в Приложение~\cref{app:B}.

\section{Conspiracy spaces}\label{sec:ch1/sec3}

Now, having obtained a meaningful isomorphism for correlation spaces, from all possible equivalence classes we can isolate those that are of interest with regard to modeling additional information asymmetry. As Aumann showed, going beyond the convex hull of the set of solutions in mixed strategies is possible if some of the players (at least two) use a correlated strategy that depends on an event about which at least one of the other players is not informed. It is natural to call such a form of mutually beneficial secret coordination of actions \emph{conspiracy}, and the signal used for synchronization "--- \emph{secret}. Let there be a secret in the correlation space $\Phi = \langle A, \Omega, \mathfrak{I}^a, \mathbb{P}, a \in A \rangle$, i.e. probability subspace $\langle \Omega, \mathfrak{S}, \mathbb{P} \rangle$. The desired information asymmetry suggests that some players observe events from $\mathfrak{S}$ (or others that correlate with them), and some "--- do not. Although theoretically one can imagine a conspiracy, the degree of involvement in which varies from player to player (someone can observe the events from $\mathfrak{S}$ partially or indirectly, through the observation of other events correlated with them), it makes sense to first consider the simplest case "--- dividing all players into <<conspirators>> $ A_{\mathfrak{S}} \subseteq A$, who observe $\mathfrak{S}$ as a whole, and <<outsiders>> $A \setminus A_{\mathfrak{S}}$, in whose field of view only events, not correlated with $\mathfrak{S}$ elements. %Получив осмысленный изоморфизм для пространств корреляции, можно выделить из всевозможных классов эквивалентности те, что представляют интерес с точки зрения моделирования дополнительной информационной асимметрии. Как показал Ауманн, выход за пределы выпуклой оболочки множества решений в смешанных стратегиях возможен, если часть игроков (не менее двух) использует коррелированную стратегию, зависящую от события, о котором не проинформирован хотя бы один из остальных игроков. Подобную форму взаимовыгодного тайного согласования действий естественно называть \emph{заговором}, а используемый для синхронизации сигнал "--- \emph{тайной}. Пусть в пространстве корреляции $\Phi = \langle A, \Omega, \mathfrak{I}^a, \mathbb{P}, a \in A \rangle$ имеется тайна, т.е. вероятностное подпространство $\langle \Omega, \mathfrak{S}, \mathbb{P} \rangle$. Искомая асимметрия информированности предполагает, что некоторые игроки наблюдают события из $\mathfrak{S}$ (или другие, коррелирующие с ними), а некоторые "--- нет. Хотя теоретически можно представить заговор, степень вовлеченности в который варьируется от игрока к игроку (кто-то может наблюдать события из $\mathfrak{S}$ частично или опосредованно, через наблюдение других коррелирующих с ними событий), имеет смысл в первую очередь рассмотреть простейший случай "--- деление всех игроков на <<заговорщиков>> $A_{\mathfrak{S}} \subseteq A$, наблюдающих $\mathfrak{S}$ целиком, и <<аутсайдеров>> $A \setminus A_{\mathfrak{S}}$, в чьём поле зрения только события, не коррелирующие с элементами $\mathfrak{S}$.

The next logical question is about the structure of the secret itself. It is quite possible to imagine how the conspirators use various sources of randomness in its role: dice throws, card deck shuffling, lottery draws, etc., so it is not obvious at first glance, if we can confine ourselves to consideration of single natural in the occurring context mechanism. A positive answer can be obtained using the correlation space mappings introduced above. If we compare all possible spaces that differ only in the secrets of the cabal $A_* \subseteq A$, the relation $\precsim$ induces a partial order on their set. The lower bound of this order will be a degenerate correlation space in which the conspiracy secret consists of a single atomic event with probability $1$ "--- such a space is mappable to any other and, obviously, cannot be used at all for strategy correlation. The upper bound's samples are more interesting "--- quintessentially, their conspiracy secrets are an arbitrary atomless \cite[с.~81]{Bogachev} spaces. Such a correlation mechanism can be easily imagined as a real-value roulette whose rotation generates a uniformly distributed random variable in unit half-interval $\left[0, 1\right)$. By dividing it into sectors of the required sizes, the conspirators observing the roulette can agree on any profile of correlated strategies in games with a finite set of outcomes. Such a universal source of randomness provides maximum freedom of choice, thus making sense as first consideration. %Следующий логичный вопрос "--- о структуре самой тайны. Вполне можно представить, как заговорщики используют в её роли самые разные источники случайности: броски игральных костей, тасование карточных колод, лотерейные розыгрыши и т.д., так что на первый взгляд неочевидно, можно ли ограничиться рассмотрением какого-то одного, естественного в контексте происходящего механизма. Положительный ответ можно получить, используя введённые выше отображения пространств корреляции. Если мы будем сравнивать всевозможные пространства, различающиеся только тайнами группы заговорщиков $A_* \subseteq A$, отношение $\precsim$ вводит на их множестве частичный порядок. Нижней гранью этого порядка будет вырожденное пространство корреляции, в котором тайна заговора состоит из единственного атомарного события с вероятностью $1$ "--- такое пространство отобразимо на любое другое и, очевидно, вообще не может быть использовано для корреляции стратегий. Верхняя грань интереснее "--- в её типичном представителе тайна заговора представляет собой произвольное безатомическое \cite[с.~81]{Bogachev} пространство. Проще всего представить такой механизм корреляции в виде вещественной рулетки, вращение которой генерирует равномерно распределённую случайную величину в единичном полуинтервале $\left[0, 1\right)$. Наблюдающие рулетку заговорщики могут, разделяя её на сектора необходимых размеров, согласовать любой набор коррелированных стратегий в играх с конечным множеством исходов. В первую очередь такой универсальный источник случайности и имеет смысл рассматривать как предоставляющий максимум свободы выбора.

Finally, one should think about correlation spaces with multiple secrets. In fact, nothing prevents players from observing several roulettes at once, choosing the opponents to correlate their strategy with depending on the situation. Moreover, a player's strategy can be tangibly dependent on more than one secret at the same time. Thus, correlation spaces consisting of distinct independent real-value roulettes, each characterized by a subset of players who can observe it, can be considered a natural subject of consideration. It remains to note that if there is two or more real-value roulettes belonging to the same circle of conspirators in the same correlation space, all but one can be discarded without damage to the model, since such duplication is obviously useless in games with finite sets of outcomes. Let us now turn to a more formal definition of the proposed concept. To do this, consider an arbitrary correlation space $\Phi = \langle A, \Omega, \mathfrak{I}^a, \mathbb{P}, a \in A \rangle$. In this space, for each non-empty group of players $A_* \subseteq A$, we define the following family of events\footnote{Here and below, $\sigma(\mathfrak{X})$ means the explement of the family of sets $\mathfrak{X}$ up to $\sigma$"~algebra}: %Наконец, следует подумать о пространствах корреляции с множественными тайнами. По сути, ничего не мешает игрокам наблюдать сразу несколько рулеток, выбирая в зависимости от ситуации, с кем из оппонентов коррелировать свою стратегию. Более того, стратегия игрока может одновременно существенно зависеть от более чем одной тайны. Таким образом, естественным предметом рассмотрения можно считать пространства корреляции, состоящие из наборов независимых вещественных рулеток, каждая из которых характеризуется подмножеством игроков, имеющих возможность её наблюдать. Остаётся заметить, что при наличии в одном пространстве корреляции двух и более вещественных рулеток, принадлежащих одному и тому же кругу заговорщиков, все кроме одной можно без ущерба для модели выкинуть, так как в играх с конечными множествами исходов такое дублирование, очевидно, бесполезно. Перейдём теперь к более формальному определению предложенной концепции. Для этого рассмотрим произвольное пространство корреляции $\Phi = \langle A, \Omega, \mathfrak{I}^a, \mathbb{P}, a \in A \rangle$. В этом пространстве для каждой непустой группы игроков $A_* \subseteq A$ определим следующее семейство событий\footnote{Здесь и далее под $\sigma(\mathfrak{X})$ понимается пополнение семейства множеств $\mathfrak{X}$ до $\sigma$"~алгебры}:
\begin{equation*}
	\mathfrak{S}_\Phi^{A_*} = \{U \in \bigcap\limits_{a \in A_*} \mathfrak{I}^a \mid \mathbb{P}(U \cap V) = \mathbb{P}(U) \mathbb{P}(V), \forall V \in \sigma(\bigcup\limits_{a \in A \setminus A_*} \mathfrak{I}^a)\}.
\end{equation*}

Thus, $\mathfrak{S}_\Phi^{A_*}$ is the set of all events such that all members of $A_*$ are aware of them, and each event is pairwise independent of all events known to nonmembers of $A_*$ even with their knowledge combined. Since an intersection of $\sigma$"~algebras is a $\sigma$"~algebra and the subset of events in a $\sigma$"~algebra independent of a given event is again a $\sigma$"~algebra, it follows that $\mathfrak{S}_\Phi^{A_*}$ is a $\sigma$"~algebra. This allows one to speak of the probability subspace $\langle \Omega, \mathfrak{S}_\Phi^{A_*}, \mathbb{P} \rangle$, which can be logically called a secret of the cabal $A_*$. We single out two cases: we say that secrets with atomless measures are \emph{complete} and secrets with trivial atomic measures with a single atom $\Omega$ are \emph{empty}. This permits one to give the following definition. %Таким образом, $\mathfrak{S}_\Phi^{A_*}$ "--- множество всех таких событий, что о них осведомлены все члены $A_*$, и каждое событие попарно независимо со всеми событиями, известными не членам $A_*$ даже при объединении их знаний. Поскольку пересечение $\sigma$"~алгебр образует $\sigma$"~алгебру, и так как подмножество независимых с некоторым событием событий $\sigma$"~алгебры также образует $\sigma$"~алгебру, то $\mathfrak{S}_\Phi^{A_*}$ "--- $\sigma$"~алгебра. Это позволяет говорить о вероятностном подпространстве $\langle \Omega, \mathfrak{S}_\Phi^{A_*}, \mathbb{P} \rangle$, которое и называется тайной группы $A_*$. Выделим два случая: тайны с безатомическими мерами мы будем называть \emph{полными}, а тайны с тривиальными атомическими мерами с единственным атомом $\Omega$ "--- \emph{пустыми}. Это даёт возможность сформулировать следующее:
\begin{definition}
	A correlation space $\langle A, \Omega, \mathfrak{I}^a, \mathbb{P}, a \in A \rangle$ is called a conspiracy space of the structure $\mathfrak{A} = \{A_1, ..., A_n\} \subseteq 2^A$ if %Пространство корреляции $\langle A, \Omega, \mathfrak{I}^a, \mathbb{P}, a \in A \rangle$ назовём пространством заговоров структуры $\mathfrak{A} = \{A_1, ..., A_n\} \subseteq 2^A$, когда
	\begin{itemize}
		\item $\{\{a\} \mid a \in A\} \subseteq \mathfrak{A}$;
		\item $\forall A_* \in \mathfrak{A}$ the secret $A_*$ is complete;
		\item $\forall A_* \notin \mathfrak{A}$ the secret $A_*$ is empty;
		\item $\mathfrak{I}^a = \sigma(\bigcup\limits_{a \in A_* \in \mathfrak{A}}\mathfrak{S}_\Phi^{A_*})$, i.e. $\mathfrak{I}^a$ is the least $\sigma$"~algebra containing all secret $\sigma$"~algebras of cabals that each player $a$ belongs to.
	\end{itemize}
\end{definition}

Simply put, conspiracy spaces are correlation spaces such that a) the secret of any cabal of players is either complete or empty; b) each player is the cabal of his own with a complete secret; c) the players do not have any knowledge about the state of nature other than that generated by the secrets of the cabals to which they belong. For illustration, let us construct the simplest example of such a space: %Проще говоря, пространствами заговоров называются такие пространства корреляции, в которых а) тайна любой группы игроков либо полна, либо пуста; б) каждый игрок в отдельности является группой с полной тайной и в) у игроков нет никаких знаний о состоянии природы, которые не порождались бы тайнами групп, которым они принадлежат. Построим для иллюстрации простейший пример такого пространства:
\begin{itemize}
	\item $A = \{1, 2, 3\}$,
	\item $\Omega = \left[0, 1\right)^2$,
	\item $\mathfrak{I}^1 = \sigma(\{\left[ 0, p_1 \right) \times \left[ 0, 1 \right) \mid 0 < p_1 \leq 1 \})$,
	\item $\mathfrak{I}^2 = \sigma(\{\left[ 0, 1 \right) \times \left[ 0, p_2 \right) \mid 0 < p_2 \leq 1 \})$,
	\item $\mathfrak{I}^3 = \sigma(\{\left[ 0, p_1 \right) \times \left[ 0, p_2 \right) \mid 0 < p_1 \leq 1, 0 < p_2 \leq 1 \})$,
	\item $\mathbb{P}$ is the Lebesgue measure. %"--- мера Лебега.
\end{itemize}

In this example, the correlation space consists of two independent real roulettes, players 1 and 2 observe one of these each, and player 3 observes both. In this case, it turns out that $\mathfrak{S}_\Phi^{\{1,3\}}$ coincides with $\mathfrak{I}^1$, $\mathfrak{S}_\Phi^{\{2,3\}}$ coincides with $\mathfrak{I}^2$, and for the other cabals $A_* \subseteq A$ the corresponding $\mathfrak{S}_\Phi^{A_*}$ is trivial. The structure of a space (or the \emph{conspiracy family}) is the set of all cabals of players with complete secrets. In the above example, the structure of the space is $\mathfrak{A} = \{\{1,3\},\{2,3\}\}$. From the point of view of feasible strategy profiles, this means that any cabal of players in the conspiracy family can use a common secret to form a correlated strategy, and players outside this cabal cannot join the choice of strategies agreed in this way. In contrast, cabals of players outside of the conspiracy family do not have the above-described opportunity. The structure of the space can be viewed as its exhaustive final description, because the following theorem holds. %В этом примере пространство корреляции состоит из двух независимых вещественных рулеток, первый и второй игроки наблюдают по одной из них, а третий наблюдает обе. При этом выходит, что $\mathfrak{S}_\Phi^{\{1,3\}}$ совпадает с $\mathfrak{I}^1$, $\mathfrak{S}_\Phi^{\{2,3\}}$ совпадает с $\mathfrak{I}^2$, а для остальных групп $A_* \subseteq A$ соответствующая $\mathfrak{S}_\Phi^{A_*}$ тривиальна. Структурой пространства (или \emph{семейством заговоров}) называется множество всех групп игроков с полными тайнами. В вышеприведённом примере структура пространства $\mathfrak{A} = \{\{1,3\},\{2,3\}\}$. С точки зрения допустимых профилей стратегий это означает, что любая группа игроков, входящая в семейство заговоров, может использовать общую тайну для формирования коррелированной стратегии, причём игроки, не входящие в эту группу, не могут присоединиться к согласованному таким образом выбору стратегий. Напротив, группы игроков, не входящие в семейство заговоров, вышеописанной возможностью не располагают. Структуру пространства можно считать его исчерпывающим конечным описанием в силу истинности следующего утверждения:
\begin{theorem} \label{the:struct}
	All conspiracy spaces of the same structure are isomorphic. %Все пространства заговоров одной структуры изоморфны.
\end{theorem}

Once again, proof of this theorem amounts to exercise in topology without close connection to the central ideas of the study and can be found in Appendix \ref{app:C}. Now that it has been established that the set of all conspiracy spaces is divided into equivalence classes, it is not difficult to suggest a way to construct a standard representative of each class from the corresponding conspiracy family. %Доказательство этой теоремы снова представляет собой упражнение в топологии без тесной связи с основными идеями работы и вынесено в приложение \ref{app:C}. Теперь, когда установлено, что множество всех пространств заговоров разбивается на классы эквивалентности, нетрудно предложить способ конструирования стандартного представителя каждого класса по соответствующему семейству заговоров:
\begin{definition}
	The standard space of a structure $\mathfrak{A} = \{A_1, A_2, ..., A_n\}$ is the correlation space $\Phi_{\mathfrak{A}} = \langle A, \Omega, \mathfrak{I}^a, \mathbb{P}, a \in A \rangle$ with the parameters %Стандартным пространством структуры $\mathfrak{A} = \{A_1, A_2, ..., A_n\}$ называется пространство корреляции $\Phi_{\mathfrak{A}} = \langle A, \Omega, \mathfrak{I}^a, \mathbb{P}, a \in A \rangle$ со следующими параметрами:
	\begin{itemize}
		\item $A = \bigcup\limits_{i=1}^n A_i$,
		\item $\Omega = \left[0, 1\right)^n$,
		\item $\mathfrak{I}^a = \sigma(\{\prod\limits_{i=1}^n\left[ 0, p_i \right) \mid \operatorname{\mathbf{if}} \: a \in A_i \: \operatorname{\mathbf{then}} \: 0 < p_i \leq 1 \: \operatorname{\mathbf{else}} \: p_i = 1 \})$,
		\item $\mathbb{P}$ is the Lebesgue measure. %"--- мера Лебега.
	\end{itemize}
\end{definition}

The set of states of nature is the $n$"~dimensional (according to the numbers of conspiracies in the family) unit cube, and the probability measure corresponds to the continuous uniform distribution. In this case, the $\sigma$"~algebra of each player is the Borel algebra in the projections onto the axes corresponding to the conspiracies they are part of and is trivial in the projections onto the other axes. The choice of a standard representative for any conspiracy families allows one to use the notation $\Gamma | \mathfrak{A}$, by which we will understand $\Gamma | \Phi_{\mathfrak{A}}$. This notation emphasizes the fact that the choice of a particular correlation space among all conspiracy spaces of the required structure is irrelevant for us, and the standard space serves as the simplest representative suitable for practical calculations. %Множество состояний природы представляет собой $n$"~мерный (по числу заговоров, входящих в семейство) единичный куб, а вероятностная мера соответствует непрерывному равномерному распределению. При этом $\sigma$"~алгебра каждого игрока борелева в проекциях на оси, соответствующие заговорам в которые он входит, и тривиальна в проекциях на остальные оси. Выделение стандартного представителя для любых семейств заговоров позволяет использовать нотацию $\Gamma | \mathfrak{A}$, под которой в дальнейшем будет пониматься $\Gamma | \Phi_{\mathfrak{A}}$. Эта нотация подчёркивает тот факт, что выбор конкретного пространства корреляции среди всех пространств заговоров необходимой структуры для нас значения не имеет, а стандартное пространство выступает в роли простейшего представителя, пригодного для практических вычислений.

\section{Three-player even-odd}\label{sec:ch1/sec4}

The game of <<three-way even"~odd>> can serve as an elementary example of a conflict sensitive to additional information asymmetry. It starts with each of the three participants secretly choosing <<eagles>> or <<tails>> on their coins and laying them on the table under their palms with the appropriate sides up. After that everyone simultaneously take off their hands and, depending on the combination, divide the fixed bank. When all three coins lie on the same side, the round is considered a draw and the players divide the pot equally. If only two of them matched, then the short-handed player is considered the loser and does not receive a share in the pot division. In matrix form, this can be described as follows: %В качестве элементарного примера конфликта, чувствительного к дополнительной информационной асимметрии, может выступать <<трёхсторонний чёт"~нечет>>. В этой игре каждый из трёх участников тайно выбирает <<орла>> или <<решку>> на своей монете и прижимает её к столу соответствующей стороной вверх, после чего все одновременно поднимают ладони и в зависимости от сложившейся комбинации делят фиксированный банк. Когда все три монеты лежат одной и той же стороной, раунд считается сыгранным вничью и игроки делят банк поровну. Если же совпали только две из них, то оказавшийся в меньшинстве игрок считается проигравшим и не получает доли при дележе банка. В матричной форме это можно описать так:
\begin{table} [htbp]
	\centering
	\begin{threeparttable}
		\caption{Three-player even-odd}
		\label{tab:coin3}
		\begin{tabular}{ |c|c|c|c|c| }
			\cline{1-2} \cline{4-5}
			\rule[-7pt]{0pt}{2em}$4, 4, 4$ &
			\rule[-7pt]{0pt}{2em}$6, 0, 6$ & \qquad\qquad\qquad &
			\rule[-7pt]{0pt}{2em}$6, 6, 0$ &
			\rule[-7pt]{0pt}{2em}$0, 6, 6$ \\
			\cline{1-2} \cline{4-5}
			\rule[-7pt]{0pt}{2em}$0, 6, 6$ &
			\rule[-7pt]{0pt}{2em}$6, 6, 0$ & \qquad\qquad\qquad &
			\rule[-7pt]{0pt}{2em}$6, 0, 6$ &
			\rule[-7pt]{0pt}{2em}$4, 4, 4$ \\
			\cline{1-2} \cline{4-5}
		\end{tabular}
	\end{threeparttable}
\end{table}

In the table \ref{tab:coin3}, the first player chooses a row, the second "--- a column, and the third "--- a matrix. The solution of this game in pure strategies is two Nash equilibria corresponding to the synchronous choices of the same sides by all players. In mixed strategies, another degenerate solution is added, with each player making an equally probable random choice between heads and tails. All these solutions obviously give the expectation of payments equal to $(4,4,4)$. In the framework of classical game theory, this would be the end of conflict analysis, but the addition of information asymmetry makes the situation more interesting. Consider the same game in the conspiracy space of structure $\{\{1,2\}\}$, i.e. in a situation where players 1 and 2 have the opportunity to use correlated strategies arranged in secrecy from player 3. Let $\alpha \in \left[0, 1\right)$ be the value of the corresponding secret roulette wheel. Conspirators can use strategies of the form $\mathbf{s}^1, \mathbf{s}^2 : \left[ 0, 1\right) \rightarrow P_{\{\mathbf{head}, \mathbf{tail}\}}$, where $P_{\{\mathbf{head}, \mathbf{tail}\}}$ denotes all possible probability measures on the set $\{\mathbf{head}, \mathbf{tail}\}$, i.e. the set of classical mixed strategies of the game under consideration. The outsider, on the other hand, has to be content with the $\mathbf{s}^3 \in P_{\{\mathbf{head}, \mathbf{tail}\}}$ strategies, since he has no access to the conspiracy roulette. In order to find themselves at a more favorable, comparing to an equal division, equilibrium point, players 1 and 2 can choose any strategies that result in an equiprobable synchronous choice of the coin side: %В таблице \ref{tab:coin3} первый игрок выбирает строку, второй "--- столбец, а третий "--- матрицу. Решением этой игры в чистых стратегиях являются два равновесия Нэша, соответствующие синхронным выборам одинаковых сторон всеми игроками. В смешанных стратегиях добавляется ещё одно вырожденное решение, когда каждый игрок делает случайный выбор между орлом и решкой с равными вероятностями. Все эти решения, очевидно, дают математическое ожидание платежей равное $(4,4,4)$. В рамках классической теории игр этим анализ конфликта и исчерпывается, однако добавление фактора информационной асимметрии делает ситуацию интереснее. Рассмотрим эту же игру в пространстве заговоров структуры $\{\{1,2\}\}$, т.е. в ситуации, когда игроки 1 и 2 имеют возможность использовать согласованные втайне от игрока 3 коррелированные стратегии. Пусть $\alpha \in \left[0, 1\right)$ "--- значение соответствующей секретной рулетки. Заговорщики могут использовать стратегии вида $\mathbf{s}^1, \mathbf{s}^2 : \left[0, 1\right) \rightarrow P_{\{\mathbf{head}, \mathbf{tail}\}}$, где под $P_{\{\mathbf{head}, \mathbf{tail}\}}$ понимаются всевозможные вероятностные меры на множестве $\{\mathbf{head}, \mathbf{tail}\}$, т.е. множество классических смешанных стратегий рассматриваемой игры. Аутсайдер же вынужден довольствоваться стратегиями $\mathbf{s}^3 \in P_{\{\mathbf{head}, \mathbf{tail}\}}$, поскольку не имеет доступа к рулетке заговора. Для того, чтобы оказаться в более выгодной по сравнению с равным дележом точке равновесия, игроки 1 и 2 могут выбрать любые стратегии, обеспечивающие им равновероятный синхронный выбор стороны монетки:
\begin{equation*}
	\mathbf{s}^1(\alpha) = \mathbf{s}^2(\alpha) = \begin{cases}
		(1, 0), & \alpha < \frac{1}{2},\\
		(0, 1), & \alpha \ge \frac{1}{2}.
	\end{cases}
\end{equation*}

Moreover, any mixed strategy of player 3, owing to independence from the conspiracy roulette, ensures that it coincides with the others in exactly half of the cases. The payouts in this situation are $(5,5,2)$, and there is no profitable individual deviation for any of the players. %При этом любая смешанная стратегия игрока 3 в силу независимости с рулеткой заговора обеспечивает ему совпадение с остальными ровно в половине случаев. Выплаты в сложившейся ситуации равны $(5,5,2)$, причём ни для одного из игроков нет выгодного индивидуального отклонения.

\section{Necessary complexity of the conspiracy model}\label{sec:ch1/sec5}

The conventional formalism of the matrix game in normal form identifies the profile of pure strategies with the game outcome "--- they are literally the same mathematical object. Almost the same can be said about its mixed extension "--- the space of profiles of mixed strategies is isomorphic to the space of outcomes, consisting of probability distributions on the game matrix that adheres to independence in the choice of rows and columns. Alas, the correlated extension as formulated by Robert Aumann messes up this rosy picture. Its space of game outcomes is even simpler than in the mixed case "--- any probability distribution on the set of elements of the game matrix goes, without additional conditions. But with strategic profiles, everything sharply becomes more complicated "--- the strategy of each player is a function, which maps the set of states of nature into the set of his pure strategies, reckoning with observance of measurability in its awareness $\sigma$"~algebra. Obviously this prevents any possibility of one-to-one correspondence between profiles and outcomes "--- depending on the correlation space parameters, some outcomes may not be achievable at all, while equiprobable events can be swapped around in the strategy domain without affecting the outcome. On top, when working with correlated strategies in Aumann's formulation, one can encounter multidimensional event spaces, Borel $\sigma$"~algebras, and other non-trivial phenomena of Kolmogorov probability theory, which probably also contributes to how reluctant game theorists are about resorting to this tool in more applied research. %Классический формализм матричной игры в нормальной форме отождествляет профиль чистых стратегий и исход розыгрыша "--- они являются буквально одним и тем же математическим объектом. Почти то же самое можно сказать и о его смешанном расширении "--- пространство профилей смешанных стратегий изоморфно пространству исходов, состоящему из вероятностных распределений на матрице игры, соблюдающих условие независимости выбора строк и столбцов. Увы, коррелированное расширение в формулировке Роберта Аумана ломает эту идиллическую картину. Пространство исходов розыгрышей в нём даже проще, чем в смешанном случае "--- распределение вероятностей на множестве элементов игровой матрицы может быть любым, без дополнительных условий. Но вот со стратегическими профилями всё резко становиться сложнее "--- стратегия каждого игрока представляет собой функцию, отображающую множество состояний природы в множество его чистых стратегий, причём с соблюдением измеримости по $\sigma$"~алгебре его информированности. Очевидно, что более ни о каком взаимно однозначном соответствии между профилями и исходами не может идти и речи "--- в зависимости от параметров пространства корреляции некоторые исходы могут оказаться вообще не достижимы, а равновероятные события можно без влияния на исход менять местами в домене стратегий. Вдобавок, работая с коррелированными стратегиями в формулировке Аумана, можно столкнуться с многомерными пространствами событий, борелевскими $\sigma$"~алгебрами и другими нетривиальными феноменами колмогоровской теории вероятностей, что, вероятно, тоже вносит свой вклад в то, сколь неохотно специалисты теории игр прибегают к этому инструменту в более прикладных исследованиях.

The conspiracy model proposed here narrows the correlated extension by extracting from a continuum of possible correlation spaces a finite (for any finite number of players) set, one for each conspiracy structure. At first glance, such a radical simplification of the parameter space gives hope that in the practical use of the model it would also be possible to do without the <<esoteric>> aspects of measure theory. Ideally, we'd like to identify in one way or another the space of strategic profiles with the set of outcomes in games with conspiracies, just as it happens in conventional formalisms. If we could, looking only at the probability distribution of the individual outcome realization in the game matrix, determine which strategy profile (or any representative from the family of indistinguishable profiles) was selected by the players, this would imply ability to also find all distributions achievable in certain deviations from the played strategy profile, thereby checking the situation for equilibrium without diving into the details of the Auman correlation model. %Предложенная здесь модель заговоров сужает коррелированное расширение, выделяя из континуума всевозможных пространств корреляции конечный (для любого конечного числа игроков) набор, по одному на каждую структуру заговоров. На первый взгляд такое радикальное упрощение пространства параметров даёт надежду на то, что и при практическом использовании модели удастся обойтись без <<эзотерических>> аспектов теории меры. В идеале хотелось бы, чтобы в играх с заговорами можно было тем или иным способом отождествить пространство стратегических профилей и множество исходов, так же, как это происходит в классических формализмах. Если бы мы могли, глядя только на распределение вероятностей реализации отдельных исходов в игровой матрице, определить, какой профиль стратегий (или любой представитель семейства неразличимых профилей) был сыгран игроками, то это означало бы, что мы можем без погружения в детали аумановской модели корреляции найти также и все распределения, достижимые в тех или иных отклонениях от сыгранного профиля стратегий, проверяя тем самым ситуацию на равновесность.

Alas, in the general case it is hardly possible "--- the loss of important information in the transition from sets of strategies to the result of correlation can be made clear, using the same tripartite even-odd described in the previous section. Imagine that the game is played in conspiracy space $\{\{1, 2\}, \{1, 2, 3\}\}$, that is, players 1 and 2 can correlate their actions both secretly from player 3 and together with him. The correlation space then turns out to consist of two roulettes $\alpha^{1, 2}$ and $\alpha^{1, 2, 3}$ observed by the players indicated in the superscripts of their designation. Consider two sets of strategies: first %Увы, в общем случае это вряд ли возможно "--- наглядно показать потерю важной информации при переходе от наборов стратегий к результату корреляции можно при помощи того же самого трёхстороннего чёт-нечета, описанного в предыдущем разделе. Представим себе, что игра ведётся в пространстве заговоров $\{\{1, 2\}, \{1, 2, 3\}\}$, то есть, игроки 1 и 2 могут коррелировать свои действия как втайне от игрока 3, так и вместе с ним. Пространство корреляции при этом оказывается состоящим из двух рулеток $\alpha^{1, 2}$ и $\alpha^{1, 2, 3}$, наблюдаемых игроками, указанными в верхних индексах их обозначения. Рассмотрим два набора стратегий: первый "---
\begin{equation*}
	\mathbf{s}^1(\alpha^{1, 2}, \alpha^{1, 2, 3}) = \mathbf{s}^2(\alpha^{1, 2}, \alpha^{1, 2, 3}) = \begin{cases}
		(1, 0), & \alpha^{1, 2, 3} < \frac{1}{2},\\
		(0, 1), & \alpha^{1, 2, 3} \ge \frac{1}{2},
	\end{cases} \mathbf{s}^3(\alpha^{1, 2, 3}) = (\frac{1}{2}, \frac{1}{2}),
\end{equation*}
and second %и второй "---
\begin{equation*}
	\mathbf{s}^1(\alpha^{1, 2}, \alpha^{1, 2, 3}) = \mathbf{s}^2(\alpha^{1, 2}, \alpha^{1, 2, 3}) = \begin{cases}
		(1, 0), & \alpha^{1, 2} < \frac{1}{2},\\
		(0, 1), & \alpha^{1, 2} \ge \frac{1}{2},
	\end{cases} \mathbf{s}^3(\alpha^{1, 2, 3}) = (\frac{1}{2}, \frac{1}{2}).
\end{equation*}

In both cases, all players make an equally probable choice between heads and tails, provided that the first two players make it synchronously, while the third "--- independently of them. The probability distribution of individual outcomes in the $2 \times 2 \times 2$ matrix from the table \ref{tab:coin3}, can be expressed for both sets like this: %В обоих случаях все игроки делают равновероятный выбор между орлом и решкой, причём первые два игрока делают его синхронно, а третий "--- независимо от них. Если записать распределение вероятностей отдельных исходов в матрице $2 \times 2 \times 2$, соответствующей игровой из таблицы \ref{tab:coin3}, то для обоих наборов это будет выглядеть так:
\begin{tabular}{ |c|c|c|c|c| }
	\cline{1-2} \cline{4-5}
	$\frac{1}{4}$ &
	$0$ & \qquad &
	$\frac{1}{4}$ &
	$0$ \\
	\cline{1-2} \cline{4-5}
	$0$ &
	$\frac{1}{4}$ & \qquad &
	$0$ &
	$\frac{1}{4}$ \\
	\cline{1-2} \cline{4-5}
\end{tabular}.
Of course, the payments here are also equal and amount to $(5, 5, 2)$. However, on a closer look, it can be seen that in the first case, players 1 and 2 for synchronization use the $\alpha^{1, 2, 3}$ roulette that they share with player 3, which allows him by changing his strategy to %Понятное дело, платежи здесь также равны и составляют $(5, 5, 2)$. Тем не менее, если присмотреться, можно обнаружить, что в первом случае игроки 1 и 2 используют для синхронизации общую с игроком 3 рулетку $\alpha^{1, 2, 3}$, что позволяет ему, изменив свою стратегию на
\begin{equation*}
	\mathbf{s}^3(\alpha^{1, 2, 3}) = \begin{cases}
		(1, 0), & \alpha^{1, 2, 3} < \frac{1}{2},\\
		(0, 1), & \alpha^{1, 2, 3} \ge \frac{1}{2},
	\end{cases}
\end{equation*}
to join them for the outcome with the probability distribution %присоединиться к ним и перейти в исход с распределением вероятностей
\begin{tabular}{ |c|c|c|c|c| }
	\cline{1-2} \cline{4-5}
	$\frac{1}{2}$ &
	$0$ & \qquad &
	$0$ &
	$0$ \\
	\cline{1-2} \cline{4-5}
	$0$ &
	$0$ & \qquad &
	$0$ &
	$\frac{1}{2}$ \\
	\cline{1-2} \cline{4-5}
\end{tabular},
where the payouts are $(4, 4, 4)$. Therefore the first point is not a Nash equilibrium. In the second case, players 1 and 2 use for synchronization the private roulette $\alpha^{1, 2}$, which outcome player 3 cannot observe. Whichever strategy he chooses, it would coincide with the strategies of the conspirators exactly in half of the cases, meaning that the payments can not change. Since the other players also do not have favorable deviations, indeed, this situation is a Nash equilibrium. %где выплаты составляют $(4, 4, 4)$. Таким образом, первая точка не является равновесием Нэша. Во втором же случае игроки 1 и 2 используют для синхронизации приватную рулетку $\alpha^{1, 2}$, исход которой игрок 3 наблюдать не может. Какую бы стратегию он ни выбрал, совпадёт со стратегиями заговорщиков она ровно в половине случаев, а значит выплаты не изменятся. Так как у остальных игроков выгодных отклонений тоже нет, эта ситуация уже будет равновесием Нэша.

Thus, we have constructed an example in which the same distribution of outcome probabilities in the game matrix corresponds to at least two sets of strategies so different that one of them is a Nash equilibrium, and the second is not. Obviously, it is impossible to combine these sets in one equivalence class in any reasonable way, which means that in the general case it is impossible to identify game situations with outcomes of the game, no matter how hard we try. Anyway, it should be noted that in simpler cases, when each player can't participate in more than one cabal, there is no way to construct such a clear and trivial counterexample "--- when solving practical problems, if the correlation manifested in the probability distribution on the game matrix could be obtained in only one way, nothing prevents us from working directly with probability distributions, and not with functions that map signals to outcomes. However, if we consider conspiracy theory as a modeling tool for information asymmetries that independent agents interactively develop in an uncontrolled environment, then such self-restraint, alas, would limit our framework without natural justification. %Таким образом, мы построили пример, в котором одному и тому же распределению вероятностей исходов в матрице игры соответствуют как минимум два настолько различных набора стратегий, что один из них является равновесием Нэша, а второй "--- нет. Очевидно, что никаким разумным способом объединить эти наборы в одном классе эквивалентности нельзя, а значит отождествление игровых ситуаций и исходов розыгрыша в общем случае невозможно, как бы мы ни старались. Следует, впрочем, заметить, что в более простых случаях, когда каждый игрок может входить не более чем в один заговор, построить такой наглядный и тривиальный контрпример уже не получится "--- если корреляция, проявляющаяся в распределении вероятностей на матрице игры, могла бы быть получена только одним способом, то при решении практических задач ничего не мешает оперировать именно распределениями вероятностей, а не функциями, отображающими сигналы в исходы. Однако, если рассматривать теорию заговоров как способ моделирования информационных асимметрий, складывающихся в ходе реальных процессов взаимодействия независимых агентов в неконтролируемой среде, то подобное самоограничение, увы, загнало бы нас в рамки, не имеющие естественного обоснования.

\FloatBarrier
