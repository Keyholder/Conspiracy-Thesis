\chapter{Collective rationality in conspiracy games}\label{ch:ch2}

\section{Task scheduling problem}\label{sec:ch2/sec1}

The concept of Nash equilibrium, being the foundation of conventional game theory models, often turns out to be an insufficiently strong formalism in itself. Analysis of multilateral conflicts often breeds situations when the set of Nash equilibria is too large to be considered a proper solution for the game. In such case, the collective rationality criteria come to the researchers' aid "--- cherry-picking the equilibrium points for strong or weak Pareto optimality in many cases significantly narrows the space of solutions due to the quite natural exclusion of obviously unfavorable for all participants. The presence of additional information asymmetry creates new difficulties, since the proposed model implies an inevitable element of motivation antagonism "--- at new equilibrium points, the increase in the payoffs of those participating in the correlation occurs at the expense of reducing the payoffs of those who cannot join it, thereby making the usual criteria of collective rationality unproductive. To demonstrate this effect, consider a trivial generalization of the canonical task scheduling problem \cite{Koutsoupias}, a prominent mark in the conceptual landscape of game theory. This conflict commonly illustrates the \emph{prices of anarchy} and \emph{prices of stability} concepts, giving perhaps the most telling example of differences between same game Nash equilibria in terms of their global optimality. However, we need to look at this game from a different angle, in which the notions of the anarchy and stability prices lose their meaning, making room for sensitivity to additional information asymmetry. %Концепция равновесия по Нэшу, будучи фундаментом классических моделей теории игр, сама по себе зачастую оказывается недостаточно сильным формализмом. При анализе многосторонних конфликтов нередко возникают ситуации, когда множество равновесий по Нэшу слишком велико, чтобы считать его полноценным решением игры. В этом случае на помощь исследователям приходят критерии коллективной рациональности "--- выбирая среди равновесных точек оптимальные по Парето или Слейтеру, иногда можно значительно сузить пространство решений за счёт вполне естественного исключения заведомо невыгодных для всех участников. Присутствие дополнительной информационной асимметрии при этом создаёт дополнительные затруднения, поскольку предлагаемая модель подразумевает неизбежный элемент антагонизма интересов "--- в новых точках равновесия увеличение выигрышей участвующих в корреляции происходит за счёт сокращения выплат тех, кто не может к ней присоединиться, тем самым делая непродуктивными обычные критерии коллективной рациональности. Для демонстрации этого эффекта рассмотрим тривиальное обобщение классической проблемы планирования заданий \cite{Koutsoupias}, занимающей важное место в концептуальном ландшафте теории игр. При помощи этого конфликта хорошо иллюстрируются понятия \emph{цены анархии} и \emph{цены стабильности}, давая вероятно самый выразительный пример того, насколько равновесия Нэша в одной и той же игре могут различаться в смысле их глобальной оптимальности. Нам, однако, необходимо взглянуть на эту игру под другим углом, при котором понятие цен анархии и стабильности теряет смысл, уступая место чувствительности к дополнительной информационной асимметрии.

Let's start with the canonical task scheduling model. The computer center has $m$ employees, each of which is assigned to perform some calculation. They have $n$ computers at their disposal, each being able to run one or more programs that perform employee calculations. The machines have architectural differences, expressed by the matrix of constants $t_i^a \ge 0$, each denoting the execution time of the employee's program $a=\overline{1,\ldots,m}$ on the computer $i=\overline{1,\ldots, n}$. Each calculation can be performed by only one device. Several programs on the same computer run sequentially, but the results of their work are produced simultaneously after the last one is stopped. Thus, the employee's benefit comes from choosing such a computer for his calculation that total execution time of all programs running on it turns out to be minimal. Let us describe what is happening in terms of the normal form game %Начнём с классической модели планирования заданий. В вычислительном центре работают $m$ сотрудников, каждому из которых поручено произвести некое вычисление. В их распоряжении находятся $n$ компьютеров, на каждом из которых может быть запущена одна или несколько программ, производящих вычисления сотрудников. Машины отличаются архитектурными особенностями, что задаётся матрицей констант $t_i^a \ge 0$, обозначающих время выполнения программы сотрудника $a=\overline{1,\ldots,m}$ на компьютере $i=\overline{1,\ldots,n}$. Каждое вычисление может производиться только одним устройством. Несколько программ на одном компьютере выполняются последовательно, но результаты их работы выводятся одновременно после остановки последней из них. Таким образом, выгода сотрудника заключается в выборе такого компьютера для своего вычисления, что для него оказывается минимальным суммарное время выполнения всех запущенных программ. Опишем происходящее в терминах игры нормальной формы
\begin{equation}\label{intro:game}
	\Gamma = \langle A, S^a, u^a(s), a \in A \rangle
\end{equation}
parametrized with:
\begin{itemize}
	\item $A = \{ 1, \ldots, m \}$;
	\item $S^1 = \ldots = S^m = \{ 1, \ldots, n \}$;
	\item $u^a(s) = -t_{s^a}(s)$, where $t_i(s) = \sum\limits_{a \in A, s^a = i} t_i^a$.
\end{itemize}

Here we should focus on the definition of the payout function. By choosing $u^a(s) = -t_{s^a}(s)$, we simulate a situation where all calculations should have been completed yesterday, and employees are directly penalized for every extra second until their results land on boss's desk. However, it is possible to simulate a less stressful working moment by taking, for example, a stepwise payout function: %Здесь следует заострить внимание на определении функции выплат. Выбрав $u^a(s) = -t_{s^a}(s)$, мы моделируем ситуацию, когда все вычисления должны были быть готовы ещё вчера, и сотрудники напрямую штрафуются за каждую лишнюю секунду, пока их результаты не лягут на стол начальству. Однако, можно смоделировать и менее напряжённый рабочий момент, взяв, например, ступенчатую функцию выплат:
\begin{equation*}%\label{intro:stairs}
	u^a(s) = \begin{cases}
		u^a_{GOOD}, &t_{s^a}(s) < t^a_{DEADLINE};\\
		u^a_{LATE}, &t_{s^a}(s) \ge t^a_{DEADLINE}.
	\end{cases}
\end{equation*}

In this case, we assign for each player $a$ the deadline $t^a_{DEADLINE}$, meeting which implies the successful completion of the task rewarded by a fixed payment $u^a_{GOOD}$ (including the bonus), and not meeting "--- $u^a_{LATE}$ (regular rate). One can come up with variety of more complex incentive schemes for employees, so let's formulate it in a general way: %При этом получается, что для каждого игрока $a$ назначен срок успешного выполнения задания $t^a_{DEADLINE}$, уложившись в который, он получает фиксированную выплату $u^a_{GOOD}$ (с учётом премии), а не уложившись "--- $u^a_{LATE}$ (обычную ставку). Можно придумать и более сложные схемы поощрения сотрудников, так что сформулируем сразу в общем виде:
\begin{equation}\label{intro:rewards}
	u^a(s) = v^a(t_{s^a}(s)),
\end{equation}
where $v^a(t)$ is a monotonically non-increasing promptness reward function assigned by employee $a$. Any normal form game built according to the scheme \ref{intro:game} with a payoff function of the form \ref{intro:rewards} is essentially a task scheduling problem, with the condition of monotonic non-increase of $v^a(t)$ being necessary, since the proof of property widely considered important for this game explicitly relies on it "--- the cost of stability being equal to $1$ \cite {Agussurja}. Let us recall that in optimization problems with selfish agents, the cost of stability is the ratio $\frac{t_{NASH}}{t_{BEST}}$, where $t_{NASH}$ is the value of the best Nash equilibria, and $t_ {BEST}$ "--- the value of the globally optimal solution. This means that in the task scheduling problem there are bound to be Nash equilibria among all situations minimizing the time until the last computer stops. However, in this study, we propose to temporarily forget about minimizing the total computation time and instead analyze new properties of the model appearing in the absence of such a monotonicity constraint. %где $v^a(t)$ "--- монотонно невозрастающая функция оплаты за срочность по заданию сотрудника $a$. Любая игра в нормальной форме, построенная по схеме \ref{intro:game} с платёжной функцией вида \ref{intro:rewards}, в сущности является проблемой планирования задач. При этом условие монотонного невозрастания $v^a(t)$ необходимо, поскольку на него в явном виде опирается доказательство считающегося важным свойства этой игры "--- равенство $1$ цены стабильности \cite{Agussurja}. Напомним, в оптимизационных задачах с эгоистичными агентами цена стабильности представляет собой соотношение $\frac{t_{NASH}}{t_{BEST}}$, где $t_{NASH}$ "--- значение наилучшего из равновесий Нэша, а $t_{BEST}$ "--- значение глобально оптимального решения. Это означает, что в проблеме планирования задач среди всех ситуаций, минимизирующих время до остановки последнего компьютера, обязательно найдутся равновесия по Нэшу. Однако, в данной работе предлагается на время забыть о минимизации общей продолжительности вычислений и вместо этого проанализировать, какие новые свойства модели могут проявиться в отсутствие такого ограничения на монотонность.

\section{Individualism penalty}\label{sec:ch2/sec2}

Imagine a data center with computers requiring complex maintenance after a shift if at least one task was run on them. If employees who meet deadlines are rewarded regardless of that, it's not hard to imagine a situation where they, trying to guarantee themselves a bonus at any cost, will scatter tasks over an unreasonably large number of computers. Faced with such a prospect, to avoid systemic underutilization of machines, management might be tempted to incentivize its employees through fines. This can be modeled by a step promptness reward function of the following form: %Представим вычислительный центр с компьютерами, требующими сложного техобслуживания после смены, если на них запускали хотя бы одну задачу. Если поощрять укладывающихся в дедлайны сотрудников невзирая на это, несложно представить ситуацию, когда они, в стремлении любой ценой гарантировать себе премию, будут раскидывать задачи по неразумно большому количеству компьютеров. Перед лицом такой перспективы у руководства может возникнуть соблазн стимулировать своих сотрудников избегать систематической недозагрузки машин при помощи штрафов. Это может быть смоделировано ступенчатой функцией оплаты за срочность следующего вида:
\begin{equation*}
	v^a(t) = \begin{cases}
		\begin{aligned}
			u^a_{HAST}, \qquad & & & \ t < t^a_{BREAKAWAY} \hspace{-10pt} & ;\\
			u^a_{GOOD}, \qquad & t^a_{BREAKAWAY} \hspace{-8pt} & \le & \ t < t^a_{DEADLINE} & ;\\
			u^a_{LATE}, \qquad & t^a_{DEADLINE} & \le & \ t & .
		\end{aligned}
	\end{cases}
\end{equation*}

Here, for each employee $a$, along with the deadline $t^a_{DEADLINE}$, which must be met in order to receive the bonus, we also fix the minimum workload of the utilized machine $t^a_{BREAKAWAY}$, which must be reached in order not to run into a fine for wasting computing resources (whereby $u^a_{HAST} < u^a_{LATE} < u^a_{GOOD}$). The size of the minimum workload can be set, for example, depending on the importance of the corresponding task "--- if an urgent result pays for the use of additional machines, then it can be made lower or even equated to zero. If, on the contrary, the task is not so important, then a large minimal workload will force such employee to mind the interests of the company and cooperate with colleagues. %Здесь для каждого сотрудника $a$ зафиксирован не только дедлайн $t^a_{DEADLINE}$, к которому требуется успеть, чтобы получить премию, но и минимальная загруженность используемого компьютера $t^a_{BREAKAWAY}$, которой нужно достигнуть, чтобы не нарваться на штраф за нерациональное расходование вычислительных ресурсов (причём $u^a_{HAST} < u^a_{LATE} < u^a_{GOOD}$). Размер минимальной загруженности может устанавливаться, например, в зависимости от важности соответствующей задачи "--- если срочное получение результата окупает использование дополнительных машин, то его можно сделать пониже или вовсе приравнять к нулю. Если же наоборот задача не так уж и важна, то большая минимальная загруженность заставит соответствующего сотрудника вспомнить об интересах фирмы и скооперироваться с коллегами.

As you can see, employees in this case have to act on a non-monotonic promptness reward function, creating some effects unusual for the canonical formulation of the task scheduling problem. First, in such a scenario, it is expected that not for every game the cost of stability will be equal to $1$. It suffices to consider a game with $2$ employees and $2$ identical computers: %Как видно, сотрудникам в данном случае приходится руководствоваться немонотонной функцией оплаты за срочность, что создаёт некоторые эффекты, несвойственные для классической формулировки проблемы планирования заданий. Во-первых, при такой постановке ожидаемо не во всякой игре цена стабильности обязана равняться $1$. Достаточно рассмотреть игру с $2$ сотрудниками и $2$ одинаковыми компьютерами:
\begin{itemize}
	\item $t_1^1 = t_2^1 = 8,\\ t_1^2 = t_2^2 = 2$;
	\item $t^1_{DEADLINE} = t^2_{DEADLINE} = 9,\\ t^1_{BREAKAWAY} = t^2_{BREAKAWAY} = 3$;
	\item $u^1_{HAST} < u^1_{LATE} < u^1_{GOOD},\\ u^2_{HAST} < u^2_{LATE} < u^2_{GOOD}$.
\end{itemize}

Since the first player, definitely not having problems with underloading, fits into the deadline only by using the computer solo, obviously, strategy combinations of both players choosing the same machine cannot be Nash equilibria. Similarly, since it is more profitable for the second player to be late with the calculations than to be punished for running an insufficiently heavy task on a separate computer ($u^a_{HAST} < u^a_{LATE}$), no Nash equilibria can be found among situations where each employee uses his own machine too. In terms of the payoff structure, the game turns out to be indistinguishable from the even-odd game, which has no solutions in pure strategies at all, having the only Nash equilibrium point with both players making independent equiprobable choice between the alternatives. Note that the most loaded machine will work either $10$ hours if their choices match, or $8$ if they do not match with the probability $\frac{1}{2}$, which gives the expected value of stability equal to $\frac{9}{8}$. In fact, when the monotonicity of the promptness reward function is abandoned, it hardly makes sense to talk about the prices of stability and anarchy at all, since games clearly not related to the search for a minimum duration of computations are now included in the class. %Поскольку первый игрок, заведомо не имеющий проблем с недозагрузкой, укладывается в дедлайн только если компьютер будет в его безраздельном пользовании, то очевидно, что сочетания стратегий, в которых оба игрока выбирают одну машину, равновесиями Нэша быть не могут. Похожим образом, поскольку второму игроку выгоднее опоздать с расчётами нежели быть наказанным за использование отдельного компьютера для недостаточно большой задачи ($u^a_{HAST} < u^a_{LATE}$), равновесиями Нэша не могут быть и те ситуации, где каждый из сотрудников использует свою машину. По структуре выплат игра оказывается неотличима от игры в чёт-нечет, вовсе не имеющей решений в чистых стратегиях и с единственным равновесием Нэша в точке независимого равновероятного выбора между альтернативами обоими игроками. При этом максимально загруженная машина с вероятностью $\frac{1}{2}$ проработает либо $10$ часов при совпадении их выбора, либо $8$ при несовпадении, что даёт математическое ожидание цены стабильности, равное $\frac{9}{8}$. По сути, при отказе от монотонности функции оплаты за срочность вряд ли имеет смысл вообще рассуждать о ценах стабильности и анархии, поскольку класс игр теперь начинает включать и такие, которые явно не имеют отношения к поискам минимума продолжительности вычислений.

\section{Mixed equilibria of $\Gamma^3_n$ game}\label{sec:ch2/sec3}

Task scheduling with nonmonotonic returns is a fairly large class of conflicts, the analysis of which in general terms is beyond the scope of this work. For a compelling demonstration of the desired effect, a specially designed example will be quite sufficient. Let's consider the game $\Gamma^3_n$ of the same general scheme as in the previous section, but slightly more complicated, with $3$ tasks of the same type and $n \ge 2$ identical computers: %Планирование заданий с немонотонной отдачей представляет собой довольно обширный класс конфликтов, анализ которого в общем виде выходит за рамки данной работы. Для наглядной демонстрации необходимого эффекта вполне достаточно будет специально сконструированного примера.  Рассмотрим игру $\Gamma^3_n$ той же общей схемы, что в предыдущем разделе, но чуть боле сложную, с $3$ однотипными заданиями и $n \ge 2$ одинаковыми компьютерами:
\begin{itemize}
	\item $t_i^1 = t_i^2 = t_i^3 = 2, i = \overline{1,n}$;
	\item $t^1_{DEADLINE} = t^2_{DEADLINE} = t^3_{DEADLINE} = 5,\\ t^1_{BREAKAWAY} = t^2_{BREAKAWAY} = t^3_{BREAKAWAY} = 3$;
	\item $u^1_{HAST} = u^2_{HAST} = u^3_{HAST} = 0,\\ u^1_{GOOD} = u^2_{GOOD} = u^3_{GOOD} = 3,\\ u^1_{LATE} = u^2_{LATE} = u^3_{LATE} = 2$
	%	\item $u^1_{HAST} < u^1_{LATE} < u^1_{GOOD},\\ u^2_{HAST} < u^2_{LATE} < u^2_{GOOD},\\ u^3_{HAST} < u^3_{LATE} < u^3_{GOOD}$.
\end{itemize}

Simply put, in the game $\Gamma^3_n$ it is most profitable to use a computer a deux "--- $4$ hours of total work time are just in the optimal gap between the underload limit and the deadline. The next most profitable option is to use one machine in threes, which results in a late fee. The least attractive is the choice of a computer running no other tasks "--- the player does not receive anything at all for such an expenditure of a public resource. At first glance, this game does not look too unusual. Nash equilibria in pure strategies are easy to find "--- all profiles $(i, i, i), i = \overline{1,n}$, and it is also obvious that no other solutions in pure strategies are possible. With mixed strategies, things get a little more interesting. %Проще говоря, в игре $\Gamma^3_n$ выгоднее всего использовать компьютер вдвоём "--- $4$ часа суммарной продолжительности работы оказываются как раз в оптимальном промежутке между границей недогруза и дедлайном. Следующий по выгоде вариант "--- использование одной машины втроём, что приводит к штрафу за опоздание. Наименее привлекателен выбор компьютера, на котором задача оказывается одна "--- за такое расходование общественного ресурса игрок не получает вообще ничего. На первый взгляд эта игра не выглядит слишком необычно. В ней без особого труда находятся равновесия Нэша в чистых стратегиях "--- все наборы $(i, i, i), i = \overline{1,n}$, причём очевидно и то, что других решений в чистых стратегиях быть не может. Со смешанными стратегиями дело становится чуть интереснее.

\begin{lemma}
	Let $T \subseteq \{1, \ldots, n\}$ be an arbitrary non-empty subset of computers. Then in the game $\Gamma^3_n$ the set of identical mixed strategies $s^1 = s^2 = s^3 = \left(\frac{[1 \in T]}{\left| T \right|}, \ldots, \frac{[n \in T]}{\left| T \right|}\right)$, where each player independently and equiprobably chooses one of the machines in the set $T$, is a Nash equilibrium.\footnote{Hereinafter the <<Iverson bracket>> notation is used to simplify and shorten the formulas: $[\textsc{true}] = 1, [\textsc{false}] = 0$.} %Пусть $T \subseteq \{1, \ldots, n\}$ "--- произвольное непустое подмножество компьютеров. Тогда в игре $\Gamma^3_n$ набор одинаковых смешанных стратегий $s^1 = s^2 = s^3 = \left(\frac{[1 \in T]}{\left| T \right|}, \ldots, \frac{[n \in T]}{\left| T \right|}\right)$, где каждый игрок независимо и равновероятно выбирает одну из машин множества $T$, является равновесием по Нэшу.\footnote{Здесь и далее для упрощения и сокращения записи используется нотация <<скобка Айверсона>>: $[\textsc{true}] = 1, [\textsc{false}] = 0$.}
\end{lemma}

\begin{proof}
	We take advantage that it suffices to check deviations only in favor of pure strategies. Having symbol $\left. s \right|^a_i$ stand for deviation from $s$ by player $a$ in favor of strategy $i$, we note that %Воспользуемся тем, что достаточно проверить отклонения только в пользу чистых стратегий. Обозначим символом $\left. s \right|^a_i$ отклонение от набора $s$ игроком $a$ в пользу стратегии $i$ и заметим, что
	\begin{align*}
		u^a(\left. s \right|^a_i) &= [i \in T] \left(\frac{(\left| T \right| - 1)^2}{\left| T \right|^2} u^a_{HAST} + 2 \frac{\left| T \right| - 1}{\left| T \right|^2} u^a_{GOOD} + \frac{1}{\left| T \right|^2} u^a_{LATE}\right)\\
		&= [i \in T] \frac{6 \left| T \right| - 4}{\left| T \right|^2}.
	\end{align*}
	
	Thus, for each player, the maximum expected payoff is achieved by deviating in favor of any computer in $T$. %Таким образом, для каждого игрока максимум ожидаемого выигрыша достигается при отклонении в пользу любого из компьютеров, входящих в $T$.
\end{proof}

As such, without deviations, the mathematical expectation of payoffs amounts to $u^a(s) = \frac{6 \left| T \right| - 4}{\left| T \right|^2}, a = \overline{1,3}$. To prove that no other Nash equilibria in mixed strategies exist, we need a couple more lemmas: %Соответственно без отклонений математическое ожидание выигрышей предстаёт в виде $u^a(s) = \frac{6 \left| T \right| - 4}{\left| T \right|^2}, a = \overline{1,3}$. Для доказательства того, что других равновесий по Нэшу в смешанных стратегиях нет, нам понадобятся ещё пара лемм:

\begin{lemma}
	In the game $\Gamma^3_n$, the set of mixed strategies $s = (s^1, s^2, s^3)$ can be a Nash equilibrium only if $s^1 = s^2 = s^3$. %В игре $\Gamma^3_n$ набор смешанных стратегий $s = (s^1, s^2, s^3)$ может быть равновесием по Нэшу только в том случае, когда $s^1 = s^2 = s^3$.
\end{lemma}

\begin{proof}
	Let $s^a = (p_1^a, \ldots, p_n^a), a = \overline{1,3}$. If the strategies of the players do not coincide, then there is a computer $i$ for which (without loss of generality) the probabilities of choosing by the first and second players are $p_i^1 > p_i^2$. Due to the elementary properties of probabilities, there is also a computer $j$, where $p_j^1 < p_j^2$. Let us write down the mathematical expectation of the same players' payoffs when they choose the computer $i$ (for the $j$ everything coincides up to an index, obviously): %Пусть $s^a = (p_1^a, \ldots, p_n^a), a = \overline{1,3}$. Если стратегии игроков не совпадают, то найдётся компьютер $i$, для которого (без потери общности) вероятности выбора первым и вторым игроком $p_i^1 > p_i^2$. В силу элементарных свойств вероятностей, найдётся также и компьютер $j$, где $p_j^1 < p_j^2$. Выпишем математическое ожидание выигрышей тех же игроков при выборе ими $i$"~го компьютера (для $j$"~го всё совпадает с точностью до индекса, очевидно):
	\begin{align*}
		u^1(\left. s \right|^1_i) &= (1 - p_i^2) (1 - p_i^3) u^1_{HAST} + (p_i^2 + p_i^3 - 2 p_i^2 p_i^3) u^1_{GOOD} + p_i^2 p_i^3 u^1_{LATE}\\ &= 3 p_i^2 + 3 p_i^3 - 4 p_i^2 p_i^3;\\
		u^2(\left. s \right|^2_i) &= 3 p_i^1 + 3 p_i^3 - 4 p_i^1 p_i^3;\\
		u^3(\left. s \right|^3_i) &= 3 p_i^1 + 3 p_i^2 - 4 p_i^1 p_i^2.
	\end{align*}
	
	About the function $f(x,y) = 3x + 3y - 4xy$ on the domain $0 \le x \le 1, 0 \le y \le 1$ one can notice the following "--- if $f(x_0, y_0) < 2$, then $f(x_0, y_0) < f(x_1, y_1)$ is true for any $x_1 > x_0, y_1 \ge y_0$ or $x_1 \ge x_0, y_1 > y_0$. Suppose $p_i ^3 \le p_j^3$ (if $p_i^3 \ge p_j^3$, the arguments are similar up to indices.) Consider the following cases: %Про функцию $f(x,y) = 3x + 3y - 4xy$ на области определения $0 \le x \le 1, 0 \le y \le 1$ можно заметить следующее "--- если $f(x_0, y_0) < 2$, то для любых $x_1 > x_0, y_1 \ge y_0$ или $x_1 \ge x_0, y_1 > y_0$ выполняется $f(x_0, y_0) < f(x_1, y_1)$. Предположим, что $p_i^3 \le p_j^3$ (если $p_i^3 \ge p_j^3$, рассуждения аналогичны с точностью до индексов). Рассмотрим следующие случаи:
	\begin{itemize}
		\item $p_i^2 < p_j^2$. Due to the elementary properties of probabilities, we can be sure that $p_i^2 < \frac{1}{2}$ and $p_i^3 \le \frac{1}{2}$, whence $u^1(\left .s \right|^1_i) < 2$, which means that $u^1(\left. s \right|^1_i) < u^1(\left. s \right|^1_j)$. Since $p_i^1 > p_i^2 \ge 0$, the first player chooses a non-optimal strategy with non-zero probability. $\bot$ %В силу элементарных свойств вероятностей мы можем быть уверены, что $p_i^2 < \frac{1}{2}$ и $p_i^3 \le \frac{1}{2}$, откуда $u^1(\left. s \right|^1_i) < 2$, а значит $u^1(\left. s \right|^1_i) < u^1(\left. s \right|^1_j)$. Поскольку $p_i^1 > p_i^2 \ge 0$, первый игрок выбирает неоптимальную стратегию с ненулевой вероятностью. $\bot$
		\item $p_i^2 \ge p_j^2$, which implies $p_i^1 > p_j^1$, and hence $u^3(\left. s \right|^3_j) < u^3(\left. s \right|^3_i)$ similarly to the previous point. If $p_j^3 > 0$, then the third player chooses a non-optimal strategy with non-zero probability. $\bot$ %из чего следует $p_i^1 > p_j^1$, а значит аналогично предыдущему пункту $u^3(\left. s \right|^3_j) < u^3(\left. s \right|^3_i)$. Если $p_j^3 > 0$, то третий игрок выбирает неоптимальную стратегию с ненулевой вероятностью. $\bot$
		\item $p_i^2 \ge p_j^2$, as in the previous case, but now $p_i^3 = p_j^3 = 0$. $p_j^1 < p_i^1$ similarly implies $u^2(\left. s \right|^2_j) < u^2(\left. s \right|^2_i)$, and so $p_j^ 2 > p_j^1 \ge 0$ implies the choice of a non-optimal strategy with non-zero probability by the second player this time. $\bot$ %как и в предыдущем случае, но теперь $p_i^3 = p_j^3 = 0$. Из $p_j^1 < p_i^1$ аналогичным образом следует $u^2(\left. s \right|^2_j) < u^2(\left. s \right|^2_i)$, и поэтому $p_j^2 > p_j^1 \ge 0$ влечёт выбор неоптимальной стратегии с ненулевой вероятностью уже вторым игроком. $\bot$
	\end{itemize}
	
	Thus, assumption that the existence of a computer for which the probability of being chosen by one player differs from the probability being chosen by another, contradicts the necessary condition of the Nash equilibrium in every case. %Таким образом, предположение о существовании компьютера, для которого вероятность выбора его одним игроком отличается от вероятности выбора другим, в любом случае противоречит необходимому условию равновесия Нэша.
\end{proof}

\begin{lemma}
	In the game $\Gamma^3_n$, a set of identical mixed strategies can be a Nash equilibrium only if all computers chosen with non-zero probability are chosen with equal probabilities. %В игре $\Gamma^3_n$ набор одинаковых смешанных стратегий может быть равновесием по Нэшу только в том случае, когда все компьютеры, выбираемые с ненулевой вероятностью, выбираются с равными вероятностями.
\end{lemma}

\begin{proof}
	Take any profile of identical strategies $(p_1, \ldots, p_n)$, where $0 < p_i < p_j$. Using the payoff formula from the previous lemma, $u^a(\left. s \right|^a_i) = 2 p_i (3 - 2 p_i)$. Again, $p_i < \frac{1}{2}$ implies $u^a(\left. s \right|^a_i) < u^a(\left. s \right|^a_j)$, and so all players have chosen a non-optimal strategy with non-zero probability, which contradicts the necessary condition of Nash equilibrium. %Возьмём любой набор, состоящий из одинаковых стратегий $(p_1, \ldots, p_n)$, где $0 < p_i < p_j$. Пользуясь формулой выплат из предыдущей леммы, $u^a(\left. s \right|^a_i) = 2 p_i (3 - 2 p_i)$. Опять же, из $p_i < \frac{1}{2}$ следует $u^a(\left. s \right|^a_i) < u^a(\left. s \right|^a_j)$, а значит все игроки выбрали неоптимальную стратегию с ненулевой вероятностью, что противоречит необходимому условию равновесия по Нэшу.
\end{proof}

Having proved that the proposed equilibrium points exhaust the mixed strategy solution space, we can construct the convex hull of the set of attainable payoff vectors. Since the set lies entirely on the line $(u, u, u)$, it suffices to find the minimum and maximum expected payoffs: %Доказав, что предложенные точки равновесия исчерпывают пространство решений в смешанных стратегиях, мы можем сконструировать выпуклую оболочку множества достижимых векторов выплат. Поскольку множество целиком лежит на прямой $(u, u, u)$, достаточно найти минимум и максимум ожидаемых выигрышей:

\begin{align*}
	\min_{\emptyset \subset T \subseteq \{1,\ldots,n\}} \frac{6 \left| T \right| - 4}{\left| T \right|^2} &= \frac{6 n - 4}{n^2};\\
	\max_{\emptyset \subset T \subseteq \{1,\ldots,n\}} \frac{6 \left| T \right| - 4}{\left| T \right|^2} &= 2.
\end{align*}

Thus, the desired convex hull is a segment connecting the points $(2, 2, 2)$ and $(\frac{6 n - 4}{n^2}, \frac{6 n - 4}{n^2} , \frac{6 n - 4}{n^2})$. If the game under consideration was not sensitive to additional information asymmetry, then its analysis would be completed "--- all players are in an equal position and, acting optimally, can expect equal payoffs from the indicated interval. However, through the lens of the conspiracy model this conflict looks much more interesting. %Таким образом искомая выпуклая оболочка представляет собой отрезок, соединяющий точки $(2, 2, 2)$ и $(\frac{6 n - 4}{n^2}, \frac{6 n - 4}{n^2}, \frac{6 n - 4}{n^2})$. Если бы рассматриваемая игра не была чувствительна к дополнительной информационной асимметрии, на этом её анализ можно было бы закончить "--- все игроки находятся в равном положении и, действуя оптимально, могут ожидать равных выигрышей из указанного промежутка. Однако, взгляд на этот конфликт через призму модели заговоров делает картину происходящего куда интереснее.

\section{Correlated equilibria of the $\Gamma^3_n$ game in the conspiracy space}\label{sec:ch2/sec4}

Let's analyze the same conflict from the standpoint of conspiracy theory by moving on to the game $\Gamma^3_n | \{\{1,2\}\}$. Now, the game $\Gamma^3_n$ is supplemented by one real-value roulette, which result is known before choosing a strategy to players $1$ and $2$, but not $3$. To obtain an equilibrium set of correlated strategies that extricate payoffs from the convex hull of the solution set in mixed strategies, it is enough for the conspirators to take any of the classical Nash equilibrium points with $\left| T \right| \ge 2$, but instead of choosing between the elements of $T \subseteq \{1, \ldots, n\}$ independently, they must divide the secret roulette into $\left| T \right|$ equal sectors and make their choice in unison, depending on the sector hit. Let us describe this more formally using the correlation space %Проанализируем этот же конфликт с позиций теории заговоров, перейдя к игре $\Gamma^3_n | \{\{1,2\}\}$. При этом игра $\Gamma^3_n$ дополняется одной вещественной рулеткой, результат вращения которой перед выбором стратегии узнают игроки $1$ и $2$, но не $3$. Для получения равновесного набора коррелированных стратегий, выводящего платежи за выпуклую оболочку множества решений в смешанных стратегиях, заговорщикам достаточно взять любую из классических точек равновесия по Нэшу с $\left| T \right| \ge 2$, но вместо того, чтобы выбирать между элементами $T \subseteq \{1, \ldots, n\}$ независимо, они должны разделить тайную рулетку на $\left| T \right|$ равных секторов и делать свой выбор синхронно, в зависимости от выпавшего сектора. Опишем это более формально, используя пространство корреляции
\begin{equation*}
	\Phi = \langle A, \Omega, \mathfrak{I}^a, \mathbb{P}, a \in A \rangle
\end{equation*}

In this case, the set of states of nature is $\Omega = \left[0, 1\right)$, player awareness $\sigma$"~algebras $\mathfrak{I}^1 = \mathfrak{I}^2$ are Borel, $\mathfrak{I}^3 = \{\emptyset, \Omega\}$ and the measure is $\mathbb{P}(X) = \left| X\right|$. The above strategies in the game $\Gamma^3_n | \Phi$ can be represented as functions that map the set of states of nature into the space of mixed strategies: %В данном случае множество состояний природы $\Omega = \left[0, 1\right)$, $\sigma$"~алгебры информированности игроков $\mathfrak{I}^1 = \mathfrak{I}^2$ "--- борелевские, $\mathfrak{I}^3 = \{\emptyset, \Omega\}$ и мера $\mathbb{P}(X) = \left| X \right|$. Вышеописанные стратегии в игре $\Gamma^3_n | \Phi$ можно представить в виде функций, отображающих множество состояний природы в пространство смешанных стратегий:
\begin{align*}
	\mathbf{s}^1(\omega) = \mathbf{s}^2(\omega) &= ([\zeta(\omega) = 1], \ldots, [\zeta(\omega) = n]); \\ \mathbf{s}^3(\omega) &= \left(\frac{[1 \in T]}{\left| T \right|}, \ldots, \frac{[n \in T]}{\left| T \right|}\right),
\end{align*}
where the function $\zeta : \Omega \rightarrow T$ common for players 1 and 2 determines the partition of the roulette into $\left| T \right|$ equal sectors. %где общая для игроков 1 и 2 функция $\zeta : \Omega \rightarrow T$ определяет разбиение рулетки на $\left| T \right|$ равных секторов.

The payouts in this profile are no longer symmetrical: %Выплаты в этом наборе уже не симметричны:
\begin{align*}
	u^1(\mathbf{s}) = u^2(\mathbf{s}) &= \frac{\left| T \right| - 1}{\left| T \right|} u^a_{GOOD} + \frac{1}{\left| T \right|} u^a_{LATE} = \frac{3 \left| T \right| - 1}{\left| T \right|};\\
	u^3(\mathbf{s}) &= \frac{\left| T \right| - 1}{\left| T \right|} u^a_{HAST} + \frac{1}{\left| T \right|} u^a_{LATE} = \frac{2}{\left| T \right|}.
\end{align*}

In this case, the situation is indeed a Nash equilibrium, since the first and second players can use as strategies any functions that map $\Omega$ into the space of probability measures on $\{1, \ldots, n\}$, while the third player has to be content only with constant ones, since the correlation mechanism does not inform him about the state of nature. Noticing that $\frac{3 \left| T \right| - 1}{\left| T \right|} > 2$ for $\left| T \right| \ge 2$, we confirm the sensitivity of the $\Gamma^3_n$ game to additional information asymmetry "--- in the new solutions, players observing a random experiment not directly related to payoffs increase their payoff compared to the best result achievable in the canonical mixed case. %При этом ситуация действительно является равновесием по Нэшу, поскольку первый и второй игроки могут в качестве стратегий использовать любые функции, отображающие $\Omega$ в пространство вероятностных мер на $\{1, \ldots, n\}$, а вот третий игрок вынужден довольствоваться только константными в виду того, что механизм корреляции не информирует его о состоянии природы. Заметив, что $\frac{3 \left| T \right| - 1}{\left| T \right|} > 2$ при $\left| T \right| \ge 2$, мы подтверждаем чувствительность игры $\Gamma^3_n$ к дополнительной информационной асимметрии "--- в новых решениях игроки, наблюдающие не связанный напрямую с выплатами случайный эксперимент, увеличивают свой выигрыш по сравнению с наилучшим результатом, достижимым в классическом смешанном случае.

\section{Collective rationality of decisions}\label{sec:ch2/sec5}

When considering the solution set of the game $\Gamma^3_n$ in ordinary mixed strategies, one should pay attention to the fact that for $\left| T \right| > 2$, the resulting Nash equilibrium points lack not only strong Pareto optimality, but even weak too. Indeed, payouts to all players at points where $\left| T \right| = 1$ or $\left| T \right| = 2$ are equal to $2$, but for larger sizes of the set $\frac{6 \left| T \right| - 4}{\left| T \right|^2} < 2$. Such circumstances endow solutions using one or two computers with a special status "--- we can expect agents familiar with the principle of collective rationality to agree not to end up in a situation totally dominated by the other solution. The question naturally follows, is it possible to filter out in a similar way the solutions, that both take into account the additional information asymmetry, and satisfy the principles of collective rationality in at least some sense? %При рассмотрении множества решений игры $\Gamma^3_n$ в обычных смешанных стратегиях следует обратить внимание на то, что при $\left| T \right| > 2$ получающиеся точки равновесия по Нэшу лишены оптимальности не только в смысле Парето, но и по Слейтеру. В самом деле, выплаты всем игрокам в точках, где $\left| T \right| = 1$ или $\left| T \right| = 2$, равны $2$, а вот при больших размерах множества $\frac{6 \left| T \right| - 4}{\left| T \right|^2} < 2$. Такие обстоятельствах наделяют решения с использованием одного или двух компьютеров особым статусом "--- можно ожидать, что агенты, знакомые с принципом коллективной рациональности, всё же сумеют договориться о том, чтобы не оказаться в ситуации, которую другое решение доминирует по всем платежам. Естественно сразу задаться вопросом, а нельзя ли и решения с учётом дополнительной информационной асимметрии отфильтровать похожим образом, выделив из них удовлетворяющие принципам коллективной рациональности хоть в каком-то смысле?

When you add a conspiracy space, consisting of the group $\{1, 2\}$, to the game it catches the attention that the classical principles of collective rationality become useless. At the new equilibrium points, the payoffs of the players 1 and 2 are now $u^1(\mathbf{s}) = u^2(\mathbf{s}) > 2$ and grow with the growth of $k$, while the utility of the third is $u^3(\mathbf{s}) < 2$ and declines, which indicates a direct antagonism of interests, making it impossible to agree on a collectively rational choice in the usual sense between a mixed equilibrium with $\left|T \right|$ equal to $1$ or $2$ and correlated solutions with different $k$. However, one can try applying a more subtle optimality criterion based on a slightly extended interpretation of what is happening in the game "--- let's call this formalism \emph{structurally coherent Nash equilibrium}. %При добавлении в игру пространства заговоров, состоящего из группы $\{1, 2\}$, сразу бросается в глаза то, что классические принципы коллективной рациональности становится бесполезны. В новых точках равновесия выигрыши первых двух игроков теперь $u^1(\mathbf{s}) = u^2(\mathbf{s}) > 2$ и растут с ростом $k$, а третьего "--- $u^3(\mathbf{s}) < 2$ и падают, что говорит о прямом антагонизме интересов. Это делает невозможным коллективно рациональный выбор в обычном смысле между смешанным равновесием с $\left| T \right|$ равным $1$ или $2$ и коррелированными решениями с различными $k$. Тем не менее, можно попытаться применить более тонкий критерий оптимальности, опирающийся на слегка расширенную интерпретацию происходящего в игре "--- назовём этот формализм \emph{структурно согласованным равновесием по Нэшу}.

Structural coherency of equilibria in games with conspiracies is the quite simple idea "--- if we assume that the groups of players included in the family of conspiracies are united not only by a common correlation mechanism, but also, in broader terms, great opportunities for coordinating actions, then among the usual Nash equilibria in correlated strategies it is possible to single out those that are resistant to deviations not only individually, but also collectively, bearing in mind exclusively the groups included in the conspiracy family. For the game $\Gamma^3_n | \{\{1,2\}\}$, specifically, equilibria allowing a deviation by players 1 and 2, mutually increasing their payoffs, will be structurally incoherent. In the most general terms, this can be expressed as follows: %Идея структурной согласованности равновесий в играх с заговорами довольно проста "--- если допустить, что входящие в семейство заговоров группы игроков объединяет не только общий механизм корреляции, но и в целом большие возможности для согласования действий, то среди обычных равновесий по Нэшу в коррелированных стратегиях можно особо выделить обладающие устойчивостью не только к индивидуальным отклонениям, но и к групповым, имея в виду исключительно входящие в семейство заговоров группы. Для игры $\Gamma^3_n | \{\{1,2\}\}$, например, это могло бы означать, что структурно несогласованными окажутся те равновесия, для которых найдётся отклонение, в котором участвуют первый и второй игроки, обоюдно увеличивая при этом свои выигрыши. В наиболее общих терминах это можно выразить так:
\begin{definition}
	In a conspiracy game $\Gamma | \mathfrak{A}$, a situation $\mathbf{s}$ is a structurally coherent equilibrium if for all conspiracies $A_* \in \mathfrak{A}$ there exist no acceptable deviations from the situation $\mathbf{s}$. %В игре с заговорами $\Gamma | \mathfrak{A}$ равновесие по Нэшу $\mathbf{s}$ называется структурно согласованным, если для всех заговоров $A_* \in \mathfrak{A}$ отсутствуют приемлемые отклонения от ситуации $\mathbf{s}$.
\end{definition}

There remains to formulate what, within the framework of this model, can be considered a deviation acceptable for various cabals. If we look at this question as a problem of multi-criteria optimization, where the criteria are the payoffs of individual participants, then two options arise: %Остаётся сформулировать, что в рамках данной модели можно считать отклонением, приемлемым для того или иного заговора. Если взглянуть на этот вопрос как на проблему многокритериальной оптимизации, где критериями являются выигрыши отдельных участников, то напрашиваются два варианта:
\begin{enumerate}
	\item Deviation is acceptable if it increases the payoff of all conspirators; (weak Pareto) %Отклонение приемлемо, если оно увеличивает выигрыш всех участников заговора; (по Слейтеру)
	\item Deviation is acceptable if it increases the payoff of at least one conspirator without decreasing the payoff of the others. (strong Pareto) %Отклонение приемлемо, если оно увеличивает выигрыш хотя бы одного участника заговора, при этом не уменьшая выигрыша остальных. (по Парето)
\end{enumerate}

Alas, for our purposes suitability of both options is questionable. It would be reasonable to expect that the addition of <<dummy>> to any conspiracy, i.e. player with a single pure strategy and constant payoff, should not change anything in the solution of the game. However, a conspiracy with such <<dummy>> in the lineup cannot have acceptable deviations in the weak Pareto sense at all, which means that its participants lose the opportunity to use collective rationality. This obviously makes the first of the proposed options too weak. On the other hand, considering the three-way even"~odd game (see \ref{sec:ch1/sec4}) with two cabals creates an unpleasant problem for the second option as well. If player 1 is in a cabal with player 2, and player 2 with player 3, then we can expect outcomes with payments formed by any mixture of $(5, 5, 2)$ and $(2, 5, 5)$, with the choice of proportion at the behest of player 2. The catch is that strong Pareto acceptability encourages player 2 in the $(5,5,2)$ situation to deviate conspiring with player 3 to improve the other's payoff. Similarly, in the $(2,5,5)$ situation, player 2 has to help player 1. In intermediate situations, player 2 can help both, which means that such altruism rules out the existence of a structurally coherent equilibrium in the problem at all, making the second version of the acceptability definition too strong. To bypass both problems, an intermediate definition of deviation should be proposed, which will be stronger than weak Pareto, but weaker than strong Pareto: %Увы, оба варианта нельзя назвать подходящими для наших целей. Разумно было бы ожидать, что добавление в любой заговор <<болвана>>, т.е. игрока с единственной чистой стратегией и константным выигрышем, не должно ничего менять в решении игры. Однако, заговор с подобным <<болваном>> в составе вообще не может иметь приемлемых отклонений в смысле Слейтера, а значит его участники теряют возможность использовать коллективную рациональность. Это делает первый из предложенных вариантов, очевидно, слишком слабым. С другой же стороны, рассмотрение игры в трёхсторонний чёт"~нечет (см. \ref{sec:ch1/sec4}) с двумя заговорами порождает неприятную проблему и для второго варианта. Если 1"~й игрок находится в заговоре со 2"~м, а 2"~й с 3"~м, то можно ожидать исходов с платежами, образованными любым смешением $(5,5,2)$ и $(2,5,5)$, причём выбор пропорции происходит по воле 2"~го игрока. Загвоздка в том, что приемлемость в смысле Парето побуждает 2"~го игрока в ситуации $(5,5,2)$ отклоняться в рамках заговора с 3"~м для улучшения чужого выигрыша. Аналогично, в ситуации $(2,5,5)$ 2"~му игроку приходится спасать 1"~го. В промежуточных же ситуациях 2"~й игрок может помочь обоим, а значит подобный альтруизм вообще исключает существование структурно согласованного равновесия в задаче, делая второй вариант определения приемлемости слишком сильным. Для обхода обоих проблем следует предложить промежуточное определение отклонения, которое будет сильнее чем по Слейтеру, но слабее чем по Парето:
\begin{definition}
	In the conspiracy game $\Gamma | \mathfrak{A}$ a situation $\mathbf{s}_* \neq \mathbf{s}$ is called a deviation from $\mathbf{s}$ acceptable for the cabal $A_* \in \mathfrak{A}$ if %В игре с заговорами $\Gamma | \mathfrak{A}$ ситуацию $\mathbf{s}_* \neq \mathbf{s}$ назовём отклонением от $\mathbf{s}$, приемлемым для заговора $A_* \in \mathfrak{A}$, если
	\begin{itemize}
		\item $\forall a \notin A_* \quad \mathbf{s}^a = \mathbf{s}_*^a$;
		\item $\forall a \in A_* \quad u^a(\mathbf{s}_*) \geq u^a(\mathbf{s})$;
		\item $\forall a : \mathbf{s}^a \neq \mathbf{s}_*^a \quad u^a(\mathbf{s}_*) > u^a(\mathbf{s})$.
	\end{itemize}
\end{definition}

If we were talking only about Nash equilibria in pure and mixed strategies, such an optimality criterion would turn out to be too strong "--- indeed, the simultaneous choice by the first and second players of the computer $i$, chosen by the third with probability $p_i^3 < 1$, gives both payoffs $(1 - p_i^3) u^a_{GOOD} + p_i^3 u^a_{LATE} = 3 - p_i^3 > 2$, which eliminates all solutions in the game $\Gamma^3_n$ For the game with conspiracies, however, the following can be formulated: %Если бы мы говорили только о равновесиях Нэша в чистых и смешанных стратегиях, такой критерий оптимальности оказался бы слишком сильным "--- действительно, одновременный выбор первым и вторым игроками компьютера $i$, выбираемого третьим с вероятностью $p_i^3 < 1$, даёт обоим выигрыш $(1 - p_i^3) u^a_{GOOD} + p_i^3 u^a_{LATE} = 3 - p_i^3 > 2$, что отсеивает вообще все решения в игре $\Gamma^3_n$. Для игры с заговорами же можно сформулировать следующее:

\begin{theorem}
	In the conspiracy game $\Gamma^3_n | \mathfrak{A}$ a structurally coherent Nash equilibrium exists for any $n$ and $\mathfrak{A}$. For non-degenerate $\mathfrak{A}$, each two-participant cabal corresponds to an unique equilibrium like that. %В игре с заговорами $\Gamma^3_n | \mathfrak{A}$ структурно согласованное равновесие Нэша существует при любых $n$ и $\mathfrak{A}$. При невырожденном $\mathfrak{A}$, каждому заговору из двух участников соответствует единственное такое равновесие.
\end{theorem}

\begin{proof}
	Three cases are possible depending on $\mathfrak{A}$: %Возможны три случая в зависимости от $\mathfrak{A}$:
	\begin{enumerate}
		\item $\mathfrak{A} = \emptyset$. With an empty family of conspiracies, group deviations are impossible, so any Nash equilibrium in pure or mixed strategies will be structurally coherent. %При пустом семействе заговоров групповые отклонения невозможны, так что любое равновесие по Нэшу в чистых или смешанных стратегиях будет структурно согласованным.
		\item $\mathfrak{A} = \{\{1, 2, 3\}\}$. For a degenerate family of conspiracies with one public correlation mechanism, the set of solutions is a convex hull of mixed equilibrium payoff vectors, which, as shown in \ref{sec:ch2/sec3}, gives a interval connecting $(2, 2, 2)$ and $(\frac{6 n - 4}{n^2}, \frac{6 n - 4}{n^2}, \frac{6 n - 4}{n^2})$. In which case, the criterion of structural coherency, obviously, filters out everything except the point $(2, 2, 2)$, corresponding to pure equilibria with the choice of any single shared strategy for everyone and mixed equilibria with an independent equiprobable choice of any pair of same strategies by each player. %При вырожденном семействе заговоров с одним публичным механизмом корреляции множество решений представляет собой выпуклую оболочку векторов платежей смешанных равновесий, что, как было показано в разделе \ref{sec:ch2/sec3}, даёт отрезок, соединяющий $(2, 2, 2)$ и $(\frac{6 n - 4}{n^2}, \frac{6 n - 4}{n^2}, \frac{6 n - 4}{n^2})$. При этом критерий структурной согласованности, очевидно, отсеивает всё, кроме точки $(2, 2, 2)$, соответствующей чистым равновесиям с выбором одной любой общей стратегии на всех и смешанным равновесиям с независимым равновероятным выбором из двух любых одних и тех же стратегий каждым игроком.
		\item $\mathfrak{A}$ contains $\{1, 2\}$, $\{1, 3\}$ or $\{2, 3\}$. As shown in the \ref{sec:ch2/sec4} section, the use of secret correlation mechanisms yields new equilibrium points with payoffs of $\frac{3k - 1}{k}$ for conspirators and $\frac{2}{k}$ for outsider, where $k$ is the number of machines involved in the strategy profile. Since for $k \ge 2$ the each of the conspirators' payoff exceeds the best conventional result, no points of mixed equilibrium will be structurally coherent. Since with the growth of $k$ the conspirators' payoffs also increase, we filter out all correlated equilibria too, except for those that maximize the income of any couple of conspirators at $k = n$. Thus, each cabal in the structure of conspiracies corresponds to one (up to permutations of roulettes) point of structurally coherent equilibrium, in which players choose equiprobably from the entire available computer pool, with the choice of cabal members always coinciding while being independent from the choice of the remaining player. %$\mathfrak{A}$ содержит $\{1, 2\}$, $\{1, 3\}$ или $\{2, 3\}$. Как было показано в разделе \ref{sec:ch2/sec4}, использование тайных механизмов корреляции даёт новые точки равновесия с выплатами $\frac{3k - 1}{k}$ участникам заговора и $\frac{2}{k}$ аутсайдеру, где $k$ "--- количество задействованных в наборе стратегий машин. Поскольку при $k \ge 2$ выигрыш каждого из заговорщиков превосходит наилучший классический результат, никакие точки смешанного равновесия структурно согласованными уже не будут. Поскольку с ростом $k$ растут и выигрыши заговорщиков, отсеиваются также и все коррелированные равновесия, кроме максимизирующих доход любой из пар заговорщиков при $k = n$. Таким образом, каждой группе, входящей в структуру заговоров, соответствует одна (с точностью до пермутаций рулеток) точка структурно согласованного равновесия, в которой игроки выбирают равновероятно из всего доступного парка машин, причём выбор членов группы всегда совпадает между собой и независим с выбором оставшегося игрока.
	\end{enumerate}
\end{proof}

For new structurally coherent equilibria, it is easy to find a fairly natural interpretation. If we imagine that two employees can coordinate their activities unbeknownst to a third, then it is not surprising that they tend to choose one computer for two in order to avoid an underload penalty, while minimizing the chance for a third player to stumble upon them coincidentally, depriving them of their promptness bonus. Logically, this is achieved when the highest probability of choosing each of the machines is minimal, i.e. with an equiprobable sweeping choice. Similarly, the goal for the third player is to maximize the smallest probability of choosing each of the computers, since he understands that the conspirators are acting together and trying to avoid meeting him, and this is also achieved in a situation of equiprobable choice from the entire computer pool. %Для новых структурно согласованных равновесий несложно подобрать вполне естественную интерпретацию. Если представить, что два сотрудника могут координировать свои действия втайне от третьего, то нет ничего неожиданного в их стремлении выбрать один компьютер на двоих, чтобы избежать штрафа за недозагруз, минимизируя при этом шанс для третьего игрока наткнуться на них по воле случая, лишив их премии за срочность. Достигается это логичным образом тогда, когда наибольшая вероятность выбора каждой из машин минимальна, т.е. при равновероятном выборе из всех. Аналогичным образом, для третьего игрока целью становится максимизация наименьшей вероятности выбора каждого из компьютеров, поскольку он понимает, что заговорщики действуют заодно и пытаются избежать встречи с ним, и это тоже достигается в ситуации равновероятного выбора из всего парка машин.

\section{Preservation of conspiracy secrets amid consensus building}\label{sec:ch2/sec6}

As is known, matrix games frequently have several points of Nash equilibrium (like coordination games), with profiles of strategies from different points being not equilibria themselves. Interpretation of such equilibrium points as game solutions requires stipulation that the players essentially choose equilibrium strategies not independently "--- the preference for one equilibrium point over another must be universal among the players. In canonical games with complete information, this does not create big problems, since the procedure leading to consensus may be wholly transparent, but when it comes to modeling conflicts with conspiracies, this issue begins to require a much more careful approach. When information asymmetry appears in the game, significantly affecting its outcome, it may become beneficial for players to change the configuration of this asymmetry (for example, notifying of the secret signal players who should not know it according to the correlation space structure), for which, in case of bad design, the same consensus mechanisms can be used. In order to understand what can go wrong, it makes sense to start with the canonical case. If we imagine an arbitrary matrix game as a real process with live players under the control of an impartial host who monitors the game protocol, then something like the following procedure can be used to obtain a Nash equilibrium: %Как известно, в матричных играх нередко наличествуют несколько точек равновесия по Нэшу (например, координационные игры), причём наборы, состоящие из стратегий, принадлежащих к разным точкам, сами равновесиями не являются. При интерпретации таких точек равновесия в качестве решений игры приходится оговаривать, что выбор игроками равновесных стратегий в сущности не является независимым "--- предпочтение одной точки равновесия другой должно носить характер консенсуса среди игроков. В классических играх с полной информацией это не создаёт больших проблем, поскольку процедура, приводящая к консенсусу, вполне может быть гласной. Однако, когда речь заходит о моделировании конфликтов с заговорами, данный вопрос начинает требовать гораздо более осторожного подхода. При появлении в игре информационной асимметрии, существенно влияющей на её исход, игрокам может становиться выгодно изменять картину этой асимметрии (оповещая, к примеру, о значении тайного сигнала игроков, которые не должны его знать в соответствии со структурой пространства корреляции), для чего при неправильном дизайне могут использоваться те самые механизмы выработки консенсуса. Для того чтобы понять, что может пойти не так, имеет смысл начать с классического случая. Если представить произвольную матричную игру в виде реального процесса с живыми игроками под управлением беспристрастного ведущего, следящего за соблюдением протокола игры, то для получения равновесия по Нэшу можно применить что-то вроде следующей процедуры:

\begin{enumerate}
	\item The host announces the payout matrix; %Ведущий объявляет матрицу выплат;
	\item Players publicly discuss the choice of strategies; %Игроки гласно обсуждают выбор стратегий;
	\item Players secretly inform the host about their moves; %Игроки втайне друг от друга извещают ведущего о своих ходах;
	\item The host announces the aggregated strategy profile; %Ведущий оглашает собранный набор стратегий;
	\item The host can choose any of the players with a non-zero probability and offer him to retract the move; %Ведущий может с ненулевой вероятностью выбрать любого из игроков и предложить ему переходить;
	\item The host calculates and announces the winnings. %Ведущий вычисляет и объявляет выигрыши.
\end{enumerate}

This procedure is quite sufficient if we are talking about canonical Nash equilibria in pure and mixed strategies. In the second case, it should only be clarified that the implementation of a specific outcome, determined by a set of mixed strategies, either does not occur at all (the host announces the mathematical expectations of the payoffs), or occurs only at the last stage. However, when correlation spaces are added to the model, the situation becomes somewhat more complicated. Since any functions mapping the signals received by the players into mixed strategies can serve as correlated strategies, it seems natural to imagine the players themselves calculating their own functions upon receiving all the relevant signals: %Этой процедуры вполне достаточно, если мы говорим о классических равновесиях Нэша в чистых и смешанных стратегиях. Во втором случае следует только уточнить, что реализация конкретного исхода, определяемого набором смешанных стратегий, либо не происходит вообще (ведущий объявляет математические ожидания выигрышей), либо происходит только на последнем этапе. Однако, при добавлении к модели пространств корреляции ситуация несколько усложняется. Поскольку коррелированными стратегиями могут быть любые функции, отображающие получаемые игроками сигналы в смешанные стратегии, кажется естественным представлять себе, как игроки сами вычисляют избранные ими же функции, получив все релевантные сигналы:

\begin{enumerate}
	\item The host announces the payout matrix; %Ведущий объявляет матрицу выплат;
	\item Players publicly discuss the choice of correlated strategies; %Игроки гласно обсуждают выбор коррелированных стратегий;
	\item The host generates a state of nature and notifies the players about the corresponding events from their awareness $\sigma$"~algebras; %Ведущий генерирует состояние природы и извещает игроков о соответствующих событиях из их $\sigma$"~алгебр информированности;
	\item Players secretly calculate mixed strategies and inform the host about their moves; %Игроки втайне друг от друга вычисляют смешанные стратегии и извещают ведущего о своих ходах;
	\item The host announces the aggregated mixed strategy profile; %Ведущий оглашает собранный набор смешанных стратегий;
	\item The host can choose any of the players with a non-zero probability and offer him to retract the move; %Ведущий может с ненулевой вероятностью выбрать любого из игроков и предложить ему переходить;
	\item The host calculates and announces the winnings. %Ведущий вычисляет и объявляет выигрыши.
\end{enumerate}

Unfortunately, such a naive approach has a significant drawback "--- it works as expected only in symmetric correlation spaces, where the awareness $\sigma$"~algebras of all players coincide. If we are talking about conspiracy spaces, then two problems arise at once. Firstly, between steps 3 and 4, some of the players may be tempted to divulge the private signal value, if this can induce players who are not aware of it to choose a strategy more profitable for the leaker. While this issue could be fixed by adding to the algorithm a ban on communication between players starting after stage 2, but, alas, this is only one of the problems. %К сожалению, такой наивный подход обладает существенным недостатком "--- он работает ожидаемым образом только в симметричных пространствах корреляции, где $\sigma$"~алгебры информированности всех игроков совпадают. Если же мы говорим о пространствах заговоров, то возникают сразу две проблемы. Во-первых, между этапами 3 и 4 у кого-то из игроков может возникнуть искушение разгласить значение приватного сигнала, если это может сподвигнуть неосведомлённых о нём игроков на выбор более выгодной для разглашающего стратегии. Этот вопрос ещё можно было бы закрыть, добавив в алгоритм запрет на коммуникацию между игроками, начинающийся после этапа 2, но, увы, это только одна из проблем.

Secondly, which is somewhat more difficult to correct, in the presence of information asymmetry, the ability of any player to change the chosen strategy at stage 6 ceases corresponding to the concept of Nash equilibrium. In the symmetrical case, between the initial choice of the mixed strategy at stage 4 and the possible deviation from it at stage 6, the player does not receive any additional information, since the publicity of the signal already allows calculating the mixed strategies chosen by other players "--- by announcing them, the host, in fact, only fixes the result of the public agreement reached at stage 2. Asymmetric correlation spaces, on the other hand, contain events about which only a part of the players are notified at stage 3. In this case, each player can reliably calculate his own mixed strategy, but not the strategies of opponents tied to signals hidden from him. In this situation, the announcement by the host of the aggregated mixed strategies at step 5 increases the knowledge of the players before one of them decides to deviate, which contradicts the idea of Nash equilibrium. To bring the above procedure in line with the modeled formalism, it must be changed in a somewhat counterintuitive way: %Во-вторых, что поправить несколько сложнее, при наличии информационной асимметрии возможность одного из игроков изменить выбранную стратегию на этапе 6 перестаёт соответствовать концепции равновесия по Нэшу. В симметричном случае между первоначальным выбором смешанной стратегии на этапе 4 и возможным отклонением от неё на этапе 6 игрок не получает никакой дополнительной информации, поскольку публичность сигнала и так позволяет вычислить избранные другими игроками смешанные стратегии "--- оглашая их, ведущий, фактически, только фиксирует результат публичной договорённости, достигнутой на этапе 2. Асимметричные пространства корреляции же содержат события, о которых на этапе 3 оповещается только часть игроков. При этом каждый игрок может достоверно вычислить свою смешанную стратегию, но не стратегии оппонентов, завязанные на скрытые от него сигналы. В этой ситуации оглашение ведущим собранных смешанных стратегий на этапе 5 увеличивает знание игроков перед принятием кем-то из них решения об отклонении, что противоречит идее равновесия по Нэшу. Для приведения вышеописанной процедуры в соответствие с моделируемым формализмом её необходимо изменить несколько противоречащим интуиции образом:

\begin{enumerate}
	\item The host announces the payout matrix; %Ведущий объявляет матрицу выплат;
	\item Players publicly discuss the choice of correlated strategies; %Игроки гласно обсуждают выбор коррелированных стратегий;
	\item Players secretly inform the host about their chosen correlated strategies; %Игроки втайне друг от друга извещают ведущего о выбранных коррелированных стратегиях;
	\item The host announces the aggregated correlated strategy profile; %Ведущий оглашает собранный набор коррелированных стратегий;
	\item The host generates a state of nature and calculates mixed strategies of players; %Ведущий генерирует состояние природы и вычисляет смешанные стратегии игроков;
	\item The host can choose any of the players with a non-zero probability, notify him about the realized events from his awareness $\sigma$"~algebras and offer him to revise mixed strategy calculated by the host; %Ведущий может с ненулевой вероятностью выбрать любого из игроков, известить его обо всех реализовавшихся событиях из его $\sigma$"~алгебры информированности и предложить ему изменить вычисленную ведущим смешанную стратегию;
	\item The host calculates and announces the winnings. %Ведущий вычисляет и объявляет выигрыши.
\end{enumerate}

Thus, in contrast to the symmetrical case, the conspiracy model disallows looking at correlated strategies as <<black boxes>> in the players' heads, that simply prompt synchronous responses to stimuli. Here, sets of correlated strategies have to be interpreted as spoken and formally fixed agreements, since the concept of Nash equilibrium implies the possibility of deviation precisely at the moment when only the intentions of the players to respond in one way or another to secret signals are generally known, but not their specific reactions yet. %Таким образом, в отличие от симметричного случая модель заговоров не позволяет смотреть на коррелированные стратегии как на <<чёрные ящики>> в головах игроков, просто подсказывающие им синхронную реакцию на раздражители. Здесь наборы коррелированных стратегий приходится интерпретировать как проговариваемые и формально фиксируемые соглашения, поскольку концепция равновесия по Нэшу подразумевает возможность отклонения именно в тот момент, когда общим знанием являются только намерения игроков реагировать тем или иным образом на тайные сигналы, но ещё не конкретные их реакции.

Interestingly, when we try to generalize this procedure to obtain structurally coherent equilibria in conspiracy spaces, we again run into a similar problem. At first glance, it would be enough to clarify only stage 6, so that the host could propose deviations from the chosen strategies both to individual players and to entire groups of conspirators. However, this only works as expected for the most simple families with non-overlapping cabals. In the case, where two cabals share participants, now group deviations cease corresponding to the formalism "--- because if, before discussing them, the leader informs each conspirator about the roulettes of all the cabals he is included in, then one of them could divulge the secret of another cabal, thereby improperly increasing the knowledge of non-participants before deviation decision. The easiest way to fix this is by moving group deviations into a separate stage, preceding the generation of the state of nature: %Интересно то, что при попытке обобщить эту процедуру для получения структурно согласованных равновесий в пространствах заговоров мы снова сталкиваемся с похожей проблемой. На первый взгляд достаточно было бы уточнить только этап 6, чтобы ведущий мог предлагать отклониться от выбранных стратегий как отдельным игрокам, так и целым группам заговорщиков. Однако, это срабатывает ожидаемым образом только для наиболее простых семейств с непересекающимися заговорами. В том же случае, когда у двух заговоров могут быть общие участники, уже групповые отклонения перестают соответствовать формализму "--- ведь если перед их обсуждением ведущий оповестит каждого заговорщика о рулетках всех заговоров, в которые тот входит, то кто-то из них тогда мог бы разгласить тайну другого заговора, тем самым неправомерно увеличив знания не входящих в него игроков до принятия решения об отклонении. Проще всего поправить это с помощью выноса групповых отклонений в отдельный этап, предшествующий генерации состояния природы:

\begin{enumerate}
	\item The host announces the payout matrix; %Ведущий объявляет матрицу выплат;
	\item Players publicly discuss the choice of correlated strategies; %Игроки гласно обсуждают выбор коррелированных стратегий;
	\item Players secretly inform the host about their chosen correlated strategies; %Игроки втайне друг от друга извещают ведущего о выбранных коррелированных стратегиях;
	\item The host announces the aggregated correlated strategy profile; %Ведущий оглашает собранный набор коррелированных стратегий;
	\item The host can choose any of the cabals with a non-zero probability and offer its members to revise chosen correlated strategies; %Ведущий может с ненулевой вероятностью выбрать любую из групп заговорщиков и предложить им изменить выбранные коррелированные стратегии;
	\item The host generates a state of nature and calculates mixed strategies of players; %Ведущий генерирует состояние природы и вычисляет смешанные стратегии игроков;
	\item The host can choose any of the players with a non-zero probability, notify him about the realized events from his awareness $\sigma$"~algebras and offer him to revise mixed strategy calculated by the host; %Ведущий может с ненулевой вероятностью выбрать любого из игроков, известить его обо всех реализовавшихся событиях из его $\sigma$"~алгебры информированности и предложить ему изменить вычисленную ведущим смешанную стратегию;
	\item The host calculates and announces the winnings. %Ведущий вычисляет и объявляет выигрыши.
\end{enumerate}

Note that stage 5 (discussion and approval of group deviation by the conspirators) is itself a multi-stage process in which those who want to increase their gain by changing strategy propose a deviation project, while the remaining conspirators can individually impose a veto if this project brings them a loss. Thus, the structural coherence of the equilibrium implies that cabals can deviate at the planning stage, before the players receive information about the state of nature, while individual deviations are possible after the correlation mechanisms comes into action, but before the announcement of actual mixed strategies played by opponents. %Заметим, что этап 5 (обсуждение и утверждение заговорщиками группового отклонения) "--- сам по себе многостадийный процесс, в котором те, кто хочет увеличить свой выигрыш сменой стратегии, предлагают проект отклонения, а остальные участники заговора имеют индивидуальное право вето, если этот проект приносит им убыток. Таким образом, структурная согласованность равновесия подразумевает, что группы заговорщиков могут отклоняться на стадии планирования, перед получением игроками информации о состоянии природы, а индивидуальные отклонения возможны уже после срабатывания механизмов корреляции, но до оглашения конкретный смешанных стратегий, сыгранных противниками.

\section{Nonmonotonic returns in other scheduling conflicts}\label{sec:ch2/sec7}

Using the $\Gamma^3_n$ game, we demonstrated, first, that task scheduling with a nonmonotonic promptness reward function can be sensitive to additional information asymmetry, and, second, that in conspiracy games, despite their inherent partial antagonism of interests, it is possible for solutions to meet the principle of collective rationality. We emphasize the usefulness of these results by noting that the task scheduling problem is much broader than the example we have considered, both in the sense of the possible values of the parameters (the matrix of coefficients $(t_i^a) \in \mathbb{R}_{\ge 0}^{m \times n}$ and reward functions $v^a : \mathbb{R}_{\ge 0} \rightarrow \mathbb{R}, a = \overline{1,m}$), and in the sense of a variety of practical applications models. In the context of this work, it makes no sense to go into too much detail on more complex cases, but in order to show the possible connection of the model with the real world outside of data centers with strange employee incentive schemes, we will try to build a couple of examples with a more substantial application domain. %При помощи игры $\Gamma^3_n$ мы продемонстрировали, во-первых, что планирование заданий с немонотонной функцией оплаты за срочность может быть чувствительно к дополнительной информационной асимметрии, и, во-вторых, что в играх с заговорами, несмотря на присущий им частичный антагонизм интересов, возможны решения, отвечающие принципу коллективной рациональности. Подчеркнём полезность этих результатов, заметив, что проблема планирования заданий гораздо шире рассмотренного нами примера, причём как в смысле возможных значений параметров (матрицы коэффициентов $(t_i^a) \in \mathbb{R}_{\ge 0}^{m \times n}$ и функций отдачи $v^a : \mathbb{R}_{\ge 0} \rightarrow \mathbb{R}, a = \overline{1,m}$), так и в смысле разнообразия практических приложений модели. В контексте этой работы нет смысла заниматься слишком подробным разбором более сложных случаев, но для того, чтобы показать возможную связь модели с реальным миром за пределами вычислительных центров со странными схемами поощрений сотрудников, попробуем построить пару примеров с более солидной предметной областью.

Let's start with economics by imagining how $m$ companies are preparing to enter the market with high-tech product offerings, while facing the choice between $n$ different open standards for the same important aspect of it. For example, it can be a variety of industrial robots and standards for their integration into a <<smart>> shop. When the company $a \in \{1, \ldots, m\}$ enters the market of the $i \in \{1, \ldots, n\}$ standard, it thereby makes a contribution to its development characterized by the vector constant $t_i^a \in \mathbb{R}_{\ge 0} \times \ldots \times \mathbb{R}_{\ge 0}$, the components of which correspond to separate independent aspects (for example, purposes which robots are fulfilling). If in the situation $s$ several companies use the same standard $i$, then by simply summing up their contributions, we can calculate the overall development index $t_i(s) = [s^1 = i] t_i^1 + \ldots + [s^m = i] t_i^m$. The expected return on investment in each of the standards is significantly affected by two discounting factors: the network effect and market saturation. %Начнём с экономики, представив себе, как $m$ компаний готовятся выйти на рынок с предложениями высокотехнологичного товара, и перед ними встаёт выбор между $n$ различными открытыми стандартами на один и тот же его важный аспект. К примеру, это могут быть разнообразные промышленные роботы и стандарты их интеграции в <<умный>> цех. Когда компания $a \in \{1, \ldots, m\}$ выходит на рынок стандарта $i \in \{1, \ldots, n\}$, она тем самым осуществляет вклад в его развитие, характеризующийся векторной константой $t_i^a \in \mathbb{R}_{\ge 0} \times \ldots \times \mathbb{R}_{\ge 0}$, компоненты которой соответствуют отдельным независимым аспектам (например, функциям, для исполнения которых приобретаются роботы). Если в ситуации $s$ одним стандартом $i$ пользуются несколько компаний, то простым суммированием их вкладов можно посчитать общий индекс развития $t_i(s) = [s^1 = i] t_i^1 + \ldots + [s^m = i] t_i^m$. На ожидаемый доход от инвестиций в каждый из стандартов существенным образом влияют два дисконтирующих фактора: сетевой эффект и насыщение рынка.

By network effect, we mean the dependence of consumer enthusiasm on the general development index of the standard "--- the function $0 \le \alpha^a(t) \le 1$ characterizes the share of buyers who are ready to purchase company $a$ robots made according to the standard with a common development index of $t$. A more developed standard always attracts more consumers, so the functions $\alpha^a(t)$ are monotonically nondecreasing, i.e. $\alpha^a(t) \le \alpha^a(t + \Delta), \forall t, \Delta \succeq (0, \ldots, 0)$. Market saturation, on the other hand, implies limited demand "--- with an excess of investment in any of the standards, the buyers' solvency is no longer enough for everyone, prices have to go down, and revenues go down with them. Accordingly, one more function $0 \le \beta^a(t) \le 1$ characterizes what fraction limits the company $a$ profit, making possible to maintain the competitiveness of its robots in the market of the standard with a common development index $t$. This function, for obvious reasons, is monotonically non-increasing, i.e. $\beta^a(t) \ge \beta^a(t + \Delta), \forall t, \Delta \succeq (0, \ldots, 0)$. Company $a$'s goal in choosing strategy $s^a$ is to maximize the combination of discount factors $u^a(s) = \alpha^a(t_{s^a}(s)) \beta^a(t_{s^a}(s))$. %Под сетевым эффектом мы понимаем зависимость покупательского энтузиазма от общего индекса развития стандарта "--- функция $0 \le \alpha^a(t) \le 1$ характеризует долю покупателей, готовых приобретать роботов компании $a$, выполненных по стандарту с общим индексом развития $t$. Более развитый стандарт всегда привлекает больше потребителей, так что функции $\alpha^a(t)$ монотонно неубывающие, т.е. $\alpha^a(t) \le \alpha^a(t + \Delta), \forall t, \Delta \succeq (0, \ldots, 0)$. Насыщение рынка, с другой стороны, подразумевает ограниченность спроса "--- при избытке инвестиций в любой из стандартов, платёжеспособности покупателей перестаёт хватать на всех, цены приходится снижать, а с ними падают и доходы. Соответственно, ещё одна функция $0 \le \beta^a(t) \le 1$ характеризует, какой долей прибыли придётся ограничиться компании $a$ для сохранения конкурентоспособности своих роботов на рынке стандарта с общим индексом развития $t$. Эта функция, по понятным причинам, монотонно невозрастающая, т.е. $\beta^a(t) \ge \beta^a(t + \Delta), \forall t, \Delta \succeq (0, \ldots, 0)$. Целью компании $a$ при выборе стратегии $s^a$ является максимизация комбинации дисконтирующих факторов $u^a(s) = \alpha^a(t_{s^a}(s)) \beta^a(t_{s^a}(s))$.

One can imagine a political interpretation of the same game. Let $n$ candidates try to be elected to some collegiate elective body (independently, without party lists), and $m$ tycoons choose which of them to campaign for in subordinate institutions. When oligarch $a$ decides to support candidate $i$, he thereby contributes to his popularity, characterized by the vector constant $t_i^a \in \mathbb{R}_{\ge 0} \times \ldots \times \mathbb{R}_{\ge 0}$, whose components correspond to electorally significant demographic groups. If in the situation $s$ candidate $i$ is supported by several oligarchs, then by simply summing up their contributions, one can obtain the overall popularity index of the candidate $t_i(s) = [s^1 = i] t_i^1 + \ldots + [s^m = i] t_i^m$. There are two discounting factors that influence the expected benefit of supporting a particular candidate: political influence and willingness to cooperate. %Можно представить и политическую интерпретацию этой же игры. Пусть в некий коллегиальный выборный орган пытаются избираться $n$ кандидатов (самостоятельно, без партийных списков), а $m$ эффективных менеджеров выбирают, за кого из них развернуть агитацию в подведомственных учреждениях. Когда олигарх $a$ принимает решение о поддержке кандидата $i$, тем самым он вносит вклад в его популярность, характеризующийся векторной константой $t_i^a \in \mathbb{R}_{\ge 0} \times \ldots \times \mathbb{R}_{\ge 0}$, компоненты которой соответствуют электорально значимым демографическим группам. Если в ситуации $s$ кандидата $i$ поддерживают несколько олигархов, то простым суммированием их вкладов можно получить общий индекс популярности кандидата $t_i(s) = [s^1 = i] t_i^1 + \ldots + [s^m = i] t_i^m$. На ожидаемую выгоду от поддержки того или иного кандидата влияют два дисконтирующих фактора: политическое влияние и готовность к сотрудничеству.

The political influence of a candidate in matters of interest to the oligarch who supported him obviously grows along with his overall popularity index, which is expressed by the function $0 \le \alpha^a(t) \le \alpha^a(t + \Delta) \le 1, \forall t, \Delta \succeq (0, \ldots, 0)$. The willingness of a candidate to cooperate with each of his supporters, on the contrary, decreases with the growth of his total popularity, which is expressed by the function $1 \ge \beta^a(t) \ge \beta^a(t + \Delta) \ge 0, \forall t, \Delta \succeq (0, \ldots, 0)$. The goal of oligarch $a$ in choosing strategy $s^a$ is to maximize the combination of discount factors $u^a(s) = \alpha^a(t_{s^a}(s)) \beta^a(t_{s^ a}(s))$. %Политическое влияние кандидата в вопросах, интересующих поддержавшего его олигарха, очевидно, растёт вместе с общим индексом его популярности, что выражается функцией $0 \le \alpha^a(t) \le \alpha^a(t + \Delta) \le 1, \forall t, \Delta \succeq (0, \ldots, 0)$. Готовность же кандидата к сотрудничеству с каждым из своих сторонников, наоборот, падает с ростом его суммарной популярности, что выражается функцией $1 \ge \beta^a(t) \ge \beta^a(t + \Delta) \ge 0, \forall t, \Delta \succeq (0, \ldots, 0)$. Целью олигарха $a$ при выборе стратегии $s^a$ является максимизация комбинации дисконтирующих факторов $u^a(s) = \alpha^a(t_{s^a}(s)) \beta^a(t_{s^a}(s))$.

Let us reduce the description of both conflicts to a matrix game in normal form: %Сведём описание обоих конфликтов к матричной игре в нормальной форме:
\begin{equation*}
	\Gamma = \langle A, S^a, u^a(s), a \in A \rangle;
\end{equation*}
\begin{equation*}
	A = \{1, \ldots, m\}, S^1 = \ldots = S^m = \{1, \ldots, n\};
\end{equation*}
\begin{equation*}
	u^a(s) = \alpha^a(t_{s^a}(s)) \beta^a(t_{s^a}(s)), a = \overline{1,m};
\end{equation*}
\begin{equation*}
	\alpha^a, \beta^a : \mathbb{R}_{\ge 0} \times \ldots \times \mathbb{R}_{\ge 0} \rightarrow [0, 1];
\end{equation*}
\begin{equation*}
	0 \le \alpha^a(t) \le \alpha^a(t + \Delta) \le 1, \forall t, \Delta \succeq (0, \ldots, 0);
\end{equation*}
\begin{equation*}
	1 \ge \beta^a(t) \ge \beta^a(t + \Delta) \ge 0, \forall t, \Delta \succeq (0, \ldots, 0);
\end{equation*}
\begin{equation*}
	t_i(s) = [s^1 = i] t_i^1 + \ldots + [s^m = i] t_i^m, i = \overline{1,n}.
\end{equation*}

It is easy to see the similarity of this game to the task scheduling problem, which we have tried to emphasize here by using the same symbols for variables and constants. In fact, the only significant difference is that in task scheduling, time $t$ was a scalar, not a vector. The promptness reward functions in the new formulations correspond to $v^a(t) = \alpha^a(t) \beta^a(t)$, the form of which determines our expectations from the outcome of the conflict. Earlier we said that the task scheduling problem analysis in general, for arbitrary $(t_i^a)$ and $(v^a)$ is an extremely difficult problem, and, of course, the transition from scalars to vectors in the domain of reward functions doesn't make things any easier. Here, the best we can do is to give a qualitative forecast for some informally described subclasses of conflict based on common sense, intuition and analogy with the special case of $\Gamma^3_n$ analyzed above. %Несложно заметить сходство этой игры с проблемой планирования заданий, что мы здесь постарались подчеркнуть использованием тех же символов для переменных и констант. Фактически, единственным значимым отличием оказывается то, что в планировании заданий время $t$ было скаляром, а не вектором. Функциям оплаты за срочность в новых формулировках соответствуют $v^a(t) = \alpha^a(t) \beta^a(t)$, форма которых и определяет наши ожидания от исхода конфликта. Ранее мы уже говорили, что анализ проблемы планирования заданий в общем виде, для произвольных $(t_i^a)$ и $(v^a)$ представляет собой чрезвычайно сложную задачу, и, конечно же, переход от скаляров к векторам в домене функций отдачи дело нисколько не упрощает. Лучшее, что тут можно сделать "--- дать качественный прогноз для некоторых неформально описанных подклассов конфликта с опорой на здравый смысл, интуицию и аналогию с разобранным выше частным случаем $\Gamma^3_n$.

To avoid overcomplication, we restrict ourselves to the case of quasiconcave reward functions $v^a(t) = \alpha^a(t) \beta^a(t)$, naturally generalizing this concept to multidimensional domains. First, denote by $T^a_{\nearrow}$ the set of all $t$ such that $t_* \prec t \Rightarrow v^a(t_*) \le v^a(t) \wedge t_* \in T^a_{\nearrow}$. Similarly, by $T^a_{\searrow}$ we denote the set of all $t$ such that $t_* \succ t \Rightarrow v^a(t_*) \le v^a(t) \wedge t_* \in T^a_{\searrow}$. These will be regions of continuous non-decreasing and non-increasing of $v^a(t)$, respectively. If these two sets cover the entire applicable domain, i.e. $T^a_{\nearrow} \cup T^a_{\searrow} = \mathbb{R}_{\ge 0} \times \ldots \times \mathbb{R}_{\ge 0}$, then the function $v^a(t)$ is quasiconcave. For similar functions, we can also denote <<ridge>> $T^a_{\sim} = T^a_{\nearrow} \cap T^a_{\searrow}$, which in the one-dimensional case corresponds to the maximum. %Во избежание переусложнения ограничимся случаем квазивогнутых функций отдачи $v^a(t) = \alpha^a(t) \beta^a(t)$, естественным образом обобщив это понятие на многомерные области определения. Для начала обозначим символом $T^a_{\nearrow}$ множество всех таких $t$, что $t_* \prec t \Rightarrow v^a(t_*) \le v^a(t) \wedge t_* \in T^a_{\nearrow}$. Аналогично, символом $T^a_{\searrow}$ обозначим множество всех таких $t$, что $t_* \succ t \Rightarrow v^a(t_*) \le v^a(t) \wedge t_* \in T^a_{\searrow}$. Это будут области непрерывного неубывания и невозрастания $v^a(t)$ соответственно. Если эти два множества покрывают всю область определения, т.е. $T^a_{\nearrow} \cup T^a_{\searrow} = \mathbb{R}_{\ge 0} \times \ldots \times \mathbb{R}_{\ge 0}$, то функция $v^a(t)$ квазивогнута. Для подобных функций также можно обозначить <<гребень>> $T^a_{\sim} = T^a_{\nearrow} \cap T^a_{\searrow}$, в одномерном случае соответствующий максимуму.

Let's try to depict logic of the conflict for the simplest case with a two-element family of plots $\mathfrak{A} = \{A_p, A_q\}, A_p \cap A_q = \emptyset, A_p \cup A_q = A$. We shorten the formulas using the following notation: %Попробуем представить логику конфликта для простейшего случая с двухэлементным семейством заговоров $\mathfrak{A} = \{A_p, A_q\}, A_p \cap A_q = \emptyset, A_p \cup A_q = A$. Сократим запись с помощью следующих обозначений:
\begin{equation*}
	t_i^{A_*} = \sum_{a \in A_*} t_i^a, \forall A_* \subseteq A;
\end{equation*}
\begin{equation*}
	\check{u}^a = \min_{1 \le i \le n} v^a(t_i^A), \hat{u}^a = \max_{1 \le i \le n} v^a(t_i^A), \forall a \in A.
\end{equation*}

Here $t_i^{A_*}$ corresponds to the $i$th total standard development (candidate's popularity) index when it is chosen by the group of players $A_* \subseteq A$. Also, for each player $a$, the base payoff interval is the interval from $\check{u}^a$ to $\hat{u}^a$, i.e. from the smallest to the largest possible payoffs from the unanimous choice of a common strategy by all conflict participants. Consider games restricted by the following preconditions: %Здесь $t_i^{A_*}$ соответствует суммарному индексу развития стандарта (популярности кандидата) $i$ при его выборе группой игроков $A_* \subseteq A$. Так же, для каждого игрока $a$ базовым платёжным интервалом называется промежуток от $\check{u}^a$ до $\hat{u}^a$, т.е. от наименьшего до наибольшего возможных выигрышей при единогласном выборе общей стратегии всеми участниками конфликта. Рассмотрим игры, ограниченные следующими условиями:
\begin{itemize}
	\item $\forall a \in A, i = \overline{1,n}, v^a(t_i^{A_p \cup \{a\}}) < \check{u}^a$, i.e. no player can reach the lower bound of their base payoff interval if only the group $A_p$ chooses the same strategy; %т.е. ни один игрок не может достигнуть нижней границы своего базового платёжного интервала, если ту же стратегию выбирает только группа $A_p$;
	\item $\forall a \in A, i = \overline{1,n}, v^a(t_i^{A_q \cup \{a\}}) > \check{u}^a$, i.e. each player overcomes the lower limit of his base payment interval when choosing any strategy together with the group $A_q$; %т.е. каждый игрок преодолевает нижнюю границу своего базового платёжного интервала при выборе любой стратегии совместно с группой $A_q$;
	\item $\forall a \in A_q, i = \overline{1,n}, t_i^{A_q} \in T^a_{\searrow}$, i.e. for all members of the group $A_q$, the choice of a common strategy brings the total index into the region of non-increasing of the reward function. %т.е. для всех членов группы $A_q$ выбор общей стратегии выводит суммарный индекс в область невозрастания функции отдачи.
\end{itemize}

Thus, we have outlined the range of situations in which the participants of the conflict are divided into two non-overlapping groups of conspirators. From the viewpoint of each conspiracy, assuming that outsiders are not involved in the game at all, it is easy to see that any strategy profile in which the conspirators choose one joint strategy will be a good candidate for Nash equilibrium. This does not mean that there cannot be other equilibria, but for the sake of clarity, we will deliberately restrict ourselves to the analysis of sets characterized by two independent probability distributions $p = (p_1, \ldots, p_n)$ and $q = (q_1, \ldots, q_n)$, where participants in the $A_p$ and $A_q$ conspiracies choose strategies $i = \overline{1,n}$ synchronously within groups but independently between groups with probabilities $p_i$ and $q_i$, respectively, using the appropriate private correlation mechanism. In this case, payments are calculated according to the formula %Таким образом мы очертили круг ситуаций, в которых участники конфликта поделены на две непересекающиеся группы заговорщиков. Если рассматривать происходящее с точки зрения каждого заговора в предположении, что аутсайдеры вообще не участвует в игре, несложно заметить, что любой стратегический набор, в котором заговорщики выбирают одну стратегию на всех, будет хорошим кандидатом в равновесия по Нэшу. Это не означает, что других равновесий не может быть, но мы в целях наглядности сознательно ограничимся анализом наборов, характеризующихся двумя независимыми распределениями вероятностей $p = (p_1, \ldots, p_n)$ и $q = (q_1, \ldots, q_n)$, где участники заговоров $A_p$ и $A_q$ синхронно внутри групп но независимо между группами выбирают стратегии $i = \overline{1,n}$ c вероятностями $p_i$ и $q_i$ соответственно, используя соответствующий приватный механизм корреляции. Выплаты при этом считаются по формуле
\begin{equation*}
	u^a(p, q) = \sum_{i=1}^n p_i ((1 - q_i) v^a(t_i^{A_p}) + q_i v^a(t_i^A)), \forall a \in A_p,
\end{equation*}
\begin{equation*}
	u^a(p, q) = \sum_{i=1}^n q_i ((1 - p_i) v^a(t_i^{A_q}) + p_i v^a(t_i^A)), \forall a \in A_q,
\end{equation*}

According to the preconditions, the $A_p$ cabal is not large enough to maximize the profits of its participants, so each of them would prefer to join the strategy chosen by the $A_q$ cabal. However, since the secret of the hostile conspiracy is not available to the players, they can only deviate from the strategy prescribed by the correlation mechanism in favor of another pure strategy. Thus, to make a profit as a result of an individual deviation from the distribution pair $(p, q)$, it is necessary and sufficient for the conspirator $a \in A_p$ to find such indices $i \neq j \in \{1, \ldots, n\}$ that for $p_i > 0$ the inequation holds%В соответствии с условиями, заговор $A_p$ недостаточно велик для максимизации прибылей своих участников, так что каждый из них предпочёл бы присоединиться к стратегии, избранной заговором $A_q$. Однако, поскольку тайна чужого заговора игрокам не доступна, от стратегии предписанной механизмом корреляции они могут отклониться только в пользу другой чистой стратегии. Таким образом, для получения прибыли в результате индивидуального отклонения от пары распределений $(p, q)$, заговорщику $a \in A_p$ необходимо и достаточно найти такие индексы $i \neq j \in \{1, \ldots, n\}$, что при $p_i > 0$ выполняется неравенство
\begin{equation*}
	(1 - q_i) v^a(t_i^{A_p}) + q_i v^a(t_i^A) < (1 - q_j) v^a(t_j^a) + q_j v^a(t_j^{A_q \cup \{a\}}).
\end{equation*}

It is easy to see that with growth of any $q_j$, the number of indices $i$ for which this inequation holds also gradually increases, and as $q_j$ approaches $1$, sooner or later it starts to hold for all $i \neq j$. By fixing an arbitrary distribution $q$, for each conspirator $a \in A_p$ one can calculate the set $S^a_q \subseteq \{1, \ldots, n\}$, which consists of strategies that allow such productive deviations. At the same time, since the participants in the $A_q$ conspiracy, deviating from the prescribed strategy, inevitably suffer losses, nothing needs to be checked for them. As a result, an arbitrary pair of distributions $(p, q)$ describes a Nash equilibrium if and only if %Несложно заметить, что с ростом любого $q_j$ постепенно увеличивается и кол-во индексов $i$, для которых выполняется это неравенство, а при приближении $q_j$ к $1$ оно рано или поздно начинает выполняться для всех $i \neq j$. Зафиксировав произвольное распределение $q$, можно для каждого заговорщика $a \in A_p$ вычислить множество $S^a_q \subseteq \{1, \ldots, n\}$, состоящее из стратегий, допускающих подобные продуктивные отклонения. При этом, поскольку участники заговора $A_q$, отклонившись от предписанной стратегии, неизбежно терпят убытки, для них проверять ничего не нужно. В результате, произвольная пара распределений $(p, q)$ описывает равновесие по Нэшу тогда и только тогда, когда
\begin{equation*}
	\forall i \in \bigcup_{a \in A_p} S^a_q, p_i = 0.
\end{equation*}

Thus, if we are talking about equilibria only in the classical Nash sense without taking into account collective rationality, then in the described confrontation, the members of a large cabal do not have to think about possible betrayal on the part of comrades-in-arms, while a small cabal must carefully choose a common strategy so that its participants were not tempted to try guessing the strategy chosen by the big one. The desire to ascertain the structural coherency of the indicated solutions gives a slightly more interesting picture. %Таким образом, если мы говорим о равновесиях только в классическом Нэшевском смысле без учёта коллективной рациональности, то в описанном противостоянии участникам большого заговора вообще не приходится думать о возможном предательстве со стороны соратников, тогда как малый заговор должен внимательно выбирать общую стратегию так, чтобы у его участников не было искушения попытаться угадать стратегию, избранную большим. Стремление же удостовериться в структурной согласованности обозначенных решений дают чуть более интересную картину.

Let us make a reservation right away that, within the established constraints, it is difficult to accurately verify the structural coherency even for a narrow set of $(p, q)$"~profiles under consideration, since, for example, collective deviations, dividing the $A_q$ cabal into two groups choosing different strategies, are quite possible. At the same time, one receives profit as a result of getting rid of unnecessary participants (see the constraint $t_i^{A_q} \in T^a_{\searrow}$, i.e., the unanimous choice points belonging to the non-increasing area of ​​the recoil function). The second one potentially increases the income by joining the strategy chosen by the $A_p$ cabal, if there is a large enough $p_i$. Of course, one can try to impose additional restrictions on the parameters of the conflict, preventing such and even more exotic deviations, but this will greatly complicate the formulation without adding too much demonstrativeness. %Сразу оговоримся, что в установленных ограничениях сложно точно убедиться в структурной согласованности даже для узкого множества рассматриваемых $(p, q)$"~наборов, поскольку вполне возможны, например, коллективные отклонения, разделяющие заговор $A_q$ на две группы, выбирающие разные стратегии. Одна при этом получает прибыль в результате избавления от лишних участников (см. ограничение $t_i^{A_q} \in T^a_{\searrow}$, т.е. принадлежность точек единогласного выбора к области невозрастания функции отдачи). Вторая же потенциально увеличивает доход, присоединившись к стратегии, избранной заговором $A_p$, если существует достаточно большое $p_i$. Можно, конечно, попытаться наложить на параметры конфликта дополнительные ограничения, предупреждающие подобные и даже более экзотические отклонения, однако это сильно усложнит постановку, не слишком добавляя иллюстративности.

Instead, we restrict ourselves to searching for only those $(p, q)$"~tuples from which there are no successful collective deviations in favor of other $(p, q)$"~tuples. The found equilibrium points can still be suspected of the lack of structural coherency, but we will at least exclude a large class of obviously inconsistent ones. So, for the profitability of the collective deviation of the small cabal, it suffices to find indices $i \neq j \in \{1, \ldots, n\}$ such that, for $p_i > 0$, for each $a \in A_p$ the following inequation holds: %Вместо этого мы ограничимся поиском только тех $(p, q)$"~наборов, от которых не существует успешных коллективных отклонений в пользу других $(p, q)$"~наборов. Найденные точки равновесия всё ещё можно будет подозревать в отсутствии структурной согласованности, однако мы хотя бы исключим большой класс заведомо несогласованных. Итак, для прибыльности коллективного отклонения малого заговора достаточно найти такие индексы $i \neq j \in \{1, \ldots, n\}$, что при $p_i > 0$ для каждого $a \in A_p$ выполняется неравенство
\begin{equation*}
	(1 - q_i) v^a(t_i^{A_p}) + q_i v^a(t_i^A) < (1 - q_j) v^a(t_j^{A_p}) + q_j v^a(t_j^A).
\end{equation*}

Similarly, for a large cabal, the deviation is successful if there are indices $i \neq j \in \{1, \ldots, n\}$ such that for $q_i > 0$ for each $a \in A_q$ the inequation holds %Аналогично, для большого заговора отклонение успешно, если есть такие индексы $i \neq j \in \{1, \ldots, n\}$, что при $q_i > 0$ для каждого $a \in A_q$ выполняется неравенство
\begin{equation*}
	(1 - p_i) v^a(t_i^{A_q}) + p_i v^a(t_i^A) < (1 - p_j) v^a(t_j^{A_q}) + p_j v^a(t_j^A).
\end{equation*}

Considering these inequations in the light of the constraints we have imposed on the parameters of the conflict, it is easy to see that collective rationality encourages both groups of players to minimize the highest probabilities of choosing individual strategies, but for opposite reasons. It is beneficial for the members of a small cabal to jointly adhere to the strategy chosen by the large cabal, the payments to whose members such an overlap of strategies obviously reduces. Translating into the language of the preestablished interpretations, a weak cartel would gladly take advantage of the standard development (or the candidate standing) selected by a large cartel, but a large cartel, on the contrary, does not relish at the prospect of sharing the limited demand in an already saturated market (or compete for attention of a politician already confident about his election) with unnecessary competitors. %Рассматривая эти неравенства в свете ограничений, наложенных нами на параметры конфликта, несложно заметить, что коллективная рациональность стимулирует обе группы игроков к минимизации наибольших вероятностей выбора отдельных стратегий, однако по противоположным причинам. Участникам малого заговора выгодно сообща присоединиться к стратегии, выбранной большим заговором, выплаты членам которого такое совпадение стратегий заведомо уменьшает. Переводя на язык условленных ранее интерпретаций, слабый картель с удовольствием воспользовался бы развитостью стандарта (или влиятельностью кандидата) избираемого крупным картелем, однако крупному картелю, напротив, совершенно не улыбается делить ограниченный спрос на и без того насыщенном рынке (или соревноваться за внимание и без того уверенного в избрании политика) с лишними конкурентами.

At the level of collectively rational decisions, the game thus becomes a kind of two-sided hide-and-seek, where one group seeks to meet another who is trying to avoid this collision, and conspiracies serve to take advantage of the combined efforts while minimizing the likelihood of undesirable companions joining the profit carve-up. In fact, the formalism of structurally coherent equilibrium in conspiracy games is not some complicated economic concept, but only the embodiment of an intuitive principle that probably had applications even in the preliterate era. It is quite possible that some hunter, noticing a wounded mammoth while walking around the tribal lands, guesstimated: <<Seemingly I won’t overwhelm him alone, but it makes no sense to call the whole tribe. I'd rather whisper in the ear of a couple of friends "--- how much honor and glory it will be to hunt so much meat for only three of us.>> The world history of conspiracies could begin with a reasoning similar to this. %На уровне коллективно рациональных решений игра, таким образом, превращается в разновидность двухсторонних пряток, где одна группа ищет встречи с другой, пытающейся этого столкновения избежать, а заговоры служат для получения преимущества от объединения усилий при минимизации вероятности присоединения к дележу прибылей нежелательных попутчиков. По сути, формализм структурно согласованного равновесия в играх с заговорами является не какой-то сложной экономической концепцией, а всего лишь воплощением интуитивного принципа, применявшегося, вероятно, ещё в дописьменную эпоху. Вполне возможно, что накой-нибудь охотник, заметив раненного мамонта при обходе племенных угодий, рассудил: <<В одиночку я его, пожалуй, не завалю, но и племя всё звать смысла нет. Шепну-ка я лучше на ушко паре друзей "--- это ж сколько почёта и славы будет, втроём столько мяса добыть.>> С похожего рассуждения и могла начаться мировая история заговоров.

\FloatBarrier
