\chapter*{Conclusion}                       % Заголовок
\addcontentsline{toc}{chapter}{Conclusion}  % Добавляем его в оглавление

%% Согласно ГОСТ Р 7.0.11-2011:
%% 5.3.3 В заключении диссертации излагают итоги выполненного исследования, рекомендации, перспективы дальнейшей разработки темы.
%% 9.2.3 В заключении автореферата диссертации излагают итоги данного исследования, рекомендации и перспективы дальнейшей разработки темы.
%% Поэтому имеет смысл сделать эту часть общей и загрузить из одного файла в автореферат и в диссертацию:

The main results of the work are as follows. %Основные результаты работы заключаются в следующем.
\begin{enumerate}
	\item Based on the analysis of the correlated equilibrium concept in the context of multilateral conflicts, the sensitivity of games to additional information asymmetry property was formulated. %На основе анализа понятия коррелированного равновесия в контексте многосторонних конфликтов было сформулировано свойство чувствительности игр к дополнительной информационной асимметрии.
	\item The study of the correlation spaces isomorphism made it possible to narrow them through introduction of the conspiracy space formalism, with the use of which it is convenient to argue about the influence of additional information asymmetry on game solutions. %Исследование изоморфизма пространств корреляции позволило ввести сужающий их формализм пространства заговоров, с применением которого удобно рассуждать о влиянии дополнительной информационной асимметрии на решения игр.
	\item Modeling the task scheduling problem under the assumption of non-monotonicity of the returns functions showed that in conspiracy spaces the concept of structurally coherent equilibria can be used as a functional substitute for the application of the conventional collective rationality criteria to Nash equilibria in mixed strategies. %Моделирование проблемы планирования заданий при допущении немонотонности функций отдачи показало, что в пространствах заговоров концепция структурной согласованности равновесий может использоваться как функциональный аналог классических критериев коллективной рациональности по отношению к равновесиям Нэша в смешанных стратегиях.
	\item To demonstrate the significance of the game sensitivity to additional information asymmetry phenomenon, a model of repeated conflicts was built taking into account the cost of calculating the next step of the strategy. %Для демонстрации значимости феномена чувствительности игр к дополнительной информационной асимметрии была построена модель повторяющихся конфликтов с учётом стоимости вычисления очередного шага стратегии.
	\item Within the framework of the constructed model, it was shown how in repeated games even without actual additional information asymmetry it is possible to use modern cryptographic primitives to construct effective punishment strategies that use sensitivity to it. %В рамках построенной модели было показано, как в повторяющихся играх можно даже без дополнительной информационной асимметрии как таковой использовать современные криптографические примитивы для конструирования эффективных стратегий наказания, использующих чувствительность к ней.
\end{enumerate}
Hopefully, this work will draw the attention of specialists to the problem of the influence of additional information asymmetry on the outcomes of multilateral conflicts. %Хочется надеяться, что эта работа привлечёт внимание специалистов к проблематике влияния дополнительной информационной асимметрии на исходы многосторонних конфликтов.

In conclusion, the author expresses his gratitude and great appreciation to the research advisor Vasin~A.\, A. for support, assistance, discussion of the results and scientific guidance. The author also thanks Morozov~V.\,V. for active participation in the work on proofs of theorems and the authors of the template *Russian-Phd-LaTeX-Dissertation-Template* for their contribution in preparing the dissertation layout. %В заключение автор выражает благодарность и большую признательность научному руководителю Васину~А.\,А. за поддержку, помощь, обсуждение результатов и научное руководство. Также автор благодарит Морозова~В.\,В. за активное участие в работе над доказательствами теорем и авторов шаблона *Russian-Phd-LaTeX-Dissertation-Template* за помощь в оформлении диссертации.
